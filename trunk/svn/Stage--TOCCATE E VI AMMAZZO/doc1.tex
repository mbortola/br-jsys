\documentclass[11pt,titlepage,a4paper]{report}

\usepackage[italian]{babel}
\usepackage{fancyhdr}
\usepackage{graphicx}
\usepackage{hyperref}

\usepackage{lastpage}
\usepackage{color}

\pagestyle{fancy}

\begin{document}  % required; doc starts here
\begin{titlepage}
\begin{center}
\textbf{
\Huge Universita degli studi di Padova\\
\Large Facolta' di scienze MM. FF. NN.\\ 
Laurea triennale in informatica.\\
 Anno Accademico 2007-2008. \\
\Large Tesi di Laurea:\\ 
Utilizzo di XML ed eXist per la gestione di file di configurazione in ambito aziendale.}


\end{titlepage}
\tableofcontents
\chapter{Sommario}
Lo scopo dello stage e' quello di utilizzare XML per rappresentare i file di configurazione utilizzati nel Software Portal Zoom.
Questi file attualmente vengono rappresentati tramite file di testo difficili da interpretare e sopratutto poco manipolabili, dato che la loro lettura e la loro modifica deve per forza avvenire in maniera sequenziale.
Convertendo questi file in formato XML si avra il vantaggio di poterli gestire molto piu facilmente.
Durante lo sviluppo di Portal Zoom questi file di configurazione hanno subito varie modifiche alla struttura e ai contenuti, tramite trasformate XSL e' possibilie aggiornare file di vecchia versione in maniera molto piu efficiente.
Successivamente associeremo a questi file un XML Schema per descriverne la struttura e per validare l'output prodotto, probabilmente affiancheremo alla validazione tramite Schema alcune semplici interrogazioni XPath.
Successivamente proveremo a far gestire questi file da un DBMS XML (nel nostro caso eXist)per avere un incremento nelle operazioni di recupero e manipolazione.
\chapter{Strumenti e tecnologie utilizzate}
\section{Sitepainter}
i-CASE(Internet Computer Aided Software Engineering) Software prodotta dalla Zucchetti che consente lo sviluppo di applicazioni business transazionali in ambiente web, fortemente interattive, con interfaccia browser.

\section{XML}
XML (eXtensible Markup Language) e' un metalinguaggio creato e gestito dalla w3c per la rappresentazione di informazione. e' derivato da una semplificazione di SGML.
\section{XML Schema}
Strunmento per la validazione di un documento XML, alternativa al DTD.
\section{XSL}
XSL (eXstensible Stylesheet Language) e' il linguaggio per la descrizione dei fogli di stile in XML, un documento XSL associato ad un documento XML puo produrre in output qualsiasi tipo di file testuale (HTML, XML, solo testo...).
\section{DOM}
DOM (Document Object Model) � una forma di rappresentazione dei documenti strutturati come modello orientato agli oggetti. Fa parte dello standar w3c per accedere a documenti XML e HTML.In DOM tutto e' trattato come un nodo, in particolare:
\begin{itemize}
\item L'intero documento e' un nodo \textit{Document};
\item Un elemento e' un nodo \textit{Element};
\item Il testo nel elemento XML e' un nodo \textit{Text};
\item Un attributo di un elemento XML e' un nodo \textit{Attribute};
\item Un commento e' un nodo \textit{Comment}.
\end{itemize}
Tramite vista ricorsiva riusciamo a creare e a leggere file XML, DOM viene supportato da diversi linguaggi di programmazione come JAVA, C++, Lisp, PHP, Python, Ruby ecc..

\section{SAX}

SAX offre un modo diverso rispetto al DOM di leggere un documento XML, permette soltanto di leggere XML ma non di creare o modificare.
La lettura di un documento XML da parte di SAX avviene in modo sequenziale, legge carattere per carattere come fosse un semplice file testuale, alla fase di lettura, o meglio di parsing, viene associato un parser che puo' sollevare eventi ogniqualvolta trova :
\begin{itemize}
\item Un tag d'apertura o chiusura;
\item Un tag di inizio o fine documento;
\item Una Processing Instruction;
\item Un errore.
\end{itemize}
Uno dei vantaggi che possiede SAX e' quello di non necessitare di tenere in memoria tutto il documento XML, dunque sarebbe preferibile utilizzarlo per la lettura di grossi file.


\section{JSON}
Acronimo di JavaScript Object Notation, il JSON � un formato adatto per lo scambio dei dati in applicazioni client-server.
Spesso viene usato in AJAXcome alternatica a XML.
E' un formato molto semplice da imparare e da utilizzare, e' supportato praticamente da tutti i linguaggi ciononostante e' indipendente da essi e dalla piattaforma utilizzata.
Supporta i tipi di dato basilari (integer, double, char, string, array, null) e consente l'annidamento di JSON tra di loro, consentando la rappresentazione anche di strutture dati complesse.
La struttura base di un oggetto JSON e' composta da una sequenza non ordinata di coppie nome-valore, nal caso di rappresentazione di un array invece abbiamo soltanto una sequenza ordinata di soli valori.
\section{Xalan}
Xalan e' uno strumento che consente di effettuare trasformate XSL in ambiente JAVA appoggiandosi a SAX e a DOM, e' un servizio fornito da Apache.

\chapter{Lavorare con XSL}
Realizzare XSL e' un operazione che puo' far storcere il naso ad alcuni programmatori.
La verbosita' tipica di XML nonche' il rigore nella definizione dei vari elementi puo' rendere di difficile comprensione il foglio di stile prodotto.
In particolare bisogna assumenre un approccio fortemente ricorsivo e basato su query XPath opportune per la definizione di elementi da elaborare.

\subsection{xsl:key}
\section{XSLT 2.0}
\subsection{xsl:for-each-group}

\chapter{Lavorare con XQuery}

\chapter{Exist}
Exist � un DBMS open source interamente basato su XML, su occupa dunque di gestire grosse moli di documenti XML (ma al suo interno pu� contenere qualsiasi tipo di file) per ottimizzare le interrogazioni, gestire i permessi di accesso e automatizzare molte delle operazioni effettuabili su file XML.
Per esempio lo sviluppo di un sito WEB basato su una base di dati in XML potrebbe venir gestito da eXist immagazzinando in esso la base dati (solitamente composta da pi� file), le varie immagini e applet, nonche i fogli di stile necessari per sviluppare la pagina HTML.
Exist � sviluppato interamente in JAVA, risulta dunque abbastanza semplice la creazione di applicazioni JAVA che si interfacciano con esso.

\section{Gestire le risorse con eXist}
La maggior parte delle operazioni per la gestione dei file pu� essere fatta direttamente tramite XQuery, eXist infatti fornisce alcune funzioni per rendere immediate operazioni che altrimenti richiederebbero una apposita sequenza di chiamante a metodi JAVA.
La funzione store per esempio permette di creare una nuova risorsae e posizionarla dove si vuole col contenuto che si vuole. si possono inoltre gestire le collezioni di risorse creando una struttura ramificata, inserendo , modificando ed eliminando a piacimento.
Inoltre � possibile applicare una trasformata XSLT da XQuery tramite la funzione trasform().
Altri metodi per una gestione specifica del testo, per le funzioni matematiche o per altro vengono definiti nel package util

\section{Altre funzionalit� di eXist}
eXist adotta gran parte delle 
\subsection{XUpdate}
\subsection{Personalizzare Trigger}
\subsection{Personalizzare gli indici}
\chapter{Riferimenti}
\end{document}
