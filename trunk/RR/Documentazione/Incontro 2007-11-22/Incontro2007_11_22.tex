\documentclass[11pt,titlepage,a4paper]{report}

%INCLUSIONE PACCHETTI
%---------------------------------------------
\usepackage[italian]{babel}
\usepackage{fancyhdr}
\usepackage{graphicx}
\graphicspath{{./pics/}}	% cartella di salvataggio immagini

% STILE DI PAGINA
%---------------------------------------------
\pagestyle{fancy}
\renewcommand{\sectionmark}[1]{\markright{\thesection.\ #1}}
\lhead{\nouppercase{\rightmark}}
\rhead{\nouppercase{\leftmark}}
\renewcommand{\chaptermark}[1]{%
\markboth{\thechapter.\ #1}{}}

%Ridefinisco lo stile plain della pagina
\fancypagestyle{plain}{%
	\lhead{\includegraphics[height=50pt]{logo.eps}}
	\chead{}
	\rhead{HappyCode inc \\ happycodeinc@gmail.com}
	\lfoot{BR-jsys}
	\cfoot{\thepage}
	\renewcommand{\headrulewidth}{1pt}
	\renewcommand{\footrulewidth}{1pt}
}
%---------------------------------------------


%INIZIO DOCUMENTO


%---------------------------------------------
% PRIMA PAGINA
%---------------------------------------------
\begin{document}

\begin{titlepage}
\begin{center}
\vspace*{0.5in}
\includegraphics{logo.eps}
\vspace*{0.2in}

{\Large \textbf{BR-jsys}}

{\Large \emph{Business Rules} per sistemi gestionali in architettura J2EE } 
\vspace{1.3in}
\par\rule{10cm}{.4pt} \par
\par\rule{12cm}{1pt} \par
\vspace*{0.5in}
\LARGE \textbf { Verbale della riunione con Gregorio Piccoli presso sede Zucchetti S.r.l. di Padova}
\vspace*{0.5in}
\par\rule{12cm}{1pt} \par
\par\rule{10cm}{.4pt} \par

\end{center}
\end{titlepage}
\vspace*{0.5in}

\thispagestyle{plain}
 \textbf{ Incontro con Gregorio Piccoli presso Sede Zucchetti  }

Data: (22 Novembre 2007) Orario: 15.00 - 17.00

Assenti: nessuno\\



\textbf{Sommario:}
In questo primo incontro si sono analizzate e approfondite le linee guida 
del progetto BR-jsys ponendo attenzione alla comprensione dei concetti principali 
che stanno alla base di un database relazionale e in particolare:

\begin{itemize}

\item Sono stati presentati i componenti dell' HappyCodeInc all' azienda proponente Zucchetti S.r.l.  
\item \'E stato chiarito il concetto Business Object: classe java che fa da intermediaria tra interfaccia e database.
\item Sono stati fatti esempi sull'implementazione di Business Object ( attraverso classi di complessit\'a pi\`u o meno limitata che sono costituite da tipi di dati semplici, array di grandezza uniforme e puntatori ad altri Business Object). 
\item Sono state elencate le principali istruzioni utilizzate nel contesto di utilizzo dei Business Object ( load, save, delete, create e replace ).
\item Chiarimenti sul funzionamento delle Businness Rules : prima di fare una qualsiasi transazione verso il database si deve eseguire un controllo. Questo viene fatto attraverso un check di dominio ( con l'integrit\'a referenziale ), eseguito dal database e un check di applicazione, eseguito dal Business Object.
\item \'E stata esplicitata da parte della Zucchetti S.r.l. la richiesta di implementare Businness Rules dinamiche.
\item  Chiarimento del concetto di API per il progetto in questione.
\item Resoconto sullo stato di sviluppo del progetto da parte della Zucchetti.
\item Sono stati fatti esempi di operazioni eseguite su vettori e scalari con particolare attenzione al fatto che la propriet\'a restava valida sia per le operazioni tra vettori sia per operazioni tra singoli elementi.
\item Utilizzo di check iterativi che consentono di verificare i dati durante il caricamento. 
\item Accenno sul funzionamento della Reflection in Java
\item Elenco delle regole che riguardano le Business Rules ( Rejector,Projector e Producer ) e relativa spiegazione.
\item Elenco dei controlli da fare ( per verificare se i dati inseriti sono scritti bene e se sono coerenti ).
\item Breve accenno sul possibile linguaggio per la validazione delle regole.

\end{itemize}
 
\textbf{Conclusioni:}
\begin{itemize}

\item   L'incontro si \`e rivelato utile per chiarire in modo approssimativo sia le aspettative dell'azienda che i mezzi da utilizzare per soddisfarle.
\item   D'accordo con il proponente si \`e stabilito, a data da destinarsi, un nuovo incontro per valutare lo stato di avanzamento di tal progetto.
\end{itemize}

\end{document}
