\documentclass[11pt,titlepage,a4paper]{report}

%INCLUSIONE PACCHETTI
%---------------------------------------------
\usepackage[italian]{babel}
\usepackage{fancyhdr}
\usepackage{graphicx}
\graphicspath{{./pics/}}	% cartella di salvataggio immagini

% STILE DI PAGINA
%---------------------------------------------
\pagestyle{fancy}
\renewcommand{\sectionmark}[1]{\markright{\thesection.\ #1}}
\lhead{\nouppercase{\rightmark}}
\rhead{\nouppercase{\leftmark}}
\renewcommand{\chaptermark}[1]{%
\markboth{\thechapter.\ #1}{}}

%Ridefinisco lo stile plain della pagina
\fancypagestyle{plain}{%
	\lhead{\includegraphics[height=50pt]{logo.eps}}
	\chead{}
	\rhead{HappyCode inc \\ happycodeinc@gmail.com}
	\lfoot{BR-jsys}
	\cfoot{\thepage}
	\renewcommand{\headrulewidth}{1pt}
	\renewcommand{\footrulewidth}{1pt}
}
%---------------------------------------------


%INIZIO DOCUMENTO


%---------------------------------------------
% PRIMA PAGINA
%---------------------------------------------
\begin{document}

\begin{titlepage}
\begin{center}
\vspace*{0.5in}
\includegraphics{logo.eps}
\vspace*{0.2in}

{\Large \textbf{BR-jsys}}

{\Large \emph{Business Rules} per sistemi gestionali in architettura J2EE } 
\vspace{1.3in}
\par\rule{10cm}{.4pt} \par
\par\rule{12cm}{1pt} \par
\vspace*{0.5in}
\LARGE \textbf {NORME DI PROGETTO}
\vspace*{0.5in}
\par\rule{12cm}{1pt} \par
{\large Versione 0.3 - 5 dicembre 2007}
\par\rule{10cm}{.4pt} \par

\end{center}
\end{titlepage}
\vspace*{0.5in}

%---------------------------------------------
% SECONDA PAGINA
%---------------------------------------------
\begin{center}
\thispagestyle{plain}
\begin{table}[htbp]
\large{
\begin{tabular}{l}
\Large{\textbf{\textsf{Capitolato: ''BR-jsys``}}} \\
\begin{tabular}{||p{6cm}||p{6cm}||}
\hline
\textbf{Data creazione:} & 12/11/07 \\
\hline
\textbf{Versione:} & 0.3 \\
\hline
\textbf{Stato del documento:} & Formale ad uso Interno \\
\hline
\textbf{Redazione:} & Filippo Carraro \\
\hline
\textbf{Revisione:} & Marco Tessarotto \\
\hline
\textbf{Approvazione:}  & Elena Trivellato\\
\hline
\end{tabular} \\
\end{tabular}
}
\end{table}

\begin{table}[hbtp]
\large{
\begin{tabular}{l}
\Large{\textbf{\textsf{Lista di distribuzione}}} \\
\begin{tabular}{||p{6cm}||p{6cm}||}
\hline
{Tutta la HappyCode inc}& Gruppo di lavoro \\
\hline
\end{tabular} \\
\end{tabular}
}
\end{table}

\begin{table}[hbtp]
\large{
\begin{tabular}{l}
\Large{\textbf{\textsf{Diario delle modifiche}}} \\
\begin{tabular}{||p{2cm}||p{3.5cm}||p{6cm}||}
\hline
\textbf{Versione} & \textbf{Data rilascio} & \textbf{Descrizione} \\
\hline
0.3 & 2007/12/03 & Revisione del documento. Modifiche al Comportamento generale e all'uso dello spazio web in Google Groups. \\
\hline
\hline
0.2 & 2007/11/26 & Revisione del documento. Correzione errori ortografici. \\
\hline
\hline
0.1 & 2007/11/12 & Stesura preliminare delle norme per i documenti. \\
\hline
\end{tabular} \\
\end{tabular}

}
\end{table}
\end{center}


%---------------------------------------------
\tableofcontents % Crea l'indice
%---------------------------------------------

\chapter{Introduzione}
\section{Scopo del documento}
Questo documento ha lo scopo di fornire una lista di norme
generali sia per il comportamento dei vari membri del gruppo,
sia per la stesura di tutta la documentazione interna
ed esterna.E' rivolto quindi a tutta la HappyCode inc.

\section{Glossario}
Il glossario viene fornito in un file esterno chiamato \textbf {Glossario\_0\_4.pdf} 

\chapter{Norme di Comportamento}
\section{Uso della Mail del Gruppo}
	La mail verr\`a utilizzata principalmente come mezzo di comunicazione 
	e di informazione tra i vari componenti del gruppo,per ricordare scadenze e  segnalare risorse utili.
	Inoltre sar\`a utilizzata come archivio per file di grosse dimensioni
	che non necessitano di continue modifiche.

\section{Uso dello spazio web in Google Groups}
	Lo spazio web in Google Groups raggiungibile tramite il seguente URL: \({http://groups.google.com/group/happycodeinc}\)
	, riservato ai soli membri del gruppo, i quali possono accedervi 
 dopo essersi autenticati sul proprio account gmail.\\
	Utilizzato per:
	\begin{itemize}
	\item Condivisione dei file necessari alla prossima revisione.
	\item Archiviazione di file e documenti condivisi,o di uso frequente.
	\item Gestione di discussioni su argomenti non chiari, o su possibili modifiche ai documenti.
	\item Esposizione, tramite discussioni, di dubbi sulla correttezza dei documenti.
	\item Aggiunta di pagine statiche contenenti risorse pertinenti i linguaggi utilizzati.
	\item Iniziale stesura di documenti condivisi. 
\end{itemize}

\section{Struttura e Uso Cartella Condivisa}
	La cartella condivisa chiamata \(HappyCodeInc\) conterr\`a al suo interno una cartella
	per ogni revisione da effettuare o effettuata.\\
	All'interno di quest'ultime si avranno, per ognuna,altre sottocartelle riguardanti
	la documentazione,il codice prodotto,ecc...\\
	Ogni documento,le sue versioni e revisioni andranno posti nella cartella corrispondente. 

\section{Comportamento Generale}
	Ad ogni componente del gruppo \`e richiesto, di partecipare alla stesura in
	modalit\`a condivisa mediante le pagine statiche di Google Groups. 
	Tutti i componenti contribuiranno ad ogni documento condiviso 
	nella parte che pi\`u li compete , il documento verr\`a poi trascritto
	all'interno del modello latex.
	Ogni documento deve poi essere caricato  nella cartella condivisa del gruppo,
	sia in formato tex sia in formato pdf.

\chapter{Norme di Documento}
\section{Nomenclatura}
In questo capitolo vengono illustrate le norme per la nomenclatura di file sorgente e di testo.
I nomi dei file devono essere significativi rispetto al loro contenuto ed inoltre:\\
\begin{itemize}
\item possono essere usati solo caratteri dell'alfabeto inglese dalla ``a'' alla ``z'' , minuscole e maiuscole.
\item non devono essere utilizzati catteri speciali.
\item non devono essere utilizzati catteri accentati.
\item i nomi utilizzati non devono contenere spazi.
\item Se il nome da utilizzare e' composto da piu' parole,
queste dovranno essere disposte come in questo esempio: \\
``EsempioDiNomeFile''
\item Alla fine del nome del file deve esserci un codice per la versione 
del documento nella forma :\\ 
``\_a\_b'' \\

dove:\\
a=versione.\\
b=revisione.\\

- La versione, numero intero che parte da 0, verr\`a aggiornata,incrementandola,
ogni volta che verr\`a aggiunta una nuova sezione al documento che tratta argomenti non presenti nel documento.

- La revisione, numero intero che parte da 1, verr\`a incrementata ogni volta che il documento verr\`a modificato 
e corretto, ma solo negli argomenti gi\`a presenti.

\end{itemize}
Esempio di un nome corretto:\\
``NormeDiProgetto\_0\_1''

\section{Impaginazione}
Questo capitolo descrive l'impaginazione dei documenti ad uso interno ed esterno.
Di seguito viene fornita la lista delle regole da seguire per la corretta stesura di un documento.

\subsection{Struttura documento}
Ogni documento \`e composto di queste parti:
\begin{enumerate}
\item Una prima pagina con il logo dell'azienda,titolo del documento,versione e data ultima modifica.
\item Una seconda parte composta da tre tabelle contenti:
	{\begin{itemize}
	\item Data creazione (anno,mese,giorno),versione,stato del documento,\\
			redazione,revisione,approvazione.
	\item Lista di Distribuzione
	\item Diario delle modifiche 
	\end{itemize}}
\item L'indice dei contenuti
\item I contenuti del documento

\end{enumerate}
E' fornito un modello di documento generale \( Modello.tex\),nel quale sono state
applicate le regole date precedentemente. E' consigliato quindi
l'aggiornamento dello stesso,aggiungendo e modificando solo i capitoli necessari.

\subsection{Intestazione/Pi\`e di pagina}
L'intestazione e il pi\`e di pagina vengono generati automaticamente corretti dal file .tex.
con i relativi loghi,il numero della pagina e i titoli dei capitoli.

\subsection{Corpo del documento}
	\subsubsection{Carattere Testo e Titoli}
	Il carattere dei documenti e dei titoli \`e quello di default, scelto nel modello iniziale. 
	Solo il titolo del documento dovr\`a essere messo tutto in maiuscolo.
	Si dovr\`a sottostare a tali scelte,che inoltre rendono il tutto pi\`u leggibile e facile da usare. 
	\subsubsection{Immagini}
	Le immagini inserite nel corpo del documento dovranno essere necessariamente salvate nella cartella pics. Inoltre tutte le immagini contenute nei pdf devono essere in formato vettoriale eps.
\section{Norme di Glossario}
	Il glossario \`e fornito in un documento chiamato \textbf{ Glossario\_a\_b.pdf}. 
	Dove ``a'' indica il numero di versione e ``b'' il numero di revisione, come descritto sopra.
\end{document}

