\documentclass[11pt,titlepage,a4paper]{report}

%INCLUSIONE PACCHETTI
%---------------------------------------------
\usepackage[italian]{babel}
\usepackage{fancyhdr}
\usepackage{graphicx}
\graphicspath{{./pics/}} % cartella di salvataggio immagini

% STILE DI PAGINA
%---------------------------------------------
\pagestyle{fancy}
\renewcommand{\sectionmark}[1]{\markright{\thesection.\ #1}}
\lhead{\nouppercase{\rightmark}}
\rhead{\nouppercase{\leftmark}}
\renewcommand{\chaptermark}[1]{%
\markboth{\thechapter.\ #1}{}}

%Ridefinisco lo stile plain della pagina
\fancypagestyle{plain}{%
	\lhead{\includegraphics[height=50pt]{logo.eps}}
	\chead{}
	\rhead{HappyCode inc \\ happycodeinc@gmail.com}
	\lfoot{BR-jsys}
	\rfoot{\dt - \lv}
	\cfoot{\thepage}
	\renewcommand{\headrulewidth}{1pt}
	\renewcommand{\footrulewidth}{1pt}
}

\begin{document}

%definizione variabili 
\newcommand{\lv}{ 1.5 } % latest version
\newcommand{\dt}{ Piano Di Qualifica }% Document title
\newcommand{\PianoDiProgetto}{ PianoDiProgetto.1.0.pdf }
\newcommand{\Glossario}{ Glossario.1.4.pdf }
%fine definizione variabili


\hyphenation{glos-sa-rio es-pli-ci-to ve-ri-fi-ca-re re-po-si-to-ry se-gna-la-ta coe-ren-za}
\begin{titlepage}\begin{center}
\vspace*{0.5in}
\includegraphics{logo.eps}
\vspace*{0.2in} \\
{\Large \textbf{BR-jsys}}
{\Large \emph{business rules} per sistemi gestionali in architettura J2EE } 
\vspace{2in} \\
\Huge \textsc{ \dt }
\par\rule{10cm}{0.4pt} \par {\large Versione \lv - \today} \\
\end{center}\end{titlepage}
\vspace*{0.5in}



\begin{center}
\thispagestyle{plain}
\begin{table}[htbp]
\large{
\begin{tabular}{l}
\Large{\textbf{\textsf{Capitolato: ''BR-jsys``}}} \\
\begin{tabular}{||p{6cm}||p{6cm}||} \hline
\textbf{Data creazione:} & 19/11/07 \\ \hline
\textbf{Versione:} & \lv \\ \hline
\textbf{Stato del documento:} & Formale, esterno \\ \hline
% ----------------------------------------------------------------------------autori
\textbf{Revisione RR}     \\ \hline
\textbf{Redazione:} & Luca Appon \\ \hline
\textbf{Revisione:} & Marco Tessarotto   \\ \hline
\textbf{Approvazione:}  & Elena Trivellato \\ \hline
\textbf{Revisione RPD}     \\ \hline
\textbf{Redazione:} & Elena Trivellato, Alessia Trivellato \\ \hline
\textbf{Revisione:} & Mattia Meroi, Marco Tessarotto \\ \hline
\textbf{Approvazione:}  & Michele Bortolato \\ \hline

\end{tabular} \\
\end{tabular}
}
\end{table}

\begin{table}[hbtp]
\large{
\begin{tabular}{l}
\Large{\textbf{\textsf{Lista di distribuzione}}} \\
\begin{tabular}{||p{6cm}||p{6cm}||} \hline
{HappyCode inc}& Gruppo di lavoro\\ \hline
{Tullio Vardanega, Renato Conte}& Rappresentanti del committente \\ \hline
{Zucchetti S.r.l}& Azienda committente\\ \hline
\end{tabular} \\
\end{tabular}
}
\end{table}

\begin{table}[hbtp]
\large{
\begin{tabular}{l}
\Large{\textbf{\textsf{Diario delle modifiche}}} \\
\begin{tabular}{||p{2cm}||p{3.5cm}||p{6cm}||}
\hline
%-------------------------------------------------------------------------------diario modifiche
\textbf{Versione} & \textbf{Data rilascio} & \textbf{Descrizione} \\ \hline
1.5 & 2008/02/10 & Aggiunta esiti nella tabella di tracciamento e ulteriore correzione al documento\\ \hline
1.4 & 2008/02/05 & Aggiunta del nome del file nel modello di documento\\ \hline
1.3 & 2008/02/05 & Completamento tabella di tracciamento Requisiti-Test
1.2 & 2008/02/04 & Correzione grammaticale documento
1.1 & 2008/01/28 & Modifica al ``Processo di ispezione''\\ \hline
1.0 & 2008/01/25 & Aggiunta delle tabelle di tracciamento Requisiti-Test\\ \hline
0.6 & 2008/01/24 & Modifica al capitolo ``Pianificazione ed esecuzione del collaudo''\\ \hline
0.5 & 2008/01/23 & Modifiche allo ``Scopo del documento''\\ \hline
0.4 & 2008/01/22 & Modifica al layout dei documenti\\ \hline
0.3 & 2007/12/21 & Documento sottoposto a revisionamento automatico\\ \hline
0.2 & 2007/12/06 & Correzione errori \\ \hline
0.1 & 2007/11/19 & Stesura preliminare del documento \\ \hline
\end{tabular} \\
\end{tabular}

}
\end{table}
\end{center}

\tableofcontents

\chapter{Introduzione}
\section{Scopo del documento}
Nel presente documento illustreremo le strategie di verifica e validazione adottate al fine di garantire la qualit\`a attesa del nostro prodotto. Si utilizzer\`a l'analisi statica, sotto forma di ispezioni del codice e dei documenti, al fine di individuare errori e difetti negli stessi. Verr\`a inoltre utilizzata l'analisi dinamica del codice, sottoforma di test, per soddisfare i requisiti quali funzionalit\`a, affidabilit\`a e usabilit\`a del prodotto ``BR-jsys''.
\begin{itemize}
\item Funzionalit\`a: \newline
Il prodotto soddisfer\`a pienamente i requisiti descritti nel documento di ``Analisi dei Requisiti''.
\item Affidabilit\`a: \newline
Il sistema sar\`a privo di errori in quanto la verifica verr\`a effettuata attraverso opportuni strumenti di controllo (test e ispezioni).
\item Usabilit\`a:
L'utente finale del sistema non dovr\`a necessariamente avere grandi competenze nel campo informatico.
\end{itemize}

\section{Scopo del prodotto}
Il prodotto richiesto verr\`a inserito nell'ambito di un progetto pi\`u ampio. Il suo scopo \`e quello di automatizzare il sistema di validazione dei dati in ingresso al database dell'applicazione principale.

\section{Glossario}
Viene fornito come documento esterno chiamato Glossario.1.4.pdf.
\section{Riferimenti}
\begin{itemize}
\item Capitolato d'appalto concorso per sistema ``BR-jsys''
\item Verbale dell'incontro con il committente ``Incontro2007-11-22.pdf''
\item Verbale dell'incontro con il committente ``Incontro2008-02-05.pdf''
\item ``Ingegneria del Software'' 8a edizione - Ian Sommerville
\item ``Analisi dei Requisiti''
\item ``Norme di Progetto''
\item ``Piano di Progetto''
\end{itemize}

\chapter[Strategia di verifica]{Visione generale della strategia di verifica}
\section[Organizzazione, pianificazione, responsabilit\`a]{Organizzazione, pianificazione strategica e temporale, responsabilit\`a}
La HappyCode inc ha  pianificato delle fasi temporali per definire, sviluppare e validare il prodotto; tali fasi saranno prese in analisi e descritte nel dettaglio nei successivi paragrafi.

\subsection{Ciclo di vita}
Le attivit\`a seguiranno un modello di tipo evolutivo che permetter\`a di intrecciare le attivit\`a di specifica, sviluppo e convalida del software, e di apportare eventuali modifiche ai documenti in tempi diversi. L'obiettivo \`e comprendere al meglio le richieste del cliente e sviluppare dunque una migliore definizione dei requisiti del sistema. Per arrivare ad un prodotto finale sar\`a quindi indispensabile lavorare in stretto contatto con il cliente, sottoponendogli periodicamente versioni parziali o prototipi del prodotto finale. Adotteremo lo sviluppo esplorativo, concentrandoci dapprima sulle parti del sistema che sono ben chiare (requisiti ben compresi) e, solo successivamente verranno aggiunte nuove parti/funzionalit\`a a fronte di chiarimenti da parte del cliente. 
\subsection{Pianificazione delle attivit\`a}
In una prima fase gli analisti discuteranno e cercheranno di comprendere al meglio il problema da risolvere, grazie soprattutto alle varie comunicazioni e incontri con il cliente. Una volta chiariti e consolidati i vari requisiti sar\`a quindi possibile la stesura del documento intitolato ``Analisi dei Requisiti'', che sar\`a la base per la fase di progettazione seguente. Queste attivit\`a potranno essere eseguite in modo parallelo su diverse parti del sistema, come anche le attivit\`a di progettazione e verifica, in modo da consentire la riduzione di tempi e costi di consegna del prodotto. Al termine delle attivit\`a di progettazione seguir\`a quella di programmazione, che produrr\`a il prototipo richiesto. L'ultima fase prima del collaudo sar\`a dedicata ad un attenta verifica di tutto il lavoro svolto nelle precedenti fasi. Il verificatore ed il responsabile di progetto sono le figure alle quali verranno affidate le responsabilit\`a pertinenti alle attivit\`a di verifica. Per una descrizione pi\`u dettagliata della pianificazione si veda il documento ``Piano di Progetto''.

\section{Risorse necessarie, risorse disponibili}
Le attivit\`a di verifica necessitano di risorse umane e tecnologiche. Il gruppo \`e composto da sette membri, ognuno dei quali durante tutto il periodo di lavoro dovr\`a assumere, in periodi di tempo diversi (in base alla disponibilit\`a di ciascuno), tutti i ruoli significativi per lo sviluppo del prodotto. L'amministratore del progetto sar\`a tenuto a supervisionare tutte le fasi di verifica e gestire le varie risorse necessarie per consentire un'attivit\`a di buon livello, senza sprechi od oneri eccessivi. Per la comunicazione tra i componenti \`e stato creato un gruppo Google, accessibile ai soli membri, raggiungibile all'URL: \textit{http://groups.google.com/group/happycodeinc}. Essendo inoltre ogni componente provvisto di un account Gmail, verr\`a utilizzato anche il sistema di chat locale per la comunicazione interattiva. Per l'archiviazione invece dei file verr\`a utilizzato un server Subversion (SVN).

\section{Strumenti, tecniche, metodi}
\subsection{Analisi Statica}
Verr\`a utilizzata durante la stesura del codice sorgente per rilevare errori, omissioni o anomalie, oltre ad incongruenze che possono sorgere tra il progetto ed i requisiti; sar\`a quindi applicata sulla struttura delle varie componenti del sistema ``BR-jsys''. Essa comprende le seguenti sottofasi:
\begin{itemize}
\item \textbf{Analisi del flusso di controllo:} Verificher\`a una corretta esecuzione del codice.
In particolare si controller\`a che non vi siano statement irraggiungibili, ossia istruzioni la cui condizione di accesso non pu\`o mai essere vera.
\item \textbf{Analisi dell'uso dei dati:} Verificher\`a un corretto utilizzo delle variabili.
In particolare si controller\`a che non vi siano variabili utilizzate prima di essere inizializzate, variabili inutilizzate, variabili sempre vere o sempre false.
\item \textbf{Verifica formale del codice:} Verificher\`a la correttezza del codice scritto. In particolare si constater\`a la correttezza totale di ogni unit\`a in modo che non conduca mai in uno stato di non terminazione.
\end{itemize}

\subsection{Analisi Dinamica}
Verr\`a applicata durante la progettazione e la stesura del codice al fine di verificare dinamicamente l'indipendenza delle singole unit\`a rispetto all'integrazione del sistema. Verr\`a testato quindi il sistema in tutti i suoi possibili casi e verranno effettuate prove per verificarne l'integrit\`a. Avverr\`a:
\begin{itemize}
\item Attraverso l'inserimento di nuove regole business in un contesto di prova, effettueremo test sulla validazione e li confronteremo con i risultati attesi.
\item Tramite opportune query di prova effettueremo test sull'utilizzo del DBMS.
\item Attraverso opportuni driver da noi progettati e sviluppati, testeremo il corretto funzionamento del DBMS e del validatore.
\item Per quanto riguarda la Gui, utilizzeremo invece stub 
\end{itemize}

\chapter[Gestione revisione]{Gestione amministrativa della revisione} 
\section{Processo di ispezione}
Le ispezioni del codice e dei documenti verranno eseguite da una squadra di almeno 4 persone che analizzeranno sistematicamente il codice e ne individueranno i possibili difetti. Parteciperanno alla riunione di ispezione: 
\begin{itemize}
\item[-]L'autore del documento ispezionato;
\item[-]i membri del gruppo tenuti a ispezionare il codice;
\item[-]il moderatore capo;
\item[-]il segretario che prender\`a nota degli errori scovati.
\end{itemize}
In una fase iniziale l'autore del codice presenter\`a alla squadra di ispezione il funzionamento dello stesso. Ognuno dei membri della squadra sar\`a poi tenuto a studiare il codice al fine di individuare difetti ed errori. I difetti individuati verranno poi annotati dal segretario durante la riunione di ispezione. 
\section{Comunicazione e risoluzione di anomalie}
Il documento redatto dal segretario dovr\`a elencare le segnalazioni in modo che sia possibile individuare il punto errato in modo semplice e non ambiguo, eventualmente citando le parti non corrette. In dettaglio il documento sar\`a costituito da una tabella contenente:
\begin{itemize}
\item Un descrizione che identifichi in modo univoco l'errore (riga del codice, nome e revisione del file).
\item I passi che hanno portato al verificarsi dell'errore.
\item La gravit\`a dell'errore e il tempo in cui deve essere corretto.
\end{itemize}
L'autore del codice sottoposto a ispezione, preso atto dei difetti individuati nello stesso, sar\`a tenuto a correggerli nei tempi indicati presentando un documento in cui indica le modifiche apportate.

\section{Trattamento delle discrepanze} 
Se dovessero verificarsi discrepanze tra le necessit\`a del cliente ed i requisiti risultanti dall'analisi, si provveder\`a ad un ulteriore analisi che avr\`a la priorit\`a sulle altre attivit\`a. Verr\`a aggiornato quindi il documento relativo all' ``Analisi dei Requisiti'' e  i documenti/pezzi di codice da esso dipendenti. In ogni caso, se fosse necessario un cambiamento, si dovr\`a prima di tutto tracciarne l'impatto sugli altri requisiti e sul progetto del sistema, valutando l'effetto della modifica proposta. Il costo della modifica verr\`a stimato in base a quante modifiche dovrebbero essere fatte al documento dei requisiti e, se opportuno, al progetto del sistema e alla sua implementazione. Completata tale analisi, si dovr\`a decidere se procedere o meno con la modifica. Sar\`a quindi compito dell'analista, in collaborazione col progettista e l'amministratore, valutare questo impatto sul lavoro gi\`a svolto e comunicare ai membri interessati le variazioni.

\section{Procedure di controllo di qualit\`a di processo}
Il controllo della qualit\`a del software interessa l'intero processo di sviluppo del prodotto in questione. A tale scopo sono stati adottati degli standard di documentazione che regolano la struttura e la presentazione dei documenti, nonch\`e degli standard di codifica e di processo. Quest'ultimi definiscono i processi da seguire durante tutto lo sviluppo software; includono definizione dei processi di specifica, progettazione e convalida, oltre ad una descrizione dei documenti che dovrebbero essere scritti durante l'esecuzione di questi processi. Il controllo di qualit\`a prevede il monitoraggio del processo software, per garantire che le procedure e gli standard di assicurazione della qualit\`a siano seguiti. Esso comprende inoltre il miglioramento del processo e la soluzione degli eventuali problemi. Quest'ultimo consiste nel comprendere i processi esistenti e modificarli per aumentare la qualit\`a del software e/o diminuire i costi e i tempi di sviluppo. Il suo obiettivo principale \`e quello di concentrarsi sul perfezionamento, per migliorare la qualit\`a del prodotto e, in particolare, per ridurre il numero di difetti nel software consegnato. Una volta ottenuto ci\`o, gli obiettivi primari diventeranno la riduzione dei costi e dei tempi.
Per avere un esteso controllo degli errori, le modifiche apportate ai documenti, in seguito a inconsistenze riscontrate durante la fase di verifica e stesura, verranno notificate secondo alcune convenzioni interne illustrate nel documento ``NormeDiProgetto''. Queste modifiche si troveranno all'inizio di ogni documento nella sezione ``Diario delle modifiche''. I verificatori sono tenuti a controllare parallelamente la documentazione sul lavoro svolto.

\chapter[Collaudo]{Pianificazione ed esecuzione del collaudo}
\section[Campagna di validazione]{Specifica della campagna di validazione}
L'attivit\`a di collaudo \`e volta a qualificare il prodotto software sviluppato secondo i requisiti da noi identificati. Il risultato di questa attivit\`a \`e determinante per procedere al rilascio del prodotto software. In base ai requisiti verranno applicate diverse tecniche di verifica.
\subsection{Tabella di Tracciabilit\`a}
\begin{tabular}{||p{4.5cm}||p{7.5cm}||}
\hline
\textbf{\textsf{F1}}& \\
\hline
{\textbf {Obiettivo Prova:}}& Creare un linguaggio per le business rules \\ \hline
{\textbf{Dipendenze:}}& nessuna \\ \hline
{\textbf{Descrizione Prova:}}& Rappresentazione di una business rule attraverso il linguaggio creato \\ \hline
{\textbf{Input:}}&  Business rule in linguaggio naturale \\ \hline
{\textbf{Output:}}& Business rule scritta nel linguaggio adottato \\ \hline
{\textbf{Esito:}}&  positivo\\ \hline
\end{tabular} \\
\\
\\
\begin{tabular}{||p{4.5cm}||p{7.5cm}||}
\hline
\textbf{\textsf{F2}}& \\
\hline
{\textbf {Obiettivo Prova:}}& Testare la correttezza sintattica di ogni business rule \\ \hline
{\textbf{Dipendenze:}}& F1 \\ \hline
{\textbf{Descrizione Prova:}}& Inserimento/eliminazione di business rules d'appoggio (sintatticamente corrette o meno), per testarne il loro stato di accettazione \\ \hline
{\textbf{Input:}}& Business Rule di appoggio \\ \hline
{\textbf{Output:}}& La stringa accettata o meno dal validatore \\ \hline
{\textbf{Esito:}}&  positivo\\ \hline
\end{tabular} \\
\\
\\
\begin{tabular}{||p{4.5cm}||p{7.5cm}||}
\hline
\textbf{\textsf{F3}}& \\
\hline
{\textbf {Obiettivo Prova:}}& Testare il corretto inserimento delle business rules nel repository\\ \hline
{\textbf{Dipendenze:}& F2 \\ \hline
{\textbf{Descrizione Prova}}&  Inserimento di varie business rules d'appoggio gi\`a validate e controllo all'interno del repository dell'avvenuto inserimento\\ \hline
{\textbf{Input:}}& Business Rules (validate) di appoggio \\ \hline
{\textbf{Output:}}& Lo stato dell'inserimento delle business rules attraverso la notifica rilasciata (positivo: inserimento avvenuto con successo, negativo: eccezione) \\ \hline
{\textbf{Esito:}}&  positivo\\ \hline
\end{tabular} \\
\\
\\

\begin{tabular}{||p{4.5cm}||p{7.5cm}||}
\hline
\textbf{\textsf{F4}}& \\
\hline
{\textbf {Obiettivo Prova:}}& Informare l'utente dell'avvenuta o meno validazione delle business rules appena inserite e del tempo trascorso, attraverso una notifica il pi\`u possibile chiara ed esaustiva\\ \hline
{\textbf{Dipendenze:}} & F2, F3, NU3 \\ \hline
{\textbf{Descrizione Prova:}}& Inserimento di varie business rules d'appoggio (corrette o meno) e controllo della corretta notifica di risposta alla richiesta di inserimento (attraverso timer nel codice) \\ \hline
{\textbf{Input}}& Business Rules (corrette o meno) di appoggio  \\ \hline
{\textbf{Output}}& Una notifica per ogni ogni business rule che si \`e provato ad inserire, con il relativo messaggio riguardante il tempo impiegato \\ \hline
{\textbf{Esito}}&   \\ \hline
\end{tabular} \\
\\
\\
\begin{tabular}{||p{4.5cm}||p{7.5cm}||}
\hline
\textbf{\textsf{F5}}& \\
\hline
{\textbf {Obiettivo Prova:}}& Testare l'efficienza in termini di tempo di risposta del DBMS di appoggio al repository \\ \hline
{\textbf{Dipendenze:}}& F2, F3, F4\\ \hline
{\textbf{Descrizione Prova:}}& Inserimento/eliminazione/modifica di varie business rules d'appoggio (sintatticamente corrette o meno) e valutazione del tempo di risposta impiegato dal DBMS. Tale valore viene visualizzato dalla Gui. \\ \hline
{\textbf{Input:}}& Business Rules (corrette o meno) di appoggio  \\ \hline
{\textbf{Output:}}& Tempo impiegato \\ \hline
{\textbf{Esito:}}&  \\ \hline
\end{tabular} \\
\\
\\
\begin{tabular}{||p{4.5cm}||p{7.5cm}||}
\hline
\textbf{\textsf{F6}}& \\
\hline
{\textbf {Obiettivo Prova:}}& Il sistema deve essere in grado di interfacciarsi con l'interprete esterno, fornendo ad esso le business rules da eseguire \\ \hline
{\textbf{Dipendenze:}}& nessuna \\ \hline
{\textbf{Descrizione Prova:}}&  Recuperare dal repository tutte le business rules associate ad un determinato business object\\ \hline
{\textbf{Input:}}&  Business object (in stringa) \\ \hline
{\textbf{Output:}}& Le business rules associate\\ \hline
{\textbf{Esito:}}&  \\ \hline
\end{tabular} \\
\\
\\
\begin{tabular}{||p{4.5cm}||p{7.5cm}||}
\hline
\textbf{\textsf{F7}}& \\
\hline
{\textbf {Obiettivo Prova:}}& Testare la correttezza della cancellazione di una business rule dal repository e controllo dei risultati ottenuti\\ \hline
{\textbf{Dipendenze:}}& nessuna \\ \hline
{\textbf{Descrizione Prova:}}&  Richiesta al DBMS di cancellare varie business rules date in input dal repository. Si tester\`a la correttezza dell'eliminazione fatta. Ogni eliminazione riuscita o meno dovr\`a essere accompagnata da una chiara notifica di evento\\ \hline
{\textbf{Input:}}& Business Rules da eliminare \\ \hline
{\textbf{Output:}}& Notifica di eliminazione (positivo: cancellazione avvenuta con successo, negativo: eccezione) \\ \hline
{\textbf{Esito:}}&  negativo \\ \hline
\end{tabular} \\
\\
\\
\begin{tabular}{||p{4.5cm}||p{7.5cm}||}
\hline
\textbf{\textsf{F8}}& \\
\hline
{\textbf {Obiettivo Prova:}}& Creare interfaccia per l'inserimento/rimozione di Business Rules e verificare il corretto funzionamento\\ \hline
{\textbf{Dipendenze:}}& F3, F4, F7 \\ \hline
{\textbf{Descrizione Prova:}}& Si creer\`a  una semplice interfaccia in linguaggio Java, con le componenti essenziali per permettere l' inserimento/rimozione di una business rule nel repository e la successiva notifica di tale evento \\ \hline
{\textbf{Input:}}& Business Rules da eliminare/inserire \\ \hline
{\textbf{Output:}}& Notifica di eliminazione o inserimento (avvenuta o meno) \\ \hline
{\textbf{Esito:}}&  positivo\\ \hline
\end{tabular} \\
\\
\\
\begin{tabular}{||p{4.5cm}||p{7.5cm}||}
\hline
\textbf{\textsf{F9}}& \\
\hline
{\textbf {Obiettivo Prova:}}& REQUISITO DEPRECATO\\ \hline
{\textbf{Dipendenze:}}& \\ \hline
{\textbf{Descrizione Prova:}}& \\ \hline
{\textbf{Input:}}& \\ \hline
{\textbf{Output:}}&  \\ \hline
{\textbf{Esito:}}&  \\ \hline
\end{tabular} \\
\\
\\
\begin{tabular}{||p{4.5cm}||p{7.5cm}||}
\hline
\textbf{\textsf{F10}}& \\
\hline
{\textbf {Obiettivo Prova:}}& Consentire al validatore di accedere ai campi del business object associato alla business rule da validare \\ \hline
{\textbf{Dipendenze:}}& F11\\ \hline
{\textbf{Descrizione Prova:}}&  Inserimento di varie business rules d'appoggio, che si riferiscono ai campi dati del business object associato, e testare la correttezza della validazione effettuata \\ \hline
{\textbf{Input:}}&  Business rules d'appoggio \\ \hline
{\textbf{Output:}}& Esito della validazione \\ \hline
{\textbf{Esito:}}&  \\ \hline
\end{tabular} \\
\\
\\
\begin{tabular}{||p{4.5cm}||p{7.5cm}||}
\hline
\textbf{\textsf{F11}}& \\
\hline
{\textbf {Obiettivo Prova:}}& Testare che ad ogni business rule sia associato un business object \\ \hline
{\textbf{Dipendenze:}}& nessuna \\ \hline
{\textbf{Descrizione Prova:}}&  Nella Gui ogni business rule ha un campo dati per il business object associato. Se tale campo (``associated Object'') viene lasciato vuoto verr\`a lanciata un'eccezione dal sistema\\ \hline
{\textbf{Input:}}&  Business rule di appoggio (con campo dati ``associated Oblect'' pieno o vuoto)\\ \hline
{\textbf{Output:}}& Una eccezione se il campo dati \`e stato lasciato vuoto, altrimenti niente \\ \hline
{\textbf{Esito:}}& positivo \\ \hli{\textbf{Dipendenze:}}& nessuna \\ \hline
\end{tabular} \\
\\
\\
\begin{tabular}{||p{4.5cm}||p{7.5cm}||}
\hline
\textbf{\textsf{NU1}}& \\
\hline
{\textbf {Obiettivo Prova:}}& Creare un linguaggio di alto livello, facile da capire per un utente con scarse conoscenze informatiche\\ \hline
{\textbf{Dipendenze:}}& F1\\ \hline
{\textbf{Descrizione Prova:}}& Creazione di business rules complesse attraverso semplici inserimenti da parte dell'utente\\ \hline
{\textbf{Input:}}& Business rule di appoggio \\ \hline
{\textbf{Output:}}& Business rule scritta nel linguaggio adottato \\ \hline
{\textbf{Esito}}&  \\ \hline
\end{tabular} \\
\\
\\
\begin{tabular}{||p{4.5cm}||p{7.5cm}||}
\hline
\textbf{\textsf{NU2}}& \\
\hline
{\textbf {Obiettivo Prova}}& Il linguaggio deve fornire funzioni primitive per le operazioni base tra i dati, come la somma, la media aritmetica e la negazione logica \\ \hline
{\textbf{Dipendenze:}}& F1 \\ \hline
{\textbf{Descrizione Prova}}&  Creazione di business rules contenenti diverse operazioni base. Se non vengono accettate tali operazioni viene lanciata un'eccezione \\ \hline
{\textbf{Input}}&  Business rule di appoggio contenenti operazioni base \\ \hline
{\textbf{Output}}& Business rule scritta nel linguaggio adottato \\ \hline
{\textbf{Esito}}&  positivo\\ \hline
\end{tabular} \\
\\
\\
\begin{tabular}{||p{4.5cm}||p{7.5cm}||}
\hline
\textbf{\textsf{NU3}}& \\
\hline
{\textbf{Obiettivo Prova}}& Verificare la chiarezza e completezza dei messaggi d'errore\\ \hline
{\textbf{Dipendenze:}}& nessuna \\ \hline
{\textbf{Descrizione Prova}}& Si tenter\`a di validare vari business objects (con errori interni e di ogni genere possibile), valutando l'efficacia del messaggio d'errore{\textbf{Dipendenze:}}& nessuna \\ \hline emesso \\ \hline
{\textbf{Input}}&  Business object (in stringa) \\ \hline %%%%%%%%%%%%%%%%%%%%%%%%%%%% CONTROLLARE
{\textbf{Output}}& Messaggio d'errore comprensibile \\ \hline
{\textbf{Esito}}&  \\ \hline
\end{tabular} \\
\\
\\
\begin{tabular}{||p{4.5cm}||p{7.5cm}||}
\hline
\textbf{\textsf{NPo1}}& \\
\hline
{\textbf{Obiettivo Prova}}& Il formato di salvataggio delle business rules e del repository dovr\`a essere XML.\\ \hline
{\textbf{Dipendenze:}}& nessuna \\ \hline
{\textbf{Descrizione Prova}}&  Creazione e salvataggio di business rules in formato XML  \\ \hline
{\textbf{Input}}&  Business rule di appoggio in XML \\ \hline
{\textbf{Output}}& L'inserimento della business rule in formato XML nel repository \\ \hline
{\textbf{Esito}}&  positivo\\ \hline
\end{tabular} \\
\\
\\
\begin{tabular}{||p{4.5cm}||p{7.5cm}||}
\hline
\textbf{\textsf{NPr1}}& \\
\hline
{\textbf{Obiettivo Prova}}& Testare la velocit\`a di interrogazione del DBMS\\ \hline
{\textbf{Dipendenze:}}& nessuna \\ \hline
{\textbf{Descrizione Prova}}& Popolamento del repository attraverso inserimento di business rules d'appoggio e interazioni con esso. Si valuter\`a l'efficienza, in termini di tempo di risposta, del DBMS usato  \\ \hline
{\textbf{Input}}& Business rule \\ \hline
{\textbf{Output}}& Valutazione del tempo impiegato in ogni interrogazione \\ \hline
{\textbf{Esito}}&  \\ \hline
\end{tabular} \\
\\
\\
\begin{tabular}{||p{4.5cm}||p{7.5cm}||}
\hline
\textbf{\textsf{NPr2}}& \\
\hlines
{\textbf{Obiettivo Prova}}& Rendere univoca l'identificazione delle business rules all'interno del repository \\ \hline
{\textbf{Dipendenze:}}& nessuna \\ \hline
{\textbf{Descrizione Prova}}& Inserimento di varie business rules d'appoggio nel repository e controllo dell'unicit\`a della loro indicizzazione \\ \hline
{\textbf{Input}}& Business object (in stringa) \\ \hline %%%%%%%%%%%%%%%%%%%% CONTROLLATE
{\textbf{Output}}& L'esito del controllo\\ \hline
{\textbf{Esito}}&  negativo \\ \hline
\end{tabular} \\
\\
\\
\begin{tabular}{||p{4.5cm}||p{7.5cm}||}
\hline
\textbf{\textsf{NQ1}}& \\
\hline
{\textbf{Obiettivo Prova}}& Testare tutte le operazioni consentite dal linguaggio creato e la loro validazione\\ \hline
{\textbf{Dipendenze:}}& F1, F3, F7 \\ \hline
{\textbf{Descrizione Prova}}& Si applicheranno a business objects esistenti tutte le operazioni che il linguaggio permette e si valutera\`a l'esito della validazione di tali operazioni, confrontandoli con i risultati aspettati.  \\ \hline
{\textbf{Input}}& Business Objects \\ \hline
{\textbf{Output}}& L'esito della validazione delle operazioni tra i business objects dati in input\\ \hline
{\textbf{Esito}}&  \\ \hline
\end{tabular} \\
\\
\\
\begin{tabular}{||p{4.5cm}||p{7.5cm}||}
\hline
\textbf{\textsf{NQ2}}& \\
\hline
{\textbf {Obiettivo Prova}}& Testare che il manuale descriva il linguaggio di definizione delle business rules \\ \hline
{\textbf{Dipendenze:}}&  F1, F8 \\ \hline
{\textbf{Descrizione Prova}}&  Creazione di un manuale utente di facile comprensione per qualsiasi tipo di utente \\ \hline
{\textbf{Input}}&  Richieste di informazioni \\ \hline  %%%%%%%%%%%%%%%%%%%%%%% CONTROLLATE
{\textbf{Output}}& Manuale utente completo e allo stesso tempo semplice di ogni componente del linguaggio di definizione delle business rules e delle funzioni della Gui \\ \hline
{\textbf{Esito}}&  \\ \hline
\end{tabular} \\
\\
\\
\begin{tabular}{||p{4.5cm}||p{7.5cm}||}
\hline
\textbf{\textsf{NQ3}}& \\
\hline
{\textbf {Obiettivo Prova}}& Descrivere le API relative al validatore e al DBMS\\ \hline
{\textbf{Dipendenze:}}& nessuna \\ \hline
{\textbf{Descrizione Prova}}&  Creazione di una descrizione dettagliata delle API relative al validatore e al DBMS\\ \hline
{\textbf{Input}}&  Richieste di informazioni \\ \hline  %%%%%%%%%%%%%%%%%%%%%% CONTROLLATE
{\textbf{Output}}& Il documento completo e semplice con la descrizione delle API  \\ \hline
{\textbf{Esito}}&  \\ \hline
\end{tabular} \\


\end{document}












