\documentclass[11pt,titlepage,a4paper]{report}

%INCLUSIONE PACCHETTI
%---------------------------------------------
\usepackage[italian]{babel}
\usepackage{fancyhdr}
\usepackage{graphicx}
\graphicspath{{./pics/}} % cartella di salvataggio immagini

% STILE DI PAGINA
%---------------------------------------------
\pagestyle{fancy}
\renewcommand{\sectionmark}[1]{\markright{\thesection.\ #1}}
\lhead{\nouppercase{\rightmark}}
\rhead{\nouppercase{\leftmark}}
\renewcommand{\chaptermark}[1]{%
\markboth{\thechapter.\ #1}{}}

%Ridefinisco lo stile plain della pagina
\fancypagestyle{plain}{%
	\lhead{\includegraphics[height=50pt]{logo.eps}}
	\chead{}
	\rhead{HappyCode inc \\ happycodeinc@gmail.com}
	\lfoot{BR-jsys}
	\cfoot{\thepage}
	\renewcommand{\headrulewidth}{1pt}
	\renewcommand{\footrulewidth}{1pt}
}

\begin{document}

%definizione variabili 
\newcommand{\lv}{1.1} % latest version
%fine definizione variabili


\hyphenation{glos-sa-rio es-pli-ci-to ve-ri-fi-ca-re re-po-si-to-ry se-gna-la-ta coe-ren-za}

\begin{titlepage}
\begin{center}
\vspace*{0.5in}
\includegraphics{logo.eps}
\vspace*{0.2in}

{\Large \textbf{BR-jsys}}
{\Large \emph{business rules} per sistemi gestionali in architettura J2EE } 
\vspace{2in}

\LARGE \textbf {PIANO DI QUALIFICA}
\par\rule{10cm}{0.4pt} \par {\large Versione \lv - \today}


\end{center}
\end{titlepage}
\vspace*{0.5in}

%---------------------------------------------
% SECONDA PAGINA
%---------------------------------------------
\begin{center}
\thispagestyle{plain}
\begin{table}[htbp]
\large{
\begin{tabular}{l}
\Large{\textbf{\textsf{Capitolato: ''BR-jsys``}}} \\
\begin{tabular}{||p{6cm}||p{6cm}||} \hline
\textbf{Data creazione:} & 19/11/07 \\ \hline
\textbf{Versione:} & \lv \\ \hline
\textbf{Stato del documento:} & formale, esterno \\ \hline
% ----------------------------------------------------------------------------autori
\textbf{Redazione:} & Luca Appon \\ \hline
\textbf{Revisione:} & Marco Tessarotto   \\ \hline
\textbf{Approvazione:}  & Elena Trivellato\\ \hline
\end{tabular} \\
\end{tabular}
}
\end{table}

\begin{table}[hbtp]
\large{
\begin{tabular}{l}
\Large{\textbf{\textsf{Lista di distribuzione}}} \\
%  -------------------------------------------------------------lista di distribuzione
\begin{tabular}{||p{6cm}||p{6cm}||} \hline
{Elena Trivellato}& Responsabile di progetto \\ \hline 
{Filippo Carraro}& Amministratore \\ \hline
{Marco Tessarotto}& Verificatore \\ \hline
{Tullio Vardanega}& Committente \\ \hline
{Renato Conte}& Committente \\ \hline
{Zucchetti S.r.l.}& Proponente \\ \hline
\end{tabular} \\
\end{tabular}
}
\end{table}

\begin{table}[hbtp]
\large{
\begin{tabular}{l}
\Large{\textbf{\textsf{Diario delle modifiche}}} \\
\begin{tabular}{||p{2cm}||p{3.5cm}||p{6cm}||}
\hline
\textbf{Versione} & \textbf{Data rilascio} & \textbf{Descrizione} \\ \hline
%-------------------------------------------------------------------------------diario modifiche
1.1 & 22/01/2008 & Modifica al layout dei documenti.\\ \hline
1.0 & 21/12/2007 & Documento sottoposto a revisionamento automatico.\\ \hline
0.2 & 06/12/2007 & Correzione errori. \\ \hline
0.1 & 19/11/2007 & Stesura preliminare del documento. \\ \hline
\end{tabular} \\
\end{tabular}

}
\end{table}
\end{center}

\tableofcontents	%Crea l'indice

\chapter{Introduzione}
\section{Scopo del documento}
Nel presente documento illustreremo le strategie di verifica e validazione adottate al fine di garantire la qualit\`a attesa del nostro prodotto. Si utilizzer\`a l'analisi statica, sotto forma di ispezioni del codice e dei documenti, al fine di individuare errori e difetti negli stessi. Verr\`a inoltre utilizzata l'analisi dinamica del codice, sottoforma di test, per soddisfare i requisiti quali funzionalit\`a, affidabilit\`a e usabilit\`a del prodotto ``BR-jsys''.
\begin{itemize}
\item Funzionalit\`a: \newline
Il prodotto soddisfer\`a pienamente i requisiti descritti nel documento di ``Analisi dei Requisiti''.
\item Affidabilit\`a: \newline
Il sistema sar\`a privo di errori in quanto la verifica verr\`a effettuata attraverso opportuni strumenti di controllo (test e ispezioni).
\item Usabilit\`a:
L'utente finale del sistema non dovr\`a necessariamente avere grandi competenze nel campo informatico.
\end{itemize}

\section{Scopo del prodotto}
Il prodotto richiesto verr\`a inserito nell'ambito di un progetto pi\`u ampio. Il suo scopo \`e quello di automatizzare il sistema di validazione dei dati in ingresso al database dell'applicazione principale.

\section{Glossario}
Viene fornito come documento esterno chiamato Glossario\_0\_4.pdf.

\chapter[Strategia di verifica]{Visione generale della strategia di verifica}

\section[Organizzazione, pianificazione, responsabilit\`a]{Organizzazione, pianificazione strategica e temporale, responsabilit\`a}
\subsection{Ciclo di vita}
Le attivit\`a seguiranno un modello di tipo evolutivo che permetter\`a di apportare, in fasi diverse, eventuali modifiche ai documenti. L'obiettivo \`e comprendere al meglio le richieste del cliente e sviluppare dunque una migliore definizione dei requisiti del sistema. Per arrivare ad un prodotto finale sar\`a quindi indispensabile lavorare in stretto contatto con il cliente, sottoponendogli periodicamente versioni parziali o prototipi del prodotto finale. Si svilupperanno prima le parti del sistema che sono ben chiare (requisiti ben compresi) e, solo successivamente si aggiungeranno nuove parti/funzionalit\`a come proposto dal cliente. 
\subsection{Pianificazione delle attivit\`a}
In una prima fase gli analisti discuteranno e cercheranno di comprendere meglio il problema da risolvere, grazie soprattutto alle varie comunicazioni e incontri con il cliente. Al termine di questa fase sar\`a possibile scrivere quindi l'analisi dei requisiti che sar\`a la base per la fase di progettazione seguente. Queste attivit\`a potranno essere eseguite in modo parallelo su diverse parti del sistema, come anche le attivit\`a di progettazione e verifica, in modo da consentire la riduzione di tempi e costi di consegna del prodotto. Al termine delle attivit\`a di progettazione seguir\`a quella di programmazione, che produrr\`a il prototipo richiesto. L'ultima fase prima del collaudo sar\`a dedicata ad un attenta verifica di tutto il lavoro svolto nelle precedenti fasi. Il verificatore ed il responsabile di progetto sono le figure alle quali verranno affidate le responsabilit\`a pertinenti alle attivit\`a di verifica. Per una descrizione pi\`u dettagliata della pianificazione si veda il documento del PianoDiProgetto\_0\_5.pdf.

\section{Risorse necessarie, risorse disponibili}
Le attivit\`a di verifica necessitano di risorse umane e tecnologiche. Il gruppo \`e composto da sette membri, ognuno dei quali durante tutto il periodo di lavoro dovr\`a assumere, in periodi di tempo diversi (in base alla disponibilit\`a e soprattutto alle competenze di ciascuno) tutti i ruoli significativi per lo sviluppo del progetto. L'amministratore del progetto sar\`a tenuto a supervisionare tutte le fasi di verifica e gestire le varie risorse necessarie per consentire un'attivit\`a di buon livello, senza sprechi od oneri eccessivi. Per la comunicazione tra i componenti \`e stato creato un gruppo Google accessibile ai soli membri del gruppo, raggiungibile all'URL: \textit{http://groups.google.com/group/happycodeinc}. Essendo inoltre ogni componente provvisto di un account Gmail, verr\`a utilizzato anche il sistema di chat locale per la comunicazione interattiva. Per l'archiviazione invece dei file verr\`a utilizzato svn, che permette inoltre anche il versionamento automatico degli stessi.

\section{Strumenti, tecniche, metodi}
\subsection{Analisi Statica}
Questa fase verr\`a applicata durante la stesura del codice sorgente per rilevare incongruenze che possono sorgere tra il progetto e i requisiti. Verr\`a applicata inoltre sulla struttura delle varie componenti del sistema ''BR-jsys''. Essa comprende le seguenti sottofasi:
\begin{itemize}
\item \textbf{Analisi del flusso di controllo:} Verificher\`a una corretta esecuzione del codice  da parte delle varie componenti del prodotto.
In particolare, si controller\`a che:
\begin{itemize}
\item[-]il validatore accetti la regola in ingresso secondo le specifiche del linguaggio scelto;
\item[-]l'esecutore verifichi la consistenza dei dati espressi dagli oggetti business.
\end{itemize}
\item \textbf{Verifica formale del codice:} Verificher\`a la correttezza del codice scritto. Si constater\`a in particolare la correttezza totale di ogni modulo in modo che non conduca mai in uno stato di non terminazione.
\end{itemize}

\subsection{Analisi Dinamica}
Questa fase verr\`a applicata durante la progettazione e la stesura del codice al fine di verificare dinamicamente l'indipendenza delle singole unit\`a rispetto all'integrazione del sistema. Verr\`a testato quindi il sistema in tutti i suoi possibili casi e verranno effettuate prove per verificarne l'integrit\`a.
\begin{itemize}
\item Attraverso l'inserimento di nuove regole business in un contesto di prova, effettueremo test sulla validazione e li confronteremo con i risultati attesi.
\item Tramite opportuni oggetti business di prova effettueremo test sui risultati ottenuti dalla validazione, e li confronteremo con i risultati attesi.
\end{itemize}

\chapter[Gestione revisione]{Gestione amministrativa della revisione} 

\section{Comunicazione e risoluzione di anomalie}
Sar\`a compito dei verificatori la comunicazione delle anomalie e avverr\`a attraverso la stesura di documenti specifici. Costoro dovranno elencare le segnalazioni in modo che sia possibile individuare il punto errato in modo semplice e non ambiguo, eventualmente citando le parti non corrette. Discorso a parte va fatto per errori grammaticali o di formattazione che potranno essere corretti direttamente dai verificatori.

\section{Trattamento delle discrepanze} 
Se dovessero verificarsi discrepanze tra le necessit\`a del cliente ed i requisiti risultanti dall'analisi si provveder\`a ad un ulteriore analisi che avr\`a la priorit\`a sulle altre attivit\`a. Si provveder\`a all'aggiornamento dei documenti di analisi dei requisiti e se necessario ai prodotti delle attivit\`a successive all'analisi. L'analista, in collaborazione col progettista e l'amministratore, valuter\`a l'impatto delle modifiche sul lavoro gi\`a svolto e comunicher\`a ai membri interessati le variazioni.

\section{Procedure di controllo di qualit\`a di processo}
Il controllo della qualit\`a del software interessa l'intero processo di sviluppo del prodotto in questione. Esso comprende il monitoraggio e il miglioramento del processo, la verifica dell'applicazione di standard e procedure prestabiliti nonch\`e la soluzione degli eventuali problemi.
Per avere un esteso controllo degli errori, le modifiche apportate ai documenti, in seguito a inconsistenze riscontrate durante la fase di verifica e stesura, verranno notificate in ogni singolo cambiamento. Queste modifiche si troveranno all'inizio di ogni documento nella sezione ''Diario delle modifiche''. I verificatori sono tenuti a controllare parallelamente la documentazione sul lavoro svolto. Il responsabile di progetto deve verificare che le obiezioni sollevate dal verificatore siano corrette senza che si presentino discrepanze o anomalie. L'approvazione spetta pertanto al responsabile di progetto.

\chapter{Pianificazione ed esecuzione del collaudo}
\section{Specifica della campagna di validazione}
L'attivit\`a di collaudo \`e volta a qualificare il prodotto software sviluppato secondo i requisiti da noi identificati. Il risultato dell'attivit\`a di collaudo \`e determinante per procedere al rilascio del prodotto software. In base ai requisiti verranno applicate diverse tecniche di verifica:
\begin{itemize}
\item Alcuni dei requisiti tracciati dovranno essere validati mediante l'inserimento di un input e la verifica dell esattezza dei dati ottenuti. In particolare i requisiti 3.1.1.1, 3.1.1.2, 3.1.1.3, 3.1.1.4, 3.1.1.5, 3.1.1.6, 3.1.1.8, 3.1.2.1, 3.2.1.1, 3.2.1.4.
\item Altri invece necessitano di una approvazione da parte del committente attraverso incontri. In particolare i requisiti 3.2.1.2, 3.2.1.5, 3.2.1.6, 3.2.2.1, 3.2.2.2, 3.2.3.1, 3.2.3.2.
\item Infine vi sar\`a la necessit\`a di verificare la gestione corretta delle anomalie da parte del progetto. In particolare per il requisito 3.1.1.7.
\end{itemize}
\chapter{Tracciabilit\`a}
\section{Tabella di Tracciabilit\'a}
\begin{tabular}{||p{4.5cm}||p{7.5cm}||}
\hline
\textbf{\textsf{F1}}& \\
\hline
{\textbf {Obbiettivo Prova}}& Creare un linguaggio ad alto livello \\ \hline
{\textbf{Descrizione Prova}}&  \\ \hline
{\textbf{Input}}&  \\ \hline
{\textbf{Output}}& \\ \hline
\end{tabular} \\
\\
\\
\begin{tabular}{||p{4.5cm}||p{7.5cm}||}
\hline
\textbf{\textsf{F2}}& \\
\hline
{\textbf {Obbiettivo Prova}}& Testare la correttezza sintattica di ogni Business Rules \\ \hline
{\textbf{Descrizione Prova}}& Tentativo di inserimento/eliminazione di varie Business Rules d'appoggio (sintatticamente corrette o meno), per testarne il loro stato di accettazione \\ \hline
{\textbf{Input}}& Business Rules di appoggio \\ \hline
{\textbf{Output}}& Lo stato di accettazione delle Business Rules attraverso la notifica rilasciata \\ \hline
\end{tabular} \\
\\
\\
\begin{tabular}{||p{4.5cm}||p{7.5cm}||}
\hline
\textbf{\textsf{F3}}& \\
\hline
{\textbf {Obbiettivo Prova}}& creare un linguaggio ad alto livelloTestare il corretto inserimento delle Business Rules nel repository\\ \hline
{\textbf{Descrizione Prova}}&  Inserimento di varie Business Rules d'appoggio gi\a' validate e controllo all'interno del repository dell'avvenuto inserimento\\ \hline
{\textbf{Input}}& Business Rules (validate) di appoggio \\ \hline
{\textbf{Output}}& Lo stato dell'inserimento delle Business Rules attraverso la notifica rilasciata \\ \hline
\end{tabular} \\
\\
\\
\begin{tabular}{||p{4.5cm}||p{7.5cm}||}
\hline
\textbf{\textsf{F4}}& \\
\hline
{\textbf {Obbiettivo Prova}}& Informare l'utente dell'avvenuta o meno validazione della Business Rule appena inserita, attraverso una notifica il pi\`u possibile chiara ed esaustiva\\ \hline
{\textbf{Descrizione Prova}}& Inserimento di varie Business Rules d'appoggio (corrette o meno), e controllo della corretta notifica di risposta alla richiesta di inserimento \\ \hline
{\textbf{Input}}& Business Rules (corrette o meno) di appoggio  \\ \hline
{\textbf{Output}}& Una notifica per ogni ogni Business Rule che si \`e provato ad inserire \\ \hline
\end{tabular} \\
\\
\\
\begin{tabular}{||p{4.5cm}||p{7.5cm}||}
\hline
\textbf{\textsf{F5}}& \\
\hline
{\textbf {Obbiettivo Prova}}& Testare l'efficienza in termini di tempo di risposta del DBMS di appoggio al repository\\ \hline
{\textbf{Descrizione Prova}}& Tentativo di inserimento/eliminazione/modifica di varie Business Rules d'appoggio (sintatticamente corrette o meno) e valutare il tempo di risposta impiegato dal DBMS \\ \hline
{\textbf{Input}}& Business Rules (corrette o meno) di appoggio  \\ \hline
{\textbf{Output}}& Notifica di inserimento/eliminazione prestando attenzione al tempo impiegato a fornira \\ \hline
\end{tabular} \\
\\
\\
\begin{tabular}{||p{4.5cm}||p{7.5cm}||}
\hline
\textbf{\textsf{F6}}& \\
\hline
{\textbf {Obbiettivo Prova}}& \\ \hline
{\textbf{Descrizione Prova}}&  \\ \hline
{\textbf{Input}}&  \\ \hline
{\textbf{Output}}& \\ \hline
\end{tabular} \\
\\
\\
\begin{tabular}{||p{4.5cm}||p{7.5cm}||}
\hline
\textbf{\textsf{F7}}& \\
\hline
{\textbf {Obbiettivo Prova}}& Tesare la correttezza della cancellazione di una Business Rule dal repository e controllo dei risultati ottenuti\\ \hline
{\textbf{Descrizione Prova}}&  Richiesta al DBMS di cancellare varie Business Rules date come input, dal repository e si tester\`a la correttezza dell'eliminazione svolta. Ogni eliminazione riuscita o meno dovr\`a essere accompagnata da una chiara notifica di evento\\ \hline
{\textbf{Input}}& Business Rules da eliminare \\ \hline
{\textbf{Output}}& Notifica di eliminazione (avvenuta o meno) \\ \hline
\end{tabular} \\
\\
\\
\begin{tabular}{||p{4.5cm}||p{7.5cm}||}
\hline
\textbf{\textsf{F8}}& \\
\hline
{\textbf {Obbiettivo Prova}}& Creare interfaccia per l'inserimento/rimozione/modifica di Business Rules e verificare il corretto funzionamento\\ \hline
{\textbf{Descrizione Prova}}& Si creer\`a  una semplice interfaccia in linguaggio Java, con le componenti essenziali per permettere l' inserimento/rimozione di una Business Rule nel repository e la successiva notifica di tale evento \\ \hline
{\textbf{Input}}& Business Rules da eliminare \\ \hline
{\textbf{Output}}& Notifica di eliminazione (avvenuta o meno) \\ \hline
\end{tabular} \\
\\
\\
\begin{tabular}{||p{4.5cm}||p{7.5cm}||}
\hline
\textbf{\textsf{F9}}& \\
\hline
{\textbf {Obbiettivo Prova}}& Creare una validatore che attesti la coerenza tra Business Rules all'interno del repository \\ \hline
{\textbf{Descrizione Prova}}& Tentativo di inserimento di Business Rules d'appoggio (coerenti o meno con quelle presenti nel repository) e verifica del corretto funzionamento del validatore. Il validatore dovr\`a rifiutare le Business Rules che entrano in conflitto e accettare le altre. Inserimento di varie Business Rules d'appoggio  \\ \hline
{\textbf{Input}}& Business Rules \\ \hline
{\textbf{Output}}& Notifica dello stato di accettazione \\ \hline
\end{tabular} \\
\\
\\
\begin{tabular}{||p{4.5cm}||p{7.5cm}||}
\hline
\textbf{\textsf{F10}}& \\
\hline
{\textbf {Obbiettivo Prova}}& Consentire al validatore di accedere ai campi del Business Object associato alla Business Rule da validare \\ \hline
{\textbf{Descrizione Prova}}&  Inserimento di varie Business Rules d'appoggio che si riferiscono a campi dati di Business Object e testare la correttezza della validazione effettuata \\ \hline
{\textbf{Input}}&  Business Rules d'appoggio \\ \hline
{\textbf{Output}}& Esisto della validazione \\ \hline
\end{tabular} \\
\\
\\
\begin{tabular}{||p{4.5cm}||p{7.5cm}||}
\hline
\textbf{\textsf{F11}}& \\
\hline
{\textbf {Obbiettivo Prova}}& \\ \hline
{\textbf{Descrizione Prova}}&  \\ \hline
{\textbf{Input}}&  \\ \hline
{\textbf{Output}}& \\ \hline
\end{tabular} \\
\\
\\
\begin{tabular}{||p{4.5cm}||p{7.5cm}||}
\hline
\textbf{\textsf{NU1}}& \\
\hline
{\textbf {Obbiettivo Prova}}& \\ \hline
{\textbf{Descrizione Prova}}&  \\ \hline
{\textbf{Input}}&  \\ \hline
{\textbf{Output}}& \\ \hline
\end{tabular} \\
\\
\\
\begin{tabular}{||p{4.5cm}||p{7.5cm}||}
\hline
\textbf{\textsf{NU2}}& \\
\hline
{\textbf {Obbiettivo Prova}}& \\ \hline
{\textbf{Descrizione Prova}}&  \\ \hline
{\textbf{Input}}&  \\ \hline
{\textbf{Output}}& \\ \hline
\end{tabular} \\
\\
\\
\begin{tabular}{||p{4.5cm}||p{7.5cm}||}
\hline
\textbf{\textsf{NU3}}& \\
\hline
{\textbf{Obbiettivo Prova}}& Verificare la chiarezza e completezza dei messaggi d'errore\\ \hline
{\textbf{Descrizione Prova}}& Si tenter\`a di validare vari Business Objects (con errori interni e di ogni genere possibile), valutando l'efficacia del messaggio d'errore emesso \\ \hline
{\textbf{Input}}&  Business Objects \\ \hline
{\textbf{Output}}& Messaggi d'errore\\ \hline
\end{tabular} \\
\\
\\
\begin{tabular}{||p{4.5cm}||p{7.5cm}||}
\hline
\textbf{\textsf{NPo1}}& \\
\hline
{\textbf{Obbiettivo Prova}}& \\ \hline
{\textbf{Descrizione Prova}}&  \\ \hline
{\textbf{Input}}&  \\ \hline
{\textbf{Output}}& \\ \hline
\end{tabular} \\
\\
\\
\begin{tabular}{||p{4.5cm}||p{7.5cm}||}
\hline
\textbf{\textsf{NPr1}}& \\
\hline
{\textbf{Obbiettivo Prova}}& Testare la velocit\`a di interrogazione del DBMS\\ \hline
{\textbf{Descrizione Prova}}& Popolamento del repository attraverso inserimento di Business Rules d'appoggio e attraverso interazioni (query) con esso, si valuter\`a l'efficienza in termini di tempo di risposta, del DBMS usato  \\ \hline
{\textbf{Input}}& Business Rules \\ \hline
{\textbf{Output}}& Valutazione del tempo impiegato di ogni interrogazione \\ \hline
\end{tabular} \\
\\
\\
\begin{tabular}{||p{4.5cm}||p{7.5cm}||}
\hline
\textbf{\textsf{NPr2}}& \\
\hline
{\textbf{Obbiettivo Prova}}& Rendere univoca l'identificazione delle Business Rules all'interno del repository \\ \hline
{\textbf{Descrizione Prova}}& Inserimento di varie Business Rules d'appoggio nel repository, e controllo dell'unicit\`a della loro indicizzazione \\ \hline
{\textbf{Input}}& Business Object \\ \hline
{\textbf{Output}}& L'esito del controllo\\ \hline
\end{tabular} \\
\\
\\
\begin{tabular}{||p{4.5cm}||p{7.5cm}||}
\hline
\textbf{\textsf{NQ1}}& \\
\hline
{\textbf{Obbiettivo Prova}}& Testare tutte le operazioni consentite dal linguaggio creato e la loro validazione\\ \hline
{\textbf{Descrizione Prova}}& Si applicheranno a Business Objects esistenti tutte le operazioni che il linguaggio permette e si valutera\`a l'esito della validazione di tali operazioni, confrontandoli con i risultati aspettati.  \\ \hline
{\textbf{Input}}& Business Objects \\ \hline
{\textbf{Output}}& L'esito della validazione delle operazioni tra i Business Objects dati in input\\ \hline
\end{tabular} \\
\\
\\
\begin{tabular}{||p{4.5cm}||p{7.5cm}||}
\hline
\textbf{\textsf{NQ2}}& \\
\hline
{\textbf {Obbiettivo Prova}}& \\ \hline
{\textbf{Descrizione Prova}}&  \\ \hline
{\textbf{Input}}&  \\ \hline
{\textbf{Output}}& \\ \hline
\end{tabular} \\
\\
\\
\begin{tabular}{||p{4.5cm}||p{7.5cm}||}
\hline
\textbf{\textsf{NQ3}}& \\
\hline
{\textbf {Obbiettivo Prova}}& \\ \hline
{\textbf{Descrizione Prova}}&  \\ \hline
{\textbf{Input}}&  \\ \hline
{\textbf{Output}}& \\ \hline
\end{tabular} \\
\section {Esiti test effettuati}
\begin{tabular}{||p{4.5cm}||p{7.5cm}||}
\hline
\textbf{\textsf{F1}}& \\
\hline
{\textbf {Obbiettivo Prova}}& \\ \hline
{\textbf{Esito}}&  \\ \hline
\end{tabular} \\
\\
\\
\begin{tabular}{||p{4.5cm}||p{7.5cm}||}
\hline
\textbf{\textsf{F2}}& \\
\hline
{\textbf {Obbiettivo Prova}}& \\ \hline
{\textbf{Esito}}&  \\ \hline
\end{tabular} \\
\\
\\
\begin{tabular}{||p{4.5cm}||p{7.5cm}||}
\hline
\textbf{\textsf{F3}}& \\
\hline
{\textbf {Obbiettivo Prova}}& \\ \hline
{\textbf{Esito}}&  \\ \hline
\end{tabular} \\
\\
\\
\begin{tabular}{||p{4.5cm}||p{7.5cm}||}
\hline
\textbf{\textsf{F4}}& \\
\hline
{\textbf {Obbiettivo Prova}}& \\ \hline
{\textbf{Esito}}&  \\ \hline
\end{tabular} \\
\\
\\
\begin{tabular}{||p{4.5cm}||p{7.5cm}||}
\hline
\textbf{\textsf{F5}}& \\
\hline
{\textbf {Obbiettivo Prova}}& \\ \hline
{\textbf{Esito}}&  \\ \hline
\end{tabular} \\
\\
\\
\begin{tabular}{||p{4.5cm}||p{7.5cm}||}
\hline
\textbf{\textsf{F6}}& \\
\hline
{\textbf {Obbiettivo Prova}}& \\ \hline
{\textbf{Esito}}&  \\ \hline
\end{tabular} \\
\\
\\
\begin{tabular}{||p{4.5cm}||p{7.5cm}||}
\hline
\textbf{\textsf{F7}}& \\
\hline
{\textbf {Obbiettivo Prova}}& \\ \hline
{\textbf{Esito}}&  \\ \hline
\end{tabular} \\
\\
\\
\begin{tabular}{||p{4.5cm}||p{7.5cm}||}
\hline
\textbf{\textsf{F8}}& \\
\hline
{\textbf {Obbiettivo Prova}}& \\ \hline
{\textbf{Esito}}&  \\ \hline
\end{tabular} \\
\\
\\
\begin{tabular}{||p{4.5cm}||p{7.5cm}||}
\hline
\textbf{\textsf{F9}} \\
\hline
{\textbf {Obbiettivo Prova}}& \\ \hline
{\textbf{Esito}}&  \\ \hline
\end{tabular} \\
\\
\\
\begin{tabular}{||p{4.5cm}||p{7.5cm}||}
\hline
\textbf{\textsf{F10}} \\
\hline
{\textbf {Obbiettivo Prova}}& \\ \hline
{\textbf{Esito}}&  \\ \hline
\end{tabular} \\
\\
\\
\begin{tabular}{||p{4.5cm}||p{7.5cm}||}
\hline
\textbf{\textsf{F11}} \\
\hline
{\textbf {Obbiettivo Prova}}& \\ \hline
{\textbf{Esito}}&  \\ \hline
\end{tabular} \\
\\
\\
\begin{tabular}{||p{4.5cm}||p{7.5cm}||}
\hline
\textbf{\textsf{NU1}} \\
\hline
{\textbf {Obbiettivo Prova}}& \\ \hline
{\textbf{Esito}}&  \\ \hline
\end{tabular} \\
\\
\\
\begin{tabular}{||p{4.5cm}||p{7.5cm}||}
\hline
\textbf{\textsf{NPo1}} \\
\hline
{\textbf {Obbiettivo Prova}}& \\ \hline
{\textbf{Esito}}&  \\ \hline
\end{tabular} \\
\\
\\
\begin{tabular}{||p{4.5cm}||p{7.5cm}||}
\hline
\textbf{\textsf{NPr1}}& \\
\hline
{\textbf {Obbiettivo Prova}}& \\ \hline
{\textbf{Esito}}&  \\ \hline
\end{tabular} \\
\\
\\
\begin{tabular}{||p{4.5cm}||p{7.5cm}||}
\hline
\textbf{\textsf{NPr2}}& \\
\hline
{\textbf {Obbiettivo Prova}}& \\ \hline
{\textbf{Esito}}&  \\ \hline
\end{tabular} \\
\\
\\
\begin{tabular}{||p{4.5cm}||p{7.5cm}||}
\hline
\textbf{\textsf{NQ1}}& \\
\hline
{\textbf {Obbiettivo Prova}}& \\ \hline
{\textbf{Esito}}&  \\ \hline
\end{tabular} \\
\\
\\
\begin{tabular}{||p{4.5cm}||p{7.5cm}||}
\hline
\textbf{\textsf{NQ1}}& \\
\hline
{\textbf {Obbiettivo Prova}}& \\ \hline
{\textbf{Esito}}&  \\ \hline
\end{tabular} \\
\\
\\
\begin{tabular}{||p{4.5cm}||p{7.5cm}||}
\hline
\textbf{\textsf{NQ2}}& \\
\hline
{\textbf {Obbiettivo Prova}}& \\ \hline
{\textbf{Esito}}&  \\ \hline
\end{tabular} \\
\\
\\
\begin{tabular}{||p{4.5cm}||p{7.5cm}||}
\hline
\textbf{\textsf{NQ3}}& \\
\hline
{\textbf {Obbiettivo Prova}}& \\ \hline
{\textbf{Esito}}&  \\ \hline
\end{tabular} \\
\end{document}












