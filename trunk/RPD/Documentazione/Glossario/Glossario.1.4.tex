\documentclass[11pt,titlepage,a4paper]{report}

%INCLUSIONE PACCHETTI
%---------------------------------------------
\usepackage[italian]{babel}
\usepackage{fancyhdr}
\usepackage{graphicx}
\graphicspath{{./pics/}} % cartella di salvataggio immagini

% STILE DI PAGINA
%---------------------------------------------
\pagestyle{fancy}
\renewcommand{\sectionmark}[1]{\markright{\thesection.\ #1}}
\lhead{\nouppercase{\rightmark}}
\rhead{\nouppercase{\leftmark}}
\renewcommand{\chaptermark}[1]{%
\markboth{\thechapter.\ #1}{}}

%Ridefinisco lo stile plain della pagina
\fancypagestyle{plain}{%
	\lhead{\includegraphics[height=50pt]{logo.eps}}
	\chead{}
	\rhead{HappyCode inc \\ happycodeinc@gmail.com}
	\lfoot{BR-jsys}
	\cfoot{\thepage}
	\renewcommand{\headrulewidth}{1pt}
	\renewcommand{\footrulewidth}{1pt}
}

\begin{document}

%definizione variabili 
\newcommand{\lv}{1.4} % latest version
%fine definizione variabili


\hyphenation{glos-sa-rio es-pli-ci-to ve-ri-fi-ca-re re-po-si-to-ry se-gna-la-ta coe-ren-za}
\begin{titlepage}
\begin{center}
\vspace*{0.5in}
\includegraphics{logo.eps}
\vspace*{0.2in}

{\Large \textbf{BR-jsys}}
{\Large \emph{business rules} per sistemi gestionali in architettura J2EE } 
\vspace{2in}

\LARGE \textbf {GLOSSARIO}
\par\rule{10cm}{0.4pt} \par {\large Versione \lv - \today}


\end{center}
\end{titlepage}
\vspace*{0.5in}

\begin{center}
\thispagestyle{plain}
\begin{table}[htbp]
\large{
\begin{tabular}{l}
\Large{\textbf{\textsf{Capitolato: ''BR-jsys``}}} \\
\begin{tabular}{||p{6cm}||p{6cm}||}
\hline
\textbf{Data creazione:} & 03/12/2007 \\ \hline
\textbf{Versione:} & \lv \\ \hline
\textbf{Stato del documento:} & Formale, esterno \\ \hline
% ----------------------------------------------------------------------------autori
\textbf{Redazione:} & Michele Bortolato\\ \hline
\textbf{Revisione:} & Marco Tessarotto\\ \hline
\textbf{Approvazione:}  & Elena Trivellato\\ \hline
\end{tabular} \\
\end{tabular}
}
\end{table}

\begin{table}[hbtp]
\large{
\begin{tabular}{l}
\Large{\textbf{\textsf{Lista di distribuzione}}} \\
\begin{tabular}{||p{6cm}||p{6cm}||} \hline
%  -------------------------------------------------------------lista di distribuzione
{Elena Trivellato}& Responsabile di progetto \\ \hline 
{Filippo Carraro}& Progettista \\ \hline
{Alessia Trivellato, Michele Bortolato}& Analisti \\ \hline
{Marco Tessarotto}& Verificatore \\ \hline
{Tullio Vardanega, Renato Conte}& Committente \\ \hline 
{Zucchetti S.r.l}& Azienda proponente\\ \hline
\end{tabular} \\
\end{tabular}
}
\end{table}
\begin{table}[hbtp]
\large{
\begin{tabular}{l}
\Large{\textbf{\textsf{Diario delle modifiche}}} \\
\begin{tabular}{||p{2cm}||p{3.5cm}||p{6cm}||} \hline
\textbf{Versione} & \textbf{Data rilascio} & \textbf{Descrizione} \\ \hline
%-------------------------------------------------------------------------------diario modifiche
1.4 & 31/01/2008 & Correzione del documento.\\ \hline
1.3 & 24/01/2008 & Inserimento di Antlr ,eXist, Use Case Diagram, Database, DBMS, JUnit, Stakeholder, Subversion, UML.\\ \hline
1.2 & 22/01/2008 & Modifica al layout dei documenti.\\ \hline
1.1 & 09/01/2008 & Aggiunto Middleware. Cancellati perch\`e non utilizzati i termini Calculator, Java Reflection, Oggetto Master, Ogetto Detail, Oggetto Master Detail, Projector .\\ \hline
1.0 & 21/12/2007 & Documento sottoposto a revisionamento automatico.\\ \hline
0.4 & 2007/12/05 & Correzione del documento. \\ \hline
0.3 & 2007/12/05 & Inserimento documento nel modello unico di layout. \\ \hline
0.2 & 2007/12/04 & Aggiunta nuovi termini. \\ \hline
0.1 & 2007/12/03 & Stesura preliminare del documento. \\ \hline

\end{tabular} \\
\end{tabular}

}
\end{table}
\end{center}
\newpage
\tableofcontents

\chapter{Introduzione}
\section{Scopo del documento}
Il seguente documento ha lo scopo di fornire una breve lista dei termini, usati nei vari documenti, che possono sembrare ambigui oppure necessitano di una definizione o di una migliore spiegazione. In alcuni casi, per maggiori informazioni, verranno messi a disposizione dei collegamenti web.
\section{Riferimenti}
\begin{itemize}
\item http://en.wikipedia.org
\end{itemize}
\chapter{A}
\section{Account Gmail:}
Account personale fornito gratuitamente da Google e attivato da ognuno dei membri della HappyCodeInc.
\section{Antlr:}
``Another Tool for Language Recognition''. \`E uno strumento generatore di parser e traduttori che permette di definire grammatiche nella sintassi Antlr.
\section{Architettura J2EE:}
Vedi J2EE.

\chapter{B}
\section{BR-jsys:}
Per BR-jsys si intende un sistema software per la gestione di business rules in ambito gestionale.
\section{Business Object:}
Anche chiamato Oggetto Business. \`E un oggetto, in un linguaggio di programmazione orientato agli oggetti, che astrae le entit\`a solitamente presenti in un database relazionale.
Per esempio, un acquisto ha bisogno di un acquirente, di un fornitore e di merci di scambio. Questi possono essere rappresentati in un database relazionale tramite entit\`a o in un programma sviluppato con un linguaggio orientato agli oggetti(come java o C++) tramite business objects.
\section{Business rule:}
Anche chiamata Regola Business. \`E una regola atta a definire dei vincoli particolari all'interno della gestione di un azienda. Un esempio di Business Rule potrebbe essere questo: ``a tutti i prodotti con prezzo superiore a 600 euro  viene applicato uno sconto dell' 5\%''.

\chapter{D}
\section{Database:}
Archivio di dati riguardanti uno stesso argomento o pi\`u argomenti correlati tra loro, strutturato in modo tale da consentire la gestione dei dati stessi da parte di applicazioni software.
\section{DBMS:}
``Database Management System''. \`E un sistema software progettato per consentire la creazione e la manipolazione efficiente di database, solitamentente da parte di pi\`u utenti.
\section{Discussioni:}
Risorsa disponibile su Google Groups. Permette lo scambio di pareri (similmente a un forum) all'interno del gruppo stesso.
\section{Driver:}
Componente attiva fittizia che serve per pilotare un modulo.

\chapter{E}
\section{eXists:}
Comando SQL che specifica una subquery per verificare l'esistenza di righe. Usa una sottoquery come condizione: la condizione è Vera se la sottoquery ritorna almeno una riga, ed è Falsa se la sottoquery non ritorna nessuna riga.

\chapter{G}
\section{Google Groups:}
Vedi Spazio web in Google Groups. 

\chapter{J}
\section{JUnit:}
\`E un unit test framework per il linguaggio di programmazione Java.
\section{J2EE:}
Il termine J2EE \`e l'acronimo di Java 2 Enterprise Edition, ossia la versione enterprise della piattaforma Java utilizzata per lo sviluppo di applicazioni server.

\chapter{M}
\section{Mail del gruppo:}
L'indirizzo di posta elettronica \textit{happycodeinc@gmail.com} creata appositamente come supporto allo sviluppo del progetto.
\section{Middleware:}
Programma informatico che funge da intermediario tra diverse applicazioni.

\chapter{O}
\section{Oggetto Business:}
Vedi Business Object.

\chapter{P}
\section{Pagine Statiche:}
Risorsa disponibile su Google Groups. Permette di definire documenti modificabili e consultabili da parte di tutto il gruppo.

\chapter{R}
\section{Regola Business:}
Vedi Business Rule.
\section{Rejector:}
Tipologia di business rule che consente di bloccare un operazione se i dati analizzati non sono accettabili.
\section{Repository:} 
Per repository si intende un archivio (nel nostro caso un file in linguaggio XML) in cui vengono memorizzati e mantenuti dei dati.
\section{Revisione}
La revisione, numero intero che parte da 1, indica se il documento \`e stato aggiornato dal suo ultimo rilascio. Un aumento nell'indice di revisione indica che il documento \`e stato modificato e/o corretto in una delle sue parti.

\chapter{S}
\section{Spazio web in Google Groups:}
Insieme delle risorse messe a disposizione da Google Groups. \`E raggiungibili dai soli membri della HappyCodeInc all'URL\\ \textit{http://groups.google.com/group/happycodeinc}. \`E comprensivo delle seguenti risorse: spazio web per la condivisione di file, gestione delle discussioni, gestione delle pagine statiche.
\section{Stakeholder:}
Soggetti ``portatori di interessi'' nei confronti di un'iniziativa economica, sia essa un'azienda o un progetto. Fanno, ad esempio, parte di questo insieme: i clienti, i fornitori, i finanziatori, i collaboratori, ma anche gruppi di interesse esterni, come i residenti di aree limitrofe all'azienda o gruppi di interesse locali.
\section{Stub:}
Componente passiva fittizia per simulare un modulo.
\section{Svn:}
``Subversion'' \`e un sistema di controllo versione automatico. 

\chapter{U}
\section{UML:}
``Unified Modeling Language'' (linguaggio di modellazione unificato). \`E un linguaggio di modellazione e specifica basato sul paradigma object-oriented.
\section{Use Case Diagram:}
In UML, gli Use Case Diagram (UCD o diagrammi dei casi d'uso) sono diagrammi dedicati alla descrizione delle funzioni o servizi offerti da un sistema, cos\`i come sono percepiti e utilizzati dagli attori che interagiscono col sistema stesso. 

\chapter{V}
\section{Versione:}
La versione, numero intero che parte da 0, serve a stabilire se il documento \`e stato modificato dal suo ultimo rilascio. Un aumento nell'indice della versione indica l'aggiunta di una nuova sezione al documento.

\end{document}
