\documentclass[11pt,titlepage,a4paper]{report}

%INCLUSIONE PACCHETTI
%---------------------------------------------
\usepackage[italian]{babel}
\usepackage{fancyhdr}
\usepackage{graphicx}
\graphicspath{{./pics/}} % cartella di salvataggio immagini

% STILE DI PAGINA
%---------------------------------------------
\pagestyle{fancy}
\renewcommand{\sectionmark}[1]{\markright{\thesection.\ #1}}
\lhead{\nouppercase{\rightmark}}
\rhead{\nouppercase{\leftmark}}
\renewcommand{\chaptermark}[1]{%
\markboth{\thechapter.\ #1}{}}

%Ridefinisco lo stile plain della pagina
\fancypagestyle{plain}{%
	\lhead{\includegraphics[height=50pt]{logo.eps}}
	\chead{}
	\rhead{HappyCode inc \\ happycodeinc@gmail.com}
	\lfoot{BR-jsys}
	\cfoot{\thepage}
	\renewcommand{\headrulewidth}{1pt}
	\renewcommand{\footrulewidth}{1pt}
}

\begin{document}

%definizione variabili 
\newcommand{\lv}{0.4} % latest version
%fine definizione variabili




\hyphenation{glos-sa-rio es-pli-ci-to ve-ri-fi-ca-re re-po-si-to-ry se-gna-la-ta coe-ren-za}
\begin{titlepage}
\begin{center}
\vspace*{0.5in}
\includegraphics{logo.eps}
\vspace*{0.2in}

{\Large \textbf{BR-jsys}}
{\Large \emph{business rules} per sistemi gestionali in architettura J2EE } 
\vspace{2in}

\LARGE \textbf {GLOSSARIO}
\par\rule{10cm}{0.4pt} \par {\large Versione \lv - \today}


\end{center}
\end{titlepage}
\vspace*{0.5in}

\begin{center}
\thispagestyle{plain}
\begin{table}[htbp]
\large{
\begin{tabular}{l}
\Large{\textbf{\textsf{Capitolato: ''BR-jsys``}}} \\
\begin{tabular}{||p{6cm}||p{6cm}||}
\hline
\textbf{Data creazione:} & 03/12/2007 \\ \hline
\textbf{Versione:} & \lv \\ \hline
\textbf{Stato del documento:} & Formale, esterno \\ \hline
% ----------------------------------------------------------------------------autori
\textbf{Redazione:} & Michele Bortolato\\ \hline
\textbf{Revisione:} & Marco Tessarotto\\ \hline
\textbf{Approvazione:}  & Elena Trivellato\\ \hline
\end{tabular} \\
\end{tabular}
}
\end{table}

\begin{table}[hbtp]
\large{
\begin{tabular}{l}
\Large{\textbf{\textsf{Lista di distribuzione}}} \\
\begin{tabular}{||p{6cm}||p{6cm}||} \hline
%  -------------------------------------------------------------lista di distribuzione
{Elena Trivellato}& Responsabile di progetto \\ \hline 
{Filippo Carraro}& Progettista \\ \hline
{Alessia Trivellato, Michele Bortolato}& Analisti \\ \hline
{Marco Tessarotto}& Verificatore \\ \hline
{Tullio Vardanega, Renato Conte}& Committente \\ \hline 
{Zucchetti S.r.l}& Azienda proponente\\ \hline
\end{tabular} \\
\end{tabular}
}
\end{table}
\begin{table}[hbtp]
\large{
\begin{tabular}{l}
\Large{\textbf{\textsf{Diario delle modifiche}}} \\
\begin{tabular}{||p{2cm}||p{3.5cm}||p{6cm}||} \hline
\textbf{Versione} & \textbf{Data rilascio} & \textbf{Descrizione} \\ \hline
%-------------------------------------------------------------------------------diario modifiche
2.0 & 21/12/2007 & Documento sottoposto a revisionamento automatico.\\ \hline
0.4 & 2007/12/05 & Correzione del documento. \\ \hline
0.3 & 2007/12/05 & Inserimento documento nel modello unico di layout. \\ \hline
0.2 & 2007/12/04 & Aggiunta nuovi termini. \\ \hline
0.1 & 2007/12/03 & Stesura preliminare del documento. \\ \hline

\end{tabular} \\
\end{tabular}

}
\end{table}
\end{center}
\newpage

\chapter{Scopo del documento}
Il seguente documento ha lo scopo di fornire una breve lista dei termini usati nei vari documenti che possono sembrare ambigui oppure necessitano di una definizione o di una migliore spiegazione. In alcuni casi vengno messi disposizione dei collegamenti web per chi necessitasse di ulteriori informazioni.
\chapter{Elenco dei termini}
\begin{itemize}

\item{\textbf{Account Gmail:}
Per account Gmail si intende l'account personale fornito gratuitamente da Google e attivato da ognuno dei membri della HappyCode Inc.}

\item{\textbf{Architettura J2EE:}
Vedi J2EE.}

\item{\textbf{BR-jsys:}
Per BR-jsys si intende un sistema software per la gestione di business rules in ambito gestionale.}

\item{\textbf{Business Object:}
Anche chiamato Oggetto Business, \`e un oggetto, in un linguaggio di programmazione orientato agli oggetti, che astrae le entit\`a solitamente presenti in un database relazionale.
Per esempio, un acquisto ha bisogno di un acquirente, di un fornitore e di merci di scambio, entrambe possono essere rappresentate in un database relazionale tramite entit\`a o in un programma sviluppato con un linguaggio orientato agli oggetti(come java o C++) tramite business objects.}

\item{\textbf{Business rule:}
Una business rule o Regola Business \`e una regola atta a definire dei vincoli particolari all'interno della gestione di un azienda. Ad esempio una Business Rule potrebbe essere questa: A tutti i prodotti con prezzo superiore a 600 euro  viene applicato uno sconto dell' 5\%.}

\item{\textbf{Calculator:}
Tipologia di business rule che fornisce un valore elaborando i dati presenti.}

\item{\textbf{Discussioni:}
Con questo termine ci si riferisce ad una delle risorse disponibili su Google Groups, la quale permette lo scambio di pareri (similmente a un forum) all'interno del gruppo stesso.}

\item{\textbf{Driver:}
Componente attiva fittizia per pilotare un modulo.}

\item{\textbf{Google Groups:}
Vedi Spazio web in Google Groups. }

\item{\textbf{J2EE:}
Il termine J2EE \`e l'acronimo di Java 2 Enterprise Edition, Ossia la versione enterprise della piattaforma Java, utilizzata per lo sviluppo di applicazioni server.}

\item{\textbf{Java Reflection:}
Caratteristica del linguaggio Java che permette di ottenere informazioni su classi e interfacce direttamente in fase di esecuzione. In particolare, \`e possibile determinare la classe a cui un oggetto appartiene, i suoi attributi, i costruttori e i metodi.
Per maggiori informazioni si rimanda al seguente link:  \begin{small}\textit{java.sun.com/javase/6/docs/api/java/lang/reflect/package-summary.html}\end{small}}

\item{\textbf{Mail del Gruppo:}
L'indirizzo di posta elettronica \textit{happycodeinc@gmail.com} creata appositamente come supporto allo sviluppo del progetto.}

\item{\textbf{Modulo:}
Componente collaudabile del software.}

\item{\textbf{Oggetto Business:}
Vedi Business Object.}

\item{\textbf{Oggetto Master:}
Oggetto che esprime in un ambiente di business rules ci\`o che in ambito relazionale \`e reappresentato da  una tabella radice che non ha vincoli di chiavi esterne.}

\item{\textbf{Oggetto Detail:}
Oggetto che esprime in un ambiente di business rules ci\`o che in ambito relazionale \`e reappresentato da  una tabella radice che ha vincoli di chiavi esterne.}

\item{\textbf{Oggetto Master/Detail:}
Tabella in ambito relazionale che possiede sottocampi composti da altre tabelle.}

\item{\textbf{Pagine Statiche:}
Con questo termine ci si riferisce ad una delle risorse disponibili su Google Groups,  la quale permette di definire documenti modificabili e consultabili da parte di tutto il gruppo.}

\item{\textbf{Projector:}
Tipologia di business rule che consente di lanciare una procedura all'accadere di un evento in un altra procedura.}

\item{\textbf{Regola Business:}
Vedi Business Rule.}

\item{\textbf{Rejector:}
Tipologia di business rule che consente di bloccare un operazione se i dati analizzati non sono accettabili.}
 
\item{\textbf{Repository:}
Per repository si intende un archivio (nel nostro caso un file in linguaggio XML) in cui vengono memorizzati e mantenuti dei dati.}

\item{\textbf{Revisione:}
La revisione, numero intero che parte da 1,  indica se il documento \`e stato aggiornato dal suo ultimo rilascio. Un aumento nell'indice di revisione indica che il documento \`e stato modificato e/o corretto in una dele sue parti}.

\item{\textbf{Stub:}
Componente passiva fittizia per simulare un modulo.}

\item{\textbf{Spazio web in Google Groups:}
Per spazio web in Google Groups si sintende l'insieme delle risorse messe a disposizione da Google Groups, raggiungibili dai soli membri della HappyCode Inc all'URL\\ \textit{http://groups.google.com/group/happycodeinc} e comprensivo delle seguenti risorse: spazio web per la condivisione di file, gestione delle discussioni, gestione delle pagine statiche.}

\item{\textbf{Versione:}
La versione, numero intero che parte da 0, serve a stabilire se il documento \`e stato modificato dal suo ultimo rilascio. Un aumento nell'indice della versione indica l'aggiunta di  una nuova sezione al documento.}

\end{itemize}
\newpage
\tableofcontents
\end{document}
