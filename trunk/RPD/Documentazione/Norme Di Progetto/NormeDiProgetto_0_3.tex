\documentclass[11pt,titlepage,a4paper]{report}

%INCLUSIONE PACCHETTI
%---------------------------------------------
\usepackage[italian]{babel}
\usepackage{fancyhdr}
\usepackage{graphicx}
\graphicspath{{./pics/}}	% cartella di salvataggio immagini

% STILE DI PAGINA
%---------------------------------------------
\pagestyle{fancy}
\renewcommand{\sectionmark}[1]{\markright{\thesection.\ #1}}
\lhead{\nouppercase{\rightmark}}
\rhead{\nouppercase{\leftmark}}
\renewcommand{\chaptermark}[1]{%
\markboth{\thechapter.\ #1}{}}

%Ridefinisco lo stile plain della pagina
\fancypagestyle{plain}{%
	\lhead{\includegraphics[height=50pt]{logo.eps}}
	\chead{}
	\rhead{HappyCode inc \\ happycodeinc@gmail.com}
	\lfoot{BR-jsys}
	\cfoot{\thepage}
	\renewcommand{\headrulewidth}{1pt}
	\renewcommand{\footrulewidth}{1pt}
}
%---------------------------------------------


%INIZIO DOCUMENTO


%---------------------------------------------
% PRIMA PAGINA
%---------------------------------------------
\begin{document}

\begin{titlepage}
\begin{center}
\vspace*{0.5in}
\includegraphics{logo.eps}
\vspace*{0.2in}

{\Large \textbf{BR-jsys}}

{\Large \emph{Business Rules} per sistemi gestionali in architettura J2EE } 
\vspace{1.3in}
\par\rule{10cm}{.4pt} \par
\par\rule{12cm}{1pt} \par
\vspace*{0.5in}
\LARGE \textbf {NORME DI PROGETTO}
\vspace*{0.5in}
\par\rule{12cm}{1pt} \par
{\large Versione 0.3 - 5 dicembre 2007}
\par\rule{10cm}{.4pt} \par

\end{center}
\end{titlepage}
\vspace*{0.5in}

%---------------------------------------------
% SECONDA PAGINA
%---------------------------------------------
\begin{center}
\thispagestyle{plain}
\begin{table}[htbp]
\large{
\begin{tabular}{l}
\Large{\textbf{\textsf{Capitolato: ``BR-jsys''}}} \\
\begin{tabular}{||p{6cm}||p{6cm}||}
\hline
\textbf{Data creazione:} & 12/11/07 \\
\hline
\textbf{Versione:} & 0.3 \\
\hline
\textbf{Stato del documento:} & Formale ad uso Interno \\
\hline
\textbf{Redazione:} & Filippo Carraro \\
\hline
\textbf{Revisione:} & Marco Tessarotto \\
\hline
\textbf{Approvazione:}  & Elena Trivellato\\
\hline
\end{tabular} \\
\end{tabular}
}
\end{table}

\begin{table}[hbtp]
\large{
\begin{tabular}{l}
\Large{\textbf{\textsf{Lista di distribuzione}}} \\
\begin{tabular}{||p{6cm}||p{6cm}||}
\hline
{Tutta la HappyCode inc}& Gruppo di lavoro \\
\hline
\end{tabular} \\
\end{tabular}
}
\end{table}

\begin{table}[hbtp]
\large{
\begin{tabular}{l}
\Large{\textbf{\textsf{Diario delle modifiche}}} \\
\begin{tabular}{||p{2cm}||p{3.5cm}||p{6cm}||}
\hline
\textbf{Versione} & \textbf{Data rilascio} & \textbf{Descrizione} \\
\hline
0.3 & 2007/12/03 & Revisione del documento. Modifiche al Comportamento generale e all'uso dello spazio web in Google Groups. \\
\hline
\hline
0.2 & 2007/11/26 & Revisione del documento. Correzione errori ortografici. \\
\hline
\hline
0.1 & 2007/11/12 & Stesura preliminare delle norme per i documenti. \\
\hline
\end{tabular} \\
\end{tabular}

}
\end{table}
\end{center}


%---------------------------------------------
\tableofcontents % Crea l'indice
%---------------------------------------------

\chapter{Introduzione}
\section{Scopo del documento}
Questo documento ha lo scopo di fornire una lista di norme generali sia per il comportamento dei vari membri del gruppo, sia per la stesura di tutta la documentazione interna ed esterna. L'obiettivo \`e quello di fissare delle norme di codifica ben precise, in modo da poter migliorare la leggibilit\`a del codice e facilitarne la comprensione. Tutti i componenti del gruppo devono attenersi a tali regole.
\section{Riferimenti}
\begin{itemize}
\item ``Ingegneria del software'' 8a edizione - Ian Sommerville 
\item Molte delle norme di codifica sono state scritte riferendosi alle convenzioni sul codice java della Sun Microsystems.
\end{itemize}
\section{Glossario}
Il glossario viene fornito come file esterno chiamato \textbf {Glossario.0.5.pdf} 

\chapter{Norme di Comportamento}
\section{Uso della Mail del Gruppo}
La mail verr\`a utilizzata principalmente come mezzo di comunicazione e di informazione tra i vari componenti del gruppo. In particolare per ricordare scadenze, segnalare risorse utili e pianificare gli incontri. 
\section{Uso dello spazio web in Google Groups}
Lo spazio web in Google Groups raggiungibile tramite il seguente URL: \({http://groups.google.com/group/happycodeinc}\), riservato ai soli membri del gruppo che possono accedervi dopo essersi autenticati sul proprio account gmail, verr\`a usato per:
\begin{itemize}
\item Gestire discussioni su argomenti non chiari o su possibili modifiche ai documenti;
\item Esporre, tramite discussioni, dubbi sulla correttezza dei documenti;
\item Aggiungere pagine statiche contenenti risorse pertinenti i linguaggi utilizzati.
\end{itemize}
\section{Comportamento Generale}
Ad ogni componente del gruppo verr\`a richiesto di partecipare alla stesura dei documenti nella parte che pi\`u li compete. Il documento verr\`a poi trascritto all'interno del modello latex. Ogni documento deve poi essere caricato  su svn, sia in formato tex sia in formato pdf.

\chapter{Norme di Documento}
\section{Nomenclatura}
In questo capitolo vengono illustrate le norme per la nomenclatura dei file sorgenti e di testo. I nomi dei file dovranno essere significativi rispetto al loro contenuto. Inoltre:
\begin{itemize}
\item Possono essere usati solo caratteri dell'alfabeto inglese dalla ``a'' alla ``z'' , minuscole e maiuscole;
\item Non devono essere utilizzati catteri speciali;
\item Non devono essere utilizzati catteri accentati;
\item I nomi utilizzati non devono contenere spazi;
\item Se il nome da utilizzare e' composto da piu' parole, queste dovranno essere disposte come in questo esempio: \\
``EsempioDiNomeFile''
\item Alla fine del nome del file deve esserci un codice per la versione del documento nella forma :\\ 
``.a.b'' \\
dove:\\
a=versione.\\
b=revisione.\\
\begin{itemize}
\item[-]La versione, numero intero che parte da 0, verr\`a aggiornata, incrementandola, ogni volta che verr\`a aggiunta una nuova sezione al documento che tratta argomenti non presenti nel documento.
\item[-]La revisione, numero intero che parte da 1, verr\`a incrementata ogni volta che il documento verr\`a modificato e corretto, ma solo negli argomenti gi\`a presenti.
\end{itemize} 
Esempio di un nome corretto:\\
``NormeDiProgetto.0.1''
\end{itemize}
\section{Impaginazione documenti}
Questa seziona descrive l'impaginazione dei documenti ad uso interno ed esterno. Di seguito viene fornita la lista delle regole da seguire per la corretta stesura di un documento.
\subsection{Struttura documento}
Ogni documento sar\`a composto di queste parti:
\begin{enumerate}
\item Una prima pagina con il logo dell'azienda, titolo del documento, versione e data ultima modifica;
\item Una seconda parte composta da tre tabelle contenti:
	{\begin{itemize}
	\item[-] Data creazione (anno,mese,giorno),versione,stato del documento, redazione,revisione,approvazione.
	\item[-] Lista di Distribuzione
	\item[-]Diario delle modifiche 
	\end{itemize}}
\item L'indice dei contenuti;
\item I contenuti del documento;
\end{enumerate}
E' fornito un modello di documento generale \( Modello.tex\), nel quale sono state applicate le regole date in precedenza. E' consigliato quindi l'aggiornamento dello stesso, aggiungendo e modificando solo i capitoli necessari.
\subsection{Intestazione/Pi\`e di pagina}
L'intestazione e il pi\`e di pagina vengono generati automaticamente corretti dal file .tex con i relativi loghi, il numero della pagina e i titoli dei capitoli.
\subsection{Corpo del documento}
\subsubsection{Carattere Testo e Titoli}
Il carattere dei documenti e dei titoli \`e quello di default, scelto nel modello iniziale. Solo il titolo del documento dovr\`a essere messo tutto in maiuscolo. Si dovr\`a sottostare a tali scelte, che inoltre rendono il tutto pi\`u leggibile e facile da usare. 
\subsubsection{Immagini}
Le immagini inserite nel corpo del documento dovranno essere necessariamente salvate nella cartella pics. Inoltre tutte le immagini contenute nei pdf devono essere in formato vettoriale eps.
\section{Norme di Glossario}
Il glossario verr\`a fornito in un documento chiamato \textbf{ Glossario.a.b.pdf}; dove ``a'' indica il numero di versione e ``b'' il numero di revisione, come descritto sopra.


\chapter{Condivisione, archiviazione e versionamento dei file}
Per gestire il lavoro collaborativo di pi\`u componenti del gruppo in contemporanea sullo stesso documento utilizzeremo SVN (Subversion) che permette la sincronizzazione delle versioni. In questo modo, ogni aggiornamento del documento inviato al repositoty (l'archivio di dati centralizzato gestito direttamente da SVN) rappresenter\`a una nuova revisione. SVN applicher\`a ad ogni modifica un numero identificativo incrementale da 1 a infinito. Attraverso il comando ``commit'' ogni componente del gruppo potr\`a inviare al repository il documento che ha appena modificato in locale (ossia la working copy). Viceversa, ``update'' \`e la fase di download della versione aggiornata del repository con aggiornamento della propria working copy. 

\chapter{File java}
\section{File sorgente}
Ogni file sorgente dovr\`a contenere una sola classe pubblica o interfaccia. I file sorgente dovranno avere il seguente ordine:
\begin{itemize}
\item Commenti iniziali;
\item Statements di Import e dichiarazioni di Package;
\item Dichiarazioni di classi e interfacce.
\end{itemize}
\subsection{Commenti iniziali}
Tutti i file sorgente dovranno inziare con un commento in stile C che elenca il nome della classe, informazioni sulla versione e la data dell'ultima modifica come nell'esempio: \newline
/* \newline
* Nome classe \newline
* \newline
* Informazioni Versione \newline
* \newline
* Data Ultima Modifica \newline
*/ \newline

\subsection{Statements di Package e di Import}
La prima linea non di commento di un file dovr\`a essere uno statement di Package. Seguiranno poi gli statement di import come nell'esempio: \\
package java.awt; \\ % DOBBIAMO METTERE UN PACCHETTO VERO NOSTRO!!!!!!!!!!
import org.w3c.dom; \\
\subsection{Dichiarazioni di Classi e di Interfacce}
Prima di tutto verranno dichiarate le varibili statiche nel seguente ordine: pubbliche, protette, quelle a livello package e poi le private. Successivamente le variabili di istanza sempre nello stesso ordine. Solo poi i costruttori seguiti dai metodi. I metodi dovranno essere raggrupati secondo la funzionalit\`a e non secondo l'accessibilit\`a.

\chapter{Indentazione}
Come unit\`a d'indentazione dovranno essere usati 4 spazi.
\subsection{Lunghezza della linea}
Una linea di codice non dovr\`a superare 80 caratteri.
\subsection{Interruzione delle linee}
Quando un'espressione non sta su una singola linea la si dovr\`a interrompe seguendo le seguenti regole:
\begin{itemize}
\item Interruzione dopo una virgola;
\item Interruzione prima di un operatore;
\item La nuova linea deve essere allineata con l'inizio dell'espressione allo stesso livello nella linea precedente;
\item Se le regole suddette portano a codice confuso oppure a codice che è troppo a ridosso del margine destro si deve invece usare una indentazione con 8 spazi.
\end{itemize}

\chapter{Commenti}
Ci potranno essere due tipi di commenti: commenti di implementazione e commenti di documentazione. 
\begin{itemize}
\item I commenti di implementazione, delimitati da /*...*/ e //, sono commenti riguardanti il codice nell'ottica della sua particolare implementazione. 
\item I commenti di documentazione, delimitati da /**...*/, descrivono la specifica del codice astraendo dall'implementazione.
\end{itemize}
\section{Formato dei commenti di implementazione}
Ci potranno essere tre stili di commenti di implementazione: a blocco, a linea singola e commenti alla fine della linea.
\subsection{Commenti a blocco}
 I commenti a blocco dovranno essere usati per descrivere files, metodi, strutture dati e algoritmi. Dovranno essere usati all'inizio di ogni file e prima di ogni metodo. Potranno essere usati anche in altre parti del file come ad esempio all'interno di un metodo. All'interno di una funzione o di un metodo dovranno essere indentati allo stesso livello del codice che descrivono. Il commento dovr\`a inoltre essere preceduto da una linea bianca che permetter\`a di separarlo dal resto del codice. \\
Esempio: \\
/* \\
* Questo \`e un commento a blocco \newline
*/ \\

\subsection{Commenti a linea singola}
I commenti a linea singola dovranno apparire in una linea singola ed essere usati per brevi commenti, indentandoli al livello del codice che segue. Se un commento non potr\`a essere scritto su una linea singola allora si dovr\`a usare un commento a blocco. Un commento a singola linea dovr\`a essere preceduto da una linea bianca. \\
Esempio: \newline
if (condition) \{ \newline
        /* Handle the condition. */ \newline
        ...... \newline
\} \newline

\subsection{Commenti alla fine della linea}
Il delimitatore di commento ``//'' potranno commentare una linea completa oppure solo una parte di essa. Non dovr\`a essere utilizzato su linee consecutive per commenti di testo per\`o potr\`a essere utilizzato su linee consecutive per eliminare (commentando) sezioni di codice.

\section{Commenti di documentazione}
I commenti di documentazione dovranno descrivere classi, interfacce, costruttori, metodi e campi dato. Ogni commento di documentazione dovr\`a essere situato all'interno dei delimitatori /**...*/. Ci dovr\`a essere un solo commento per ogni classe, interfaccia o membro. Non dovranno essere posizionati all'interno del blocco di definizione di un metodo o di un costruttore. Questo tipo di commento dovr\`a apparire subito prima della dichiarazione:\\
Esempio:\newline
/** \newline
* Commento di documentazione \newline
*/ \newline
public class Esempio \{ \newline
    ....... \newline
\} \newline

\chapter{Dichiarazioni}
Per facilitare i commenti ci dovr\`a essere una sola dichiarazione per linea. Non si dovranno inoltre mettere tipi differenti sulla stessa linea. Le variabili locali dovranno essere inizializzate dove sono dichiarate se ci\`o \`e possibile. Le dichiarazioni dovranno essere posizionate solamente all'inizio dei blocchi di codice. L'unica eccezione che potr\`a essere fatta è per gli indici dei cicli for; in questo caso la dichiarazione potr\`a essere fatta nello statement del ciclo for. Non ci dovranno inoltre essere dichiarazioni che rendono invisibili altre dichiarazioni a livello più alto; in pratica non si dovr\`a usare un nome di variabile già usato in un blocco soprastante.
\subsection{Dichiarazioni di Classi o di Interfacce}
Nella codifica di classi ed interfacce dovranno essere seguite le seguenti regole:
\begin{itemize}
\item Non ci dovranno essere spazi tra il nome di un metodo e le parentesi che apre la sua lista dei parametri;
\item La parentesi graffa che apre un blocco di codice dovr\`a apparire alla fine della stessa linea dello statement di dichiarazione;
\item La parentesi graffa che chiude un blocco di codice dovr\`a apparire su una linea singola a sé stante, indentata in modo da poter corrispondere la corrispondente parentesi di apertura del blocco, ad eccezione del caso quando lo statement è vuoto nel qual caso;
\item La parentesi di chiusura apparir\`a subito dopo quella di apertura;
\item I metodi dovranno essere separati da una linea bianca.
\end{itemize}

\chapter{Linee e spazi bianchi}
Le linee bianche dovranno essere utilizzate per migliorare la leggibilit\`a del codice. Nei seguenti casi dovranno sempre essere usate due linee bianche:
\begin{itemize}
\item Tra sezioni di un file sorgente;
\item Tra definizioni di classi ed interfacce.
\end{itemize}
Nei seguenti casi si dovr\`a usare una una sola linea bianca:
\begin{itemize}
\item Tra metodi;
\item Tra le variabili locali in un metodo e il suo primo statement;
\item Prima di un commento a blocco o un commento a linea singola;
\item Tra sezioni logiche all'interno di un metodo in modo da migliorare la leggibilità.
\end{itemize}
Nelle seguenti situazioni dovranno essere usati spazi bianchi:
\begin{itemize}
\item Una parola chiave seguita da una parentesi dovr\`a essere separata da uno spazio bianco. Da notare che uno spazio bianco non dovr\`a essere usato tra il nome di un metodo e la sua parentesi di apertura;
\item Uno spazio bianco dovr\`a apparire dopo le virgole nelle liste degli argomenti;
\item Tutti gli operatori binari ad eccezione di “.” dovranno essere separati dagli operandi con spazi bianchi;
\item Le espressioni in uno statement for dovranno essere separate da spazi bianchi.
\end{itemize}


\end{document}

