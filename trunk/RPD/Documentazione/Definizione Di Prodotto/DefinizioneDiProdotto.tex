\documentclass[11pt,titlepage,a4paper]{report}

%INCLUSIONE PACCHETTI
%---------------------------------------------
\usepackage[italian]{babel}
\usepackage{fancyhdr}
\usepackage{graphicx}
\graphicspath{{./pics/}} % cartella di salvataggio immagini

% STILE DI PAGINA
%---------------------------------------------
\pagestyle{fancy}
\renewcommand{\sectionmark}[1]{\markright{\thesection.\ #1}}
\lhead{\nouppercase{\rightmark}}
\rhead{\nouppercase{\leftmark}}
\renewcommand{\chaptermark}[1]{%
\markboth{\thechapter.\ #1}{}}

%Ridefinisco lo stile plain della pagina
\fancypagestyle{plain}{%
	\lhead{\includegraphics[height=50pt]{logo.eps}}
	\chead{}
	\rhead{HappyCode inc \\ happycodeinc@gmail.com}
	\lfoot{BR-jsys}
	\cfoot{\thepage}
	\renewcommand{\headrulewidth}{1pt}
	\renewcommand{\footrulewidth}{1pt}
}

\begin{document}

%definizione variabili 
\newcommand{\lv}{0.2} % latest version
\newcommand{\dt}{ Definizione Di Prodotto }% Document title
\newcommand{\Glossario}{ Glossario.1.4.pdf }
%fine definizione variabili



\hyphenation{glos-sa-rio es-pli-ci-to ve-ri-fi-ca-re re-po-si-to-ry se-gna-la-ta coe-ren-za}
\begin{titlepage}\begin{center}
\vspace*{0.5in}
\includegraphics{logo.eps}
\vspace*{0.2in} \\
{\Large \textbf{BR-jsys}}
{\Large \emph{business rules} per sistemi gestionali in architettura J2EE } 
\vspace{2in} \\
\Huge \textsc{ \dt }
\par\rule{10cm}{0.4pt} \par {\large Versione \lv - \today} \\
\end{center}\end{titlepage}
\vspace*{0.5in}

\begin{center}
\thispagestyle{plain}
\begin{table}[htbp]
\large{
\begin{tabular}{l}
\Large{\textbf{\textsf{Capitolato: ''BR-jsys``}}} \\
\begin{tabular}{||p{6cm}||p{6cm}||}
\hline
\textbf{Data creazione:} & 18/11/2007 \\ \hline
\textbf{Versione:} & \lv \\ \hline
\textbf{Stato del documento:} & Formale, esterno \\ \hline
% ----------------------------------------------------------------------------autori
\textbf{Redazione:} & Carraro Filippo \\ \hline
\textbf{Revisione:} &    \\ \hline
\textbf{Approvazione:}  & Elena Trivellato \\ \hline
\end{tabular} \\
\end{tabular}
}
\end{table}

\begin{table}[hbtp]
\large{
\begin{tabular}{l}
\Large{\textbf{\textsf{Lista di distribuzione}}} \\
\begin{tabular}{||p{6cm}||p{6cm}||} \hline
%  -------------------------------------------------------------lista di distribuzione
{Elena Trivellato}& Responsabile di progetto \\ \hline 
{Filippo Carraro}& Progettista \\ \hline
{Alessia Trivellato, Michele Bortolato}& Analisti \\ \hline
{Marco Tessarotto}& Verificatore \\ \hline
{Tullio Vardanega, Renato Conte}& Committente \\ \hline 
{Zucchetti S.r.l}& Azienda proponente\\ \hline
\end{tabular} \\
\end{tabular}
}
\end{table}
\begin{table}[hbtp]
\large{
\begin{tabular}{l}
\Large{\textbf{\textsf{Diario delle modifiche}}} \\
\begin{tabular}{||p{2cm}||p{3.5cm}||p{6cm}||} \hline
\textbf{Versione} & \textbf{Data rilascio} & \textbf{Descrizione} \\ \hline
%-------------------------------------------------------------------------------diario modifiche
0.2 & 26/01/2007 & Documento sottoposto a revisionamento automatico.\\ \hline
0.1 & 25/01/2007 & Stesura preliminare del documento. \\ \hline

\end{tabular} \\
\end{tabular}

}
\end{table}
\end{center}
\newpage



\chapter{Introduzione}
\section{Scopo del documento}
Nel presente documento illustreremo tutte le componenti del sistema software da realizzare. Come punto di partenza prenderemo la Specifica Tecnica, comprensiva dei diagrammi delle classi. Amplieremo quest'ultimi e ne aggiungeremo degli altri che ci permetteranno di descrivere in modo dettagliato tutta la struttura del prodotto ``Br-jsys''.
\section{Scopo del prodotto}
Il prodotto richiesto verr\`a inserito nell'ambito di un progetto pi\`u ampio. Il suo scopo \`e quello di automatizzare il sistema di validazione dei dati in ingresso al database dell'applicazione principale.

\section{Glossario}
Viene fornito come documento esterno chiamato Glossario.1.4.pdf.

\section{Riferimenti}
\section{Riferimenti}
\begin{itemize}
\item Capitolato d'appalto concorso per sistema ``BR-jsys''
\item Verbale dell'incontro con il committente ``Incontro2007-11-22.pdf''
\item Verbale dell'incontro con il committente ``Incontro2008-02-05.pdf''
\item ``Analisi dei Requisiti''
\item ``Piano di Qualifica''
\item ``Norme di Progetto''
\item ``Specifica tecnica''
\item ``Ingegneria del Software'' 8a edizione - Ian Sommerville
\item ``The Definitive ANTLR Reference''
\end{itemize}
\chapter{Standard di progetto}
\section{Standard di progettazione architetturale}
Per le rappresentazioni grafiche dell'architettura del nostro sistema, ci baseremo sulle regole definite da UML 1.5. In particolare useremo:
\begin{itemize}
\item Diagrammi delle classi;
\item Diagrammi degli oggetti;
\item Diagrammi delle attivit\`a;
\item Diagramma delle componenti.
\end{itemize}
\section{Standard di documentazione del codice}
Le specifiche per la documentazione del codice sono descritte nel documento ``NormeDiProgetto.x.x.x.pdf''.
\section{Standard di denominazione di entit\`a e relazioni}
\section{Standard di programmazione}
\section{Strumenti di lavoro}

\chapter{Specifica delle componenti}
\chapter{Appendice}
\section{Codice Sorgente}
\section{Tracciamento della relazione componenti - requisiti}
\newpage

\tableofcontents
\end{document}
