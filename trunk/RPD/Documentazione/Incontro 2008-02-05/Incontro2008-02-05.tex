\documentclass[11pt,titlepage,a4paper]{report}

%INCLUSIONE PACCHETTI
%---------------------------------------------
\usepackage[italian]{babel}
\usepackage{fancyhdr}
\usepackage{graphicx}
\graphicspath{{./pics/}}	% cartella di salvataggio immagini

% STILE DI PAGINA
%---------------------------------------------
\pagestyle{fancy}
\renewcommand{\sectionmark}[1]{\markright{\thesection.\ #1}}
\lhead{\nouppercase{\rightmark}}
\rhead{\nouppercase{\leftmark}}
\renewcommand{\chaptermark}[1]{%
\markboth{\thechapter.\ #1}{}}

%Ridefinisco lo stile plain della pagina
\fancypagestyle{plain}{%
	\lhead{\includegraphics[height=50pt]{logo.eps}}
	\chead{}
	\rhead{HappyCode inc \\ happycodeinc@gmail.com}
	\lfoot{BR-jsys}
	\cfoot{\thepage}
	\renewcommand{\headrulewidth}{1pt}
	\renewcommand{\footrulewidth}{1pt}
}
%---------------------------------------------


%INIZIO DOCUMENTO


%---------------------------------------------
% PRIMA PAGINA
%---------------------------------------------
\begin{document}

\begin{titlepage}
\begin{center}
\vspace*{0.5in}
\includegraphics{logo.eps}
\vspace*{0.2in}

{\Large \textbf{BR-jsys}}

{\Large \emph{Business Rules} per sistemi gestionali in architettura J2EE } 
\vspace{1.3in}
\par\rule{10cm}{.4pt} \par
\par\rule{12cm}{1pt} \par
\vspace*{0.5in}
\LARGE \textbf {Verbale della riunione del 17 gennaio 2008}
\vspace*{0.5in}
\par\rule{12cm}{1pt} \par
\par\rule{10cm}{.4pt} \par

\end{center}
\end{titlepage}
\vspace*{0.5in}
ggio a beneficio di un interprete che usa la rappresentazione per interpretare frasi prodotte in quel linguaggio.
\thispagestyle{plain}
Il secondo incontro con l'azienda proponente aveva lo scopo di mostrare e far approvare il lavoro fin qui svolto e sciogliere gli ultimi dubbi sul capitolato.

\paragraph{Approvazioni}
\begin{itemize}

\item \`E stata approvata la grammatica proposta.
\item \`E stato approvato l'utilizzo di ANTLR come strumento per la generazione del parser.
\item \`E stato approvato l'utilizzo di ANTLRWorks come strumento di appoggio per il debugging del parser.

\end{itemize}

\paragraph{Modifiche al capitolato}

\begin{itemize}

\item Si dovr\`a utilizzare un DBMS esterno come stumento di interfacciamento tra il repository e il sistema, al fine di ridurre al minimo i tempi per le transazioni.

\item Ogni business rule avr\`ggio a beneficio di un interprete che usa la rappresentazione per interpretare frasi prodotte in quel linguaggio.a un proprio business object associato.

\item Ogni business rule presente nel repository dovr\`a avere un' etichetta che la distingue dalle altre.

\item Si dov\`a dotare il linguaggio della funzionalit\`a \textit{message} da porre alla fine di ogni operazione di confronto. Se in fase di eggio a beneficio di un interprete che usa la rappresentazione per interpretare frasi prodotte in quel linguaggio.secuzione verr\`a violata un'operazione di confronto che ha associato un \textit{message}, allora verr\`a visualizzato il messaggio testuale al suo interno.

\item Si dovr\`a dotare il sistema della funzionalit\`a di cancellazione di una business rule.

\item Il requisito di controllo coerenza verr\`a convertito da obbligatorio a opzionale.

\item L'azienda proponente ci fornir\`a un esecutore. Esso interagir\`a col nostro sistema inviando richieste di business rules, da noi fornite tramite opportune invocazioni del DBMS usato per colloquiare con il repository. Una volta ricevute le regole, l'esecutore le applicher\`a ai suoi business objects. Di conseguenza non dovremo pi\`u preoccuparci di realizzare l'esecutore di business rules.
\end{itemize}

\end{document}
