\documentclass[11pt,titlepage,a4paper]{report}

%INCLUSIONE PACCHETTI
%---------------------------------------------
\usepackage[italian]{babel}
\usepackage{fancyhdr}
\usepackage{graphicx}
\graphicspath{{./pics/}}	% cartella di salvataggio immagini

% STILE DI PAGINA
%---------------------------------------------
\pagestyle{fancy}
\renewcommand{\sectionmark}[1]{\markright{\thesection.\ #1}}
\lhead{\nouppercase{\rightmark}}
\rhead{\nouppercase{\leftmark}}
\renewcommand{\chaptermark}[1]{%
\markboth{\thechapter.\ #1}{}}

%Ridefinisco lo stile plain della pagina
\fancypagestyle{plain}{%
	\lhead{\includegraphics[height=50pt]{logo.eps}}
	\chead{}
	\rhead{HappyCode inc \\ happycodeinc@gmail.com}
	\lfoot{BR-jsys}
	\cfoot{\thepage}
	\renewcommand{\headrulewidth}{1pt}
	\renewcommand{\footrulewidth}{1pt}
}
%---------------------------------------------


%INIZIO DOCUMENTO


%---------------------------------------------
% PRIMA PAGINA
%---------------------------------------------
\begin{document}

\begin{titlepage}
\begin{center}
\vspace*{0.5in}
\includegraphics{logo.eps}
\vspace*{0.2in}

{\Large \textbf{BR-jsys}}

{\Large \emph{Business Rules} per sistemi gestionali in architettura J2EE } 
\vspace{1.3in}
\par\rule{10cm}{.4pt} \par
\par\rule{12cm}{1pt} \par
\vspace*{0.5in}
\LARGE \textbf {Verbale della riunione del 05 febbraio 2008}
\vspace*{0.5in}
\par\rule{12cm}{1pt} \par
\par\rule{10cm}{.4pt} \par

\end{center}
\end{titlepage}
\vspace*{0.5in}
Il terzo incontro con l'azienda proponente aveva lo scopo di mostrare e far approvare il lavoro fin qui svolto e consegnare un primo prototipo del progetto ``BR-jsys''.
\thispagestyle{plain}

\paragraph{Approvazioni}
\begin{itemize}
\item \`E stato approvato l'uso di eXist come DBMS esterno per ridurre al minimo i tempi per le tranzizioni.
\item Sono state approvate le modifiche all' implementazione delle business rule. Ognuna avr\`a cos\`i un business object associato e un' etichetta per distinguerla dalle altre.
\item \`E stato approvata la funzionalit\`a message.
\item \`E stata approvata la funzionalit\`a di cancellazione di una business rule dal repository.
\item Ci \`e stato fornito dal proponente un interprete minimale come comunicato nel precedente incontro.
\item Approvato il prototipo e le sue funzionalit\`a.
\end{itemize}

\paragraph{Modifiche al prototipo}
\begin{itemize}
\item Si dovr\`a creare un' interfaccia grafica user-friendly con le seguenti funzionalit\`a:\\
\textbf{-Inserimento:} per rinserire una business rule nel repository.\\
\textbf{-Cancellazione:} per cancellare una business rule dal repository.\\
\textbf{-RecuperaRegola:} per ricercare e reuperare una business rule se presente nel repository.\\
\textbf{-Sandbox: }struttura con due \textit{textarea}, la prima per immettere una query e la seconda per visualizzare l'esito della query e una StatusBar che fornir\`a i dati dell'operazione svolta (per esempio tempo impiegato, numero di business rules trovate ecc).\\
\textbf{-RecuperaDaBO: }simile a RecuperaRegola con la differenza che recupera tutte le regole associate ad un business object.\\
\end{itemize}

\end{document}
