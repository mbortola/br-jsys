\documentclass[11pt,titlepage,a4paper]{report}

%INCLUSIONE PACCHETTI
%---------------------------------------------
\usepackage[italian]{babel}
\usepackage{fancyhdr}
\usepackage{graphicx}
\graphicspath{{./pics/}} % cartella di salvataggio immagini

% STILE DI PAGINA
%---------------------------------------------
\pagestyle{fancy}
\renewcommand{\sectionmark}[1]{\markright{\thesection.\ #1}}
\lhead{\nouppercase{\rightmark}}
\rhead{\nouppercase{\leftmark}}
\renewcommand{\chaptermark}[1]{%
\markboth{\thechapter.\ #1}{}}

%Ridefinisco lo stile plain della pagina
\fancypagestyle{plain}{%
	\lhead{\includegraphics[height=50pt]{logo.eps}}
	\chead{}
	\rhead{HappyCode inc \\ happycodeinc@gmail.com}
	\lfoot{BR-jsys}
	\rfoot{\dt - \lv}
	\cfoot{\thepage}
	\renewcommand{\headrulewidth}{1pt}
	\renewcommand{\footrulewidth}{1pt}
}

\begin{document}


%definizione variabili 
\newcommand{\lv}{ 2.1 } % latest version
\newcommand{\dt}{ Piano Di Progetto }% Document title
\newcommand{\Glossario}{ Glossario.1.4.pdf }
%fine definizione variabili

\hyphenation{glos-sa-rio es-pli-ci-to ve-ri-fi-ca-re re-po-si-to-ry se-gna-la-ta coe-ren-za}
\begin{titlepage}\begin{center}

\vspace*{0.5in}
\includegraphics{logo.eps}
\vspace*{0.2in} \\
{\Large \textbf{BR-jsys}}

{\Large \emph{business rules} per sistemi gestionali in architettura J2EE } 
\vspace{2in} \\
\Huge \textsc{ \dt }
\par\rule{10cm}{0.4pt} \par {\large Versione \lv - \today} \\
\end{center}\end{titlepage}
\vspace*{0.5in}


\begin{center}
\thispagestyle{plain}
\begin{table}[htbp]\section{Riferimenti}
\begin{itemize}
\item Capitolato d'appalto concorso per sistema ``BR-jsys'';
\item Verbale dell'incontro con il committente ``Incontro2007-11-22.pdf'';
\item Verbale dell'incontro con il committente ``Incontro2008-02-05.pdf'';
\item ``Ingegneria del Software'' 8a edizione - Ian Sommerville.
\end{itemize}
\large{
\begin{tabular}{l}
\Large{\textbf{\textsf{Capitolato: ''BR-jsys``}}} \\
\begin{tabular}{||p{6cm}||p{6cm}||} \hline
\textbf{Data creazione:} & 2007/11/21 \\ \hline
\textbf{Versione:} & \lv \\ \hline
% ----------------------------------------------------------------------------autori
\textbf{Stato del documento:} & formale, esterno \\ \hline
\textbf{Revisione RR} &           \\ \hline
\textbf{Redazione:} & Elena Trivellato \\ \hline
\textbf{Revisione:} & Marco Tessarotto \\ \hline
\textbf{Approvazione:} Elena Trivellato & \\ \hline
\textbf{Revisione RPD} &    \\ \hline
\textbf{Redazione:} & Mattia Meroi \\ \hline
\textbf{Revisione:} & Elena Trivellato, Filippo Carraro \\ \hline
\textbf{Approvazione:}  & Mattia Meroi \\  \hline
\end{tabular} \\
\end{tabular}
}
\end{table}

\begin{table}[hbtp]
\large{
\begin{tabular}{l}
\Large{\textbf{\textsf{Lista di distribuzione}}} \\
\begin{tabular}{||p{6cm}||p{6cm}||} \hline
%  -------------------------------------------------------------lista di distribuzione
{HappyCode inc}& Gruppo di lavoro\\ \hline
{Tullio Vardanega, Renato Conte}& Rappresentanti del committente \\ \hline 
{Zucchetti S.r.l}& Azienda committente\\ \hline
\end{tabular} \\
\end{tabular}
}
\end{table}
\begin{table}[hbtp]
\large{
\begin{tabular}{l}
\Large{\textbf{\textsf{Diario delle modifiche}}} \\

\begin{tabular}{||p{2cm}||p{3.5cm}||p{6cm}||} \hline
%-------------------------------------------------------------------------------diario modifiche
\textbf{Versione} & \textbf{Data rilascio} & \textbf{Descrizione} \\ \hline
2.1 & 2008/02/14 & Modifica grafici.\\ \hline
2.0 & 2008/02/13 & Aggiunta tabelle carico effettivo in fase di progettazione.\\ \hline
1.5 & 2008/02/11 & Modifica layout tabelle.\\ \hline 
1.4 & 2008/02/05 & Aggiunta del nome del file nel modello di documento.\\ \hline
1.3 & 2008/01/25 & Aggiunta tabelle delle ore effettivamente impiegate in fase di analisi \\ \hline
1.2 & 2008/01/23 & Aggiunto carico totale delle risorse preventivo \\ \hline
1.1 & 2008/01/22 & Modifica al layout dei documenti.\\ \hline
1.0 & 2007/12/21 & Documento sottoposto a revisionamento automatico.\\ \hline
0.5 & 2007/12/05 & Aggiunto riferimento diagramma di Gantt \\ \hline
0.4 & 2007/11/29 & Stesura completa del documento. \\ \hline
0.3 & 2007/11/25 & Assegnazione dei ruoli ai componenti del gruppo. \\ \hline
0.2 & 2007/11/23 & Riviste al ribasso le ore di programmazione. \\ \hline
0.1 & 2007/11/21 & Stesura preliminare del documento. \\ \hline


\end{tabular} \\
\end{tabular}

}
\end{table}
\end{center}


\tableofcontents 


\chapter{Introduzione}
\section{Scopo del documento}
Questo documento rappresenta il piano di progetto preliminare
del capitolato d'appalto per il sistema "Business Rules per sistemi
gestionali in architettura J2EE BR-jsys". 
Qui \`e riportata la divisione dei ruoli e il costo complessivo del
sistema in base ai ruoli e alle ore impegnate.

\section{Glossario}
Il glossario viene fornito come file esterno chiamato \Glossario .

\section{Riferimenti}
\begin{itemize}
\item Capitolato d'appalto concorso per sistema ``BR-jsys'';
\item Verbale dell'incontro con il committente ``Incontro2007-11-22.pdf'';
\item Verbale dell'incontro con il committente ``Incontro2008-02-05.pdf'';
\item ``Ingegneria del Software'' 8a edizione - Ian Sommerville.
\end{itemize}

\chapter{Ruoli}
\section{Definizione ruoli}
Nella tabella sottostante riportiamo l'impegno complessivo, in base ai ruoli di progetto, nelle quattro macrofasi previste dal ciclo di vita del nostro software. Quest'ultima
\`e stata aggiornata al 15 febbraio al termine della fase di progettazione. Inseriamo quindi il valore effettivo di ore impiegate, riportando tra parentesi la differenza rispetto a quanto preventivato relativamente alle prime due fasi.
Rispetto alle previsioni in fase di analisi sono aumentate considerevolmente 
le ore dei progettisti ed in fase di progettazione sono aumentate le ore 
dei programmatori, a scapito di quelle dei progettisti. 
Tutto ci\`o \`e avvenuto a seguito dell'utilizzo del generatore di parser che ha accelerato notevolmente i tempi, consentendoci di soffermarci di pi\`u sulla progettazione gi\`a in fase di analisi. Di conseguenza in fase di progettazione abbiamo gi\`a a disposizione un piccolo prototipo software. 


\begin{table}[hbtp]
\large{
\begin{tabular}{l}
\Large{\textbf{\textsf{Tabella dei Ruoli}}} \\
\begin{tabular}{||p{3cm}||p{2.5cm}||p{2.5cm}||p{2cm}||p{2cm}||}
\hline 
\textbf{Ruoli} & \textbf{Analisi} & \textbf{Progett.} & \textbf{Sviluppo} & \textbf{Verifica}\\
\hline

{Responsabile}&10&10&10&9 \\ 
\hline 
{Amministratore} &10&10&10&10\\ 
\hline
{Analista}& 60 \footnotesize{(-2)}&20&5&0 \\
\hline
{Progettista}&29 \footnotesize{(+24)}&60 \footnotesize{(-15)}&25&0 \\
\hline
{Programmatore}&0&10 \footnotesize{(+5)}&60&55 \\
\hline
{Verificatore}& 30&30&90&95 \\
\hline
{Totale}& 137 \footnotesize{(+22)}&140 \footnotesize{(-10)}&200&169 \\
\hline
\end{tabular} \\

\end{tabular}
}

\end{table}

Il seguente grafico mostra in che misura incide ciascun ruolo per ogni macrofase e la differenza tra ore previste ed ore effettive nelle due fasi (analisi e progettazione) gi\`a portate a termine.
Vediamo graficamente in questo modo come ogni ruolo di progetto contribuisce a formare 
il totale delle ore di ciascuna delle quattro fasi di lavoro:
\begin{center}
\includegraphics [width=1\textwidth] {ruoliperFasi.eps}
\end{center}

\section{Incidenza percentuale}
Riportiamo in questa tabella il peso percentuale di ciascun ruolo nelle ore
complessive previste di realizzazione del prodotto. Le cifre tra parentesi indicano di quanto il dato effettivo riportato si discosta dalle previsioni iniziali.
\begin{table}[hbtp]
\large{
\begin{tabular}{l}
\Large{\textbf{\textsf{Tabella delle percentuali dei ruoli}}} \\
\begin{tabular}{||p{6cm}||p{4cm}||}
\hline

\textbf{Ruoli} & \textbf{Percentuali}\\
\hline
{Responsabile}&6\\ 
\hline 
{Amministratore} &6\\ 
\hline
{Analista} &13 \\
\hline
{Progettista} &18 \footnotesize{(+1)}\\
\hline
{Programmatore} &19\\
\hline
{Verificatore} &38 \footnotesize{(-1)} \\
\hline
{Totale} &100 \\
\hline

\end{tabular} \\
\end{tabular}
}
\end{table}

Evidenziamo la distribuzione delle ore aggiornata al 15 febbraio tra i vari ruoli con un grafico; a differenza di prima qui abbiamo una visione globale riguardante 
tutto il ciclo di vita del software:


\begin{center}
\includegraphics [width=1\textwidth] {orePercentuali.eps}
\end{center}


\section{Assegnazione dei ruoli e delle ore a ciascun membro}
Nelle seguenti tabelle vediamo la ripartizione delle ore nelle quattro macro-fasi rispetto ai ruoli di progetto. Per le fasi gi\`a concluse (analisi e progettazione) riportiamo le ore effettive indicando tra parentesi in che misura si discostano dalle previsioni iniziali.
\begin{table}[hbtp]
\large{
\begin{tabular}{l}
\Large{\textbf{\textsf{Fase di Analisi (Consuntivo) - 1}}} \\
\begin{tabular}{||p{3.5cm}||p{2cm}||p{2cm}||p{2cm}||p{2cm}||}
\hline
\textbf{Membro} & \textbf{Respon.} & \textbf{Ammin.} & \textbf{Analista}

& \textbf{Progett.}\\
\hline
{Appon Luca}&0&5&8 \footnotesize{(-1)}&3 \footnotesize{(+3)} \\ 
\hline 
{Bortolato Michele} &2&0&9&7 \footnotesize{(+5)}\\ 
\hline
{Carraro Filippo}&0&5&8&4 \footnotesize{(+4)} \\
\hline
{Meroi Mattia}&6&0&6 \footnotesize{(-1)}&3 \footnotesize{(+3)}\\
\hline
{Tessarotto Marco} &0&0&9&7 \footnotesize{(+4)}\\
\hline
{Trivellato Alessia} &0&0&9&3 \footnotesize{(+3)} \\
\hline
{Trivellato Elena} &2&0&9&2 \footnotesize{(+2)} \\
\hline
{Totale}& 10&10&58 \footnotesize{(-2)}&29 \footnotesize{(+24)} \\
\hline

\end{tabular} \\
\end{tabular}
}
\end{table}

\begin{table}[hbtp]
\large{
\begin{tabular}{l}
\Large{\textbf{\textsf{Fase di analisi (Consuntivo) - 2}}} \\
\begin{tabular}{||p{3.5cm}||p{2cm}||p{2cm}||p{2cm}||p{2cm}||}
\hline
\textbf{Membro} & \textbf{Program} & \textbf{Verif.} & \textbf{Totale}\\ \hline
{Appon Luca}&0&3&17 \\ \hline 
{Bortolato Michele} &0&3&16\\ \hline
{Carraro Filippo}&0&3&16 \\ \hline
{Meroi Mattia}&0&4&17\\ \hline
{Tessarotto Marco} &0&4&16\\ \hline
{Trivellato Alessia} &0&7&16 \\ \hline
{Trivellato Elena} &0&6&17 \\ \hline
{Totale} &0&30&115 \\ \hline
\end{tabular} \\
\end{tabular}
}
\end{table}


\begin{table}[hbtp]
\large{
\begin{tabular}{l}
\Large{\textbf{\textsf{Fase di Progettazione (Consuntivo) - 1}}} \\
\begin{tabular}{||p{3.5cm}||p{2cm}||p{2cm}||p{2cm}||p{2cm}||}
\hline

\textbf{Membro} & \textbf{Respon.} & \textbf{Ammin.} & \textbf{Analista}
& \textbf{Progett.}\\
\hline
{Appon Luca}&4&0&2&8 \footnotesize{(-2)} \\ 
\hline 
{Bortolato Michele} &4&0&2&8 \footnotesize{(-2)}\\ 
\hline
{Carraro Filippo}&0&0&4&9 \footnotesize{(-3)} \\
\hline
{Meroi Mattia}&0&5&3&8 \footnotesize{(-2)}\\
\hline
{Tessarotto Marco} &0&0&4&9 \footnotesize{(-1)}\\
\hline
{Trivellato Alessia} &0&5&2&9 \footnotesize{(-3)} \\
\hline
{Trivellato Elena} &2&0&3&9 \footnotesize{(-2)} \\
\hline
{Totale}& 10&10&20&60 \footnotesize{(-15)} \\
\hline

\end{tabular} \\
\end{tabular}
}
\end{table}

\begin{table}[hbtp]
\large{
\begin{tabular}{l}
\Large{\textbf{\textsf{Fase di Progettazione (Consuntivo) - 2}}} \\
\begin{tabular}{||p{3.5cm}||p{2cm}||p{2cm}||p{2cm}||p{2cm}||}
\hline

\textbf{Membro} & \textbf{Program} & \textbf{Verif.} & \textbf{Totale}\\
\hline
{Appon Luca}&0&5&21 \\ 
\hline 
{Bortolato Michele} &2 \footnotesize{(+2)}&6&22\\ 
\hline
{Carraro Filippo}&2 \footnotesize{(+2)}&5&21 \\
\hline
{Meroi Mattia}&0&3&21\\
\hline
{Tessarotto Marco} &4 \footnotesize{(-1)}&3&22\\
\hline
{Trivellato Alessia} &1 \footnotesize{(+1)}&3&22 \\
\hline
{Trivellato Elena} &1 \footnotesize{(+1)}&5&21 \\
\hline
{Totale}&10 \footnotesize{(+5)}&30&150 \\
\hline

\end{tabular} \\
\end{tabular}
}
\end{table}


\begin{table}[hbtp]
\large{
\begin{tabular}{l}
\Large{\textbf{\textsf{Fase di Sviluppo (Preventivo) - 1}}} \\
\begin{tabular}{||p{3.5cm}||p{2cm}||p{2cm}||p{2cm}||p{2cm}||}
\hline
\textbf{Membro} & \textbf{Respon.} & \textbf{Ammin.} & \textbf{Analista} & \textbf{Progett.}\\ \hline
{Appon Luca}&0&0&2&5 \\ \hline 
{Bortolato Michele} &0&2&0&3\\ \hline
{Carraro Filippo}&5&0&0&4 \\ \hline
{Meroi Mattia}&0&0&3&4\\ \hline
{Tessarotto Marco} &5&6&0&3\\ \hline
{Trivellato Alessia} &0&2&0&3 \\ \hline
{Trivellato Elena} &0&0&0&3 \\ \hline
{Totale}& 10&10&5&25 \\ \hline
\end{tabular} \\
\end{tabular}
}
\end{table}

\begin{table}[hbtp]
\large{
\begin{tabular}{l}
\Large{\textbf{\textsf{Fase di Sviluppo (Preventivo) - 2}}} \\
\begin{tabular}{||p{3.5cm}||p{2cm}||p{2cm}||p{2cm}||p{2cm}||}
\hline
\textbf{Membro} & \textbf{Program} & \textbf{Verif.} & \textbf{Totale}\\ \hline
{Appon Luca}&10&12&29 \\ \hline 
{Bortolato Michele} &12&10&27\\ \hline
{Carraro Filippo}&8&12&29 \\ \hline
{Meroi Mattia}&12&9&28\\ \hline
{Tessarotto Marco} &4&11&29\\ \hline
{Trivellato Alessia} &7&17&29 \\ \hline
{Trivellato Elena} &7&19&29 \\ \hline
{Totale}& 60&90&200 \\ \hline
\end{tabular} \\
\end{tabular}
}
\end{table}


\begin{table}[hbtp]
\large{
\begin{tabular}{l}
\Large{\textbf{\textsf{Fase di Verifica (Preventivo) - 1}}} \\
\begin{tabular}{||p{3.5cm}||p{2cm}||p{2cm}||p{2cm}||p{2cm}||} \hline
\textbf{Membro} & \textbf{Respon.} & \textbf{Ammin.} & \textbf{Analista} & \textbf{Progett.}\\ \hline
{Appon Luca}&2&0&0&0 \\ \hline 
{Bortolato Michele} &0&5&0&0\\ \hline
{Carraro Filippo}&0&0&0&0 \\ \hline
{Meroi Mattia}&0&0&0&0\\ \hline
{Tessarotto Marco} &0&0&0&0\\ \hline
{Trivellato Alessia} &5&0&0&0 \\ \hline
{Trivellato Elena} &2&5&0&0 \\ \hline
{Totale}& 9&10&0&0 \\ \hline
\end{tabular} \\
\end{tabular}
}
\end{table}

\begin{table}[hbtp]
\large{
\begin{tabular}{l}
\Large{\textbf{\textsf{Fase di Verifica (Preventivo) - 2}}} \\
\begin{tabular}{||p{3.5cm}||p{2cm}||p{2cm}||p{2cm}||p{2cm}||} \hline
\textbf{Membro} & \textbf{Program} & \textbf{Verif.} & \textbf{Totale}\\ \hline
{Appon Luca}&6&16&24 \\ \hline
{Bortolato Michele} &3&17&25\\ \hline
{Carraro Filippo}&10&14&24 \\ \hline
{Meroi Mattia}&7&17&24\\ \hline
{Tessarotto Marco} &7&17&249\\ \hline
{Trivellato Alessia} &11&8&24 \\ \hline
{Trivellato Elena} &11&6&24 \\ \hline
{Totale} &55&95&169 \\ \hline
\end{tabular} \\
\end{tabular}
}
\end{table}

\newpage

\section{Carico totale di ore per ciascun componente}
In questa sezione vediamo in che misura (numero di ore) ogni componente collabora alla 
realizzazione del progetto in ogni macrofase e globalmente. Come prima a fianco delle ore effettive indicheremo lo scostamento dalle previsioni.
Il bilancio delle ore ``previste-effettive'' ha evidenziato un leggero aumento del carico complessivo di lavoro per ciascun membro del gruppo quantificabile mediamente in 1-2 ore ciascuno.


\begin{table}[hbtp]
\large{
\begin{tabular}{l}
\Large{\textbf{\textsf{Carico totale delle risorse (Consuntivo al 15/02/08) - 1}}} \\

\begin{tabular}{||p{3.5cm}||p{2cm}||p{2cm}||p{2cm}||p{2cm}||}
\hline
\textbf{Membro} & \textbf{Respon.} & \textbf{Ammin.} & \textbf{Analista}
& \textbf{Progett.}\\
\hline
{Appon Luca}&6&5&12 \footnotesize{(-1)}&16 \footnotesize{(+1)} \\ 
\hline 
{Bortolato Michele} &6&7&11&18 \footnotesize{(+3)}\\ 
\hline
{Carraro Filippo}&5&5&12&17 \footnotesize{(+1)} \\
\hline
{Meroi Mattia}&6&5&12 \footnotesize{(-1)}&15 \footnotesize{(+1)}\\
\hline
{Tessarotto Marco} &5&6&13&19 \footnotesize{(+3)}\\
\hline
{Trivellato Alessia} &5&7&11&18 \\
\hline
{Trivellato Elena} &6&5&12&16 \\
\hline
{Totale}& 39&40&83 \footnotesize{(-2)}&114 \footnotesize{(+9)} \\
\hline



\end{tabular} \\
\end{tabular}
}
\end{table}

\begin{table}[hbtp]
\large{
\begin{tabular}{l}
\Large{\textbf{\textsf{Carico totale delle risorse (Consuntivo al 15/02/08) - 2}}} \\

\begin{tabular}{||p{3.5cm}||p{2cm}||p{2cm}||p{2cm}||p{2cm}||}
\hline
\textbf{Membro} & \textbf{Program} & \textbf{Verif.} & \textbf{Totale}\\
\hline
{Appon Luca}&16&36&91 \\ 
\hline 
{Bortolato Michele} &17 \footnotesize{(+2)}&36&95 \footnotesize{(+5)}\\ 
\hline
{Carraro Filippo}&20 \footnotesize{(+2)}&34&93 \footnotesize{(+3)} \\
\hline
{Meroi Mattia}&19&33&90\\
\hline
{Tessarotto Marco} &15 \footnotesize{(-1)}&35&93 \footnotesize{(+2)}\\
\hline
{Trivellato Alessia} &19 \footnotesize{(+1)}&35&92 \footnotesize{(+1)} \\
\hline
{Trivellato Elena} &19 \footnotesize{(+1)}&36&92 \footnotesize{(+1)} \\
\hline
{Totale} &125 \footnotesize{(+5)}&245&646 \footnotesize{(+12)} \\
\hline

\end{tabular} \\
\end{tabular}
}
\end{table}


\chapter{Costi}
\section{Costo orario per ogni ruolo}
Evidenziamo il costo orario di ciascun ruolo di progetto
\begin{table}[hbtp]
\large{
\begin{tabular}{l}
\Large{\textbf{\textsf{Tabella dei costi orari}}} \\

\begin{tabular}{||p{6cm}||p{5cm}||}
\hline
\textbf{Ruoli} & \textbf{Costo orario in Euro}\\
\hline
{Responsabile}&30,00\\ 
\hline 
{Amministratore} &18,00\\ 
\hline
{Analista} &25,00 \\
\hline
{Progettista} &20,00 \\
\hline
{Programmatore} &15,00\\
\hline
{Verificatore} &15,00 \\
\hline

\end{tabular} \\
\end{tabular}
}
\end{table}

\section{Costo totale per ogni ruolo}
Evidenziamo il costo totale di ciascun ruolo di progetto e la sua incidenza
percentuale sul totale di spesa prevista.
Vediamo che l'incremento di ore effettive %&2&0&3&11 
rispetto a quelle previste ha causato
anche un aumento del costo complessivo del lavoro di euro 205.00.

\begin{table}[hbtp]
\large{

\begin{tabular}{l|r|l}
\Large{\textbf{\textsf{Tabella dei costi Totali (Consuntivo al 15/02/08)}}} \\
\begin{tabular}{||p{4cm}||p{5cm}||p{3cm}||}
\hline
\textbf{Ruoli} & \textbf{Costo Totale}& \textbf{Costo Percentuale}\\
\hline
{Responsabile}&1.170,00&10\\ 
\hline 
{Amministratore} &720,00&6\\ 
\hline
{Analista} &2.075,00 \footnotesize{(-50,00)}&17 \footnotesize{(-1)} \\
\hline
{Progettista} &2.280,00 \footnotesize{(+180,00)}&21 \footnotesize{(+3)} \\
\hline
{Programmatore} &1.875,00 \footnotesize{(+75,00)}&15 \footnotesize{(-1)}\\
\hline
{Verificatore} &3.675,00&31 \footnotesize{(-1)} \\
\hline
{Totale} &11.795,00 \footnotesize{(+205,00)}&100 \\
\hline

\end{tabular} \\
\end{tabular}
}
\end{table}


Evidenziamo come il costo di ciascun ruolo incide in percentuale
sul totale con un grafico:
\begin{center}
\includegraphics [width=1\textwidth] {costiPercentuali.eps}
\end{center}


\chapter{Diagramma di Gantt}
Alleghiamo due files denominati rispettivamente:
\begin {itemize} 
\item DiagrammaGantt\_0\_1.png che rappresenta l'impiego delle risorse nel tempo 
visualizzate giorno per giorno
\item DiagrammaGanttSettimanale\_0\_1.png che invece mostra l'impiego delle risorse
pi\'u globalmente organizzato per settimane
\end{itemize}
\begin{center}
\includegraphics [width=1\textwidth] {confronto-ruoli-fasi.eps}
\end{center}
fdgdfgfg
\begin{center}
\includegraphics [width=1\textwidth] {confronti-costi-totale.eps}
\end{center}
dfgdfgfdggf
\begin{center}
\includegraphics [width=1\textwidth] {confronto-ore-totale.eps}
\end{center}
\end{document}
