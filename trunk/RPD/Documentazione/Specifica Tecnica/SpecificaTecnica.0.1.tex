\documentclass[11pt,titlepage,a4paper]{report}

%INCLUSIONE PACCHETTI
%---------------------------------------------
\usepackage[italian]{babel}
\usepackage{fancyhdr}
\usepackage{graphicx}
\graphicspath{{./pics/}} % cartella di salvataggio immagini

% STILE DI PAGINA
%---------------------------------------------
\pagestyle{fancy}
\renewcommand{\sectionmark}[1]{\markright{\thesection.\ #1}}
\lhead{\nouppercase{\rightmark}}
\rhead{\nouppercase{\leftmark}}
\renewcommand{\chaptermark}[1]{%
\markboth{\thechapter.\ #1}{}}

%Ridefinisco lo stile plain della pagina
\fancypagestyle{plain}{%
	\lhead{\includegraphics[height=50pt]{logo.eps}}
	\chead{}
	\rhead{HappyCode inc \\ happycodeinc@gmail.com}
	\lfoot{BR-jsys}
	\cfoot{\thepage}
	\renewcommand{\headrulewidth}{1pt}
	\renewcommand{\footrulewidth}{1pt}
}

\begin{document}

%definizione variabili 
\newcommand{\lv}{1.1} % latest version
%fine definizione variabili

\hyphenation{re-qui-si-ti in-se-ri-men-to}

\begin{titlepage}
\begin{center}
\vspace*{0.5in}
\includegraphics{logo.eps}
\vspace*{0.2in}

{\Large \textbf{BR-jsys}}
{\Large \emph{business rules} per sistemi gestionali in architettura J2EE } 
\vspace{2in}

\LARGE \textbf {SPECIFICA TECNICA}
\par\rule{10cm}{0.4pt} \par {\large Versione \lv - \today}


\end{center}
\end{titlepage}
\vspace*{0.5in}



\begin{center}
\thispagestyle{plain}
\begin{table}[htbp]
\large{
\begin{tabular}{l}
\Large{\textbf{\textsf{Capitolato: ''BR-jsys``}}} \\
\begin{tabular}{||p{6cm}||p{6cm}||} \hline
\textbf{Data creazione:} & 23/01/08 \\ \hline
\textbf{Versione:} & \lv  \\ \hline
% ----------------------------------------------------------------------------autori
\textbf{Stato del documento:} & formale, esterno \\ \hline
\textbf{Redazione:} & Luca Appon e Mattia Meroi \\ \hline
\textbf{Revisione:} & Marco Tessarotto   \\ \hline
\textbf{Approvazione:}  & Elena Trivellato\\ \hline
\end{tabular} \\
\end{tabular}
}
\end{table}

\begin{table}[hbtp]
\large{
\begin{tabular}{l}
\Large{\textbf{\textsf{Lista di distribuzione}}} \\
%  -------------------------------------------------------------lista di distribuzione
\begin{tabular}{||p{6cm}||p{6cm}||} \hline
{Elena Trivellato}& Responsabile di progetto \\ \hline 
{Filippo Carraro}& Amministratore \\ \hline
{Marco Tessarotto}& Verificatore \\ \hline
{Tullio Vardanega}& Committente \\ \hline
{Renato Conte}& Committente \\ \hline
{Zucchetti S.r.l.}& Proponente \\ \hline
\end{tabular} \\
\end{tabular}
}
\end{table}

\begin{table}[hbtp]
\large{
\begin{tabular}{l}
\Large{\textbf{\textsf{Diario delle modifiche}}} \\
\begin{tabular}{||p{2cm}||p{3.5cm}||p{6cm}||}
\hline
\textbf{Versione} & \textbf{Data rilascio} & \textbf{Descrizione} \\ \hline
%-------------------------------------------------------------------------------diario modifiche
1.1 & 22/01/2008 & Modifica al layout dei documenti.\\ \hline
1.0 & 21/12/2007 & Documento sottoposto a revisionamento automatico.\\ \hline
0.1 & 2007/11/19 & Stesura preliminare del documento. \\ \hline
\end{tabular} \\
\end{tabular}

}
\end{table}
\end{center}
\newpage

\chapter{Introduzione}
\section{Scopo del documento}
Il presente documento fornisce indicazioni da rispettare per la realizzazione del prodotto software br-jsys sulla base dell' AnalisiDeiRequisiti e descrive l'archittettura indicandone (e descrivendone) le componenti utilizzate.
\section{Scopo del prodotto}
Per lo scopo del prodotto si faccia riferimento al paragrafo 1.1 dell' AnalisiDeiRequisiti
\section{Riferimenti}
\subsection{Riferimenti normativi}
\begin{itemize}
\item Capitolato d'appalto BR-jsys
\item AnalisiDeiRequisiti 1.3 
\item NormeDiProgetto 0.3
\end{itemize}
\subsection{Riferimenti informativi}
\begin{itemize}
\item The Definitive ANTLR Reference
\item Ingegneria del software 8a edizione - Ian Sommerville 
\item VANNO INSERITI GLI ALTRI RIFERIMENTI UTILIZZATI!!!!
\end{itemize}
\section{Glossario}
Viene fornito come documento esterno chiamato Glossario\_0\_4.pdf.

\chapter{Definizione del prodotto}




\chapter{Descrizione dei singoli componenti}




\chapter{Stime di fattibilit\`a e di bisogno di risorse}




\chapter{Tracciamento della relazione componenti - requisiti}







\newpage
\tableofcontents

\end{document}
	
