\documentclass[11pt,titlepage,a4paper]{report}

%INCLUSIONE PACCHETTI
%---------------------------------------------
\usepackage[italian]{babel}
\usepackage{fancyhdr}
\usepackage{graphicx}
\graphicspath{{./pics/}}    % cartella di salvataggio immagini

% STILE DI PAGINA
%---------------------------------------------
\pagestyle{fancy}
\renewcommand{\sectionmark}[1]{\markright{\thesection.\ #1}}
\lhead{\nouppercase{\rightmark}}
\rhead{\nouppercase{\leftmark}}
\renewcommand{\chaptermark}[1]{%
\markboth{\thechapter.\ #1}{}}

%Ridefinisco lo stile plain della pagina
\fancypagestyle{plain}{%
	\lhead{\includegraphics[height=50pt]{logo.eps}}
	\chead{}
	\rhead{HappyCode inc \\ happycodeinc@gmail.com}
	\lfoot{BR-jsys}
	\rfoot{\dt - \lv}
	\cfoot{\thepage}
	\renewcommand{\headrulewidth}{5pt}
	\renewcommand{\footrulewidth}{0.4pt}
}


\begin{document}


%definizione variabili 
\newcommand{\lv}{ 0.5 } % latest version
\newcommand{\dt}{ Specifica Tecnica }% Document title
\newcommand{\Glossario}{ Glossario.1.4.pdf }
\newcommand{\br}{business rule}
\newcommand{\brs}{business rules}
\newcommand{\bo}{business object}
\newcommand{\bos}{business objects}
\newcommand{\re}{repository}
\newcommand{\brp}{BusinessRuleParser}
\newcommand{\brl}{BusinessRuleLexer}
%fine definizione variabili


\hyphenation{glos-sa-rio es-pli-ci-to ve-ri-fi-ca-re re-po-si-to-ry se-gna-la-ta coe-ren-za}
\begin{titlepage}\begin{center}
\vspace*{0.5in}
\includegraphics{logo.eps}
\vspace*{0.2in} \\
{\Large \textbf{BR-jsys}}
{\Large \emph{business rules} per sistemi gestionali in architettura J2EE } 
\vspace{2in} \\
\Huge \textsc{ \dt }
\par\rule{10cm}{0.4pt} \par {\large Versione \lv - \today} \\
\end{center}\end{titlepage}
\vspace*{0.5in}


\begin{center}
\thispagestyle{plain}
\begin{table}[htbp]
\large{
\begin{tabular}{l}
\Large{\textbf{\textsf{Capitolato: ''BR-jsys``}}} \\
\begin{tabular}{||p{6cm}||p{6cm}||} \hline
\textbf{Data creazione:} & 18/11/2007 \\ \hline
\textbf{Versione:} & \lv \\ \hline
\textbf{Stato del documento:} & formale, esterno \\ \hline
% ----------------------------------------------------------------------------autori
\textbf{Redazione:} & Alessia Trivellato ­18/11/2007 \\ \hline
\textbf{Revisione:} &    \\ \hline
\textbf{Approvazione:}  & \\ \hline
\end{tabular} \\
\end{tabular}
}
\end{table}
\begin{table}[hbtp]
\large{
\begin{tabular}{l}
\Large{\textbf{\textsf{Lista di distribuzione}}} \\
\begin{tabular}{||p{6cm}||p{6cm}||} \hline
%  -------------------------------------------------------------lista di distribuzione
{HappyCode inc}& Gruppo di lavoro\\ \hline
{Tullio Vardanega, Renato Conte}& Rappresentanti del committente \\ \hline 
{Zucchetti S.r.l}& Azienda committente\\ \hline
\end{tabular} \\
\end{tabular}
}
\end{table}

\begin{table}[hbtp]
\large{
\begin{tabular}{l}
\Large{\textbf{\textsf{Diario delle modifiche}}} \\
\begin{tabular}{||p{2cm}||p{3.5cm}||p{6cm}||} \hline
%-------------------------------------------------------------------------------diario modifiche
\textbf{Versione} & \textbf{Data rilascio} & \textbf{Descrizione} \\ \hline
0.5 & 2008/02/05 & Aggiunta del nome del file nel modello di documento.\\ \hline
0.4 & 22/01/2008 & Modifica al layout dei documenti.\\ \hline
0.3 & 22/12/2007 & Aggiornamento requisiti. \\ \hline \hline
0.2 & 21/12/2007 & Documento sottoposto a revisionamento automatico.\\ \hline
0.1 & 18/12/2007 & Stesura preliminare del documento. \\ \hline

\end{tabular} \\
\end{tabular}

}
\end{table}
\end{center}
\newpage
\tableofcontents
\chapter{Introduzione}
\section{Scopo del documento}
Il presente documento si ripropone di descrivere il sistema software ``BR-jsys'', sulla base del documento di Analisi dei Requisiti, dal punto di vista architetturale. Lo strumento da noi adottato sar\`a il linguaggio UML, sottoforma di diagrammi delle classi. L'approccio adottato nell'illustrare il sistema sar\`a di tipo ``Bottom-up'' Fornir\`a inoltre una visione pi\`u dettagliata delle componenti da realizzare.
Verr\`a definito il contesto d'uso del sistema e fornita la decomposizione di questo in componenti principali.
\section{Scopo del prodotto}
Per lo scopo del prodotto si faccia riferimento al documento di Analisi dei Requisiti.
\section{Glossario}
Viene fornito come documento esterno chiamato \Glossario .
\section{Riferimenti}
\subsection{Riferimenti normativi}
\begin{itemize}
\item Capitolato d'appalto BR-jsys
\item AnalisiDeiRequisiti
\item NormeDiProgetto
\item PianoDiProgetto
\item PianoDiQualifica
\item ``Ingegneria del software'' 8a edizione - Ian Sommerville 
\item ``The Definitive ANTLR Reference''
\end{itemize}
%Ho tolto i riferimenti informativi

\chapter{Definizione del prodotto}
\subsection{Metodo e formalismo di specifica}
%Copiato di sana pianta dalla Nonsolocodice
Abbiamo attuato la progettazione del prodotto con l'uso di diagrammi UML 2.1 in quanto linguaggio internazionalmente riconosciuto e standardizzato.
Il software di cui l’azienda ha fatto uso per la creazione dei diagrammi e' ArgoUML 0.24. L'utilizzo di tale software deriva dalla completezza
di strumenti di cui e' fornito, i quali soddisfano in pieno le nostre necessita'.

\subsection{Presentazione dell'architettura generale del sistema e identificazione dei componenti architetturali di alto livello}
Lo schema generale del prodotto è specificato nell'immagine sottostante.
%Immagine simile a quella che abbiamo mandato a tullio
Abbiamo deciso di suddividere il prodotto in quattro macrocomponenti:
\begin{enumerate}
 \item GUI:Fornisce all'utente un interfaccia grafica minimale consentendogli di effettuare le operazioni di inserimento, cancellazione e interrogazione del repository in maniera user-friendly.
\item Business Rule: Componente che rappresenta la \br che l'utente dichiara e che deve essere passata al validatore per compilarla.
\item Validatore: Componente in grado di accettare in input una \br, di validarla, e di inserirla, se scritta correttamente, nel \re.
\item Comunicatore: Componente in grado di connettersi al DBMS esterno e di comunicare con esso tramite i formalismi del linguaggio XQuery.
Sar\`a appunto tramite Query che il DBMS effettuer\`a operazioni di insermento, cancellazione o di ricerca nel \re qualora un componente lo richiedesse.
\end{enumerate}
Questa suddivisione, per quanto minimale, consente di individuare le componenti che verranno ampliamente descritte nel successivo capitolo.

\chapter{Descrizione dei singoli componenti}
Qui di seguito analizzeremo separatamente le quattro macrocomponenti elencate il capitolo precedente e ne daremo una descrizione pi\`u approfondita.
\section{GUI}
\subsection{Diagramma delle classi}
\subsection{Gui}
%Diagramma della classe GUI...comprensivo di package
\paragraph{Tipo, obiettivo e funzione del componente}
\paragraph{Relazioni d'uso di altre componenti}
\paragraph{Interfacce con e relazioni di uso da altre componenti}
\paragraph{Attivit\`a svolte e dati trattati}

\section{Validatore}
\subsection{Diagramma delle classi}
\subsection{Validator}%facade!!!->va descritto
\paragraph{Tipo, obiettivo e funzione del componente}
\paragraph{Relazioni d'uso di altre componenti}
\paragraph{Interfacce con e relazioni di uso da altre componenti}
\paragraph{Attivit\`a svolte e dati trattati}

\subsection{BusinessRuleLexer}
\paragraph{Tipo, obiettivo e funzione del componente}
Questa componente, derivata da org.antlr.runtime.Lexer, non \`e altro che una classe wrapper per la stringa che rappresenta la \br che aggiunge ad essa funzionalit\`a necessarie per il successivo parsing.
\paragraph{Relazioni d'uso di altre componenti}
Nessuna.
\paragraph{Interfacce con e relazioni di uso da altre componenti}
La componente BusinessRuleParser necessita la componente BusinessRuleLexer che verr\`a approfondita successivamente.
\paragraph{Attivit\`a svolte e dati trattati}
BusinessRuleLexer mette a disposizione vari metodi per la lettura del testo della \br, nonch\`e per la gestione di eventuali eccezzioni avvenute in fase di validazione.
%voglio i vostri pareri
\textit{\textbf{Nota:}Questa classe \`e stata creata utilizzando uno strumento automatico per la generazione di parser data in input la specifica di una grammatica.}
\subsection{BusinessRuleParser}
\paragraph{Tipo, obiettivo e funzione del componente}
Questo componente, derivata da org.antlr.runtime.Parser, effettua il parsing della stringa che rappresenta la \br. Effettua il controllo sintatico della regola, il controllo semantico facendo un controllo sui tipi dei dati siano essi costanti oppure campi dati di \bos. Mentre effettua la validazione BusinessRuleParser genera l'albero di pasing secondo le specifiche presenti nel controllo semantico. Infine \`e in grado di dare informazioni accurate riguardo eventuali errori in fase di validazione.
\paragraph{Relazioni d'uso di altre componenti}
Questa componente necessita di un TokenStream fornitogli indirettamente da BusinessRuleLexer. Nonchè ha bisogno della componente eccezzione TypeCollisionException per trattare gli errori in fase di validazione.
\paragraph{Interfacce con e relazioni di uso da altre componenti}
\brp viene messa in relazione con XMLParser che verr\`a trattata successivamente.
\`E utilizzata inoltre dalla componente Validator.
\paragraph{Attivit\`a svolte e dati trattati}
\brp mette a disposizione vari metodi per effettuare il parsing e per i test semantici, nonch\`e dispone di numerosi campi dato per la ricognizione dei token.
\textit{\textbf{Nota:}Questa classe \`e stata creata utilizzando uno strumento automatico per la generazione di parser data in input la specifica di una grammatica.}
\subsection{TypeCollisionException}
\paragraph{Tipo, obiettivo e funzione del componente}
Questa componente, derivata dalla classe org.antlr.runtime.RecognitionException, permentte di ricavare informazioni sugli eventuali errori avvenuti in fase di parsing della \br, essa in definitiva aggiunge alla sua superclasse la possibilit\`a di riportare informazioni su errori di tipo.
\paragraph{Relazioni d'uso di altre componenti}
Nessuna.
\paragraph{Interfacce con e relazioni di uso da altre componenti}
La componente TypeCollisionException \`e utilizzata da \brp per sollevare eccezioni da errori sul controlo dei tipi.
\paragraph{Attivit\`a svolte e dati trattati}
Viene ridefinito il metodo printStackTrace() e messo a disposizione un costruttore per avere informazioni sui tipi che si sono rivelati incompatibili.
\subsection{XMLParser}%probabilmente è un facade anche questo
\paragraph{Tipo, obiettivo e funzione del componente}
\paragraph{Relazioni d'uso di altre componenti}
\paragraph{Interfacce con e relazioni di uso da altre componenti}
\paragraph{Attivit\`a svolte e dati trattati}

\subsection{BusinessObjects}%è un package
\paragraph{Tipo, obiettivo e funzione del componente}
\paragraph{Relazioni d'uso di altre componenti}
\paragraph{Interfacce con e relazioni di uso da altre componenti}
\paragraph{Attivit\`a svolte e dati trattati}

\section{Business Rule}
\subsection{Diagramma delle classi}
\subsection{BusinessRule}
\paragraph{Tipo, obiettivo e funzione del componente}
\paragraph{Relazioni d'uso di altre componenti}
\paragraph{Interfacce con e relazioni di uso da altre componenti}
\paragraph{Attivit\`a svolte e dati trattati}

\section{Comunicatore}
\subsection{Diagramma delle classi}
\subsection{Communicator}
\paragraph{Tipo, obiettivo e funzione del componente}
\paragraph{Relazioni d'uso di altre componenti}
\paragraph{Interfacce con e relazioni di uso da altre componenti}
\paragraph{Attivit\`a svolte e dati trattati}

\subsection{GUICommunicator}
\paragraph{Tipo, obiettivo e funzione del componente}
\paragraph{Relazioni d'uso di altre componenti}
\paragraph{Interfacce con e relazioni di uso da altre componenti}
\paragraph{Attivit\`a svolte e dati trattati}

\subsection{InterpreterCommunicator}
\paragraph{Tipo, obiettivo e funzione del componente}
\paragraph{Relazioni d'uso di altre componenti}
\paragraph{Interfacce con e relazioni di uso da altre componenti}
\paragraph{Attivit\`a svolte e dati trattati}

\subsection{ValidatorCommunicator}
\paragraph{Tipo, obiettivo e funzione del componente}
\paragraph{Relazioni d'uso di altre componenti}
\paragraph{Interfacce con e relazioni di uso da altre componenti}
\paragraph{Attivit\`a svolte e dati trattati}

\begin{itemize}
\item Il validatore delle business-rules \`e un modulo software. Il suo obiettivo \`e verificare se la business-rules pu\`o essere inserita nel repository e, in caso affermativo tradurla in formato in XML comprensibile all'interprete esterno, altrimenti segnalare l'anomalia con la rispettiva eccezione. Il nostro validatore ha un'architettura che comprende i seguenti sotto-componenti:
\begin{enumerate}
\item un analizzatore lessicale (lexer) che prende i token del nostro linguaggio e li traduce in una forma interna;
%\item una tabella dei simboli che contiene le informazioni sui nomi delle entit\`a (nomi delle classi, nomi degli oggetti, etc.) utilizzate nel testo %che si sta traducendo;

\item un analizzatore sintattico (parser) che controlla la sintassi del linguaggio che si sta traducendo: utilizza la grammatica del linguaggio da noi definita e costruisce l'albero sintattico;
%\item un albero sintattico, ovvero una struttura interna che rappresenta il programma da compilare;
%NON DICHIARANDO VARIABILI NON ABBIAMO BISONGO DI UNA TABELLA DEI SIMBOLI DURANTE IL PARSING
%\item un analizzatore semantico che utilizza le informazioni dell'albero sintattico per verificare la correttezza semantica del testo in ingresso;
%PER QUANTO RIGUARDA LA SEMANTICA O LA FA IL CONVERTITORE XML O LA FA IL PARSER
\item un generatore di codice che ``cammina'' sull'albero e genera il codice XML da inserire nel repository.
\end{enumerate}

\item Interfaccia grafica (GUI) consentir\`a all'utente le seguenti operazioni:
\begin{enumerate}
\item Inserimento di una nuova business rule;
\item Cancellazione di business rule presenti nel repository;
\item Esecuzione di query definite dall'utente;
\end{enumerate}

\end{itemize}

\subsection{Relazioni d'uso di altre componenti}
\begin{itemize}
\item{Il Validatore dialoga con il modulo di comunicazione nella fase di inserimento delle business rule validata. }
Il modulo di comunicazione dialoga con il DBMS attraverso l'invio di richieste di inserimento, cancellazione o recupero di informazioni. Ha relazioni inoltre con l'interprete esterno attraverso cui accetta richieste di business rules relative a un determinato business object. Successivamente il modulo di connessione chiama il DBMS richiedendo le opportune business rules che verranno restituite e inviate all'interprete. Il validatore dialoga con il DBMS esterno chiedendo di inserire le regole precedentemente validate. La GUI invece dialoga direttamente con il modulo di connessione durante la fase di cancellazione di una business rules e, direttamente con il validatore durante l'inserimento.
\end{itemize}
\subsection{Interfacce con e relazioni di uso da altre componenti}
Un interfaccia ``fornisce'' i servizi dal componente, ovvero i metodi che possono essere chiamati da un utente del programma. ``Specifica '' inoltre quali servizi devono essere forniti da altri componenti del sistema. Se questi non sono disponibili il componente non funzioner\`a senza per\`o compromettere l'indipendenza o la consegnabilit\`a di una componente, poich\`e non richiede che sia utilizzato uno specifico componente per fornire i servizi. Nel nostro caso:
\begin{itemize}
\item La componente GUI richieder\`a che l'interfaccia fornita comprenda i metodi per inserire, cancellare e cercare le business rules. Richieder\`a quindi un'interfaccia di gestione e un'interfaccia dati.
\item La componente validatore richieder\`a due interfaccie: di input e di output. La prima deve colloquiare tra l'utente e il validatore, la seconda dal validatore al modulo di connessione.
\item La componente modulo di connessione richieder\`a che l'interfaccia colloqui con l'interprete esterno.
\end{itemize}

\chapter{Descrizione dei diagrammi di attivit\`a, di sequenza e collaborazione}%attenzione...dobbiamo farli!

\chapter{Stime di fattibilit\`a e di bisogno di risorse}
Dopo aver analizzato il problema attraverso schemi progetturali sono state individuate le risorse necessarie  per la realizzazione del prodotto. Attraverso l'utilizzo di software open source siamo riusciti a contenere i costi e contemporaneamente a rendere disponibili tutte le risorse necessarie.
Le risorse necessarie ai nostri componenti per affrontare le varie problematiche di comunicazione, sviluppo del codice, gestione degli archivi, verifica dei documenti e del sistema sono state descritte nel documento Piano di Qualifica. Nella fase di specifica tecnica e successivamente di progettazione verrano utilizzati diagrammi UML delle classi e degli oggetti realizzati con Dia. Lo sviluppo avr\`a luogo singolarmente o in piccoli gruppi a seconda della natura della problematica da affrontare. Il documento/codice avr\`a in ogni modo un unico proprietario incaricato di renderlo pubblico tramite server SVN.
%Parlare ASSOLUTAMENTE di ANTLR e di ANTLRWORKS nonchè di poseidon, eclipse e se la elena per fare la gui utilizza roba esterna dichiararlo...ma poseidon non è + open source!!
\chapter{Tracciamento della relazione componenti-requisiti}



\end{document}

    
