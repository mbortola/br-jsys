
\documentclass[11pt,titlepage,a4paper]{report}

\usepackage[italian]{babel}
\usepackage{fancyhdr}
\usepackage{graphicx}
\usepackage{hyperref}

%\usepackage{lastpage} % total page count

\usepackage{color}
\usepackage{lastpage} % total page count

\graphicspath{{./pics/}} % cartella di salvataggio immagini

\pagestyle{fancy}
\renewcommand{\sectionmark}[1]{\markright{\thesection.\ #1}}
\lhead{\nouppercase{\rightmark}}
\rhead{\nouppercase{\leftmark}}
\renewcommand{\chaptermark}[1]{%
\markboth{\thechapter.\ #1}{}}


\fancypagestyle{plain}{%
	\lhead{\includegraphics[height=50pt]{logo.eps}}
	\chead{}
	\rhead{HappyCode inc \\ happycodeinc@gmail.com}
	\lfoot{BR-jsys}
	\cfoot{\thepage\ / \pageref{LastPage}}
	\rfoot{\dt - \lv}
	\renewcommand{\headrulewidth}{1pt}
	\renewcommand{\footrulewidth}{1pt}
}
	\lhead{\includegraphics[height=50pt]{logo.eps}}
	\chead{}
	\rhead{HappyCode inc \\ happycodeinc@gmail.com}
	\lfoot{BR-jsys}
	\cfoot{\thepage\ / \pageref{LastPage}}
	\rfoot{\dt - \lv}
	\renewcommand{\headrulewidth}{1pt}
	\renewcommand{\footrulewidth}{1pt}

\hypersetup{
    colorlinks=true,       % false: boxed links; true: colored links
   linkcolor=[rgb]{0.11,0.55,0.83},          % color of internal links
    urlcolor=cyan           % color of external links
}
\definecolor{err}{rgb}{0.9,0.1,0.1}

% fine layout% layout
\begin{document}
%definizione variabili 
\newcommand{\lv}{ 1.3 } % latest version
\newcommand{\dt}{ Test Report }% Document title
%common variables
\newcommand{\br}{\underline{business rule}}
\newcommand{\brs}{\underline{business rules}}
\newcommand{\bo}{\underline{business object}}
\newcommand{\bos}{\underline{business objects}}
\newcommand{\rp}{\underline{repository}}
\newcommand{\brp}{BusinessRuleParser}
\newcommand{\brl}{BusinessRuleLexer}
\newcommand{\BR}{\underline{BusinessRule}}

%nomi dei componenti
\newcommand{\AT}{Alessia Trivellato}
\newcommand{\ET}{Elena Trivellato}
\newcommand{\FC}{Filippo Carraro}
\newcommand{\LA}{Luca Appon}
\newcommand{\MB}{Michele Bortolato}
\newcommand{\MT}{Marco Tessarotto}
\newcommand{\MM}{Mattia Meroi}%altre variabili
% ultime versioni dei documenti da modificare solo alla fine
\newcommand{\AR}{AnalisiDeiRequisiti.2.6.pdf}
\newcommand{\DdP}{DefinizioneDiProdotto.0.9.pdf}
\newcommand{\G}{ Glossario.1.8.pdf }
\newcommand{\NdP}{NormeDiProgetto.2.0.pdf}
\newcommand{\PdQ}{ PianoDiQualifica.1.4.pdf }
\newcommand{\PdP}{ PianoDiProgetto.1.7.pdf }
\newcommand{\ST}{SpecificaTecnica.1.5.pdf}
\newcommand{\TR}{TestReport.0.7.pdf}
\newcommand{\MU}{ManualeUtente.0.3.pdf}%nomi documenti
%fine definizione variabili
\hyphenation{
 a-na-lo-go
 as-so-cia-zio-ne
 %attività non si può inserire come tutte le parole accentate che vanno messe nel testo semplice scritte at\-ti\-vi\-tà o come variabile
 coe-ren-za
 com-po-nen-ti
 con-si-glia-bi-le
 des-crit-te
 des-cri-zio-ni
 di-a-gram-ma
 di-a-gram-mi
 e-le-men-to
 e-se-gui-re
 e-si-sten-ti
 es-pli-ci-to
 glo-bal-men-te
 glos-sa-rio
 in-se-ri-men-to
 li-vel-lo
 ne-ces-sa-rio
 per-met-te-re
 re-po-si-to-ry
 re-vi-sio-na-men-to
 ri-chies-te
 se-le-zio-na-ta
 se-gna-la-ta
 va-li-da-zio-ne
 va-ria-bi-li
 ve-ri-fi-ca-re
 vi-sua-liz-za-te
 e-ven-tua-li
 o-pe-ra-zio-ne
 ar-chi-via-zio-ne
 mo-di-fi-ca
 ar-chi-vio
 des-cri-zio-ne
 pa-ren-te-si
 i-ni-zia
}


%sillabazione

\begin{titlepage}\begin{center}
\vspace*{0.5in}
\includegraphics{logo.eps}
\vspace*{0.2in} \\
{\Large \textbf{BR-jsys}}
{\Large \emph{business rules} per sistemi gestionali in architettura J2EE } 
\vspace{2in} \\
\Huge \textsc{ \dt }
\par\rule{10cm}{0.4pt} \par {\large Versione \lv - \today} \\
\end{center}\end{titlepage}

\vspace*{0.5in}%pagina del titolo


\begin{center}
\thispagestyle{plain}
\begin{table}[htbp]
\large{
\begin{tabular}{l}
\Large{\textbf{\textsf{Capitolato: ''BR-jsys``}}} \\
\begin{tabular}{|p{6cm}|p{6cm}|}
\hline
\textbf{Data creazione:} & 02/03/2008 \\ \hline
\textbf{Versione:} & \lv \\ \hline
\textbf{Stato del documento:} & Formale, esterno \\ \hline
% ----------------------------------------------------------------------------autori
& \\ \hline
\textbf{Revisione RQ} & \\ \hline
\textbf{Redazione:} & \MT \\ \hline
\textbf{Revisione:} & \MB \\ \hline
\textbf{Approvazione:}  & \AT  \\ \hline
& \\ \hline
\textbf{Revisione RA} & \\ \hline
\textbf{Redazione:} & \ET \\ \hline
\textbf{Revisione:} & \AT \\ \hline
\textbf{Approvazione:}  & \MT  \\ \hline
\end{tabular} \\
\end{tabular}
}
\end{table}

\begin{table}[hbtp]
\large{
\begin{tabular}{l}
\Large{\textbf{\textsf{Lista di distribuzione}}} \\
\begin{tabular}{|p{6cm}|p{6cm}|} \hline
%  -------------------------------------------------------------lista di distribuzione
{HappyCode inc}& Gruppo di lavoro \\ \hline
{Tullio Vardanega, Renato Conte}& Committenti \\ \hline 
{Zucchetti S.r.l}& Azienda proponente\\ \hline
\end{tabular} \\
\end{tabular}
}
\end{table}
\begin{table}[hbtp]

\Large{\textbf{\textsf{Diario delle modifiche}}} \\
\begin{small}
\begin{tabular}[t]{|p{1,2cm}|p{1.9cm}|p{2.9cm}|p{5cm}|} \hline
Versione & Data & Autore & Descrizione \\ \hline
%-------------------------------------------------------------------------------diario modifiche
1.3 & 14/03/2008 & \MT & Esiti dei test di convalida. \\ \hline
1.2 & 14/03/2008 & \ET & Aggiunta degli esiti di JUnit. \\ \hline
1.1 & 14/03/2008 & \MT & Esiti dei test di regressione. \\ \hline
1.0 & 13/03/2008 & \MT & Correzione dei test errati. \\ \hline
0.9 & 12/03/2008 & \AT & Modifiche ai test case dopo l'introduzione del ';' nella nuova grammatica. \\ \hline
0.8 & 11/03/2008 & \ET & Aggiunta sezione ``Analisi statica'' e tabelle degli errori. \\ \hline
0.7 & 07/03/2008 & \MT & Aggiunta dei 'test case' per le integrazioni di componenti. Aggiunta esiti dei test.\\ \hline
0.6 & 07/03/2008 & \MM & Aggiunta la tabella di tracciamento 'requisiti - test'.\\ \hline
0.5 & 06/03/2008 & \LA & Aggiunto Sommario.\\ \hline
0.4 & 05/03/2008 & \MT & Aggiunta delle tabelle per contenere gli esiti, e dei 'test case' per il 'comunicatore' \\ \hline
0.3 & 04/03/2008 & \MT & Modifica alla nomenclatura dei test.\\ \hline
0.2 & 03/03/2008 & \MT & Aggiunta dei 'test case' per il 'validatore' .\\ \hline
0.1 & 02/03/2008 & \MT & Stesura iniziale del documento.\\ \hline
\end{tabular} \\
\end{small}


\end{table}
\end{center}
\newpage
\tableofcontents

\chapter*{Sommario}
\addcontentsline{toc}{chapter}{I Sommario}
Il presente documento illustra le strategie utilizzate per testare il prodotto BR-jsys in ogni sua componente.

\chapter*{Glossario}
\addcontentsline{toc}{chapter}{II Glossario}
Per gli altri termini non citati nella tabella soprastante si fa riferimento al file esterno \G. I termini presenti in questo documento la cui descrizione \`e riportata nel glossario sono evidenziati mediante sottolineatura.

\chapter{Introduzione}

\section{Scopo del documento}
Il presente documento viene redatto al fine di illustrare la progettazione dei \textit{test case} nel dettaglio indicando un insieme di input e il corrispondente insieme di \textit{output attesi}. Verranno inoltre riportati nel presente documento gli esiti dei test eseguiti sul sistema ``Br-jsys'' al fine di non appesantire la lettura del documento \PdQ .

\section{Definizioni, acronimi, abbreviazioni}
Nella tabella di seguito vengono riportate tutte le abbreviazioni utlizzate nel documento, comprese le sigle utilizzate per l'identificazione dei test.
\begin{center}
\begin{tabular}{|p{3.0cm}|p{8.5cm}|} \hline
\textbf{Abbreviazione} & \textbf{Significato} \\ \hline

VT\textit{\#} & Con VT seguito da un numero intero sono indicati i test che riguardano la 'macrocomponente validatore'.\\ \hline
CT\textit{\#} & Con CT seguito da un numero intero sono indicati i test che riguardano la 'macrocomponente comunicatore'.\\ \hline
GT\textit{\#} & Con GT seguito da un numero intero sono indicati i test che riguardano la 'macrocomponente gui'.\\ \hline
CVT\textit{\#} & Con CVT seguito da un numero intero sono indicati i test di integrazione tra 'validatore' e 'comunicatore'.\\ \hline
GVT\textit{\#} & Con GVT seguito da un numero intero sono indicati i test integrazione tra 'gui' e 'validatore'.\\ \hline
S\textit{\#} & Con S seguito da un numero intero sono indicati i test di integrazione del sistema.\\ \hline
RT\textit{} & Con RT si intende la riesecuzione di un test (test di regressione) avvenuta a fronte di una modifica al codice.\\ \hline

\end{tabular} \\
\end{center}

\chapter{Analisi Statica del codice}

L'analisi statica automatica sui file Java \`e stata effettuata con l'utilizzo di JiveLint 1.21, uno strumento in grado di trovare codice e variabili inutilizzate oltre ad altri bugs/errori nel codice.
Di seguito riportiamo gli errori riscontrati nel nostro codice.

Per rendere le tabelle pi\`u leggibili gli stessi errori in righe diverse dello stesso file.java seguiranno la seguente notazione: \\
NomeClasse.java:riga1-riga2-..-rigan Rule(codice) Errore. \\

\textbf{BusinessRule.java}
\begin{center}
\begin{tabular}{|p{12cm}|} \hline
BusinessRule.java:26 Rule(1022) Line is too long. Prefer lines no longer than 80 characters. \\ \hline
BusinessRule.java:26 Rule(1038) Declare parameters final. \\ \hline
\end{tabular}
\end{center}

\textbf{Communicator.java}
\begin{center}
\begin{tabular}{|p{12cm}|} \hline
Communicator.java:6-7-8 Rule(1002) Wildcard import declaration. \\ \hline
Communicator.java:53-91-93-95 Rule(1011) Output on System.out or System.err. \\ \hline
Communicator.java:35-108 Rule(1038) Declare parameters final. \\ \hline
\end{tabular}
\end{center}

\textbf{GUICommunicator.java}
\begin{center}
\begin{tabular}{|p{12cm}|} \hline
GUICommunicator.java:3 Rule(1002) Wildcard import declaration. \\ \hline
GUICommunicator.java:121 Rule(1005) Type in catch clause too general. \\ \hline
GUICommunicator.java:64-123 Rule(1011) Output on System.out or System.err. \\ \hline
GUICommunicator.java:25-49-77 Rule(1038) Declare parameters final. \\ \hline
\end{tabular}
\end{center}

\textbf{InterpreterCommunicator.java}
\begin{center}
\begin{tabular}{|p{12cm}|} \hline
InterpreterCommunicator.java:3-4 Rule(1002) Wildcard import declaration. \\ \hline
InterpreterCommunicator.java:69 Rule(1011) Output on System.out or System.err. \\ \hline
InterpreterCommunicator.java:27-51 Rule(1038) Declare parameters final.\\ \hline
\end{tabular}
\end{center}

\textbf{ValidatorCommunicator.java}
\begin{center}
\begin{tabular}{|p{12cm}|} \hline
ValidatorCommunicator.java:3 Rule(1002) Wildcard import declaration. \\ \hline
ValidatorCommunicator.java:65 Rule(1011) Output on System.out or System.err. \\ \hline
ValidatorCommunicator.java:23-48 Rule(1038) Declare parameters final. \\ \hline
\end{tabular}
\end{center}

\textbf{Gui.java}
\begin{center}
\begin{tabular}{|p{12cm}|} \hline
Gui.java:4-5-6 Rule(1002) Wildcard import declaration.\\ \hline
Gui.java:666-731-808 Rule(1005) Type in catch clause too general.\\ \hline
Gui.java:668 Rule(1011) Output on System.out or System.err.\\ \hline
Gui.java:652-674-693-701-709-737-770-792-814 Rule(1038) Declare parameters final.\\ \hline
\end{tabular}
\end{center}

\textbf{Validator.java}
\begin{center}
\begin{tabular}{|p{12cm}|} \hline
Validator.java:3 Rule(1002) Wildcard import declaration.\\ \hline
Validator.java:31-46 Rule(1038) Declare parameters final.\\ \hline
\end{tabular}
\end{center}

\textbf{TypeCollisionExceptions.java}
\begin{center}
\begin{tabular}{|p{12cm}|} \hline
TypeCollisionException.java:18 Rule(1038) Declare parameters final.\\ \hline
\end{tabular}
\end{center}

\textbf{XMLParser.java}
\begin{center}
\begin{tabular}{|p{12cm}|} \hline
XMLParser.java:3-5-6-11 Rule(1002) Wildcard import declaration.\\ \hline
XMLParser.java:96-128-138 Rule(1011) Output on System.out or System.err.\\ \hline
XMLParser.java:30-43-88 Rule(1038) Declare parameters final.\\ \hline
\end{tabular}
\end{center}

\textbf{Archivio.java}
\begin{center}
\begin{tabular}{|p{12cm}|} \hline
Archivio.java:1 Rule(1037) No @author javadoc tag found.\\ \hline
\end{tabular}
\end{center}

\textbf{Articolo.java}
\begin{center}
\begin{tabular}{|p{12cm}|} \hline
Articolo.java:1 Rule(1037) No @author javadoc tag found.\\ \hline
\end{tabular}
\end{center}

\textbf{SubArch.java}
\begin{center}
\begin{tabular}{|p{12cm}|} \hline
SubArch.java:1 Rule(1037) No @author javadoc tag found.\\ \hline
\end{tabular}
\end{center}

Tali errori nel codice sorgente sono stati tutti corretti. Una nuova esecuzione del programma non ha rilevato ulteriori difetti nel codice.

\chapter{Analisi dinamica}
Per analizzare il comportamento del prodotto durante la sua fase di esecuzione, abbiamo progettato alcune batterie di test atte a valutarne la correttezza di avanzamento, attraverso dati di input prestabiliti che dovevano produrre risultati attesi.
Ogni \textit{test case} \`e formato da una tripla <ingresso,uscita,ambiente>, dove l'ambiente \`e costituito dalla funzionalit\`a da testare.
Ogni \textit{test case} \`e documentato con descrizione d'uso e un elenco di requisiti soddisfatti. 


\section{Test sulle componenti}
Vengono di seguito riportati i test riguardanti le macrocomponenti del sistema \underline{br-jsys}. Questi test si occupano di testare il corretto funzionamento della componente, nonch\`e di mostrare il soddisfacimento di alcuni requisiti.

\subsection{Test sul validatore} 
% DA CONTOLLARE :
Con l'ausilio di test funzionali (\underline{blackbox}) si analizzer\`a il flusso di esecuzione del codice, inserendo dati in input (gli attributi di seguito elencati), capaci di provocare l'esito atteso.  Attraverso questa tecnica abbiamo valutato il corretto comportamento del programma attraverso situazioni limite. 
%PRIMA ERA SCRITTO COSI': La macrocomponente validatore verr\`a testata a scatola aperta (\underline{whitebox}), inserendo in input al validatore un oggetto \textit{\br\ } con gli attributi di seguito specificati e confrontando l'output del validatore con l'output atteso.\\
%MA QUESTA E' LA DEFINIZIONE DEI TEST FUNZIONALI (BLACK BOX)
\\
\begin{Large}\textbf{VT1a}\end{Large} \\
Si testa una \br\ semplice per verificare che siano accettati tutti i campi dati.\\
Requisiti soddisfatti dall'esecuzione: F1,F2,F10,F11,NU3
\begin{center}
\begin{tabular}{|p{5cm}|p{6cm}|} \hline
\textbf{Attributo \br} & \textbf{Valore} \\ \hline
Nome & BR\\ \hline
Commento & UnCommento\\ \hline
Business object associato & Articolo\\ \hline
Regola & entrate=0; AND NOT(scontato)=true;\\ \hline
\end{tabular} \\
\end{center}
\begin{center}
\begin{tabular}{|p{11cm}|} \hline
\textbf{Output atteso}\\ \hline
\textless BusinessRule associated=``Articolo'' comment=``Commento'' name=``BR'' rule=``entrate=0 AND NOT(scontato)=true''\textgreater \\
\textless AstRuleVersion value=``(AND (= entrate 0) (= (NOT scontato) true))''/\textgreater \\
 \textless OBool type=``AND''\textgreater \\
 \textless Conf type=``=''\textgreater \\
 \textless FIELD\textgreater entrate\textless /FIELD\textgreater \\
 \textless FLOAT\textgreater 0\textless /FLOAT\textgreater \\
 \textless /Conf\textgreater \\
 \textless Conf type=``=''\textgreater \\
\textless BoFun type=``NOT''\textgreater \\
 \textless FIELD\textgreater scontato\textless /FIELD\textgreater \\
\textless /BoFun\textgreater \\
 \textless BOOL\textgreater true\textless /BOOL\textgreater \\
\textless /Conf\textgreater \\
\textless /OBool\textgreater \\
\textless /BusinessRule\textgreater \\
 \hline
\end{tabular} \\
\end{center}

\begin{Large}\textbf{VT1b}\end{Large} \\
Si testa una \br\ semplice per verificare che il commento pu\`o non essere inserito.\\
Requisiti soddisfatti dall'esecuzione: F1,F2,F10
\begin{center}
\begin{tabular}{|p{5cm}|p{6cm}|} \hline
\textbf{Attributo \br} & \textbf{Valore} \\ \hline
Nome & BR\\ \hline
Commento & \\ \hline
Business object associato & Articolo\\ \hline
Regola & entrate=3; AND uscite=4;\\ \hline
\end{tabular} \\
\end{center}
\begin{center}
\begin{tabular}{|p{11cm}|} \hline
\textbf{Output atteso}\\ \hline
\textless BusinessRule associated=``Articolo'' comment=``'' name=``BR'' rule=``entrate=3 AND uscite=4''\textgreater \\
 \textless AstRuleVersion value=``(AND (= entrate 3) (= uscite 4))''/\textgreater \\
\textless OBool type=``AND''\textgreater \\
\textless Conf type=``=``\textgreater \\
\textless FIELD\textgreater entrate\textless /FIELD\textgreater \\
\textless FLOAT\textgreater 3\textless /FLOAT\textgreater \\
\textless /Conf\textgreater \\
\textless Conf type=``=``\textgreater \\
\textless FIELD\textgreater uscite\textless /FIELD\textgreater \\
\textless FLOAT\textgreater 4\textless /FLOAT\textgreater \\
\textless /Conf\textgreater \\
\textless /OBool\textgreater \\
\textless /BusinessRule\textgreater \\
 \hline
\end{tabular} \\
\end{center}

\begin{Large}\textbf{VT1c}\end{Large} \\
Si testa una \br\ semplice per verificare che il nome non \`e opzionale.\\
Requisiti soddisfatti dall'esecuzione: F1,F2,F10
\begin{center}
\begin{tabular}{|p{5cm}|p{6cm}|} \hline
\textbf{Attributo \br} & \textbf{Valore} \\ \hline
Nome & \\ \hline
Commento & Commento\\ \hline
Business object associato & Articolo\\ \hline
Regola & entrate=5; AND uscite=3;\\ \hline
\end{tabular} \\
\end{center}
\begin{center}
\begin{tabular}{|p{11cm}|} \hline
\textbf{Output atteso}\\ \hline
Errore sintattico\\
 \hline
\end{tabular} \\
\end{center}

\begin{Large}\textbf{VT1d}\end{Large} \\
Si testa una \br\ semplice per verificare che il \bo\ non \`e opzionale.\\
Requisiti soddisfatti dall'esecuzione: F1,F2,F10,F11
\begin{center}
\begin{tabular}{|p{5cm}|p{6cm}|} \hline
\textbf{Attributo \br} & \textbf{Valore} \\ \hline
Nome & BR \\ \hline
Commento & Commento\\ \hline
Business object associato & \\ \hline
Regola & entrate=3; OR uscite=5;\\ \hline
\end{tabular} \\
\end{center}
\begin{center}
\begin{tabular}{|p{11cm}|} \hline
\textbf{Output atteso}\\ \hline
Errore sintattico\\
 \hline
\end{tabular} \\
\end{center}

\begin{Large}\textbf{VT1e}\end{Large} \\
Si testa una \br\ semplice per verificare che la regola stessa non \`e opzionale.\\
Requisiti soddisfatti dall'esecuzione: F1,F2
\begin{center}
\begin{tabular}{|p{5cm}|p{6cm}|} \hline
\textbf{Attributo \br} & \textbf{Valore} \\ \hline
Nome & BR \\ \hline
Commento & Commento\\ \hline
Business object associato & Articolo \\ \hline
Regola & \\ \hline
\end{tabular} \\
\end{center}
\begin{center}
\begin{tabular}{|p{11cm}|} \hline
\textbf{Output atteso}\\ \hline
Errore sintattico\\
 \hline
\end{tabular} \\
\end{center}

\begin{Large}\textbf{VT1f}\end{Large} \\
Si testa una \br\ semplice per verificare che non vengano accettate \br\ vuote.\\
Requisiti soddisfatti dall'esecuzione: F1,F2
\begin{center}
\begin{tabular}{|p{5cm}|p{6cm}|} \hline
\textbf{Attributo \br} & \textbf{Valore} \\ \hline
Nome &  \\ \hline
Commento & \\ \hline
Business object associato &  \\ \hline
Regola & \\ \hline
\end{tabular} \\
\end{center}
\begin{center}
\begin{tabular}{|p{11cm}|} \hline
\textbf{Output atteso}\\ \hline
Errore sintattico\\
 \hline
\end{tabular} \\
\end{center}

\begin{Large}\textbf{VT2a}\end{Large} \\
Si testa una \br\ per verificare che vengano ben interpretate le parentesi tra 'AND' e 'OR'.\\
Requisiti soddisfatti dall'esecuzione: F1,F2,F10
\begin{center}
\begin{tabular}{|p{5cm}|p{6cm}|} \hline
\textbf{Attributo \br} & \textbf{Valore} \\ \hline
Nome & BR \\ \hline
Commento & Commento\\ \hline
Business object associato & Articolo \\ \hline
Regola & (uscite=5; OR uscite=5;)  AND ( uscite=5; OR uscite=5;)  \\ \hline
\end{tabular} \\
\end{center}
\begin{center}
\begin{tabular}{|p{11cm}|} \hline
\textbf{Output atteso}\\ \hline
\textless BusinessRule associated=``Articolo'' comment=``Commento'' name=``BR'' rule=``(uscite=5 OR uscite=5)  AND ( uscite=5 OR uscite=5)''\textgreater\\
 \textless AstRuleVersion value=``(AND (OR (= uscite 5) (= uscite 5)) (OR (= uscite 5) (= uscite 5)))''/\textgreater \\
\textless OBool type=``AND''\textgreater\\
\textless OBool type=``OR''\textgreater\\
\textless Conf type=``=``\textgreater \\
\textless FIELD\textgreater uscite\textless /FIELD\textgreater\\
 \textless FLOAT\textgreater 5\textless /FLOAT\textgreater\\
 \textless /Conf\textgreater\\
\textless Conf type=``=``\textgreater \\
\textless FIELD\textgreater uscite\textless /FIELD\textgreater\\
 \textless FLOAT\textgreater 5\textless /FLOAT\textgreater\\
 \textless /Conf\textgreater\\
\textless /OBool\textgreater \\
\textless OBool type=``OR''\textgreater\\
\textless Conf type=``=``\textgreater \\
\textless FIELD\textgreater uscite\textless /FIELD\textgreater\\
 \textless FLOAT\textgreater 5\textless /FLOAT\textgreater\\
 \textless /Conf\textgreater\\
\textless Conf type=``=``\textgreater \\
\textless FIELD\textgreater uscite\textless /FIELD\textgreater\\
 \textless FLOAT\textgreater 5\textless /FLOAT\textgreater\\
 \textless /Conf\textgreater\\
\textless /OBool\textgreater \\
\textless /OBool\textgreater \\
\textless /BusinessRule\textgreater \\
 \hline
\end{tabular} \\
\end{center}

\begin{Large}\textbf{VT2b}\end{Large} \\
Si testa una \br\ per verificare che vengano corretamente riportato il susseguirsi di 'AND' e 'OR'.\\
Requisiti soddisfatti dall'esecuzione: F1,F2,F10
\begin{center}
\begin{tabular}{|p{5cm}|p{6cm}|} \hline
\textbf{Attributo \br} & \textbf{Valore} \\ \hline
Nome & BR \\ \hline
Commento & Commento\\ \hline
Business object associato & Articolo \\ \hline
Regola & uscite=5; AND uscite=5; OR uscite=5; AND uscite=5; OR uscite=5; AND uscite=5; OR uscite=5; AND uscite=5; AND uscite=5; AND uscite=5; OR uscite=5; OR uscite=5; AND uscite=5; OR uscite=5; OR uscite=5; AND uscite=5; \\ \hline
\end{tabular} \\
\end{center}
\begin{center}
\begin{tabular}{|p{11cm}|} \hline
\textbf{Output atteso}\\ \hline
\textless BusinessRule associated=``Articolo'' comment=``Commento'' name=``BR'' rule=``uscite=5 OR uscite=5  AND  uscite=5 OR uscite=5''\textgreater\\
 \textless AstRuleVersion value=``(OR (AND (OR (= uscite 5) (= uscite 5)) (= uscite 5)) (= uscite 5))''/\textgreater \\
\textless OBool type=``OR''\textgreater\\
\textless OBool type=``AND''\textgreater\\
\textless OBool type=``OR''\textgreater\\
\textless Conf type=``=``\textgreater \\
\textless FIELD\textgreater uscite\textless /FIELD\textgreater\\
 \textless FLOAT\textgreater 5\textless /FLOAT\textgreater\\
 \textless /Conf\textgreater\\
\textless Conf type=``=``\textgreater \\
\textless FIELD\textgreater uscite\textless /FIELD\textgreater\\
 \textless FLOAT\textgreater 5\textless /FLOAT\textgreater\\
 \textless /Conf\textgreater\\
\textless /OBool\textgreater \\
\textless Conf type=``=``\textgreater \\
\textless FIELD\textgreater uscite\textless /FIELD\textgreater\\
 \textless FLOAT\textgreater 5\textless /FLOAT\textgreater\\
 \textless /Conf\textgreater\\
\textless /OBool\textgreater \\
\textless Conf type=``=``\textgreater \\
\textless FIELD\textgreater uscite\textless /FIELD\textgreater\\
 \textless FLOAT\textgreater 5\textless /FLOAT\textgreater\\
 \textless /Conf\textgreater\\
\textless /OBool\textgreater \\
\textless /BusinessRule\textgreater \\
 \hline
\end{tabular} \\
\end{center}

\begin{Large}\textbf{VT2c}\end{Large} \\
Si testa una \br\ per verificare che possano essere assenti 'AND' e 'OR'.\\
Requisiti soddisfatti dall'esecuzione: F1,F2,F10
\begin{center}
\begin{tabular}{|p{5cm}|p{6cm}|} \hline
\textbf{Attributo \br} & \textbf{Valore} \\ \hline
Nome & BR \\ \hline
Commento & Commento\\ \hline
Business object associato & Articolo \\ \hline
Regola & entrate=5+2;\\ \hline
\end{tabular} \\
\end{center}
\begin{center}
\begin{tabular}{|p{11cm}|} \hline
\textbf{Output atteso}\\ \hline
\textless BusinessRule associated=``Articolo'' comment=``Commento'' name=``BR'' rule=``entrate=5+2''\textgreater \\
\textless AstRuleVersion value=``(= entrate (+ 5 2))''/\textgreater \\
\textless Bconf type=``=``\textgreater\\
 \textless FIELD\textgreater entrate\textless /FIELD\textgreater \\
\textless OFloat type=``+''\textgreater \\
\textless FLOAT\textgreater 5\textless /FLOAT\textgreater \\
\textless FLOAT\textgreater 2\textless /FLOAT\textgreater\\
 \textless /OFloat\textgreater \\
\textless /Bconf\textgreater \\
\textless /BusinessRule\textgreater \\
 \hline
\end{tabular} \\
\end{center}

\begin{Large}\textbf{VT3a}\end{Large} \\
Si testa una \br\ per verificare che il 'message' sia opzionale.\\
Requisiti soddisfatti dall'esecuzione: F1,F2,F10
\begin{center}
\begin{tabular}{|p{5cm}|p{6cm}|} \hline
\textbf{Attributo \br} & \textbf{Valore} \\ \hline
Nome & BR \\ \hline
Commento & Commento\\ \hline
Business object associato & Articolo \\ \hline
Regola & entrate=SUM(prezzoBase);\\ \hline
\end{tabular} \\
\end{center}
\begin{center}
\begin{tabular}{|p{11cm}|} \hline
\textbf{Output atteso}\\ \hline
\textless BusinessRule associated=``Articolo'' comment=``Commento'' name=``BR'' rule=``entrate=SUM(prezzoBase)''\textgreater \\
\textless AstRuleVersion value=``(= entrate (SUM prezzoBase))''/\textgreater \\
\textless Bconf type=``=``\textgreater \\
\textless FIELD\textgreater entrate\textless /FIELD\textgreater \\
\textless FlFun type=``SUM''\textgreater \\
\textless FIELD\textgreater prezzoBase\textless /FIELD\textgreater \\
\textless /FlFun\textgreater \\
\textless /Bconf\textgreater \\
\textless /BusinessRule\textgreater \\
 \hline
\end{tabular} \\
\end{center}

\begin{Large}\textbf{VT3b}\end{Large} \\
Si testa una \br\ per verificare che sia possibile inserire un 'message'.\\
Requisiti soddisfatti dall'esecuzione: F1,F2,F10,NU3
\begin{center}
\begin{tabular}{|p{5cm}|p{6cm}|} \hline
\textbf{Attributo \br} & \textbf{Valore} \\ \hline
Nome & BR \\ \hline
Commento & Commento\\ \hline
Business object associato & Articolo \\ \hline
Regola & entrate=AVG(prezzoBase)+2 message(``errore nel confronto'');\\ \hline
\end{tabular} \\
\end{center}
\begin{center}
\begin{tabular}{|p{11cm}|} \hline
\textbf{Output atteso}\\ \hline
\textless BusinessRule associated=``Articolo'' comment=``Commento'' name=``BR'' rule=``entrate=AVG(prezzoBase)+2 message(\&quot;errore nel confronto\&quot;)''\textgreater \\
\textless AstRuleVersion value=``(= entrate (+ (AVG prezzoBase) 2) (message \&quot;errore nel confronto\&quot;))''/\textgreater\\
 \textless Bconf type=``=``\textgreater \\
\textless FIELD\textgreater entrate\textless /FIELD\textgreater \\
\textless OFloat type=``+''\textgreater \\
\textless FlFun type=``AVG''\textgreater \\
\textless FIELD\textgreater prezzoBase\textless /FIELD\textgreater \\
\textless /FlFun\textgreater \\
\textless FLOAT\textgreater 2\textless /FLOAT\textgreater \\
\textless /OFloat\textgreater \\
\textless Msg type=``message''\textgreater \\
\textless STRING\textgreater ``errore nel confronto''\textless /STRING\textgreater\\
 \textless /Msg\textgreater\\
 \textless /Bconf\textgreater \\
\textless /BusinessRule\textgreater \\
 \hline
\end{tabular} \\
\end{center}

\begin{Large}\textbf{VT4}\end{Large} \\
Si testa una \br\ per verificare che sia possibile fare confronti del tipo '\textless', '\textgreater', '\textless =', '\textgreater =', '=', '!='.\\
Requisiti soddisfatti dall'esecuzione: F1,F2,F10
\begin{center}
\begin{tabular}{|p{5cm}|p{6cm}|} \hline
\textbf{Attributo \br} & \textbf{Valore} \\ \hline
Nome & BR \\ \hline
Commento & Commento\\ \hline
Business object associato & Articolo \\ \hline
Regola & uscite\textless2; AND uscite\textgreater 2; AND uscite\textless=2; AND uscite\textgreater =2; AND uscite=2; AND uscite!=2;\\ \hline
\end{tabular} \\
\end{center}
\begin{center}
\begin{tabular}{|p{11cm}|} \hline
\textbf{Output atteso}\\ \hline
\textless BusinessRule associated=``Articolo'' comment=``Commento'' name=``BR'' rule=``uscite\&lt;2 AND uscite\&gt; 2 AND uscite\&lt;=2 AND uscite\&gt;=2 AND uscite=2 AND uscite!=2''\textgreater\\
 \textless AstRuleVersion value=``(AND (AND (AND (AND (AND (= uscite 2) (!= uscite 2)) (\&gt;= uscite 2)) (\&lt;= uscite 2)) (\&gt; uscite 2)) (\&lt; uscite 2))''/\textgreater \\
\textless OBool type=``AND''\textgreater \textless OBool type=``AND''\textgreater\\
\textless OBool type=``AND''\textgreater \textless OBool type=``AND''\textgreater\\
\textless OBool type=``AND''\textgreater\\
\textless Conf type=``=``\textgreater \\
\textless FIELD\textgreater uscite\textless /FIELD\textgreater\\
 \textless FLOAT\textgreater 5\textless /FLOAT\textgreater\\
 \textless /Conf\textgreater\\
\textless Conf type=``!=``\textgreater \\
\textless FIELD\textgreater uscite\textless /FIELD\textgreater\\
 \textless FLOAT\textgreater 5\textless /FLOAT\textgreater\\
 \textless /Conf\textgreater\\
\textless /OBool\textgreater \\
\textless Conf type=``\&gt;=``\textgreater \\
\textless FIELD\textgreater uscite\textless /FIELD\textgreater\\
 \textless FLOAT\textgreater 5\textless /FLOAT\textgreater\\
 \textless /Conf\textgreater\\
\textless /OBool\textgreater \\
\textless Conf type=``\&lt;=``\textgreater \\
\textless FIELD\textgreater uscite\textless /FIELD\textgreater\\
 \textless FLOAT\textgreater 5\textless /FLOAT\textgreater\\
 \textless /Conf\textgreater\\
\textless /OBool\textgreater \\
\textless Conf type=``\&gt;``\textgreater \\
\textless FIELD\textgreater uscite\textless /FIELD\textgreater\\
 \textless FLOAT\textgreater 5\textless /FLOAT\textgreater\\
 \textless /Conf\textgreater\\
\textless /OBool\textgreater \\
\textless Conf type=``\&lt;``\textgreater \\
\textless FIELD\textgreater uscite\textless /FIELD\textgreater\\
 \textless FLOAT\textgreater 5\textless /FLOAT\textgreater\\
 \textless /Conf\textgreater\\
\textless /OBool\textgreater \\
\textless /BusinessRule\textgreater \\
 \hline
\end{tabular} \\
\end{center}


\begin{Large}\textbf{VT5a}\end{Large} \\
Si testa una \br\ per verificare che sia possibile fare operazioni aritmetiche semplici tra scalari.\\
Requisiti soddisfatti dall'esecuzione: F1,F2,F10
\begin{center}
\begin{tabular}{|p{5cm}|p{6cm}|} \hline
\textbf{Attributo \br} & \textbf{Valore} \\ \hline
Nome & BR \\ \hline
Commento & Commento\\ \hline
Business object associato & Articolo \\ \hline
Regola & uscite=(2+2-(3*4))/5; \\ \hline
\end{tabular} \\
\end{center}
\begin{center}
\begin{tabular}{|p{11cm}|} \hline
\textbf{Output atteso}\\ \hline
\textless BusinessRule associated=``Articolo'' comment=``Commento'' name=``BR'' rule=``uscite=(2+2-(3*4))/5''\textgreater \\
\textless AstRuleVersion value=``(= uscite (/ (- (+ 2 2) (* 3 4)) 5))''/\textgreater \\
 \textless Bconf type=``=``\textgreater \\
\textless FIELD\textgreater uscite\textless /FIELD\textgreater \\
\textless OFloat type=``/''\textgreater \\
\textless OFloat type=``-''\textgreater \\
\textless OFloat type=``+''\textgreater \\
\textless FLOAT\textgreater 2\textless /FLOAT\textgreater \\
\textless FLOAT\textgreater 2\textless /FLOAT\textgreater \\
\textless /OFloat\textgreater \\
\textless OFloat type=``*''\textgreater \\
\textless FLOAT\textgreater 3\textless /FLOAT\textgreater\\
\textless FLOAT\textgreater 4\textless /FLOAT\textgreater \\
\textless /OFloat\textgreater \\
\textless /OFloat\textgreater \\
\textless FLOAT\textgreater 5\textless /FLOAT\textgreater \\
\textless /OFloat\textgreater \\
\textless /Bconf\textgreater \\
\textless /BusinessRule\textgreater \\
 \hline
\end{tabular} \\
\end{center}

\begin{Large}\textbf{VT5b}\end{Large} \\
Si testa una \br\ per verificare che sia possibile fare operazioni aritmetiche semplici tra scalari e campi dati scalari.\\
Requisiti soddisfatti dall'esecuzione: F1,F2,F10
\begin{center}
\begin{tabular}{|p{5cm}|p{6cm}|} \hline
\textbf{Attributo \br} & \textbf{Valore} \\ \hline
Nome & BR \\ \hline
Commento & Commento\\ \hline
Business object associato & Articolo \\ \hline
Regola & uscite=(2+entrate-(uscite*4))/entrate; \\ \hline
\end{tabular} \\
\end{center}
\begin{center}
\begin{tabular}{|p{11cm}|} \hline
\textbf{Output atteso}\\ \hline
\textless BusinessRule associated=``Articolo'' comment=``Commento'' name=``BR'' rule=``uscite=(2+entrate-(uscite*4))/entrate''\textgreater\\
 \textless AstRuleVersion value=``(= uscite (/ (- (+ 2 entrate) (* uscite 4)) entrate))''/\textgreater\\
 \textless Bconf type=``=``\textgreater \\
\textless FIELD\textgreater uscite\textless /FIELD\textgreater \\
\textless OFloat type=``/''\textgreater \\
\textless OFloat type=``-''\textgreater \\
\textless OFloat type=``+''\textgreater \\
\textless FLOAT\textgreater 2\textless /FLOAT\textgreater \\
\textless FIELD\textgreater entrate\textless /FIELD\textgreater \\
\textless /OFloat\textgreater \\
\textless OFloat type=``*''\textgreater \\
\textless FIELD\textgreater uscite\textless /FIELD\textgreater \\
\textless FLOAT\textgreater 4\textless /FLOAT\textgreater \\
\textless /OFloat\textgreater\\
 \textless /OFloat\textgreater \\
\textless FIELD\textgreater entrate\textless /FIELD\textgreater \\
\textless /OFloat\textgreater\\
 \textless /Bconf\textgreater \\
\textless /BusinessRule\textgreater \\
 \hline
\end{tabular} \\
\end{center}

\begin{Large}\textbf{VT5c}\end{Large} \\
Si testa una \br\ per verificare che sia possibile fare operazioni aritmetiche semplici tra campi dati scalari.\\
Requisiti soddisfatti dall'esecuzione: F1,F2,F10
\begin{center}
\begin{tabular}{|p{5cm}|p{6cm}|} \hline
\textbf{Attributo \br} & \textbf{Valore} \\ \hline
Nome & BR \\ \hline
Commento & Commento\\ \hline
Business object associato & Articolo \\ \hline
Regola & uscite=(uscite+entrate-(uscite*entrate))/entrate; \\ \hline
\end{tabular} \\
\end{center}
\begin{center}
\begin{tabular}{|p{11cm}|} \hline
\textbf{Output atteso}\\ \hline
\textless BusinessRule associated=``Articolo'' comment=``Commento'' name=``BR'' rule=``uscite=(uscite+entrate-(uscite*entrate))/entrate''\textgreater\\
 \textless AstRuleVersion value=``(= uscite (/ (- (+ uscite entrate) (* uscite entrate)) entrate))''/\textgreater \\
\textless Bconf type=``=``\textgreater \\
\textless FIELD\textgreater uscite\textless /FIELD\textgreater\\
 \textless OFloat type=``/''\textgreater \\
\textless OFloat type=``-''\textgreater \\
\textless OFloat type=``+''\textgreater \\
\textless FIELD\textgreater uscite\textless /FIELD\textgreater\\
 \textless FIELD\textgreater entrate\textless /FIELD\textgreater \\
\textless /OFloat\textgreater \\
\textless OFloat type=``*''\textgreater \\
\textless FIELD\textgreater uscite\textless /FIELD\textgreater \\
\textless FIELD\textgreater entrate\textless /FIELD\textgreater \\
\textless /OFloat\textgreater \\
\textless /OFloat\textgreater \\
\textless FIELD\textgreater entrate\textless /FIELD\textgreater \\
\textless /OFloat\textgreater \\
\textless /Bconf\textgreater \\
\textless /BusinessRule\textgreater \\
 \hline
\end{tabular} \\
\end{center}

\begin{Large}\textbf{VT6a}\end{Large} \\
Si testa una \br\ per verificare che sia possibile fare le operazioni aritmetiche semplici tra scalare e vettore.\\
Requisiti soddisfatti dall'esecuzione: F1,F2,F10,NU1
\begin{center}
\begin{tabular}{|p{5cm}|p{6cm}|} \hline
\textbf{Attributo \br} & \textbf{Valore} \\ \hline
Nome & BR \\ \hline
Commento & Commento\\ \hline
Business object associato & Articolo \\ \hline
Regola & uscite=(2+prezzoBase) +(prezzoBase*4) +(prezzoBase/4) +(prezzoBase-3); \\ \hline
\end{tabular} \\
\end{center}
\begin{center}
\begin{tabular}{|p{11cm}|} \hline
\textbf{Output atteso}\\ \hline
\textless BusinessRule associated=``Articolo'' comment=``Commento'' name=``BR'' rule=``uscite=(2+prezzoBase)+(prezzoBase*4) +(prezzoBase/4)+(prezzoBase-3)''\textgreater\\
 \textless AstRuleVersion value=``(= uscite (+ (+ (+ (+ 2 prezzoBase) (* prezzoBase 4)) (/ prezzoBase 4)) (- prezzoBase 3)))''/\textgreater \\
\textless Bconf type=``=``\textgreater \\
\textless FIELD\textgreater uscite\textless /FIELD\textgreater \\
\textless OFloat type=``+''\textgreater \\
\textless OFloat type=``+''\textgreater \\
\textless OFloat type=``+''\textgreater \\
\textless OFloat type=``+''\textgreater \\
\textless FLOAT\textgreater 2\textless /FLOAT\textgreater \\
\textless FIELD\textgreater prezzoBase\textless /FIELD\textgreater \\
\textless /OFloat\textgreater \\
\textless OFloat type=``*''\textgreater \\
\textless FIELD\textgreater prezzoBase\textless /FIELD\textgreater \\
\textless FLOAT\textgreater 4\textless /FLOAT\textgreater \\
\textless /OFloat\textgreater \\
\textless /OFloat\textgreater \\
\textless OFloat type=``/''\textgreater\\
 \textless FIELD\textgreater prezzoBase\textless /FIELD\textgreater \\
\textless FLOAT\textgreater 4\textless /FLOAT\textgreater \\
\textless /OFloat\textgreater \\
\textless /OFloat\textgreater \\
\textless OFloat type=``-''\textgreater \\
\textless FIELD\textgreater prezzoBase\textless /FIELD\textgreater \\
\textless FLOAT\textgreater 3\textless /FLOAT\textgreater \\
\textless /OFloat\textgreater\\
 \textless /OFloat\textgreater\\
 \textless /Bconf\textgreater \\
\textless /BusinessRule\textgreater \\
 \hline
\end{tabular} \\
\end{center}

\begin{Large}\textbf{VT6b}\end{Large} \\
Si testa una \br\ per verificare che sia possibile fare le operazioni aritmetiche semplici tra campi dati scalari e vettori.\\
Requisiti soddisfatti dall'esecuzione: F1,F2,F10,NU1
\begin{center}
\begin{tabular}{|p{5cm}|p{6cm}|} \hline
\textbf{Attributo \br} & \textbf{Valore} \\ \hline
Nome & BR \\ \hline
Commento & Commento\\ \hline
Business object associato & Articolo \\ \hline
Regola & (entrate+prezzoBase) +(prezzoBase*uscite) +(prezzoBase/entrate) +(prezzoBase-uscite) = 0;\\ \hline
\end{tabular} \\
\end{center}
\begin{center}
\begin{tabular}{|p{11cm}|} \hline
\textbf{Output atteso}\\ \hline
\textless BusinessRule associated``Articolo'' comment``Comment'' name``BR'' rule``(entrate+prezzoBase)+ (prezzoBase*uscite)+ (prezzoBase/entrate)+ (prezzoBase-uscite)=0''\textgreater\\
\textless AstRuleVersion value``(= (+ (+ (+ (+ entrate prezzoBase) (* prezzoBase prezzoBase)) (/ prezzoBase entrate)) (- prezzoBase prezzoBase)) 0)''/\textgreater\\
\textless Bconf type````\textgreater\\
\textless OFloat type``+''\textgreater\\
\textless OFloat type``+''\textgreater\\
\textless OFloat type``+''\textgreater\\
\textless OFloat type``+''\textgreater\\
\textless FIELD\textgreater entrate\textless /FIELD\textgreater\\
\textless FIELD\textgreater prezzoBase\textless /FIELD\textgreater\\
\textless /OFloat\textgreater\\
\textless OFloat type``*''\textgreater\\
\textless FIELD\textgreater prezzoBase\textless /FIELD\textgreater\\
\textless FIELD\textgreater uscite\textless /FIELD\textgreater\\
\textless /OFloat\textgreater\\
\textless /OFloat\textgreater\\
\textless OFloat type``/''\textgreater\\
\textless FIELD\textgreater prezzoBase\textless /FIELD\textgreater\\
\textless FIELD\textgreater entrate\textless /FIELD\textgreater\\
\textless /OFloat\textgreater\\
\textless /OFloat\textgreater\\
\textless OFloat type``-''\textgreater\\
\textless FIELD\textgreater prezzoBase\textless /FIELD\textgreater\\
\textless FIELD\textgreater uscite\textless /FIELD\textgreater\\
\textless /OFloat\textgreater\\
\textless /OFloat\textgreater\\
\textless FLOAT\textgreater0\textless /FLOAT\textgreater\\
\textless /Bconf\textgreater\\
\textless /BusinessRule\textgreater\\
 \hline
\end{tabular} \\
\end{center}

\begin{Large}\textbf{VT7}\end{Large} \\
Si testa una \br\ per verificare che sia possibile fare le operazioni aritmetiche semplici tra vettori omogenei.\\
Requisiti soddisfatti dall'esecuzione: F1,F2,F10,NU1
\begin{center}
\begin{tabular}{|p{5cm}|p{6cm}|} \hline
\textbf{Attributo \br} & \textbf{Valore} \\ \hline
Nome & BR \\ \hline
Commento & Commento\\ \hline
Business object associato & Articolo \\ \hline
Regola & (prezzoBase+prezzoBase) +(prezzoBase*prezzoBase) +(prezzoBase/prezzoBase) +(prezzoBase-prezzoBase) =0; \\ \hline
\end{tabular} \\
\end{center}
\begin{center}
\begin{tabular}{|p{11cm}|} \hline
\textbf{Output atteso}\\ \hline
\textless BusinessRule associated``Articolo'' comment``Comment'' name``BR'' rule``(prezzoBase+prezzoBase)+ (prezzoBase*prezzoBase)+ (prezzoBase/prezzoBase)+ (prezzoBase-prezzoBase)=0''\textgreater\\
\textless AstRuleVersion value``(= (+ (+ (+ (+ prezzoBase prezzoBase) (* prezzoBase prezzoBase)) (/ prezzoBase prezzoBase)) (- prezzoBase prezzoBase)) 0)''/\textgreater\\
\textless Bconf type````\textgreater\\
\textless OFloat type``+''\textgreater\\
\textless OFloat type``+''\textgreater\\
\textless OFloat type``+''\textgreater\\
\textless OFloat type``+''\textgreater\\
\textless FIELD\textgreater prezzoBase\textless /FIELD\textgreater\\
\textless FIELD\textgreater prezzoBase\textless /FIELD\textgreater\\
\textless /OFloat\textgreater\\
\textless OFloat type``*''\textgreater\\
\textless FIELD\textgreater prezzoBase\textless /FIELD\textgreater\\
\textless FIELD\textgreater prezzoBase\textless /FIELD\textgreater\\
\textless /OFloat\textgreater\\
\textless /OFloat\textgreater\\
\textless OFloat type``/''\textgreater\\
\textless FIELD\textgreater prezzoBase\textless /FIELD\textgreater\\
\textless FIELD\textgreater prezzoBase\textless /FIELD\textgreater\\
\textless /OFloat\textgreater\\
\textless /OFloat\textgreater\\
\textless OFloat type``-''\textgreater\\
\textless FIELD\textgreater prezzoBase\textless /FIELD\textgreater\\
\textless FIELD\textgreater prezzoBase\textless /FIELD\textgreater\\
\textless /OFloat\textgreater\\
\textless /OFloat\textgreater\\
\textless FLOAT\textgreater0\textless /FLOAT\textgreater\\
\textless /Bconf\textgreater\\
\textless /BusinessRule\textgreater\\
 \hline
\end{tabular} \\
\end{center}

\begin{Large}\textbf{VT8a}\end{Large} \\
Si testa una \br\ per verificare che sia possibile fare le operazioni aritmetiche semplici tra matrici omogenee.\\
Requisiti soddisfatti dall'esecuzione: F1,F2,F10,NU1
\begin{center}
\begin{tabular}{|p{5cm}|p{6cm}|} \hline
\textbf{Attributo \br} & \textbf{Valore} \\ \hline
Nome & BR \\ \hline
Commento & Commento\\ \hline
Business object associato & Articolo \\ \hline
Regola & (matricePrezzi+matricePrezzi) +(matricePrezzi*matricePrezzi) +(matricePrezzi/matricePrezzi) +(matricePrezzi-matricePrezzi) =0; \\ \hline
\end{tabular} \\
\end{center}
\begin{center}
\begin{tabular}{|p{11cm}|} \hline
\textbf{Output atteso}\\ \hline
\textless BusinessRule associated``Articolo'' comment``Comment'' name``BR'' rule``(matricePrezzi+matricePrezzi)+ (matricePrezzi*matricePrezzi)+ (matricePrezzi/matricePrezzi)+ (matricePrezzi-matricePrezzi)=0''\textgreater\\
\textless AstRuleVersion value``(= (+ (+ (+ (+ matricePrezzi matricePrezzi) (* matricePrezzi matricePrezzi)) (/ matricePrezzi matricePrezzi)) (- matricePrezzi matricePrezzi)) 0)''/\textgreater\\
\textless Bconf type````\textgreater\\
\textless OFloat type``+''\textgreater\\
\textless OFloat type``+''\textgreater\\
\textless OFloat type``+''\textgreater\\
\textless OFloat type``+''\textgreater\\
\textless FIELD\textgreater matricePrezzi\textless /FIELD\textgreater\\
\textless FIELD\textgreater matricePrezzi\textless /FIELD\textgreater\\
\textless /OFloat\textgreater\\
\textless OFloat type``*''\textgreater\\
\textless FIELD\textgreater matricePrezzi\textless /FIELD\textgreater\\
\textless FIELD\textgreater matricePrezzi\textless /FIELD\textgreater\\
\textless /OFloat\textgreater\\
\textless /OFloat\textgreater\\
\textless OFloat type``/''\textgreater\\
\textless FIELD\textgreater matricePrezzi\textless /FIELD\textgreater\\
\textless FIELD\textgreater matricePrezzi\textless /FIELD\textgreater\\
\textless /OFloat\textgreater\\
\textless /OFloat\textgreater\\
\textless OFloat type``-''\textgreater\\
\textless FIELD\textgreater matricePrezzi\textless /FIELD\textgreater\\
\textless FIELD\textgreater matricePrezzi\textless /FIELD\textgreater\\
\textless /OFloat\textgreater\\
\textless /OFloat\textgreater\\
\textless FLOAT\textgreater0\textless /FLOAT\textgreater\\
\textless /Bconf\textgreater\\
\textless /BusinessRule\textgreater\\
 \hline
\end{tabular} \\
\end{center}

\begin{Large}\textbf{VT8b}\end{Large} \\
Si testa una \br\ per verificare che sia possibile fare le operazioni aritmetiche semplici tra matrici e scalari.\\
Requisiti soddisfatti dall'esecuzione: F1,F2,F10,NPr2,NQ1
\begin{center}
\begin{tabular}{|p{5cm}|p{6cm}|} \hline
\textbf{Attributo \br} & \textbf{Valore} \\ \hline
Nome & BR \\ \hline
Commento & Commento\\ \hline
Business object associato & Articolo \\ \hline
Regola & (matricePrezzi+2) +(matricePrezzi*2) +(matricePrezzi/2) +(matricePrezzi-2) =0; \\ \hline
\end{tabular} \\
\end{center}
\begin{center}
\begin{tabular}{|p{11cm}|} \hline
\textbf{Output atteso}\\ \hline
\textless BusinessRule associated``Articolo'' comment``Comment'' name``BR'' rule``(matricePrezzi+2)+ (matricePrezzi*2)+ (matricePrezzi/2)+ (matricePrezzi-2)=0''\textgreater\\
\textless AstRuleVersion value``(= (+ (+ (+ (+ matricePrezzi 2) (* matricePrezzi 2)) (/ matricePrezzi 2)) (- matricePrezzi 2)) 0)''/\textgreater\\
\textless Bconf type````\textgreater\\
\textless OFloat type``+''\textgreater\\
\textless OFloat type``+''\textgreater\\
\textless OFloat type``+''\textgreater\\
\textless OFloat type``+''\textgreater\\
\textless FIELD\textgreater matricePrezzi\textless /FIELD\textgreater\\
\textless FLOAT\textgreater 2\textless /FLOAT\textgreater\\
\textless /OFloat\textgreater\\
\textless OFloat type``*''\textgreater\\
\textless FIELD\textgreater matricePrezzi\textless /FIELD\textgreater\\
\textless FLOAT\textgreater 2\textless /FLOAT\textgreater\\
\textless /OFloat\textgreater\\
\textless /OFloat\textgreater\\
\textless OFloat type``/''\textgreater\\
\textless FIELD\textgreater matricePrezzi\textless /FIELD\textgreater\\
\textless FLOAT\textgreater 2\textless /FLOAT\textgreater\\
\textless /OFloat\textgreater\\
\textless /OFloat\textgreater\\
\textless OFloat type``-''\textgreater\\
\textless FIELD\textgreater matricePrezzi\textless /FIELD\textgreater\\
\textless FLOAT\textgreater 2\textless /FLOAT\textgreater\\
\textless /OFloat\textgreater\\
\textless /OFloat\textgreater\\
\textless FLOAT\textgreater0\textless /FLOAT\textgreater\\
\textless /Bconf\textgreater\\
\textless /BusinessRule\textgreater\\
 \hline
\end{tabular} \\
\end{center}

\begin{Large}\textbf{VT8c}\end{Large} \\
Si testa una \br\ per verificare che sia possibile fare le operazioni aritmetiche semplici tra matrici e campi dati scalari.\\
Requisiti soddisfatti dall'esecuzione: F1,F2,F10,NU1,NPr2,NQ1
\begin{center}
\begin{tabular}{|p{5cm}|p{6cm}|} \hline
\textbf{Attributo \br} & \textbf{Valore} \\ \hline
Nome & BR \\ \hline
Commento & Commento\\ \hline
Business object associato & Articolo \\ \hline
Regola & (matricePrezzi+entrate) +(matricePrezzi*entrate) +(matricePrezzi/entrate) +(matricePrezzi-entrate)=0; \\ \hline
\end{tabular} \\
\end{center}
\begin{center}
\begin{tabular}{|p{11cm}|} \hline
\textbf{Output atteso}\\ \hline
\textless BusinessRule associated``Articolo'' comment``Comment'' name``BR'' rule``(matricePrezzi+entrate)+ (matricePrezzi*entrate)+ (matricePrezzi/entrate)+ (matricePrezzi-entrate)=0''\textgreater\\
\textless AstRuleVersion value``(= (+ (+ (+ (+ matricePrezzi entrate) (* matricePrezzi entrate)) (/ matricePrezzi entrate)) (- matricePrezzi entrate)) 0)''/\textgreater\\
\textless Bconf type````\textgreater\\
\textless OFloat type``+''\textgreater\\
\textless OFloat type``+''\textgreater\\
\textless OFloat type``+''\textgreater\\
\textless OFloat type``+''\textgreater\\
\textless FIELD\textgreater matricePrezzi\textless /FIELD\textgreater\\
\textless FIELD\textgreater entrate\textless /FIELD\textgreater\\
\textless /OFloat\textgreater\\
\textless OFloat type``*''\textgreater\\
\textless FIELD\textgreater matricePrezzi\textless /FIELD\textgreater\\
\textless FIELD\textgreater entrate\textless /FIELD\textgreater\\
\textless /OFloat\textgreater\\
\textless /OFloat\textgreater\\
\textless OFloat type``/''\textgreater\\
\textless FIELD\textgreater matricePrezzi\textless /FIELD\textgreater\\
\textless FIELD\textgreater entrate\textless /FIELD\textgreater\\
\textless /OFloat\textgreater\\
\textless /OFloat\textgreater\\
\textless OFloat type``-''\textgreater\\
\textless FIELD\textgreater matricePrezzi\textless /FIELD\textgreater\\
\textless FIELD\textgreater entrate\textless /FIELD\textgreater\\
\textless /OFloat\textgreater\\
\textless /OFloat\textgreater\\
\textless FLOAT\textgreater0\textless /FLOAT\textgreater\\
\textless /Bconf\textgreater\\
\textless /BusinessRule\textgreater\\
 \hline
\end{tabular} \\
\end{center}

\begin{Large}\textbf{VT9a}\end{Large} \\
Si testa una \br\ per verificare che sia impossibile fare le operazioni aritmetiche semplici tra matrici e vettori.\\
Requisiti soddisfatti dall'esecuzione: F1,F2,F10,NU1
\begin{center}
\begin{tabular}{|p{5cm}|p{6cm}|} \hline
\textbf{Attributo \br} & \textbf{Valore} \\ \hline
Nome & BR \\ \hline
Commento & Commento\\ \hline
Business object associato & Articolo \\ \hline
Regola & (matricePrezzi+prezzoBase) +(matricePrezzi*prezzoBase) +(matricePrezzi/prezzoBase) +(matricePrezzi-prezzoBase); \\ \hline
\end{tabular} \\
\end{center}
\begin{center}
\begin{tabular}{|p{11cm}|} \hline
\textbf{Output atteso}\\ \hline
Operazione non consentita\\
 \hline
\end{tabular} \\
\end{center}

\begin{Large}\textbf{VT9b}\end{Large} \\
Si testa una \br\ per verificare che sia impossibile fare le operazioni aritmetiche semplici tra matrici non omogenee.\\
Requisiti soddisfatti dall'esecuzione:F1,F2,F10
\begin{center}
\begin{tabular}{|p{5cm}|p{6cm}|} \hline
\textbf{Attributo \br} & \textbf{Valore} \\ \hline
Nome & BR \\ \hline
Commento & Commento\\ \hline
Business object associato & Articolo \\ \hline
Regola & (matricePrezzi+grafico) +(matricePrezzi*grafico) +(matricePrezzi/grafico) +(matricePrezzi-grafico); \\ \hline
\end{tabular} \\
\end{center}
\begin{center}
\begin{tabular}{|p{11cm}|} \hline
\textbf{Output atteso}\\ \hline
Operazione non consentita\\
 \hline
\end{tabular} \\
\end{center}

\begin{Large}\textbf{VT9c}\end{Large} \\
Si testa una \br\ per verificare che sia impossibile fare le operazioni aritmetiche semplici tra matrici, vettori e scalari di tipo diverso.\\
Requisiti soddisfatti dall'esecuzione: F1,F2,F10
\begin{center}
\begin{tabular}{|p{5cm}|p{6cm}|} \hline
\textbf{Attributo \br} & \textbf{Valore} \\ \hline
Nome & BR \\ \hline
Commento & Commento\\ \hline
Business object associato & Articolo \\ \hline
Regola & matricePrezzi=``ArticoloUno''; AND prezziScontati=3 OR uscite=true; \\ \hline
\end{tabular} \\
\end{center}
\begin{center}
\begin{tabular}{|p{11cm}|} \hline
\textbf{Output atteso}\\ \hline
Operazione non consentita\\
 \hline
\end{tabular} \\
\end{center}

\begin{Large}\textbf{VT10}\end{Large} \\
Si testa una \br\ per verificare che sia ben intepretata la precedenza tra gli operatori.\\
Requisiti soddisfatti dall'esecuzione: F1,F2,F10
\begin{center}
\begin{tabular}{|p{5cm}|p{6cm}|} \hline
\textbf{Attributo \br} & \textbf{Valore} \\ \hline
Nome & BR \\ \hline
Commento & Commento\\ \hline
Business object associato & Articolo \\ \hline
Regola & uscite=2*3*4+4-3*5-5+3/2*3; \\ \hline
\end{tabular} \\
\end{center}
\begin{center}
\begin{tabular}{|p{11cm}|} \hline
\textbf{Output atteso}\\ \hline
\textless BusinessRule associated=``Articolo'' comment=``Commento'' name=``BR'' rule=``uscite=2*3*4+4-3*5-5+3/2*3''\textgreater\\
 \textless AstRuleVersion value=``(= uscite (+ (- (- (+ (* (* 2 3) 4) 4) (* 3 5)) 5) (* (/ 3 2) 3)))''/\textgreater\\
 \textless Bconf type=``=``\textgreater\\
 \textless FIELD\textgreater uscite\textless /FIELD\textgreater\\
 \textless OFloat type=``+''\textgreater\\
 \textless OFloat type=``-''\textgreater\\
 \textless OFloat type=``-''\textgreater \\
\textless OFloat type=``+''\textgreater \\
\textless OFloat type=``*''\textgreater\\
 \textless OFloat type=``*''\textgreater\\
 \textless FLOAT\textgreater 2\textless /FLOAT\textgreater\\
 \textless FLOAT\textgreater 3\textless /FLOAT\textgreater\\
 \textless /OFloat\textgreater\\
 \textless FLOAT\textgreater 4\textless /FLOAT\textgreater\\
 \textless /OFloat\textgreater\\
 \textless FLOAT\textgreater 4\textless /FLOAT\textgreater\\
 \textless /OFloat\textgreater\\
 \textless OFloat type=``*''\textgreater\\
 \textless FLOAT\textgreater 3\textless /FLOAT\textgreater\\
 \textless FLOAT\textgreater 5\textless /FLOAT\textgreater\\
 \textless /OFloat\textgreater\\
 \textless /OFloat\textgreater\\
 \textless FLOAT\textgreater 5\textless /FLOAT\textgreater\\
 \textless /OFloat\textgreater\\
 \textless OFloat type=``*''\textgreater\\
 \textless OFloat type=``/''\textgreater\\
 \textless FLOAT\textgreater 3\textless /FLOAT\textgreater\\
 \textless FLOAT\textgreater 2\textless /FLOAT\textgreater\\
 \textless /OFloat\textgreater\\
 \textless FLOAT\textgreater 3\textless /FLOAT\textgreater\\
 \textless /OFloat\textgreater\\
 \textless /OFloat\textgreater\\
 \textless /Bconf\textgreater\\
 \textless /BusinessRule\textgreater \\
 \hline
\end{tabular} \\
\end{center}

\begin{Large}\textbf{VT11}\end{Large} \\
Si testa una \br\ per verificare che siano ben intepretate le parentesi.\\
Requisiti soddisfatti dall'esecuzione: F1,F2,F10
\begin{center}
\begin{tabular}{|p{5cm}|p{6cm}|} \hline
\textbf{Attributo \br} & \textbf{Valore} \\ \hline
Nome & BR \\ \hline
Commento & Commento\\ \hline
Business object associato & Articolo \\ \hline
Regola & uscite=((((2*((3*4)+(4)-(3*5)-5)+3)/2)*3)+7); \\ \hline
\end{tabular} \\
\end{center}
\begin{center}
\begin{tabular}{|p{11cm}|} \hline
\textbf{Output atteso}\\ \hline
\textless BusinessRule associated=``Articolo'' comment=``Commento'' name=``BR'' rule=``uscite=((((2*((3*4)+(4)-(3*5)-5)+3)/2)*3)+7)''\textgreater\\
 \textless AstRuleVersion value=``(= uscite (+ (* (/ (+ (* 2 (- (- (+ (* 3 4) 4) (* 3 5)) 5)) 3) 2) 3) 7))''/\textgreater\\
 \textless Bconf type=``=``\textgreater \\
\textless FIELD\textgreater uscite\textless /FIELD\textgreater\\
 \textless OFloat type=``+''\textgreater\\
 \textless OFloat type=``*''\textgreater \\
\textless OFloat type=``/''\textgreater \\
\textless OFloat type=``+''\textgreater \\
\textless OFloat type=``*''\textgreater \\
\textless FLOAT\textgreater 2\textless /FLOAT\textgreater \\
\textless OFloat type=``-''\textgreater \\
\textless OFloat type=``-''\textgreater \\
\textless OFloat type=``+''\textgreater\\
 \textless OFloat type=``*''\textgreater \\
\textless FLOAT\textgreater 3\textless /FLOAT\textgreater \\
\textless FLOAT\textgreater 4\textless /FLOAT\textgreater \\
\textless /OFloat\textgreater \\
\textless FLOAT\textgreater 4\textless /FLOAT\textgreater \\
\textless /OFloat\textgreater \\
\textless OFloat type=``*''\textgreater \\
\textless FLOAT\textgreater 3\textless /FLOAT\textgreater \\
\textless FLOAT\textgreater 5\textless /FLOAT\textgreater \\
\textless /OFloat\textgreater \textless /OFloat\textgreater \\
\textless FLOAT\textgreater 5\textless /FLOAT\textgreater \\
\textless /OFloat\textgreater \textless /OFloat\textgreater \\
\textless FLOAT\textgreater 3\textless /FLOAT\textgreater \\
\textless /OFloat\textgreater \\
\textless FLOAT\textgreater 2\textless /FLOAT\textgreater \\
\textless /OFloat\textgreater \\
\textless FLOAT\textgreater 3\textless /FLOAT\textgreater\\
 \textless /OFloat\textgreater \\
\textless FLOAT\textgreater 7\textless /FLOAT\textgreater \\
\textless /OFloat\textgreater \\
\textless /Bconf\textgreater\\
 \textless /BusinessRule\textgreater \\
 \hline
\end{tabular} \\
\end{center}

\begin{Large}\textbf{VT12a}\end{Large} \\
Si testa una \br\ per verificare che sia impossibile effettuare operazioni tra le stringhe.\\
Requisiti soddisfatti dall'esecuzione: F1,F2,F10
\begin{center}
\begin{tabular}{|p{5cm}|p{6cm}|} \hline
\textbf{Attributo \br} & \textbf{Valore} \\ \hline
Nome & BR \\ \hline
Commento & Commento\\ \hline
Business object associato & Articolo \\ \hline
Regola & ``ArticoloUno''\textgreater``ArticoloUno''; AND ``ArticoloUno''\textgreater=``ArticoloUno''; AND ``ArticoloUno''\textless``ArticoloUno''; AND ``ArticoloUno''\textless=``ArticoloUno''; AND ``ArticoloUno''*nome;  AND ``ArticoloUno''-nome;  AND ``ArticoloUno''/nome;  AND ``ArticoloUno''+nome;\\ \hline
\end{tabular} \\
\end{center}
\begin{center}
\begin{tabular}{|p{11cm}|} \hline
\textbf{Output atteso}\\ \hline
Operazione non consentita\\
 \hline
\end{tabular} \\
\end{center}

\begin{Large}\textbf{VT12b}\end{Large} \\
Si testa una \br\ per verificare che sia possibile effettuare confronti semplici ('=', '!=') tra le stringhe.\\
Requisiti soddisfatti dall'esecuzione: F1,F2,F10
\begin{center}
\begin{tabular}{|p{5cm}|p{6cm}|} \hline
\textbf{Attributo \br} & \textbf{Valore} \\ \hline
Nome & BR \\ \hline
Commento & Commento\\ \hline
Business object associato & Articolo \\ \hline
Regola & nome=``ArticoloUno'' AND nome!=``ArticoloDue''\\ \hline
\end{tabular} \\
\end{center}
\begin{center}
\begin{tabular}{|p{11cm}|} \hline
\textbf{Output atteso}\\ \hline
\textless BusinessRule associated=``Articolo'' comment=``Commento'' name=``BR'' rule=``nome=\&quot;ArticoloUno\&quot;  AND nome!=\&quot;ArticoloDue\&quot;''\textgreater\\
 \textless AstRuleVersion value=``(AND (= nome \&quot;ArticoloUno\&quot;) (!= nome \&quot;ArticoloDue\&quot;))''/\textgreater\\
 \textless OBool type=``AND''\textgreater\\
 \textless Conf type=``=``\textgreater\\
 \textless FIELD\textgreater nome\textless /FIELD\textgreater\\
 \textless STRING\textgreater ''ArticoloUno''\textless /STRING\textgreater\\
 \textless /Conf\textgreater\\
 \textless Conf type=``!=``\textgreater\\
 \textless FIELD\textgreater nome\textless /FIELD\textgreater\\
 \textless STRING\textgreater ''ArticoloDue''\textless /STRING\textgreater\\
 \textless /Conf\textgreater\\
 \textless /OBool\textgreater\\
 \textless /BusinessRule\textgreater \\
 \hline
\end{tabular} \\
\end{center}

\begin{Large}\textbf{VT13a}\end{Large} \\
Si testa una \br\ per verificare che sia impossibile effettuare operazioni aritmetiche tra booleani.\\
Requisiti soddisfatti dall'esecuzione: F1,F2,F10
\begin{center}
\begin{tabular}{|p{5cm}|p{6cm}|} \hline
\textbf{Attributo \br} & \textbf{Valore} \\ \hline
Nome & BR \\ \hline
Commento & Commento\\ \hline
Business object associato & Articolo \\ \hline
Regola & (scontato+3) - (scontato*4)/scontato;\\ \hline
\end{tabular} \\
\end{center}
\begin{center}
\begin{tabular}{|p{11cm}|} \hline
\textbf{Output atteso}\\ \hline
Operazione non consentita\\
 \hline
\end{tabular} \\
\end{center}

\begin{Large}\textbf{VT13b}\end{Large} \\
Si testa una \br\ per verificare che sia possibile effettuare operazioni logiche tra booleani.\\
Requisiti soddisfatti dall'esecuzione: F1,F2,F10
\begin{center}
\begin{tabular}{|p{5cm}|p{6cm}|} \hline
\textbf{Attributo \br} & \textbf{Valore} \\ \hline
Nome & BR \\ \hline
Commento & Commento\\ \hline
Business object associato & Articolo \\ \hline
Regola & ((scontato\textbar \textbar scontato)\&\& scontato) = true; AND (true\textbar \textbar false\&\& NOT(true))=scontato;\\ \hline
\end{tabular} \\
\end{center}
\begin{center}
\begin{tabular}{|p{11cm}|} \hline
\textbf{Output atteso}\\ \hline
\textless BusinessRule associated=``Articolo'' comment=``Commento'' name=``BR'' rule=``((scontato||scontato) \&amp;\&amp; scontato) = true AND (true || false \&amp;\&amp; NOT(true))=scontato''\textgreater \\
\textless AstRuleVersion value=``(AND (= (\&amp;\&amp; (|| scontato scontato) scontato) true) (= (\&amp;\&amp; (|| true false) (NOT true)) scontato))''/\textgreater\\
 \textless OBool type=``AND''\textgreater\\
 \textless Conf type=``=``\textgreater\\
 \textless OBool type=``\&amp;\&amp;''\textgreater \\
\textless OBool type=``||''\textgreater\\
 \textless FIELD\textgreater scontato\textless /FIELD\textgreater\\
 \textless FIELD\textgreater scontato\textless /FIELD\textgreater\\
 \textless /OBool\textgreater\\
 \textless FIELD\textgreater scontato\textless /FIELD\textgreater\\
 \textless /OBool\textgreater\\
 \textless BOOL\textgreater true\textless /BOOL\textgreater\\
 \textless /Conf\textgreater\\
 \textless Conf type=``=``\textgreater\\
 \textless OBool type=``\&amp;\&amp;''\textgreater\\
 \textless OBool type=``||''\textgreater\\
 \textless BOOL\textgreater true\textless /BOOL\textgreater\\
 \textless BOOL\textgreater false\textless /BOOL\textgreater\\
 \textless /OBool\textgreater\\
 \textless BoFun type=``NOT''\textgreater\\
 \textless BOOL\textgreater true\textless /BOOL\textgreater\\
 \textless /BoFun\textgreater\\
 \textless /OBool\textgreater\\
 \textless FIELD\textgreater scontato\textless /FIELD\textgreater\\
 \textless /Conf\textgreater\\
 \textless /OBool\textgreater\\
 \textless /BusinessRule\textgreater \\
 \hline
\end{tabular} \\
\end{center}

\begin{Large}\textbf{VT14}\end{Large} \\
Si testa una \br\ per verificare che sia possibile effettuare funzioni aritmetiche semplici come 'AVG', 'COUNT', 'SUM'.\\
Requisiti soddisfatti dall'esecuzione: F1,F2,F10,NU2
\begin{center}
\begin{tabular}{|p{5cm}|p{6cm}|} \hline
\textbf{Attributo \br} & \textbf{Valore} \\ \hline
Nome & BR \\ \hline
Commento & Commento\\ \hline
Business object associato & Articolo \\ \hline
Regola & SUM(prezzoBase) = 12; AND COUNT(entrate)=1; AND AVG(prezzoBase)=4;\\ \hline
\end{tabular} \\
\end{center}
\begin{center}
\begin{tabular}{|p{11cm}|} \hline
\textbf{Output atteso}\\ \hline
\textless BusinessRule associated=``Articolo'' comment=``Commento'' name=``BR'' rule=``SUM(prezzoBase) = 12 AND COUNT(entrate)=1, AND AVG(prezzoBase)=4 ''\textgreater\\
 \textless AstRuleVersion value=``(AND (= (SUM prezzoBase) 12) (= (COUNT entrate) 1))''/\textgreater\\
 \textless OBool type=``AND''\textgreater\\
 \textless Conf type=``=``\textgreater\\
 \textless FlFun type=``SUM''\textgreater\\
 \textless FIELD\textgreater prezzoBase\textless /FIELD\textgreater\\
 \textless /FlFun\textgreater\\
 \textless FLOAT\textgreater 12\textless /FLOAT\textgreater\\
 \textless /Conf\textgreater\\
 \textless Conf type=``=``\textgreater\\
 \textless Count type=``COUNT''\textgreater\\
 \textless FIELD\textgreater entrate\textless /FIELD\textgreater\\
 \textless /Count\textgreater\\
 \textless FLOAT\textgreater 1\textless /FLOAT\textgreater\\
 \textless /Conf\textgreater\\
 \textless /OBool\textgreater\\
 \textless /BusinessRule\textgreater \\
 \hline
\end{tabular} \\
\end{center}

\begin{Large}\textbf{VT15a}\end{Large} \\
Si testa una \br\ per verificare che sia impossibile inserire regole prive di un operatore di confronto.\\
Requisiti soddisfatti dall'esecuzione: F1,F2,F10
\begin{center}
\begin{tabular}{|p{5cm}|p{6cm}|} \hline
\textbf{Attributo \br} & \textbf{Valore} \\ \hline
Nome & BR \\ \hline
Commento & Commento\\ \hline
Business object associato & Articolo \\ \hline
Regola & SUM(prezzoBase);\\ \hline
\end{tabular} \\
\end{center}
\begin{center}
\begin{tabular}{|p{11cm}|} \hline
\textbf{Output atteso}\\ \hline
Errore sintattico\\
 \hline
\end{tabular} \\
\end{center}

\begin{Large}\textbf{VT15b}\end{Large} \\
Si testa una \br\ per verificare che sia impossibile inserire regole prive di uno dei due elementi da confrontare\\
Requisiti soddisfatti dall'esecuzione: F1,F2,F10
\begin{center}
\begin{tabular}{|p{5cm}|p{6cm}|} \hline
\textbf{Attributo \br} & \textbf{Valore} \\ \hline
Nome & BR \\ \hline
Commento & Commento\\ \hline
Business object associato & Articolo \\ \hline
Regola & SUM(prezzoBase)=;\\ \hline
\end{tabular} \\
\end{center}
\begin{center}
\begin{tabular}{|p{11cm}|} \hline
\textbf{Output atteso}\\ \hline
Errore sintattico\\
 \hline
\end{tabular} \\
\end{center}


\begin{Large}\textbf{VT15c}\end{Large} \\
Si testa una \br\ per verificare che sia impossibile inserire espressioni prive di operandi\\
Requisiti soddisfatti dall'esecuzione: F1,F2,F10
\begin{center}
\begin{tabular}{|p{5cm}|p{6cm}|} \hline
\textbf{Attributo \br} & \textbf{Valore} \\ \hline
Nome & BR \\ \hline
Commento & Commento\\ \hline
Business object associato & Articolo \\ \hline
Regola & 12 = 2entrate;\\ \hline
\end{tabular} \\
\end{center}
\begin{center}
\begin{tabular}{|p{11cm}|} \hline
\textbf{Output atteso}\\ \hline
Errore sintattico\\
 \hline
\end{tabular} \\
\end{center}

\begin{Large}\textbf{VT15d}\end{Large} \\
Si testa una \br\ per verificare che sia impossibile inserire espressioni prive di operandi\\
Requisiti soddisfatti dall'esecuzione: F1,F2,F10
\begin{center}
\begin{tabular}{|p{5cm}|p{6cm}|} \hline
\textbf{Attributo \br} & \textbf{Valore} \\ \hline
Nome & BR \\ \hline
Commento & Commento\\ \hline
Business object associato & Articolo \\ \hline
Regola & uscite = 2 entrate;\\ \hline
\end{tabular} \\
\end{center}
\begin{center}
\begin{tabular}{|p{11cm}|} \hline
\textbf{Output atteso}\\ \hline
Errore sintattico\\
 \hline
\end{tabular} \\
\end{center}

\begin{Large}\textbf{VT15e}\end{Large} \\
Si testa una \br\ per verificare che sia impossibile inserire regole non complete.\\
Requisiti soddisfatti dall'esecuzione: F1,F2,F10
\begin{center}
\begin{tabular}{|p{5cm}|p{6cm}|} \hline
\textbf{Attributo \br} & \textbf{Valore} \\ \hline
Nome & BR \\ \hline
Commento & Commento\\ \hline
Business object associato & Articolo \\ \hline
Regola & uscite = 2 AND;\\ \hline
\end{tabular} \\
\end{center}
\begin{center}
\begin{tabular}{|p{11cm}|} \hline
\textbf{Output atteso}\\ \hline
Errore sintattico\\
 \hline
\end{tabular} \\
\end{center}

\begin{Large}\textbf{VT15f}\end{Large} \\
Si testa una \br\ per verificare che sia impossibile inserire parentesizzazioni errate.\\
Requisiti soddisfatti dall'esecuzione: F1,F2,F10
\begin{center}
\begin{tabular}{|p{5cm}|p{6cm}|} \hline
\textbf{Attributo \br} & \textbf{Valore} \\ \hline
Nome & BR \\ \hline
Commento & Commento\\ \hline
Business object associato & Articolo \\ \hline
Regola & ((2+4)=uscite;\\ \hline
\end{tabular} \\
\end{center}
\begin{center}
\begin{tabular}{|p{11cm}|} \hline
\textbf{Output atteso}\\ \hline
Errore sintattico\\
 \hline
\end{tabular} \\
\end{center}

\begin{Large}\textbf{VT16}\end{Large} \\
Si testa una \br\ per verificare che sia supportato l'utilizzo dei numeri negativi.\\
Requisiti soddisfatti dall'esecuzione: F1, F2, F10.
\begin{center}
\begin{tabular}{|p{5cm}|p{6cm}|} \hline
\textbf{Attributo \br} & \textbf{Valore} \\ \hline
Nome & BR \\ \hline
Commento & Commento\\ \hline
Business object associato & Articolo \\ \hline
Regola & uscite=((((\textasciitilde2*((\textasciitilde3*\textasciitilde4)+(\textasciitilde4)-(\textasciitilde3*\textasciitilde5)-5)+3)/\textasciitilde2)*\textasciitilde3)+7); \\ \hline
\end{tabular} \\
\end{center}
\begin{center}
\begin{tabular}{|p{11cm}|} \hline
\textbf{Output atteso}\\ \hline
\textless BusinessRule associated=``Articolo'' comment=``Commento'' name=``BR'' rule=``uscite=((((\textasciitilde2*((\textasciitilde3 *\textasciitilde4)+(\textasciitilde4)-(\textasciitilde3*\textasciitilde5) -5)+3)/\textasciitilde2)*\textasciitilde3)+7)''\textgreater \\
\textless AstRuleVersion value=``(= uscite (+ (* (/ (+ (* \textasciitilde2 (- (- (+ (* \textasciitilde3 \textasciitilde4) \textasciitilde4) (* \textasciitilde3 \textasciitilde5)) 5)) 3) \textasciitilde2) \textasciitilde3) 7))''/\textgreater\\ 
\textless Bconf type=``=``\textgreater \\
\textless FIELD\textgreater uscite\textless /FIELD\textgreater \\
\textless OFloat type=``+''\textgreater \\
\textless OFloat type=``*''\textgreater \\
\textless OFloat type=``/''\textgreater \\
\textless OFloat type=``+''\textgreater \\
\textless OFloat type=``*''\textgreater \\
\textless FLOAT\textgreater \textasciitilde2\textless /FLOAT\textgreater \\
\textless OFloat type=``-''\textgreater \\
\textless OFloat type=``-''\textgreater \\
\textless OFloat type=``+''\textgreater \\
\textless OFloat type=``*''\textgreater \\
\textless FLOAT\textgreater \textasciitilde3\textless /FLOAT\textgreater \\
\textless FLOAT\textgreater \textasciitilde4\textless /FLOAT\textgreater \\
\textless /OFloat\textgreater \\
\textless FLOAT\textgreater \textasciitilde4\textless /FLOAT\textgreater \\
\textless /OFloat\textgreater \\
\textless OFloat type=``*''\textgreater \\
\textless FLOAT\textgreater \textasciitilde3\textless /FLOAT\textgreater \\
\textless FLOAT\textgreater \textasciitilde5\textless /FLOAT\textgreater \\
\textless /OFloat\textgreater \\
\textless /OFloat\textgreater \\
\textless FLOAT\textgreater 5\textless /FLOAT\textgreater \\
\textless /OFloat\textgreater \\
\textless /OFloat\textgreater \\
\textless FLOAT\textgreater 3\textless /FLOAT\textgreater \\
\textless /OFloat\textgreater \\
\textless FLOAT\textgreater \textasciitilde2\textless /FLOAT\textgreater \\
\textless /OFloat\textgreater \\
\textless FLOAT\textgreater \textasciitilde3\textless /FLOAT\textgreater \\
\textless /OFloat\textgreater \\
\textless FLOAT\textgreater 7\textless /FLOAT\textgreater \\
\textless /OFloat\textgreater \\
\textless /Bconf\textgreater \\
\textless /BusinessRule\textgreater \\
 \hline
\end{tabular} \\
\end{center}

\begin{Large}\textbf{VT17}\end{Large} \\
Si testa una \br\ per verificare che siano ben tradotti tutti i caratteri non ammessi in un documento xml.\\
Requisiti soddisfatti dall'esecuzione: F1,F2,F10,NPo1
\begin{center}
\begin{tabular}{|p{5cm}|p{6cm}|} \hline
\textbf{Attributo \br} & \textbf{Valore} \\ \hline
Nome & \`a \`e \`i \`o \`u \`A \`E \`I \`O \`U ``test'' , 'test', \& , \textgreater , \textless \\ \hline
Commento & \`a \`e \`i \`o \`u \`A \`E \`I \`O \`U ``test'' , 'test', \& , \textgreater , \textless \\ \hline
Business object associato & Articolo \\ \hline
Regola & scontato=true; \\ \hline
\end{tabular} \\
\end{center}
\begin{center}
\begin{tabular}{|p{11cm}|} \hline
\textbf{Output atteso}\\ \hline
\textless BusinessRule associated=``Articolo'' comment=``\&agrave; \&egrave; \&igrave; \&ograve; \&ugrave; \&Agrave; \&Egrave; \&Igrave; \&Ograve; \&Ugrave; \&quot;test\&quot;, 'test', \&amp;, \&gt;, \&lt;'' name=``\&agrave; \&egrave; \&igrave; \&ograve; \&ugrave; \&Agrave; \&Egrave; \&Igrave; \&Ograve; \&Ugrave; \&quot;test\&quot;, 'test', \&amp;, \&gt;, \&lt;'' rule=``scontato=true''\textgreater\\
 \textless AstRuleVersion value=``(= scontato true)''/\textgreater\\
\textless Bconf type=``=``\textgreater\\
 \textless FIELD\textgreater scontato\textless /FIELD\textgreater\\
 \textless BOOL\textgreater true\textless /BOOL\textgreater\\
 \textless /Bconf\textgreater\\
 \textless /BusinessRule\textgreater \\
 \hline
\end{tabular} \\
\end{center}

\begin{Large}\textbf{VT18}\end{Large} \\
Si testa una \br\ per verificare che siano ammessi solo alcuni caratteri nel message.\\
Requisiti soddisfatti dall'esecuzione: F1,F2,F10,NU3
\begin{center}
\begin{tabular}{|p{5cm}|p{6cm}|} \hline
\textbf{Attributo \br} & \textbf{Valore} \\ \hline
Nome & BR \\ \hline
Commento & Commento \\ \hline
Business object associato & Articolo \\ \hline
Regola & scontato=true message(``abc.. ABC.. 0..9 .?!\^\_-\textasciitilde''); \\ \hline
\end{tabular} \\
\end{center}
\begin{center}
\begin{tabular}{|p{11cm}|} \hline
\textbf{Output atteso}\\ \hline
\textless BusinessRule associated=``Articolo'' comment=``Comment'' name=``BR'' rule=``scontato=true message(\&quote;abc.. ABC.. 0..9 .?!\&\#94; \&\#95;- \&\#126;\&quote;)''\textgreater\\
 \textless AstRuleVersion value=``(= scontato true) message(\&quote;abc.. ABC.. 0..9 .?!\&\#94; \&\#95;- \&\#126; \&quote;)''/\textgreater\\
\textless Bconf type=``=``\textgreater\\
 \textless FIELD\textgreater scontato\textless /FIELD\textgreater\\
 \textless BOOL\textgreater true\textless /BOOL\textgreater\\
 \textless /Bconf\textgreater\\
 \textless /BusinessRule\textgreater \\
 \hline
\end{tabular} \\
\end{center}

\begin{Large}\textbf{VT19}\end{Large} \\
Si testa una \br\ per verificare che non siano ammessi altri caratteri nel message.\\
Requisiti soddisfatti dall'esecuzione: F1,F2,F10
\begin{center}
\begin{tabular}{|p{5cm}|p{6cm}|} \hline
\textbf{Attributo \br} & \textbf{Valore} \\ \hline
Nome & BR \\ \hline
Commento & Commento \\ \hline
Business object associato & Articolo \\ \hline
Regola & scontato=true message(``\# []\{\}|\/=+-*\@\$\%''); \\ \hline
\end{tabular} \\
\end{center}
\begin{center}
\begin{tabular}{|p{11cm}|} \hline
\textbf{Output atteso}\\ \hline
Errore sintattico\\
 \hline
\end{tabular} \\
\end{center}

\begin{Large}\textbf{VT20}\end{Large} \\
Si testa una \br\ per verificare che non siano ammesse regole malformate.\\
Requisiti soddisfatti dall'esecuzione: F1,F2,F10
\begin{center}
\begin{tabular}{|p{5cm}|p{6cm}|} \hline
\textbf{Attributo \br} & \textbf{Valore} \\ \hline
Nome & BR \\ \hline
Commento & Commento \\ \hline
Business object associato & Articolo \\ \hline
Regola & scontato=true +; \\ \hline
\end{tabular} \\
\end{center}
\begin{center}
\begin{tabular}{|p{11cm}|} \hline
\textbf{Output atteso}\\ \hline
Errore sintattico\\
 \hline
\end{tabular} \\
\end{center}

\subsection{Test sul comunicatore}
La macrocomponente comunicatore verr\`a testata a scatola aperta (\underline{whitebox}), inserendo in input al comunicatore una stringa contenente una query \underline{XQuery} e verificando che la modifica sia stata effettuata nel repository. Non verranno testate le query non accettate dal linguaggio \underline{XQuery}, in quanto gia` garantite dal \underline{DBMS} \underline{eXist}. Per un corretto esito dei test \`e necessario che il \textit{repository} abbia una determinata situazione iniziale, illustrata nel riquadro sottostante. \`E importante che la situazione iniziale del repository sia quella descritta di seguito, e che i test sul comunicatore siano eseguiti nell'ordine descritto. Per i motivi sopra elencati \`e molto importante che venga utilizzata l'apposita classe di test. \\
\\

\begin{center}
\begin{tabular}{|p{11cm}|} \hline
\textless BusinessRule associated=``Articolo'' comment=``Commento'' name=``BR'' rule=``entrate=0 AND NOT(scontato)=true''\textgreater \\
\textless AstRuleVersion value=``(AND (= entrate 0) (= (NOT scontato) true))''/\textgreater \\
 \textless OBool type=``AND''\textgreater \\
 \textless Conf type=``=''\textgreater \\
 \textless FIELD\textgreater entrate\textless /FIELD\textgreater \\
 \textless FLOAT\textgreater 0\textless /FLOAT\textgreater \\
 \textless /Conf\textgreater \\
 \textless Conf type=``=''\textgreater \\
\textless BoFun type=``NOT''\textgreater \\
 \textless FIELD\textgreater scontato\textless /FIELD\textgreater \\
\textless /BoFun\textgreater \\
 \textless BOOL\textgreater true\textless /BOOL\textgreater \\
\textless /Conf\textgreater \\
\textless /OBool\textgreater \\
\textless /BusinessRule\textgreater\\ \hline
\end{tabular} \\
\end{center}

\begin{center}
\begin{tabular}{|p{11cm}|} \hline
\textless BusinessRule associated=``Articolo'' comment=``Commento'' name=``BR1'' rule=``entrate=0 AND NOT(scontato)=true''\textgreater \\
\textless AstRuleVersion value=``(AND (= entrate 0) (= (NOT scontato) true))''/\textgreater \\
 \textless OBool type=``AND''\textgreater \\
 \textless Conf type=``=''\textgreater \\
 \textless FIELD\textgreater entrate\textless /FIELD\textgreater \\
 \textless FLOAT\textgreater 0\textless /FLOAT\textgreater \\
 \textless /Conf\textgreater \\
 \textless Conf type=``=''\textgreater \\
\textless BoFun type=``NOT''\textgreater \\
 \textless FIELD\textgreater scontato\textless /FIELD\textgreater \\
\textless /BoFun\textgreater \\
 \textless BOOL\textgreater true\textless /BOOL\textgreater \\
\textless /Conf\textgreater \\
\textless /OBool\textgreater \\
\textless /BusinessRule\textgreater \\ \hline
\end{tabular} \\
\end{center}

\begin{center}
\begin{tabular}{|p{11cm}|} \hline
\textless BusinessRule associated=``Articolo'' comment=``'' name=``BR2'' rule=``entrate=0 AND NOT(scontato)=true''\textgreater \\
\textless AstRuleVersion value=``(AND (= entrate 0) (= (NOT scontato) true))''/\textgreater \\
 \textless OBool type=``AND''\textgreater \\
 \textless Conf type=``=''\textgreater \\
 \textless FIELD\textgreater entrate\textless /FIELD\textgreater \\
 \textless FLOAT\textgreater 0\textless /FLOAT\textgreater \\
 \textless /Conf\textgreater \\
 \textless Conf type=``=''\textgreater \\
\textless BoFun type=``NOT''\textgreater \\
 \textless FIELD\textgreater scontato\textless /FIELD\textgreater \\
\textless /BoFun\textgreater \\
 \textless BOOL\textgreater true\textless /BOOL\textgreater \\
\textless /Conf\textgreater \\
\textless /OBool\textgreater \\
\textless /BusinessRule\textgreater \\ \hline
\end{tabular} \\
\end{center}

\begin{center}
\begin{tabular}{|p{11cm}|} \hline
\textless BusinessRule associated=``Archivio'' comment=``'' name=``BR3'' rule=``entrate=0 AND NOT(scontato)=true''\textgreater \\
\textless AstRuleVersion value=``(AND (= entrate 0) (= (NOT scontato) true))''/\textgreater \\
 \textless OBool type=``AND''\textgreater \\
 \textless Conf type=``=''\textgreater \\
 \textless FIELD\textgreater entrate\textless /FIELD\textgreater \\
 \textless FLOAT\textgreater 0\textless /FLOAT\textgreater \\
 \textless /Conf\textgreater \\
 \textless Conf type=``=''\textgreater \\
\textless BoFun type=``NOT''\textgreater \\
 \textless FIELD\textgreater scontato\textless /FIELD\textgreater \\
\textless /BoFun\textgreater \\
 \textless BOOL\textgreater true\textless /BOOL\textgreater \\
\textless /Conf\textgreater \\
\textless /OBool\textgreater \\
\textless /BusinessRule\textgreater \\ \hline
\end{tabular} \\
\end{center}


\begin{Large}\textbf{CT1}\end{Large} \\
Si testa il comunicatore con una query che ritorni tutte le \br\ che hanno l'attributo name che comincia con BR.\\
Requisiti soddisfatti dall'esecuzione: F3,F5,F7,F8,NPo1,NPr2,NU3
\begin{center}
\begin{tabular}{|p{11cm}|} \hline
\textbf{Query eseguita}\\ \hline
 for \$br in //BusinessRule[starts-with(\@name,'BR')] return \$br\\ \hline
\end{tabular} \\
\end{center}
\begin{center}
\begin{tabular}{|p{11cm}|} \hline
\textbf{Output Atteso}\\ \hline
L'elenco delle \br\ in \underline{xml} come scritte nell'esempio di \rp\ sopra illustrato. In particolare dovranno essere visibili le \br\ : BR, BR1, BR2, BR3.\\ \hline
\end{tabular} \\
\end{center}

\begin{Large}\textbf{CT2}\end{Large} \\
Si testa il comunicatore con una query che cancelli tutte le \br\ il cui nome \`e BR1.\\
Requisiti soddisfatti dall'esecuzione: F3,F5,F7,F8,NPo1,NPr2,NU3
\begin{center}
\begin{tabular}{|p{11cm}|} \hline
\textbf{Query eseguita}\\ \hline
for \$br in //BusinessRule[\@name='BR1'] return update delete \$br;\\ \hline
\end{tabular} \\
\end{center}
\begin{center}
\begin{tabular}{|p{11cm}|} \hline
\textbf{Output Atteso}\\ \hline
La query BR1 deve essere rimossa dal \rp.\\ \hline
\end{tabular} \\
\end{center}

\begin{Large}\textbf{CT3}\end{Large} \\
Si testa il comunicatore con una query che ritorni il nome e la regola associata di tutte le \br\  che sono associate al \bo\ 'Articolo' e non hanno il commento.\\
Requisiti soddisfatti dall'esecuzione: F3,F5,F7,F8,NPo1,NPr2,NU3
\begin{center}
\begin{tabular}{|p{11cm}|} \hline
\textbf{Query eseguita}\\ \hline
for \$br in //BusinessRule[\@associated='Articolo' and not(\@comment)] return concat(\$br[\@name], \$br[\@rule])\\ \hline
\end{tabular} \\
\end{center}
\begin{center}
\begin{tabular}{|p{11cm}|} \hline
\textbf{Output Atteso}\\ \hline
Deve essere visibile la concatenazione del nome della regola seguito dalla regola stessa. In particolare ci\`o deve essere vero per le \br: BR, BR1.\\ \hline
\end{tabular} \\
\end{center}

\begin{Large}\textbf{CT4}\end{Large} \\
Si testa il comunicatore con una query che ritorni il nome e la regola associata di tutte le \br\  che non hanno il commento.\\
Requisiti soddisfatti dall'esecuzione: F3,F5,F7,F8,NPo1,NPr2,NU3
\begin{center}
\begin{tabular}{|p{11cm}|} \hline
\textbf{Query eseguita}\\ \hline
for \$br in //BusinessRule[not(\@comment)]return concat(\$br[\@name], \$br[\@rule])\\ \hline
\end{tabular} \\
\end{center}
\begin{center}
\begin{tabular}{|p{11cm}|} \hline
\textbf{Output Atteso}\\ \hline
Deve essere visibile la concatenazione del nome della regola seguito dalla regola stessa. In particolare ci\`o deve essere vero per le \br: BR2, BR3.\\ \hline
\end{tabular} \\
\end{center}

\begin{Large}\textbf{CT5}\end{Large} \\
Si testa il comunicatore con una query che inserisca una determinata \br .\\
Requisiti soddisfatti dall'esecuzione: F3,F5,F7,F8,NPo1,NPr2,NU3
\begin{center}
\begin{tabular}{|p{11cm}|} \hline
\textbf{Query eseguita}\\ \hline
update insert "\textless BusinessRule associated=``Archivio'' comment=``'' name=``BR3'' rule=``entrate=0 AND NOT(scontato)=true''\textgreater 
\textless AstRuleVersion value=``(AND (= entrate 0) (= (NOT scontato) true))''/\textgreater 
 \textless OBool type=``AND''\textgreater 
 \textless Conf type=``=''\textgreater 
 \textless FIELD\textgreater entrate \textless /FIELD\textgreater 
 \textless FLOAT\textgreater 0 \textless /FLOAT\textgreater 
 \textless /Conf\textgreater 
 \textless Conf type=``='' \textgreater 
\textless BoFun type=``NOT'' \textgreater 
 \textless FIELD\textgreater scontato \textless /FIELD\textgreater 
\textless /BoFun\textgreater 
 \textless BOOL\textgreater true \textless /BOOL\textgreater 
\textless /Conf\textgreater 
\textless /OBool\textgreater 
\textless /BusinessRule\textgreater " into /BusinessRules\\ \hline
\end{tabular} \\
\end{center}

\subsection{Test sulla GUI}
La macrocomponente gui verr\`a testata a scatola aperta (\underline{whitebox}) e manualmente, verificando il funzionamento di ogni sua componente grafica. Verranno testati tutti i bottoni che lanciano nuove finestre. Si tester\`a che ogni bottone invochi il metodo appropiato.\\
\\
\begin{Large}\textbf{GT1a}\end{Large} \\
Si testa il bottone 'Inserisci Business Rule' e si testa che la finestra 'Inserisci' venga aperta.
Requisiti soddisfatti dall'esecuzione: F4,F8,NU3 \\
 \\
\begin{Large}\textbf{GT1b}\end{Large} \\
Si testa il bottone 'Rimuovi Business Rule' e si testa che la finestra 'Rimuovi' venga aperta.
Requisiti soddisfatti dall'esecuzione: F4,F8,NU3\\
 \\
\begin{Large}\textbf{GT1c}\end{Large} \\
Si testa il bottone 'Sandbox' e si testa che la finestra 'Sandbox' venga aperta.
Requisiti soddisfatti dall'esecuzione: F4,F8,NU3\\
 \\
\begin{Large}\textbf{GT2}\end{Large} \\
Si testa che tutti i campi textbox siano scrivibili.
Requisiti soddisfatti dall'esecuzione: F4,F8,NU3\\
 \\
\begin{Large}\textbf{GT3}\end{Large} \\
Si testa che ogni altro bottone della gui invochi il metodo corretto secondo le specifiche del sistema.
Requisiti soddisfatti dall'esecuzione: F4,F8,NU3\\
 \\
\section{Test di integrazione}
Vengono di seguito riportati i test riguardanti le integrazioni delle varie componenti. Questi test hanno come precondizione l'esecuzione dei test sulle singole macrocomponenti, di conseguenza sono mirati a scoprire difetti nell'interfacciamento e nell'interazione tra diverse componenti, nonch\`e di mostrare il soddisfacimento di alcuni requisiti.

\subsection{Integrazione validatore, comunicatore}
L'integrazione tra 'validatore' e 'comunicatore' verr\`a testata a scatola aperta (\underline{whitebox}), inserendo in input al validatore un oggetto \textit{\br\ } con gli attributi di seguito specificati e verificando che l'output sia quello atteso. Essendo gi\`a state testate separatamente le 2 componenti ci si limiter\`a a testare le sole funzionalit\`a risultanti dalla loro integrazione.\\

\begin{Large}\textbf{CVT1a}\end{Large} \\
Si testa che una \br\ non presente nel \rp\ venga inserita ritornando 'true'.
Requisiti soddisfatti dall'esecuzione: NPr2. \\
\begin{center}
\begin{tabular}{|p{11cm}|} \hline
\textbf{Business rule da inserire}\\ \hline
\textless BusinessRule associated=``Articolo'' comment=``'' name=``BRUnique'' rule=``entrate=0''\textgreater \\
\textless AstRuleVersion value=``(= entrate 0)''/\textgreater \\
 \textless Conf type=``=''\textgreater \\
 \textless FIELD\textgreater entrate \textless /FIELD\textgreater \\
 \textless FLOAT\textgreater 0 \textless /FLOAT\textgreater \\
 \textless /Conf\textgreater \\
\textless /BusinessRule\textgreater \\ \hline
\end{tabular} \\
\end{center}

\begin{Large}\textbf{CVT1b}\end{Large} \\
Si deve testare l'impedimento da parte del validatore di inserire nel \rp\ \br\ con lo stesso nome. Per fare ci\`o si riesegue l'inserimento della \br\ sopra illustrata e si testa che il valore ritornato sia 'false'.
Requisiti soddisfatti dall'esecuzione: NPr2.\\
 \\
\subsection{Integrazione gui, validatore}
L'integrazione tra 'gui' e 'validatore' verr\`a testata a scatola aperta (\underline{whitebox}), inserendo da gui  un oggetto \textit{\br\ } con gli attributi di seguito specificati e verificando che l'output sia quello atteso. Essendo gi\`a state testate separatamente le due componenti ci si limiter\`a a testare le sole funzionalit\`a risultanti dalla loro integrazione.\\

\begin{Large}\textbf{GVT1}\end{Large} \\
Si testa che il bottone 'Valida' invochi la validazione della \br\ appena inserita.
Requisiti soddisfatti dall'esecuzione: F2, F8.\\
\begin{center}
\begin{tabular}{|p{11cm}|} \hline
\textbf{Business rule da inserire}\\ \hline
\textless BusinessRule associated=``Articolo'' comment=``'' name=``BR5'' rule=``entrate=0''\textgreater \\
\textless AstRuleVersion value=``(= entrate 0)''/\textgreater \\
 \textless Conf type=``=''\textgreater \\
 \textless FIELD\textgreater entrate \textless /FIELD\textgreater \\
 \textless FLOAT\textgreater 0 \textless /FLOAT\textgreater \\
 \textless /Conf\textgreater \\
\textless /BusinessRule\textgreater \\ \hline
\end{tabular} \\
\end{center}


\subsection{Integrazione del sistema}
L'integrazione dell'intero sistema verr\`a testata a scatola aperta (\underline{whitebox}), testando le sole funzionalit\`a risultanti dalla completa integrazione del sistema.\\
 \\
\begin{Large}\textbf{S1}\end{Large} \\
Si testa che la maschera di login della gui accetti username e password corretti, stabilendo la connessione col \rp.
Requisiti soddisfatti dall'esecuzione: \\
\begin{center}
\begin{tabular}{|p{11cm}|} \hline
\textbf{Input previsto}\\ \hline
Username e password corretti.\\ \hline
\textbf{Output Atteso}\\ \hline
Visualizzazione della finestra principale della gui.\\ \hline
\end{tabular} \\
\end{center}

\begin{Large}\textbf{S2}\end{Large} \\
Si testa che la maschera di login della gui  non accetti username e password errati, impedendo la connessione col \rp\ e riportando un messaggio di errore.
Requisiti soddisfatti dall'esecuzione: F5. \\
 \\
\begin{center}
\begin{tabular}{|p{11cm}|} \hline
\textbf{Input previsto}\\ \hline
Username e password errati.\\ \hline
\textbf{Output Atteso}\\ \hline
Visualizzazione di un messaggio d'errore.\\ \hline
\end{tabular} \\
\end{center}

\section{Test di regressione}
Con test di regressione si intende la ripetizione di alcuni test case a fronte di una o pi\`u modifiche nel software. Ogni volta che una componente viene modificata vanno ripetuti (mediante l'utilizzo del sistema di automazione dei test) tutti i test riguardanti la componente specifica nonch\`e i test per la sua integrazione con altre componenti fino all'esecuzione dei test di integrazione del sistema.
Negli esiti dei test, per non appesantire la lettura, non vengono riportati tutti i test di regressione ma soltanto quelli che abbiano prodotto un output o un esito differente.

\section{Tracciamento requisiti - test}
Nella seguente tabella vengono evidenziati i test necessari al soddisfacimento di ogni requisito. Gli identificativi dei test sono quelli riportati nel capitolo 'Descrizione dei test case'{link}. Mentre gli identificativi dei singoli requisiti sono quelli descritti nel documento ``\AR'' .
 \begin{center}
\begin{tabular}{|p{3cm}|p{8cm}|} \hline
\textbf{Requisito} & \textbf{Test che lo soddisfano}\\ \hline
F1  & VT1a, VT1b, VT1c, VT1d, VT1e, VT1f, VT2a, VT2b, VT2c, VT3a, VT3b, VT4, VT5a, VT5b, VT5c, VT6a, VT6b, VT7, VT8a, VT8b, VT8c, VT9a, VT9b, VT9c, VT10, VT11, VT12a, VT12b, VT13a, VT13b, VT14, VT15a, VT15b, VT15c, VT15d, VT15e, VT15f, VT16, VT17, VT18, VT19, VT20\\ \hline
F2  & GVT1, VT1a, VT1b, VT1c, VT1d, VT1e, VT1f, VT2a, VT2b, VT2c, VT3a, VT3b, VT4, VT5a, VT5b, VT5c, VT6a, VT6b, VT7, VT8a, VT8b, VT8c, VT9a, VT9b, VT9c, VT10, VT11, VT12a, VT12b, VT13a, VT13b, VT14, VT15a, VT15b, VT15c, VT15d, VT15e, VT15f, VT16, VT17, VT18, VT19, VT20\\ \hline
F3  & CT1, CT2, CT3, CT4, CT5 \\ \hline
F4  & GT1a, GT1b, GT1c, GT2, GT3\\ \hline
F5  & S1, S2, CT1, CT2, CT3, CT4, CT5 \\ \hline
F6  &  - \\ \hline
F7  & CT1, CT2, CT3, CT4, CT5  \\ \hline
F8  & GVT1, CT1, CT2, CT3, CT4, CT5, GT1a, GT1b, GT1c, GT2, GT3  \\ \hline
F9  &  DEPRECATO\\ \hline
F10 & VT1a,VT1b,VT1c,VT1d, VT2a, VT2b, VT2c, VT3a, VT3b, VT4, VT5a, VT5b, VT5c, VT6a, VT6b, VT7, VT8a, VT8b, VT8c, VT9a, VT9b, VT9c, VT10, VT11, VT12a, VT12b, VT13a, VT13b, VT14, VT15a, VT15b, VT15c, VT15d, VT15e, VT15f, VT16, VT17, VT18, VT19, VT20 \\ \hline
F11 & VT1a,VT1d \\ \hline
NU1 & VT6a, VT6b, VT7, VT8a, VT8c, VT9a\\ \hline
NU2 & VT14\\ \hline
NU3 & VT1a, VT3b, VT18, GT1a, GT1b, GT1c, GT2, GT3\\ \hline
NPo1 &  VT17, CT1, CT2, CT3, CT4, CT5 \\ \hline
NPr2 &  CVT1a, CVT1b,VT8b, VT8c, CT1, CT2, CT3, CT4, CT5 \\ \hline
NQ1 & VT8b, VT8c  \\ \hline
NQ2 &  - \\ \hline
NQ3 &  - \\ \hline
\end{tabular} \\
\end{center}

\chapter{Automatizzazione dei test}
Essendo la fase di test una fase costosa e laboriosa del processo software, si \`e deciso di utilizzare tools automatici per l'esecuzione dei test.
Per fare questo ci siamo appoggiati ad un framework di lavoro denominato \textit{JUnit}, il quale ci ha permesso di costruire batterie di test strutturalmente omogenei  con i quali invocare metodi e funzionalit\`a di una o pi\`u componenti, pi\`u o meno integrate tra loro. 
Per alcuni test \`e stato necessario l'intervento manuale del verificatore il quale, interagendo direttamente con l'interfaccia grafica del programma, ne ha guidato la corretta esecuzione.  

\section{Dettaglio dei Test eseguiti con \textit{JUnit}}
Riportiamo di seguito i test eseguiti con \textit{JUnit} e gli eventuali errors o failures emersi.
Per ogni classe sono stati eseguiti i test indicati di default da JUnit ed i test relativi alle funzionalit\`a proprie.
\begin{center}
\textbf{Test sulla classe BusinessRule}
\begin{tabular}{|p{6,5 cm}|p{3 cm}|} \hline
\textbf{Metodo/Costruttore} & \textbf{Failure/Errors} \\ \hline
BusinessRuleTest(String) & nessuno \\ \hline
testBusinessRule() & nessuno \\ \hline
setUp() & nessuno \\ \hline
tearDown() & nessuno \\ \hline
\end{tabular}
\end{center}

\begin{center}
\textbf{Test sulla classe Communicator}
\begin{tabular}{|p{6,5 cm}|p{3 cm}|} \hline
\textbf{Metodo/Costruttore} & \textbf{Failure/Errors} \\ \hline
CommunicatorTester(String) & nessuno \\ \hline
testCommunicator() & nessuno \\ \hline
testMakeQuery() & nessuno \\ \hline
setUp() & nessuno \\ \hline
tearDown() & nessuno \\ \hline
\end{tabular}
\end{center}

\begin{center}
\textbf{Test sulla classe GUICommunicator}
\begin{tabular}{|p{6,5 cm}|p{3 cm}|} \hline
\textbf{Metodo/Costruttore} & \textbf{Failure/Errors} \\ \hline
GUICommunicatorTester(String) & nessuno \\ \hline
testGUICommunicator() & nessuno \\ \hline
testDeleteRuleByName() & nessuno \\ \hline
testGetListRules() & nessuno \\ \hline
testMakeQuery() & nessuno \\ \hline
setUp() & nessuno \\ \hline
tearDown() & nessuno \\ \hline
\end{tabular}
\end{center}

\begin{center}
\textbf{Test sulla classe InterpreterCommunicator}
\begin{tabular}{|p{6,5 cm}|p{3 cm}|} \hline
\textbf{Metodo/Costruttore} & \textbf{Failure/Errors} \\ \hline
InterpreterCommunicatorTester(String) & nessuno \\ \hline
testInterpreterCommunicator() & nessuno \\ \hline
testGetRules() & nessuno \\ \hline
setUp() & nessuno \\ \hline
tearDown() & nessuno \\ \hline
\end{tabular}
\end{center}

\begin{center}
\textbf{Test sulla classe ValidatorCommunicator}
\begin{tabular}{|p{6,5 cm}|p{3 cm}|} \hline
\textbf{Metodo/Costruttore} & \textbf{Failure/Errors} \\ \hline
ValidatorCommunicatorTester(String) & nessuno \\ \hline
testValidatorCommunicator() & nessuno \\ \hline
testInsertRule() & nessuno \\ \hline
setUp() & nessuno \\ \hline
tearDown() & nessuno \\ \hline
\end{tabular}
\end{center}

\begin{center}
\textbf{Test sulla classe Gui}
\begin{tabular}{|p{6,5 cm}|p{3 cm}|} \hline
\textbf{Metodo/Costruttore} & \textbf{Failure/Errors} \\ \hline
GuiTest(String) & nessuno \\ \hline
testGui() & nessuno \\ \hline
testMain() & nessuno \\ \hline
setUp() & nessuno \\ \hline
tearDown() & nessuno \\ \hline
\end{tabular}
\end{center}

\begin{center}
\textbf{Test sulla classe Validator}
\begin{tabular}{|p{6,5 cm}|p{3 cm}|} \hline
\textbf{Metodo/Costruttore} & \textbf{Failure/Errors} \\ \hline
ValidatorTest(String, String) & nessuno \\ \hline
testValidate(BusinessRule) & nessuno \\ \hline
setUp() & nessuno \\ \hline
tearDown() & nessuno \\ \hline
\end{tabular}
\end{center}

\begin{center}
\textbf{Test sulla classe TypeCollisionException}
\begin{tabular}{|p{6,5 cm}|p{3 cm}|} \hline
\textbf{Metodo/Costruttore} & \textbf{Failure/Errors} \\ \hline
TypeCollisionExceptionTest(String) & nessuno \\ \hline
testToString() & nessuno \\ \hline
setUp() & nessuno \\ \hline
tearDown() & nessuno \\ \hline
\end{tabular}
\end{center}

\begin{center}
\textbf{Test sulla classe XMLParser}
\begin{tabular}{|p{6,5 cm}|p{3 cm}|} \hline
\textbf{Metodo/Costruttore} & \textbf{Failure/Errors} \\ \hline
XMLParserTest(String[]) & nessuno \\ \hline
testScanAST(Tree, Element, Document) & nessuno \\ \hline
testHandleTagName(String) & nessuno \\ \hline
testParse(Tree, BusinessRule) & nessuno \\ \hline
setUp() & nessuno \\ \hline
tearDown() & nessuno \\ \hline
\end{tabular}
\end{center}

\chapter{Esiti dei test}

\section{Esiti dei test dei difetti}
I \textit{test dei difetti} servono ad individuare possibili difetti o errori del codice su cui vengono testati. Per questo l'esito di un test dei difetti \`e segnato come negativo ogniqualvolta non evidenzia un difetto, viene altres\`i segnato come positivo l'esito di quei test che abbiano individuato dei difetti nel software.
Vengono di seguito elencati gli esiti dei test effetuati. I test  sono presentati in forma tabulare, indicando il codice identificativo del test, seguito dal suo esito. I test risultati positivi sono evidenziati in colore \textcolor{err}{rosso}. I test di regressione, come illustrato precedentemente, sono riportati solo dove abbiano prodotto un output o un esito differente dalla loro precedente esecuzione. I test di regressione sono evidenziati nelle tabelle sottostanti dalla sigla RT e sono colorati in \textcolor{rt}{blu} .


\subsection{Test sul validatore}
\begin{center}
\begin{tabular}{|p{1cm}|p{1.6cm}|p{8.4cm}|} \hline
\textbf{Id} & \textbf{Esito} & \textbf{Note} \\ \hline
VT1a & negativo & \\ \hline
VT1b & negativo & \\ \hline
VT1c & \textcolor{err}{positivo} & Vengono inserite \br\ con nome=''.\\ \cline{2-3}
& \textcolor{rt}{negativo} & RT  Viene effettuato un controllo per evitare stringhe vuote.\\ \hline
VT1d & negativo & \\ \hline
VT1e & negativo & \\ \hline
VT1f & negativo & \\ \hline
VT2a & \textcolor{err}{positivo} & La parentesizzazione che coinvolge un intera operazione di confronto non \`e ancora stata introdotta nella grammatica del linguaggio.\\ \cline{2-3}
& \textcolor{rt}{negativo} & RT  \\ \hline
VT2b & \textcolor{err}{positivo} & Non devono comparire nel codice \underline{XML} tag 'OpRule'.\\ \cline{2-3}
& \textcolor{rt}{negativo} & RT  \\ \hline
VT2c & negativo & \\ \hline
VT3a & negativo & \\ \hline
VT3b & negativo & \\ \hline
VT4 & negativo & \\ \hline
VT5a & negativo & \\ \hline
VT5b & negativo & \\ \hline
VT5c & negativo & \\ \hline
VT6a & negativo & \\ \hline
VT6b & negativo & \\ \hline
VT7 & negativo & \\ \hline
VT8a & negativo & \\ \hline
VT8b & negativo & \\ \hline
VT8c & negativo & \\ \hline
VT9a & negativo & \\ \hline
VT9b & negativo & \\ \hline
VT9c & negativo & \\ \hline
VT10 & negativo & \\ \hline
VT11 & negativo & \\ \hline
VT12a & negativo & \\ \hline
VT12b & negativo & \\ \hline
VT13a & negativo & \\ \hline
VT13b & negativo & \\ \hline
VT14 & negativo & \\ \hline
\end{tabular} \\
\end{center}

\begin{center}
\begin{tabular}{|p{1cm}|p{1.6cm}|p{8.4cm}|} \hline
\textbf{Id} & \textbf{Esito} & \textbf{Note} \\ \hline
VT15a & negativo & \\ \hline
VT15b & negativo & \\ \hline
VT15c & \textcolor{err}{positivo} &  \\ \cline{2-3}
& \textcolor{rt}{negativo} & RT Introduzione dei ';' nella grammatica del linguaggio \\ \hline
VT15d & \textcolor{err}{positivo} & \\ \cline{2-3}
& \textcolor{rt}{negativo} & RT Introduzione dei ';' nella grammatica del linguaggio \\ \hline
VT15e & negativo & \\ \hline
VT15f & negativo & \\ \hline
VT16 & negativo & \\ \hline
VT17 & \textcolor{err}{positivo} & Le lettere accentate non sono ancora state aggiunte alla grammatica del linguaggio.\\ \cline{2-3}
& \textcolor{rt}{negativo} & RT  \\ \hline
VT18 & \textcolor{err}{positivo} & La grammatica del linguaggio non prevede ancora l'utilizzo di numeri da 0 a 9 nelle stringhe.\\ \cline{2-3}
& \textcolor{rt}{negativo} & RT  \\ \hline
VT19 & negativo & \\ \hline
VT20 & negativo & \\ \hline
\end{tabular} \\
\end{center}

\subsection{Test sul comunicatore}
\begin{center}
\begin{tabular}{|p{1cm}|p{1.6cm}|p{8.4cm}|} \hline
\textbf{Id} & \textbf{Esito} & \textbf{Note} \\ \hline
CT1 & negativo & \\ \hline
CT2 & negativo & \\ \hline
CT3 & negativo & \\ \hline
CT4 & negativo & \\ \hline
CT5 & negativo & \\ \hline
\end{tabular} \\
\end{center}

\subsection{Test sulla gui}
\begin{center}
\begin{tabular}{|p{1cm}|p{1.6cm}|p{8.4cm}|} \hline
\textbf{Id} & \textbf{Esito} & \textbf{Note} \\ \hline
GT1a & negativo & \\ \hline
GT1b & negativo & \\ \hline
GT1c & negativo & \\ \hline
GT2 & negativo & \\ \hline
GT3 & negativo & \\ \hline
\end{tabular} \\
\end{center}

\subsection{Integrazione validatore, comunicatore}
\begin{center}
\begin{tabular}{|p{1cm}|p{1.6cm}|p{8.4cm}|} \hline
\textbf{Id} & \textbf{Esito} & \textbf{Note} \\ \hline
CVT1a & negativo & \\ \hline
CVT1b & negativo & \\ \hline
\end{tabular} \\
\end{center}

\subsection{Integrazione gui, validatore}
\begin{center}
\begin{tabular}{|p{1cm}|p{1.6cm}|p{8.4cm}|} \hline
\textbf{Id} & \textbf{Esito} & \textbf{Note} \\ \hline
GVT1 & negativo & \\ \hline
\end{tabular} \\
\end{center}

\subsection{Integrazione del sistema}
\begin{center}
\begin{tabular}{|p{1cm}|p{1.6cm}|p{8.4cm}|} \hline
\textbf{Id} & \textbf{Esito} & \textbf{Note} \\ \hline
S1 & negativo & \\ \hline
S2 & \textcolor{err}{positivo} & In sitemi unix permette di accedere al \rp\ anche inserendo username e password errati.\\ \cline{2-3}
& \textcolor{rt}{negativo} & RT Esito errato del pprecedente test. I dati di autenticazione accettati sono sempre quelli inseriti durante l'installazione di \underline{eXist} \\ \hline
\end{tabular} \\
\end{center}


\chapter{Esiti dei test di convalida}
I \textit{test di convalida} servono a dimostrare che tutti i requisiti sono stati soddisfatti. Per questo l'esito di un test di convalida \`e segnato come positivo ogniqualvolta dimostra che un requisito \`e soddisfatto, viene altres\`i segnato come negativo l'esito dei test che hanno fallito. 
 \begin{center}
\begin{tabular}{|p{2cm}|p{7cm}|p{2cm}|} \hline
\textbf{Requisito} & \textbf{Test che lo soddisfano} & \textbf{Esito}\\ \hline
\textbf{F1}  & VT1a, VT1b, VT1c, VT1d, VT1e, VT1f & positivi\\ \cline{2-3}
& VT2a, VT2b, VT2c, VT3a, VT3b & positivi\\ \cline{2-3}
& VT4 & positivo\\ \cline{2-3}
& VT5a, VT5b, VT5c & positivi\\ \cline{2-3}
&  VT6a, VT6b & positivi\\ \cline{2-3}
&  VT7 & positivo\\ \cline{2-3}
&  VT8a, VT8b, VT8c & positivi\\ \cline{2-3}
&  VT9a, VT9b, VT9c & positivi\\ \cline{2-3}
&  VT10 & positivo\\ \cline{2-3}
&  VT11 & positivo\\ \cline{2-3}
&  VT12a, VT12b & positivi\\ \cline{2-3}
&  VT13a, VT13b & positivi\\ \cline{2-3}
&  VT14 & positivo\\ \cline{2-3}
&  VT15a, VT15b, VT15c, VT15d, VT15e, VT15f & positivi\\ \cline{2-3}
&  VT16 & positivo\\ \cline{2-3}
&  VT17 & positivo\\ \cline{2-3}
&  VT18 & positivo\\ \cline{2-3}
&  VT19 & positivo\\ \cline{2-3}
&  VT20 & positivo\\ \hline
\end{tabular} \\
\end{center}

 \begin{center}
\begin{tabular}{|p{2cm}|p{7cm}|p{2cm}|} \hline
\textbf{F2}  & GVT1 & positivo\\ \cline{2-3}
&  VT1a, VT1b, VT1c, VT1d, VT1e, VT1f & positivi\\ \cline{2-3}
&  VT2a, VT2b, VT2c & positivi\\ \cline{2-3}
&  VT3a, VT3b & positivi\\ \cline{2-3}
&  VT4 & positivo\\ \cline{2-3}
&  VT5a, VT5b, VT5c & positivi\\ \cline{2-3}
&  VT6a, VT6b & positivi\\ \cline{2-3}
&  VT7 & positivo\\ \cline{2-3}
&  VT8a, VT8b, VT8c & positivi\\ \cline{2-3}
&  VT9a, VT9b, VT9c & positivi\\ \cline{2-3}
&  VT10 & positivo\\ \cline{2-3}
&  VT11 & positivo\\ \cline{2-3}
&  VT12a, VT12b & positivi\\ \cline{2-3}
&  VT13a, VT13b & positivi\\ \cline{2-3}
&  VT14 & positivo\\ \cline{2-3}
&  VT15a, VT15b, VT15c, VT15d, VT15e, VT15f & positivi\\ \cline{2-3}
&  VT16 & positivo\\ \cline{2-3}
&  VT17 & positivo\\ \cline{2-3}
&  VT18 & positivo\\ \cline{2-3}
&  VT19 & positivo\\ \cline{2-3}
&  VT20 & positivo\\ \hline
\end{tabular} \\
\end{center}

 \begin{center}
\begin{tabular}{|p{2cm}|p{7cm}|p{2cm}|} \hline
\textbf{F3 } & CT1, CT2, CT3, CT4, CT5 & positivi \\ \hline
\end{tabular} \\
\end{center}

 \begin{center}
\begin{tabular}{|p{2cm}|p{7cm}|p{2cm}|} \hline
\textbf{F4}  & GT1a, GT1b, GT1c, GT2, GT3 & positivi\\ \hline
\end{tabular} \\
\end{center}

 \begin{center}
\begin{tabular}{|p{2cm}|p{7cm}|p{2cm}|} \hline
\textbf{F5}  & S1, S2 & positivi\\ \cline{2-3}
& CT1, CT2, CT3, CT4, CT5 & positivi\\ \hline
\end{tabular} \\
\end{center}

 \begin{center}
\begin{tabular}{|p{2cm}|p{7cm}|p{2cm}|} \hline
\textbf{F6}  &  - & -\\ \hline
\end{tabular} \\
\end{center}

 \begin{center}
\begin{tabular}{|p{2cm}|p{7cm}|p{2cm}|} \hline
\textbf{F7}  & CT1, CT2, CT3, CT4, CT5 & positivi \\ \hline
\end{tabular} \\
\end{center}

 \begin{center}
\begin{tabular}{|p{2cm}|p{7cm}|p{2cm}|} \hline
\textbf{F8}  & GVT1 & positivo\\ \cline{2-3}
&  CT1, CT2, CT3, CT4, CT5 & positivi\\ \cline{2-3}
&  GT1a, GT1b, GT1c, GT2, GT3 & positivi \\ \hline
\end{tabular} \\
\end{center}

 \begin{center}
\begin{tabular}{|p{2cm}|p{7cm}|p{2cm}|} \hline
\textbf{F9}  &  DEPRECATO & -\\ \hline
\end{tabular} \\
\end{center}

 \begin{center}
\begin{tabular}{|p{2cm}|p{7cm}|p{2cm}|} \hline
\textbf{F10} & VT1a,VT1b,VT1c,VT1d & positivi\\ \cline{2-3}
&  VT2a, VT2b, VT2c & positivi\\ \cline{2-3}
&  VT3a, VT3b & positivi\\ \cline{2-3}
&  VT4 & positivo\\ \cline{2-3}
&  VT5a, VT5b, VT5c & positivi\\ \cline{2-3}
&  VT6a, VT6b & positivi\\ \cline{2-3}
&  VT7 & positivo\\ \cline{2-3}
&  VT8a, VT8b, VT8c & positivi\\ \cline{2-3}
&  VT9a, VT9b, VT9c & positivi\\ \cline{2-3}
&  VT10 & positivo\\ \cline{2-3}
&  VT11 & positivo\\ \cline{2-3}
&  VT12a, VT12b & positivi\\ \cline{2-3}
&  VT13a, VT13b & positivi\\ \cline{2-3}
&  VT14 & positivo\\ \cline{2-3}
&  VT15a, VT15b, VT15c, VT15d, VT15e, VT15f & positivi\\ \cline{2-3}
&  VT16 & positivo\\ \cline{2-3}
&  VT17 & positivo\\ \cline{2-3}
&  VT18 & positivo\\ \cline{2-3}
&  VT19 & positivo\\ \cline{2-3}
&  VT20 & positivo \\ \hline
\end{tabular} \\
\end{center}

 \begin{center}
\begin{tabular}{|p{2cm}|p{7cm}|p{2cm}|} \hline
\textbf{F11} & VT1a,VT1d & positivi\\ \hline
\end{tabular} \\
\end{center}

 \begin{center}
\begin{tabular}{|p{2cm}|p{7cm}|p{2cm}|} \hline
\textbf{NU1} & VT6a, VT6b& positivi\\ \cline{2-3}
& VT7& positivo\\ \cline{2-3}
&  VT8a, VT8c& positivi\\ \cline{2-3}
&  VT9a & positivo \\ \hline
\end{tabular} \\
\end{center}

 \begin{center}
\begin{tabular}{|p{2cm}|p{7cm}|p{2cm}|} \hline
\textbf{NU2} & VT14 & positivo\\ \hline
\end{tabular} \\
\end{center}

 \begin{center}
\begin{tabular}{|p{2cm}|p{7cm}|p{2cm}|} \hline
\textbf{NU3} & VT1a& positivo\\ \cline{2-3}
&  VT3b& positivo\\ \cline{2-3}
&  VT18& positivo\\ \cline{2-3}
&  GT1a, GT1b, GT1c, GT2, GT3 & positivi\\ \hline
\end{tabular} \\
\end{center}

 \begin{center}
\begin{tabular}{|p{2cm}|p{7cm}|p{2cm}|} \hline
\textbf{NPo1} &  VT17& positivo\\ \cline{2-3}
&  CT1, CT2, CT3, CT4, CT5 & positivi\\ \hline
\end{tabular} \\
\end{center}

 \begin{center}
\begin{tabular}{|p{2cm}|p{7cm}|p{2cm}|} \hline
\textbf{NPr2} &  CVT1a, CVT1b& positivi\\ \cline{2-3}
& VT8b, VT8c & positivi\\ \cline{2-3}
&  CT1, CT2, CT3, CT4, CT5 & positivi \\ \hline
\end{tabular} \\
\end{center}

 \begin{center}
\begin{tabular}{|p{2cm}|p{7cm}|p{2cm}|} \hline
\textbf{NQ1} & VT8b, VT8c & positivi \\ \hline
\end{tabular} \\
\end{center}

 \begin{center}
\begin{tabular}{|p{2cm}|p{7cm}|p{2cm}|} \hline
\textbf{NQ2} &  - & -\\ \hline
\end{tabular} \\
\end{center}

 \begin{center}
\begin{tabular}{|p{2cm}|p{7cm}|p{2cm}|} \hline
\textbf{NQ3} &  - & -\\ \hline
\end{tabular} \\
\end{center}
\end{document}
