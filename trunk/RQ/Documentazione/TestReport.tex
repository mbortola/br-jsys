
\documentclass[11pt,titlepage,a4paper]{report}

\usepackage[italian]{babel}
\usepackage{fancyhdr}
\usepackage{graphicx}
\usepackage{hyperref}

%\usepackage{lastpage} % total page count

\usepackage{color}
\usepackage{lastpage} % total page count

\graphicspath{{./pics/}} % cartella di salvataggio immagini

\pagestyle{fancy}
\renewcommand{\sectionmark}[1]{\markright{\thesection.\ #1}}
\lhead{\nouppercase{\rightmark}}
\rhead{\nouppercase{\leftmark}}
\renewcommand{\chaptermark}[1]{%
\markboth{\thechapter.\ #1}{}}


\fancypagestyle{plain}{%
	\lhead{\includegraphics[height=50pt]{logo.eps}}
	\chead{}
	\rhead{HappyCode inc \\ happycodeinc@gmail.com}
	\lfoot{BR-jsys}
	\cfoot{\thepage\ / \pageref{LastPage}}
	\rfoot{\dt - \lv}
	\renewcommand{\headrulewidth}{1pt}
	\renewcommand{\footrulewidth}{1pt}
}
	\lhead{\includegraphics[height=50pt]{logo.eps}}
	\chead{}
	\rhead{HappyCode inc \\ happycodeinc@gmail.com}
	\lfoot{BR-jsys}
	\cfoot{\thepage\ / \pageref{LastPage}}
	\rfoot{\dt - \lv}
	\renewcommand{\headrulewidth}{1pt}
	\renewcommand{\footrulewidth}{1pt}

\hypersetup{
    colorlinks=true,       % false: boxed links; true: colored links
   linkcolor=[rgb]{0.11,0.55,0.83},          % color of internal links
    urlcolor=cyan           % color of external links
}
\definecolor{err}{rgb}{0.9,0.1,0.1}

% fine layout% layout
\begin{document}
%definizione variabili 
\newcommand{\lv}{ 0.3 } % latest version
\newcommand{\dt}{ Test Report }% Document title
%common variables
\newcommand{\br}{\underline{business rule}}
\newcommand{\brs}{\underline{business rules}}
\newcommand{\bo}{\underline{business object}}
\newcommand{\bos}{\underline{business objects}}
\newcommand{\rp}{\underline{repository}}
\newcommand{\brp}{BusinessRuleParser}
\newcommand{\brl}{BusinessRuleLexer}
\newcommand{\BR}{\underline{BusinessRule}}

%nomi dei componenti
\newcommand{\AT}{Alessia Trivellato}
\newcommand{\ET}{Elena Trivellato}
\newcommand{\FC}{Filippo Carraro}
\newcommand{\LA}{Luca Appon}
\newcommand{\MB}{Michele Bortolato}
\newcommand{\MT}{Marco Tessarotto}
\newcommand{\MM}{Mattia Meroi}%altre variabili
% ultime versioni dei documenti da modificare solo alla fine
\newcommand{\AR}{AnalisiDeiRequisiti.2.6.pdf}
\newcommand{\DdP}{DefinizioneDiProdotto.0.9.pdf}
\newcommand{\G}{ Glossario.1.8.pdf }
\newcommand{\NdP}{NormeDiProgetto.2.0.pdf}
\newcommand{\PdQ}{ PianoDiQualifica.1.4.pdf }
\newcommand{\PdP}{ PianoDiProgetto.1.7.pdf }
\newcommand{\ST}{SpecificaTecnica.1.5.pdf}
\newcommand{\TR}{TestReport.0.7.pdf}
\newcommand{\MU}{ManualeUtente.0.3.pdf}%nomi documenti
%fine definizione variabili
\hyphenation{
 a-na-lo-go
 as-so-cia-zio-ne
 %attività non si può inserire come tutte le parole accentate che vanno messe nel testo semplice scritte at\-ti\-vi\-tà o come variabile
 coe-ren-za
 com-po-nen-ti
 con-si-glia-bi-le
 des-crit-te
 des-cri-zio-ni
 di-a-gram-ma
 di-a-gram-mi
 e-le-men-to
 e-se-gui-re
 e-si-sten-ti
 es-pli-ci-to
 glo-bal-men-te
 glos-sa-rio
 in-se-ri-men-to
 li-vel-lo
 ne-ces-sa-rio
 per-met-te-re
 re-po-si-to-ry
 re-vi-sio-na-men-to
 ri-chies-te
 se-le-zio-na-ta
 se-gna-la-ta
 va-li-da-zio-ne
 va-ria-bi-li
 ve-ri-fi-ca-re
 vi-sua-liz-za-te
 e-ven-tua-li
 o-pe-ra-zio-ne
 ar-chi-via-zio-ne
 mo-di-fi-ca
 ar-chi-vio
 des-cri-zio-ne
 pa-ren-te-si
 i-ni-zia
}


%sillabazione

\begin{titlepage}\begin{center}
\vspace*{0.5in}
\includegraphics{logo.eps}
\vspace*{0.2in} \\
{\Large \textbf{BR-jsys}}
{\Large \emph{business rules} per sistemi gestionali in architettura J2EE } 
\vspace{2in} \\
\Huge \textsc{ \dt }
\par\rule{10cm}{0.4pt} \par {\large Versione \lv - \today} \\
\end{center}\end{titlepage}

\vspace*{0.5in}%pagina del titolo


\begin{center}
\thispagestyle{plain}
\begin{table}[htbp]
\large{
\begin{tabular}{l}
\Large{\textbf{\textsf{Capitolato: ''BR-jsys``}}} \\
\begin{tabular}{|p{6cm}|p{6cm}|}
\hline
\textbf{Data creazione:} & 02/03/2008 \\ \hline
\textbf{Versione:} & \lv \\ \hline
\textbf{Stato del documento:} & Formale, esterno \\ \hline
% ----------------------------------------------------------------------------autori
\textbf{Revisione RQ} & \\ \hline
\textbf{Redazione:} & \MT \\ \hline
\textbf{Revisione:} &  \\ \hline
\textbf{Approvazione:}  &  \\ \hline
\end{tabular} \\
\end{tabular}
}
\end{table}

\begin{table}[hbtp]
\large{
\begin{tabular}{l}
\Large{\textbf{\textsf{Lista di distribuzione}}} \\
\begin{tabular}{|p{6cm}|p{6cm}|} \hline
%  -------------------------------------------------------------lista di distribuzione
{HappyCode inc}& Gruppo di lavoro \\ \hline
{Tullio Vardanega, Renato Conte}& Committenti \\ \hline 
{Zucchetti S.r.l}& Azienda proponente\\ \hline
\end{tabular} \\
\end{tabular}
}
\end{table}
\begin{table}[hbtp]

\Large{\textbf{\textsf{Diario delle modifiche}}} \\
\begin{small}
\begin{tabular}[t]{|p{1,2cm}|p{1.9cm}|p{2.9cm}|p{5cm}|} \hline
Versione & Data & Autore & Descrizione \\ \hline
%-------------------------------------------------------------------------------diario modifiche
0.5 & 06/03/2008 & \LA & Aggiunto Sommario.\\ \hline
0.4 & 05/03/2008 & \MT & Aggiunta delle tabelle per gli esiti e della tabella di tracciamento 'requisiti - test'.\\ \hline
0.3 & 04/03/2008 & \MT & Modifica alla nomenclatura dei test.\\ \hline
0.2 & 03/03/2008 & \MT & Aggiunta dei 'test case'.\\ \hline
0.1 & 02/03/2008 & \MT & Stesura iniziale del documento.\\ \hline
\end{tabular} \\
\end{small}


\end{table}
\end{center}
\newpage
\tableofcontents

\chapter*{Sommario}
\addcontentsline{toc}{chapter}{ I Sommario}
Il presente documento illustra le strategie utilizzate per testare il prodotto BR-jsys in ogni sua componente.

\chapter*{Glossario}
\addcontentsline{toc}{chapter}{II Glossario}
Per gli altri termini non citati nella tabella soprastante si fa riferimento al file esterno \G. I termini presenti in questo documento la cui descrizione \`e riportata nel glossario sono evidenziati mediante sottolineatura.

\chapter{Introduzione}

\section{Scopo del documento}
Il presente documento viene redatto al fine di illustrare la progettazione dei \textit{test case} nel dettaglio indicando un insieme di input e il corrispondente insieme di \textit{output attesi}. Verranno inoltre riportati nel presente documento gli esiti dei test eseguiti sul sistema ``Br-jsys'' al fine di non appesantire la lettura del documento \PdQ .

\section{Definizioni, acronimi, abbreviazioni}
Nella tabella di seguito vengono riportate tutte le abbreviazioni utlizzate nel documento, comprese le sigle utilizzate per l'identificazione dei test.
\begin{center}
\begin{tabular}{||p{3.0cm}||p{8.5cm}||} \hline
\textbf{Abbreviazione} & \textbf{Significato} \\ \hline

VT\textit{\#} & Con VT seguito da un numero intero sono indicati i test che riguardano la 'macrocomponente validatore'.\\ \hline
CT\textit{\#} & Con CT seguito da un numero intero sono indicati i test che riguardano la 'macrocomponente comunicatore'.\\ \hline
CVT\textit{\#} & Con CVT seguito da un numero intero sono indicati i test di integrazione tra 'validatore' e 'comunicatore'.\\ \hline
GVT\textit{\#} & Con GVT seguito da un numero intero sono indicati i test integrazione tra 'gui' e 'validatore'.\\ \hline
S\textit{\#} & Con S seguito da un numero intero sono indicati i test di integrazione del sistema.\\ \hline
RT\textit{} & Con RT si intende la riesecuzione di un test (test di regressione) avvenuta a fronte di una modifica al codice.\\ \hline

\end{tabular} \\
\end{center}

\chapter{Descrizione dei test case}
Vengono di seguito presentati i \textit{test case}. Ogni test case \`e composto da un input, un output atteso. Ogni test case \`e inoltre accompagnato da un breve testo che identifica lo scopo del test, e dall'elenco dei requisiti soddisfatti dal test.

\section{Test sulle componenti}
Vengono di seguito riportati i test riguardanti le macrocomponenti del sistema \underline{br-jsys}. Questi test si occupano di testare il corretto funzionamento della componente, nonch\`e di mostrare il soddisfacimento di alcuni requisiti.

\subsection{Test sul validatore}
La macrocomponente validatore verr\`a testata a scatola aperta ( whitebox ), inserendo in input al validatore un oggetto \textit{\br\ } con gli attributi di seguito specificati e confrontando l'output del validatore con l'output atteso.\\
\\
\begin{Large}\textbf{VT1a}\end{Large} \\
Si testa una \br\ semplice per verificare che siano accettati tutti i campi dati.\\
Requisiti soddisfatti dall'esecuzione:
\begin{center}
\begin{tabular}{|p{5cm}|p{6cm}|} \hline
\textbf{Attributo \br} & \textbf{Valore} \\ \hline
Nome & BR\\ \hline
Commento & UnCommento\\ \hline
Business object associato & Articolo\\ \hline
Regola & entrate=0 AND NOT(scontato)\\ \hline
\end{tabular} \\
\end{center}
\begin{center}
\begin{tabular}{|p{11cm}|} \hline
\textbf{Output atteso}\\ \hline
...\\
 \hline
\end{tabular} \\
\end{center}

\begin{Large}\textbf{VT1b}\end{Large} \\
Si testa una \br\ semplice per verificare che il commento pu\`o non essere inserito.\\
Requisiti soddisfatti dall'esecuzione:
\begin{center}
\begin{tabular}{|p{5cm}|p{6cm}|} \hline
\textbf{Attributo \br} & \textbf{Valore} \\ \hline
Nome & BR\\ \hline
Commento & \\ \hline
Business object associato & Articolo\\ \hline
Regola & entrate=3 AND NOT(uscite=4)\\ \hline
\end{tabular} \\
\end{center}
\begin{center}
\begin{tabular}{|p{11cm}|} \hline
\textbf{Output atteso}\\ \hline
...\\
 \hline
\end{tabular} \\
\end{center}

\begin{Large}\textbf{VT1c}\end{Large} \\
Si testa una \br\ semplice per verificare che il nome non \`e opzionale.\\
Requisiti soddisfatti dall'esecuzione:
\begin{center}
\begin{tabular}{|p{5cm}|p{6cm}|} \hline
\textbf{Attributo \br} & \textbf{Valore} \\ \hline
Nome & \\ \hline
Commento & Commento\\ \hline
Business object associato & Articolo\\ \hline
Regola & entrate=5 AND uscite=3\\ \hline
\end{tabular} \\
\end{center}
\begin{center}
\begin{tabular}{|p{11cm}|} \hline
\textbf{Output atteso}\\ \hline
...\\
 \hline
\end{tabular} \\
\end{center}

\begin{Large}\textbf{VT1d}\end{Large} \\
Si testa una \br\ semplice per verificare che il \bo\ non \`e opzionale.\\
Requisiti soddisfatti dall'esecuzione:
\begin{center}
\begin{tabular}{|p{5cm}|p{6cm}|} \hline
\textbf{Attributo \br} & \textbf{Valore} \\ \hline
Nome & BR \\ \hline
Commento & Commento\\ \hline
Business object associato & \\ \hline
Regola & entrate=3 OR uscite=5\\ \hline
\end{tabular} \\
\end{center}
\begin{center}
\begin{tabular}{|p{11cm}|} \hline
\textbf{Output atteso}\\ \hline
...\\
 \hline
\end{tabular} \\
\end{center}

\begin{Large}\textbf{VT1e}\end{Large} \\
Si testa una \br\ semplice per verificare che la regola stessa non \`e opzionale.\\
Requisiti soddisfatti dall'esecuzione:
\begin{center}
\begin{tabular}{|p{5cm}|p{6cm}|} \hline
\textbf{Attributo \br} & \textbf{Valore} \\ \hline
Nome & BR \\ \hline
Commento & Commento\\ \hline
Business object associato & Articolo \\ \hline
Regola & \\ \hline
\end{tabular} \\
\end{center}
\begin{center}
\begin{tabular}{|p{11cm}|} \hline
\textbf{Output atteso}\\ \hline
...\\
 \hline
\end{tabular} \\
\end{center}

\begin{Large}\textbf{VT1f}\end{Large} \\
Si testa una \br\ semplice per verificare che non vengano accettate \br\ vuote.\\
Requisiti soddisfatti dall'esecuzione:
\begin{center}
\begin{tabular}{|p{5cm}|p{6cm}|} \hline
\textbf{Attributo \br} & \textbf{Valore} \\ \hline
Nome &  \\ \hline
Commento & \\ \hline
Business object associato &  \\ \hline
Regola & \\ \hline
\end{tabular} \\
\end{center}
\begin{center}
\begin{tabular}{|p{11cm}|} \hline
\textbf{Output atteso}\\ \hline
...\\
 \hline
\end{tabular} \\
\end{center}

\begin{Large}\textbf{VT2a}\end{Large} \\
Si testa una \br\ per verificare che vengano ben interpretate le parentesi tra 'AND' e 'OR'.\\
Requisiti soddisfatti dall'esecuzione:
\begin{center}
\begin{tabular}{|p{5cm}|p{6cm}|} \hline
\textbf{Attributo \br} & \textbf{Valore} \\ \hline
Nome & BR \\ \hline
Commento & Commento\\ \hline
Business object associato & Articolo \\ \hline
Regola & uscite=5 AND uscite=5 OR uscite=5 AND ( uscite=5 OR uscite=5 AND ( uscite=5 OR (uscite=5 AND uscite=5 AND uscite=5 AND uscite=5) OR uscite=5) OR uscite=5 AND ( uscite=5 OR uscite=5) OR uscite=5) AND uscite=5 \\ \hline
\end{tabular} \\
\end{center}
\begin{center}
\begin{tabular}{|p{11cm}|} \hline
\textbf{Output atteso}\\ \hline
...\\
 \hline
\end{tabular} \\
\end{center}

\begin{Large}\textbf{VT2b}\end{Large} \\
Si testa una \br\ per verificare che vengano corretamente riportato il susseguirsi di 'AND' e 'OR'.\\
Requisiti soddisfatti dall'esecuzione:
\begin{center}
\begin{tabular}{|p{5cm}|p{6cm}|} \hline
\textbf{Attributo \br} & \textbf{Valore} \\ \hline
Nome & BR \\ \hline
Commento & Commento\\ \hline
Business object associato & Articolo \\ \hline
Regola & uscite=5 AND uscite=5 OR uscite=5 AND uscite=5 OR uscite=5 AND uscite=5 OR uscite=5 AND uscite=5 AND uscite=5 AND uscite=5 OR uscite=5 OR uscite=5 AND uscite=5 OR uscite=5 OR uscite=5 AND uscite=5 \\ \hline
\end{tabular} \\
\end{center}
\begin{center}
\begin{tabular}{|p{11cm}|} \hline
\textbf{Output atteso}\\ \hline
...\\
 \hline
\end{tabular} \\
\end{center}

\begin{Large}\textbf{VT2c}\end{Large} \\
Si testa una \br\ per verificare che possano essere assenti 'AND' e 'OR'.\\
Requisiti soddisfatti dall'esecuzione:
\begin{center}
\begin{tabular}{|p{5cm}|p{6cm}|} \hline
\textbf{Attributo \br} & \textbf{Valore} \\ \hline
Nome & BR \\ \hline
Commento & Commento\\ \hline
Business object associato & Articolo \\ \hline
Regola & entrate=5+2\\ \hline
\end{tabular} \\
\end{center}
\begin{center}
\begin{tabular}{|p{11cm}|} \hline
\textbf{Output atteso}\\ \hline
...\\
 \hline
\end{tabular} \\
\end{center}

\begin{Large}\textbf{VT3a}\end{Large} \\
Si testa una \br\ per verificare che il 'message' sia opzionale.\\
Requisiti soddisfatti dall'esecuzione:
\begin{center}
\begin{tabular}{|p{5cm}|p{6cm}|} \hline
\textbf{Attributo \br} & \textbf{Valore} \\ \hline
Nome & BR \\ \hline
Commento & Commento\\ \hline
Business object associato & Articolo \\ \hline
Regola & entrate=SUM(prezzoBase)\\ \hline
\end{tabular} \\
\end{center}
\begin{center}
\begin{tabular}{|p{11cm}|} \hline
\textbf{Output atteso}\\ \hline
...\\
 \hline
\end{tabular} \\
\end{center}

\begin{Large}\textbf{VT3b}\end{Large} \\
Si testa una \br\ per verificare che sia possibile inserire un 'message'.\\
Requisiti soddisfatti dall'esecuzione:
\begin{center}
\begin{tabular}{|p{5cm}|p{6cm}|} \hline
\textbf{Attributo \br} & \textbf{Valore} \\ \hline
Nome & BR \\ \hline
Commento & Commento\\ \hline
Business object associato & Articolo \\ \hline
Regola & entrate=AVG(prezzoBase)+2 message(``errore nel confronto'')\\ \hline
\end{tabular} \\
\end{center}
\begin{center}
\begin{tabular}{|p{11cm}|} \hline
\textbf{Output atteso}\\ \hline
...\\
 \hline
\end{tabular} \\
\end{center}

\begin{Large}\textbf{VT4}\end{Large} \\
Si testa una \br\ per verificare che sia possibile fare confronti del tipo '\textless', '\textgreater', '\textless =', '\textgreater =', '=', '!='.\\
Requisiti soddisfatti dall'esecuzione:
\begin{center}
\begin{tabular}{|p{5cm}|p{6cm}|} \hline
\textbf{Attributo \br} & \textbf{Valore} \\ \hline
Nome & BR \\ \hline
Commento & Commento\\ \hline
Business object associato & Articolo \\ \hline
Regola & uscite\textless2 AND entrate\textgreater 3 OR (uscite\textless=2 AND entrate\textgreater =3 OR (uscite=5 AND entrate!=2) )\\ \hline
\end{tabular} \\
\end{center}
\begin{center}
\begin{tabular}{|p{11cm}|} \hline
\textbf{Output atteso}\\ \hline
...\\
 \hline
\end{tabular} \\
\end{center}


\begin{Large}\textbf{VT5a}\end{Large} \\
Si testa una \br\ per verificare che sia possibile fare operazioni aritmetiche semplici tra scalari.\\
Requisiti soddisfatti dall'esecuzione:
\begin{center}
\begin{tabular}{|p{5cm}|p{6cm}|} \hline
\textbf{Attributo \br} & \textbf{Valore} \\ \hline
Nome & BR \\ \hline
Commento & Commento\\ \hline
Business object associato & Articolo \\ \hline
Regola & uscite=(2+2-(3*4))/5 \\ \hline
\end{tabular} \\
\end{center}
\begin{center}
\begin{tabular}{|p{11cm}|} \hline
\textbf{Output atteso}\\ \hline
...\\
 \hline
\end{tabular} \\
\end{center}

\begin{Large}\textbf{VT5b}\end{Large} \\
Si testa una \br\ per verificare che sia possibile fare operazioni aritmetiche semplici tra scalari e campi dati scalari.\\
Requisiti soddisfatti dall'esecuzione:
\begin{center}
\begin{tabular}{|p{5cm}|p{6cm}|} \hline
\textbf{Attributo \br} & \textbf{Valore} \\ \hline
Nome & BR \\ \hline
Commento & Commento\\ \hline
Business object associato & Articolo \\ \hline
Regola & uscite=(2+entrate-(uscite*4))/entrate \\ \hline
\end{tabular} \\
\end{center}
\begin{center}
\begin{tabular}{|p{11cm}|} \hline
\textbf{Output atteso}\\ \hline
...\\
 \hline
\end{tabular} \\
\end{center}

\begin{Large}\textbf{VT5c}\end{Large} \\
Si testa una \br\ per verificare che sia possibile fare operazioni aritmetiche semplici tra campi dati scalari.\\
Requisiti soddisfatti dall'esecuzione:
\begin{center}
\begin{tabular}{|p{5cm}|p{6cm}|} \hline
\textbf{Attributo \br} & \textbf{Valore} \\ \hline
Nome & BR \\ \hline
Commento & Commento\\ \hline
Business object associato & Articolo \\ \hline
Regola & uscite=(2+entrate-(uscite*4))/entrate \\ \hline
\end{tabular} \\
\end{center}
\begin{center}
\begin{tabular}{|p{11cm}|} \hline
\textbf{Output atteso}\\ \hline
...\\
 \hline
\end{tabular} \\
\end{center}

\begin{Large}\textbf{VT6a}\end{Large} \\
Si testa una \br\ per verificare che sia possibile fare la operazioni aritmetiche semplici tra scalare e vettore.\\
Requisiti soddisfatti dall'esecuzione:
\begin{center}
\begin{tabular}{|p{5cm}|p{6cm}|} \hline
\textbf{Attributo \br} & \textbf{Valore} \\ \hline
Nome & BR \\ \hline
Commento & Commento\\ \hline
Business object associato & Articolo \\ \hline
Regola & uscite=(2+prezzoBase) +(prezzoBase*4) +(prezzoBase/4) +(prezzoBase-3) \\ \hline
\end{tabular} \\
\end{center}
\begin{center}
\begin{tabular}{|p{11cm}|} \hline
\textbf{Output atteso}\\ \hline
...\\
 \hline
\end{tabular} \\
\end{center}

\begin{Large}\textbf{VT6b}\end{Large} \\
Si testa una \br\ per verificare che sia possibile fare la operazioni aritmetiche semplici tra campi dati scalari e vettori.\\
Requisiti soddisfatti dall'esecuzione:
\begin{center}
\begin{tabular}{|p{5cm}|p{6cm}|} \hline
\textbf{Attributo \br} & \textbf{Valore} \\ \hline
Nome & BR \\ \hline
Commento & Commento\\ \hline
Business object associato & Articolo \\ \hline
Regola & (entrate+prezzoBase) +(prezzoBase*uscite) +(prezzoBase/entrate) +(prezzoBase-uscite) \\ \hline
\end{tabular} \\
\end{center}
\begin{center}
\begin{tabular}{|p{11cm}|} \hline
\textbf{Output atteso}\\ \hline
...\\
 \hline
\end{tabular} \\
\end{center}

\begin{Large}\textbf{VT7}\end{Large} \\
Si testa una \br\ per verificare che sia possibile fare la operazioni aritmetiche semplici tra vettori omogenei.\\
Requisiti soddisfatti dall'esecuzione:
\begin{center}
\begin{tabular}{|p{5cm}|p{6cm}|} \hline
\textbf{Attributo \br} & \textbf{Valore} \\ \hline
Nome & BR \\ \hline
Commento & Commento\\ \hline
Business object associato & Articolo \\ \hline
Regola & (prezzoBase+prezzoBase) +(prezzoBase*prezzoBase) +(prezzoBase/prezzoBase) +(prezzoBase-prezzoBase) \\ \hline
\end{tabular} \\
\end{center}
\begin{center}
\begin{tabular}{|p{11cm}|} \hline
\textbf{Output atteso}\\ \hline
...\\
 \hline
\end{tabular} \\
\end{center}

\begin{Large}\textbf{VT8a}\end{Large} \\
Si testa una \br\ per verificare che sia possibile fare la operazioni aritmetiche semplici tra matrici omogenee.\\
Requisiti soddisfatti dall'esecuzione:
\begin{center}
\begin{tabular}{|p{5cm}|p{6cm}|} \hline
\textbf{Attributo \br} & \textbf{Valore} \\ \hline
Nome & BR \\ \hline
Commento & Commento\\ \hline
Business object associato & Articolo \\ \hline
Regola & (matricePrezzi+matricePrezzi) +(matricePrezzi*matricePrezzi) +(matricePrezzi/matricePrezzi) +(matricePrezzi-matricePrezzi) \\ \hline
\end{tabular} \\
\end{center}
\begin{center}
\begin{tabular}{|p{11cm}|} \hline
\textbf{Output atteso}\\ \hline
...\\
 \hline
\end{tabular} \\
\end{center}

\begin{Large}\textbf{VT8b}\end{Large} \\
Si testa una \br\ per verificare che sia possibile fare la operazioni aritmetiche semplici tra matrici e scalari.\\
Requisiti soddisfatti dall'esecuzione:
\begin{center}
\begin{tabular}{|p{5cm}|p{6cm}|} \hline
\textbf{Attributo \br} & \textbf{Valore} \\ \hline
Nome & BR \\ \hline
Commento & Commento\\ \hline
Business object associato & Articolo \\ \hline
Regola & (matricePrezzi+2) +(matricePrezzi*3) +(matricePrezzi/4) +(matricePrezzi-5) \\ \hline
\end{tabular} \\
\end{center}
\begin{center}
\begin{tabular}{|p{11cm}|} \hline
\textbf{Output atteso}\\ \hline
...\\
 \hline
\end{tabular} \\
\end{center}

\begin{Large}\textbf{VT8c}\end{Large} \\
Si testa una \br\ per verificare che sia possibile fare la operazioni aritmetiche semplici tra matrici e campi dati scalari.\\
Requisiti soddisfatti dall'esecuzione:
\begin{center}
\begin{tabular}{|p{5cm}|p{6cm}|} \hline
\textbf{Attributo \br} & \textbf{Valore} \\ \hline
Nome & BR \\ \hline
Commento & Commento\\ \hline
Business object associato & Articolo \\ \hline
Regola & (matricePrezzi+entrate) +(matricePrezzi*entrate) +(matricePrezzi/entrate) +(matricePrezzi-entrate) \\ \hline
\end{tabular} \\
\end{center}
\begin{center}
\begin{tabular}{|p{11cm}|} \hline
\textbf{Output atteso}\\ \hline
...\\
 \hline
\end{tabular} \\
\end{center}

\begin{Large}\textbf{VT9a}\end{Large} \\
Si testa una \br\ per verificare che sia impossibile fare la operazioni aritmetiche semplici tra matrici e vettori.\\
Requisiti soddisfatti dall'esecuzione:
\begin{center}
\begin{tabular}{|p{5cm}|p{6cm}|} \hline
\textbf{Attributo \br} & \textbf{Valore} \\ \hline
Nome & BR \\ \hline
Commento & Commento\\ \hline
Business object associato & Articolo \\ \hline
Regola & (matricePrezzi+prezzoBase) +(matricePrezzi*prezzoBase) +(matricePrezzi/prezzoBase) +(matricePrezzi-prezzoBase) \\ \hline
\end{tabular} \\
\end{center}
\begin{center}
\begin{tabular}{|p{11cm}|} \hline
\textbf{Output atteso}\\ \hline
...\\
 \hline
\end{tabular} \\
\end{center}

\begin{Large}\textbf{VT9b}\end{Large} \\
Si testa una \br\ per verificare che sia impossibile fare la operazioni aritmetiche semplici tra matrici non omogenee.\\
Requisiti soddisfatti dall'esecuzione:
\begin{center}
\begin{tabular}{|p{5cm}|p{6cm}|} \hline
\textbf{Attributo \br} & \textbf{Valore} \\ \hline
Nome & BR \\ \hline
Commento & Commento\\ \hline
Business object associato & Articolo \\ \hline
Regola & (matricePrezzi+grafico) +(matricePrezzi*grafico) +(matricePrezzi/grafico) +(matricePrezzi-grafico) \\ \hline
\end{tabular} \\
\end{center}
\begin{center}
\begin{tabular}{|p{11cm}|} \hline
\textbf{Output atteso}\\ \hline
...\\
 \hline
\end{tabular} \\
\end{center}

\begin{Large}\textbf{VT9c}\end{Large} \\
Si testa una \br\ per verificare che sia impossibile fare la operazioni aritmetiche semplici tra matrici, vettori e scalari di tipo diverso.\\
Requisiti soddisfatti dall'esecuzione:
\begin{center}
\begin{tabular}{|p{5cm}|p{6cm}|} \hline
\textbf{Attributo \br} & \textbf{Valore} \\ \hline
Nome & BR \\ \hline
Commento & Commento\\ \hline
Business object associato & Articolo \\ \hline
Regola & matricePrezzi=``ArticoloUno'' AND prezziScontati=3 OR uscite=true \\ \hline
\end{tabular} \\
\end{center}
\begin{center}
\begin{tabular}{|p{11cm}|} \hline
\textbf{Output atteso}\\ \hline
...\\
 \hline
\end{tabular} \\
\end{center}

\begin{Large}\textbf{VT10}\end{Large} \\
Si testa una \br\ per verificare che sia ben inteprtata la precedenza tra gli operatori.\\
Requisiti soddisfatti dall'esecuzione:
\begin{center}
\begin{tabular}{|p{5cm}|p{6cm}|} \hline
\textbf{Attributo \br} & \textbf{Valore} \\ \hline
Nome & BR \\ \hline
Commento & Commento\\ \hline
Business object associato & Articolo \\ \hline
Regola & uscite=2*3*4+4-3*5-5+3/2*3 \\ \hline
\end{tabular} \\
\end{center}
\begin{center}
\begin{tabular}{|p{11cm}|} \hline
\textbf{Output atteso}\\ \hline
...\\
 \hline
\end{tabular} \\
\end{center}

\begin{Large}\textbf{VT11}\end{Large} \\
Si testa una \br\ per verificare che siano ben intepretate le parentesi.\\
Requisiti soddisfatti dall'esecuzione:
\begin{center}
\begin{tabular}{|p{5cm}|p{6cm}|} \hline
\textbf{Attributo \br} & \textbf{Valore} \\ \hline
Nome & BR \\ \hline
Commento & Commento\\ \hline
Business object associato & Articolo \\ \hline
Regola & uscite=((((2*((3*4)+(4)-(3*5)-5)+3)/2)*3)+7) \\ \hline
\end{tabular} \\
\end{center}
\begin{center}
\begin{tabular}{|p{11cm}|} \hline
\textbf{Output atteso}\\ \hline
...\\
 \hline
\end{tabular} \\
\end{center}

\begin{Large}\textbf{VT12a}\end{Large} \\
Si testa una \br\ per verificare che sia impossibile effettuare operazioni tra le stringhe.\\
Requisiti soddisfatti dall'esecuzione:
\begin{center}
\begin{tabular}{|p{5cm}|p{6cm}|} \hline
\textbf{Attributo \br} & \textbf{Valore} \\ \hline
Nome & BR \\ \hline
Commento & Commento\\ \hline
Business object associato & Articolo \\ \hline
Regola & ``ArticoloUno''\textgreater``ArticoloUno'' AND ``ArticoloUno''\textgreater=``ArticoloUno'' AND ``ArticoloUno''\textless``ArticoloUno'' AND ``ArticoloUno''\textless=``ArticoloUno'' AND ``ArticoloUno''*nome  AND ``ArticoloUno''-nome  AND ``ArticoloUno''/nome  AND ``ArticoloUno''+nome\\ \hline
\end{tabular} \\
\end{center}
\begin{center}
\begin{tabular}{|p{11cm}|} \hline
\textbf{Output atteso}\\ \hline
...\\
 \hline
\end{tabular} \\
\end{center}

\begin{Large}\textbf{VT12b}\end{Large} \\
Si testa una \br\ per verificare che sia possibile effettuare confronti semplici ('=', '!=') tra le stringhe.\\
Requisiti soddisfatti dall'esecuzione:
\begin{center}
\begin{tabular}{|p{5cm}|p{6cm}|} \hline
\textbf{Attributo \br} & \textbf{Valore} \\ \hline
Nome & BR \\ \hline
Commento & Commento\\ \hline
Business object associato & Articolo \\ \hline
Regola & nome=``ArticoloUno'' AND nome!=``ArticoloDue''\\ \hline
\end{tabular} \\
\end{center}
\begin{center}
\begin{tabular}{|p{11cm}|} \hline
\textbf{Output atteso}\\ \hline
...\\
 \hline
\end{tabular} \\
\end{center}

\begin{Large}\textbf{VT13a}\end{Large} \\
Si testa una \br\ per verificare che sia impossibile effettuare operazioni aritmetiche tra booleani.\\
Requisiti soddisfatti dall'esecuzione:
\begin{center}
\begin{tabular}{|p{5cm}|p{6cm}|} \hline
\textbf{Attributo \br} & \textbf{Valore} \\ \hline
Nome & BR \\ \hline
Commento & Commento\\ \hline
Business object associato & Articolo \\ \hline
Regola & (scontato+3) - (scontato*4)/scontato\\ \hline
\end{tabular} \\
\end{center}
\begin{center}
\begin{tabular}{|p{11cm}|} \hline
\textbf{Output atteso}\\ \hline
...\\
 \hline
\end{tabular} \\
\end{center}

\begin{Large}\textbf{VT13b}\end{Large} \\
Si testa una \br\ per verificare che sia possibile effettuare operazioni logiche tra booleani.\\
Requisiti soddisfatti dall'esecuzione:
\begin{center}
\begin{tabular}{|p{5cm}|p{6cm}|} \hline
\textbf{Attributo \br} & \textbf{Valore} \\ \hline
Nome & BR \\ \hline
Commento & Commento\\ \hline
Business object associato & Articolo \\ \hline
Regola & ((scontato||scontato)\&\& scontato) = true AND (true||false\&\&NOT(true))=scontato\\ \hline
\end{tabular} \\
\end{center}
\begin{center}
\begin{tabular}{|p{11cm}|} \hline
\textbf{Output atteso}\\ \hline
...\\
 \hline
\end{tabular} \\
\end{center}

\begin{Large}\textbf{VT14}\end{Large} \\
Si testa una \br\ per verificare che sia possibile effettuare funzioni aritmetiche semplici come 'AVG', 'COUNT', 'SUM'.\\
Requisiti soddisfatti dall'esecuzione:
\begin{center}
\begin{tabular}{|p{5cm}|p{6cm}|} \hline
\textbf{Attributo \br} & \textbf{Valore} \\ \hline
Nome & BR \\ \hline
Commento & Commento\\ \hline
Business object associato & Articolo \\ \hline
Regola & SUM(prezzoBase) = 12 AND COUNT(entrate)=1, AND AVG(prezzoBase)=4\\ \hline
\end{tabular} \\
\end{center}
\begin{center}
\begin{tabular}{|p{11cm}|} \hline
\textbf{Output atteso}\\ \hline
...\\
 \hline
\end{tabular} \\
\end{center}

\begin{Large}\textbf{VT15a}\end{Large} \\
Si testa una \br\ per verificare che sia impossibile inserire regole prive di un operatore di confronto.\\
Requisiti soddisfatti dall'esecuzione:
\begin{center}
\begin{tabular}{|p{5cm}|p{6cm}|} \hline
\textbf{Attributo \br} & \textbf{Valore} \\ \hline
Nome & BR \\ \hline
Commento & Commento\\ \hline
Business object associato & Articolo \\ \hline
Regola & SUM(prezzoBase)\\ \hline
\end{tabular} \\
\end{center}
\begin{center}
\begin{tabular}{|p{11cm}|} \hline
\textbf{Output atteso}\\ \hline
...\\
 \hline
\end{tabular} \\
\end{center}

\begin{Large}\textbf{VT15b}\end{Large} \\
Si testa una \br\ per verificare che sia impossibile inserire regole prive di uno dei due elementi da confrontare\\
Requisiti soddisfatti dall'esecuzione:
\begin{center}
\begin{tabular}{|p{5cm}|p{6cm}|} \hline
\textbf{Attributo \br} & \textbf{Valore} \\ \hline
Nome & BR \\ \hline
Commento & Commento\\ \hline
Business object associato & Articolo \\ \hline
Regola & SUM(prezzoBase)=\\ \hline
\end{tabular} \\
\end{center}
\begin{center}
\begin{tabular}{|p{11cm}|} \hline
\textbf{Output atteso}\\ \hline
...\\
 \hline
\end{tabular} \\
\end{center}


\begin{Large}\textbf{VT15c}\end{Large} \\
Si testa una \br\ per verificare che sia impossibile inserire espressioni prive di operandi\\
Requisiti soddisfatti dall'esecuzione:
\begin{center}
\begin{tabular}{|p{5cm}|p{6cm}|} \hline
\textbf{Attributo \br} & \textbf{Valore} \\ \hline
Nome & BR \\ \hline
Commento & Commento\\ \hline
Business object associato & Articolo \\ \hline
Regola & 12 = 2entrate\\ \hline
\end{tabular} \\
\end{center}
\begin{center}
\begin{tabular}{|p{11cm}|} \hline
\textbf{Output atteso}\\ \hline
...\\
 \hline
\end{tabular} \\
\end{center}

\begin{Large}\textbf{VT15d}\end{Large} \\
Si testa una \br\ per verificare che sia impossibile inserire espressioni prive di operandi\\
Requisiti soddisfatti dall'esecuzione:
\begin{center}
\begin{tabular}{|p{5cm}|p{6cm}|} \hline
\textbf{Attributo \br} & \textbf{Valore} \\ \hline
Nome & BR \\ \hline
Commento & Commento\\ \hline
Business object associato & Articolo \\ \hline
Regola & uscite = 2 entrate\\ \hline
\end{tabular} \\
\end{center}
\begin{center}
\begin{tabular}{|p{11cm}|} \hline
\textbf{Output atteso}\\ \hline
...\\
 \hline
\end{tabular} \\
\end{center}

\begin{Large}\textbf{VT15e}\end{Large} \\
Si testa una \br\ per verificare che sia impossibile inserire regole non complete.\\
Requisiti soddisfatti dall'esecuzione:
\begin{center}
\begin{tabular}{|p{5cm}|p{6cm}|} \hline
\textbf{Attributo \br} & \textbf{Valore} \\ \hline
Nome & BR \\ \hline
Commento & Commento\\ \hline
Business object associato & Articolo \\ \hline
Regola & uscite = 2 AND\\ \hline
\end{tabular} \\
\end{center}
\begin{center}
\begin{tabular}{|p{11cm}|} \hline
\textbf{Output atteso}\\ \hline
...\\
 \hline
\end{tabular} \\
\end{center}

\begin{Large}\textbf{VT15f}\end{Large} \\
Si testa una \br\ per verificare che sia impossibile inserire parentesizzazioni errate.\\
Requisiti soddisfatti dall'esecuzione:
\begin{center}
\begin{tabular}{|p{5cm}|p{6cm}|} \hline
\textbf{Attributo \br} & \textbf{Valore} \\ \hline
Nome & BR \\ \hline
Commento & Commento\\ \hline
Business object associato & Articolo \\ \hline
Regola & ((2+4)=uscite\\ \hline
\end{tabular} \\
\end{center}
\begin{center}
\begin{tabular}{|p{11cm}|} \hline
\textbf{Output atteso}\\ \hline
...\\
 \hline
\end{tabular} \\
\end{center}

\begin{Large}\textbf{VT16}\end{Large} \\
Si testa una \br\ per verificare che siano supportato l'utilizzo dei numeri negativi.\\
Requisiti soddisfatti dall'esecuzione:
\begin{center}
\begin{tabular}{|p{5cm}|p{6cm}|} \hline
\textbf{Attributo \br} & \textbf{Valore} \\ \hline
Nome & BR \\ \hline
Commento & Commento\\ \hline
Business object associato & Articolo \\ \hline
Regola & uscite=((((~2*((~3*~4)+(~4)-(~3*~5)-5)+3)/~2)*~3)+7) \\ \hline
\end{tabular} \\
\end{center}
\begin{center}
\begin{tabular}{|p{11cm}|} \hline
\textbf{Output atteso}\\ \hline
...\\
 \hline
\end{tabular} \\
\end{center}

\begin{Large}\textbf{VT17}\end{Large} \\
Si testa una \br\ per verificare che siano ben tradotti tutti i caratteri non ammessi in un documento xml.\\
Requisiti soddisfatti dall'esecuzione:
\begin{center}
\begin{tabular}{|p{5cm}|p{6cm}|} \hline
\textbf{Attributo \br} & \textbf{Valore} \\ \hline
Nome & \`a \`e \`i \`o \`u \`A \`E \`I \`O \`U ``test'' , 'test', \& , \textgreater , \textless \\ \hline
Commento & \`a \`e \`i \`o \`u \`A \`E \`I \`O \`U ``test'' , 'test', \& , \textgreater , \textless \\ \hline
Business object associato & Articolo \\ \hline
Regola & scontato=true \\ \hline
\end{tabular} \\
\end{center}
\begin{center}
\begin{tabular}{|p{11cm}|} \hline
\textbf{Output atteso}\\ \hline
...\\
 \hline
\end{tabular} \\
\end{center}

\begin{Large}\textbf{VT18}\end{Large} \\
Si testa una \br\ per verificare che siano ammessi solo alcuni caratteri nel message.\\
Requisiti soddisfatti dall'esecuzione:
\begin{center}
\begin{tabular}{|p{5cm}|p{6cm}|} \hline
\textbf{Attributo \br} & \textbf{Valore} \\ \hline
Nome & BR \\ \hline
Commento & Commento \\ \hline
Business object associato & Articolo \\ \hline
Regola & scontato=true message(``abcABC .?!^_-~'') \\ \hline
\end{tabular} \\
\end{center}
\begin{center}
\begin{tabular}{|p{11cm}|} \hline
\textbf{Output atteso}\\ \hline
...\\
 \hline
\end{tabular} \\
\end{center}











\subsection{Test sul comunicatore}
La macrocomponente comunicatore verr\`a testata a scatola aperta ( whitebox ), inserendo in input al comunicatore un stringa contenente una query \underline{XPath} e confrontando l'output del comunicatore con l'output atteso.\\
\\
\begin{Large}\textbf{CT1}\end{Large} \\
Requisiti soddisfatti dall'esecuzione: F1 etc..
\begin{center}
\begin{tabular}{|p{11cm}|} \hline
\textbf{Query eseguita}\\ \hline
Business object associato.\\ \hline
\end{tabular} \\
\end{center}
\begin{center}
\begin{tabular}{|p{11cm}|} \hline
\textbf{Output atteso}\\ \hline
ewerafdjkofsdlkfjsdl;fjdlsjfdkfjdkfj\\
sdfdffd\\
 \hline
\end{tabular} \\
\end{center}

\section{Test di integrazione}
Vengono di seguito riportati i test riguardanti le integrazioni delle varie componenti. Questi test hanno come precondizione l'esecuzione dei test sulle singole macrocomponenti, di conseguenza sono mirati a scoprire difetti nell'interfacciamento e nell'interazione tra diverse componenti, nonch\`e di mostrare il soddisfacimento di alcuni requisiti.

\subsection{Integrazione validatore, comunicatore}
L'integrazione tra 'validatore' e 'comunicatore' verr\`a testata a scatola aperta ( whitebox ), inserendo in input al validatore un oggetto \textit{\br\ } con gli attributi di seguito specificati e verificando che l'output sia quello atteso. Essendo gi\`a state testate separatamente le 2 componenti ci si limiter\`a a testare le sole funzionalit\`a risultanti dalla loro integrazione.\\
\subsection{Integrazione gui, validatore}
L'integrazione tra 'gui' e 'validatore' verr\`a testata a scatola aperta ( whitebox ), inserendo da gui  un oggetto \textit{\br\ } con gli attributi di seguito specificati e verificando che l'output sia quello atteso. Essendo gi\`a state testate separatamente le 2 componenti ci si limiter\`a a testare le sole funzionalit\`a risultanti dalla loro integrazione.\\

\subsection{Integrazione del sistema}
L'integrazionedell'intero sistema verr\`a testata a scatola aperta ( whitebox ), testando le sole funzionalit\`a risultanti dalla completa integrazione del sistema.\\

\section{Test di regressione}
Con test di regressione si intende la riescuzione di alcuni test case a fronte di una o pi\`u modifiche nel codice software. Ogni qual volta una componente venga modificata vanno rieseguiti (mediante l'utilizzo del sistema di automazione dei test) tutti i test riguardanti la componente specifica nonch\`e i test riguardanti l'integrazione della componente sessa con altre componenti fino all'esecuzione dei test di integrazione del sistema.
Negli esiti dei test, per non appesantire la lettura non vengono riportatitutti i test di regressione ma solo i test di regressione che abbiano prodotto un output o un esito differente.

\section{Tracciamento requisiti - test}
Nella seguente tabella vengono evidenziati i test necessari al soddisfacimento di ogni requisito. Gli identificativi dei test sono quelli riportati nel capitolo 'Descrizione dei test case'{link}. Mentre gli identificativi dei singoli requisiti sono quelli descritti nel documento \AR .
 \begin{center}
\begin{tabular}{|p{3cm}|p{8cm}|} \hline
\textbf{Requisito} & \textbf{Test che lo soddisfano}\\ \hline
F1 & CT1, ED2 \\ \hline
\end{tabular} \\
\end{center}



\chapter{Esiti dei test}
\section{Automatizzazione dei test}
Essendo la fase di test una fase costosa e laboriosa del processo software, si \`e deciso di automatizzare il pi\`e possibile l'esecuzione dei test. L'esecuzione dei test viene automatizzata mediante del semplice codice che si limita a invocare metodi e funzionalit\`a di una componente (o di alcune componenti integrate fra loro). Tale codice inizialmente si occupa di passare ai metodi delle componenti da testare degli opportuni parametri (alcuni errati ed altri corretti secondo gli input dei test case), confrontando poi l'output prodotto con quello atteso.
Alcuni test si rende necessario eseguirli a mano, come ad esempio i test riguardanti la componente \textit{gui}, tali test sono comunque una parte minima di tutti i test da eseguire non incidendo eccessivamente sul costo dell'intera fase di test.
\section{Esiti dei test dei difetti}
I \textit{test dei difetti} servono ad individuare possibili difetti o errori del codice su cui vengono testati. Per questo l'esito di un test dei difetti \`e segnato come negativo ogni qual volta non evidenzia un difetto, viene altres\`i segnato come positivo l'esito di quei test che abbiano individuato dei difetti nel software.
Vengono di seguito elencati gli esiti dei test effetuati. I test  sono presentati in forma tabellare, indicando il codice identificativo del test, seguito dall'esito del test. I test di regressione come illustrato precedentemente sono riportati solo dove abbiano prodotto un output o un esito differente dalla loro precedente esecuzione. I test di regressione sono evidenziati nelle tabelle sottostanti dalla sigla RT.


\subsection{Test sul validatore}
\begin{center}
\begin{tabular}{|p{2cm}|p{4cm}|p{5cm}|} \hline
\textbf{Id} & \textbf{Esito} & \textbf{Note} \\ \hline
VT1 & fallito & errore del picchio\\ \hline
\end{tabular} \\
\end{center}

\subsection{Test sul comunicatore}
\subsection{Integrazione validatore, comunicatore}
\subsection{Integrazione gui, validatore}
\subsection{Integrazione del sistema}

\chapter{Esiti dei test di convalida}
I \textit{test di convalida} servono ad dimostrare che tutti i requisiti sono stati soddisfatti. Per questo l'esito di un test di convalida \`e segnato come positivo ogni qual volta dimostra che un requisito \`e soddisfatto, viene altres\`i segnato come negativo l'esito di quei test che hanno fallito.
I test di convalida sono rimandati alla fase di verifica e validazione.
\end{document}
