
\documentclass[11pt,titlepage,a4paper]{report}

\usepackage[italian]{babel}
\usepackage{fancyhdr}
\usepackage{graphicx}
\usepackage{hyperref}

%\usepackage{lastpage} % total page count

\usepackage{color}
\usepackage{lastpage} % total page count

\graphicspath{{./pics/}} % cartella di salvataggio immagini

\pagestyle{fancy}
\renewcommand{\sectionmark}[1]{\markright{\thesection.\ #1}}
\lhead{\nouppercase{\rightmark}}
\rhead{\nouppercase{\leftmark}}
\renewcommand{\chaptermark}[1]{%
\markboth{\thechapter.\ #1}{}}


\fancypagestyle{plain}{%
	\lhead{\includegraphics[height=50pt]{logo.eps}}
	\chead{}
	\rhead{HappyCode inc \\ happycodeinc@gmail.com}
	\lfoot{BR-jsys}
	\cfoot{\thepage\ / \pageref{LastPage}}
	\rfoot{\dt - \lv}
	\renewcommand{\headrulewidth}{1pt}
	\renewcommand{\footrulewidth}{1pt}
}
	\lhead{\includegraphics[height=50pt]{logo.eps}}
	\chead{}
	\rhead{HappyCode inc \\ happycodeinc@gmail.com}
	\lfoot{BR-jsys}
	\cfoot{\thepage\ / \pageref{LastPage}}
	\rfoot{\dt - \lv}
	\renewcommand{\headrulewidth}{1pt}
	\renewcommand{\footrulewidth}{1pt}

\hypersetup{
    colorlinks=true,       % false: boxed links; true: colored links
   linkcolor=[rgb]{0.11,0.55,0.83},          % color of internal links
    urlcolor=cyan           % color of external links
}
\definecolor{err}{rgb}{0.9,0.1,0.1}

% fine layout% layout
\begin{document}
%definizione variabili 
\newcommand{\lv}{ 0.1 } % latest version
\newcommand{\dt}{ Test Report }% Document title
%common variables
\newcommand{\br}{\underline{business rule}}
\newcommand{\brs}{\underline{business rules}}
\newcommand{\bo}{\underline{business object}}
\newcommand{\bos}{\underline{business objects}}
\newcommand{\rp}{\underline{repository}}
\newcommand{\brp}{BusinessRuleParser}
\newcommand{\brl}{BusinessRuleLexer}
\newcommand{\BR}{\underline{BusinessRule}}

%nomi dei componenti
\newcommand{\AT}{Alessia Trivellato}
\newcommand{\ET}{Elena Trivellato}
\newcommand{\FC}{Filippo Carraro}
\newcommand{\LA}{Luca Appon}
\newcommand{\MB}{Michele Bortolato}
\newcommand{\MT}{Marco Tessarotto}
\newcommand{\MM}{Mattia Meroi}%altre variabili
% ultime versioni dei documenti da modificare solo alla fine
\newcommand{\AR}{AnalisiDeiRequisiti.2.6.pdf}
\newcommand{\DdP}{DefinizioneDiProdotto.0.9.pdf}
\newcommand{\G}{ Glossario.1.8.pdf }
\newcommand{\NdP}{NormeDiProgetto.2.0.pdf}
\newcommand{\PdQ}{ PianoDiQualifica.1.4.pdf }
\newcommand{\PdP}{ PianoDiProgetto.1.7.pdf }
\newcommand{\ST}{SpecificaTecnica.1.5.pdf}
\newcommand{\TR}{TestReport.0.7.pdf}
\newcommand{\MU}{ManualeUtente.0.3.pdf}%nomi documenti
%fine definizione variabili
\hyphenation{
 a-na-lo-go
 as-so-cia-zio-ne
 %attività non si può inserire come tutte le parole accentate che vanno messe nel testo semplice scritte at\-ti\-vi\-tà o come variabile
 coe-ren-za
 com-po-nen-ti
 con-si-glia-bi-le
 des-crit-te
 des-cri-zio-ni
 di-a-gram-ma
 di-a-gram-mi
 e-le-men-to
 e-se-gui-re
 e-si-sten-ti
 es-pli-ci-to
 glo-bal-men-te
 glos-sa-rio
 in-se-ri-men-to
 li-vel-lo
 ne-ces-sa-rio
 per-met-te-re
 re-po-si-to-ry
 re-vi-sio-na-men-to
 ri-chies-te
 se-le-zio-na-ta
 se-gna-la-ta
 va-li-da-zio-ne
 va-ria-bi-li
 ve-ri-fi-ca-re
 vi-sua-liz-za-te
 e-ven-tua-li
 o-pe-ra-zio-ne
 ar-chi-via-zio-ne
 mo-di-fi-ca
 ar-chi-vio
 des-cri-zio-ne
 pa-ren-te-si
 i-ni-zia
}


%sillabazione

\begin{titlepage}\begin{center}
\vspace*{0.5in}
\includegraphics{logo.eps}
\vspace*{0.2in} \\
{\Large \textbf{BR-jsys}}
{\Large \emph{business rules} per sistemi gestionali in architettura J2EE } 
\vspace{2in} \\
\Huge \textsc{ \dt }
\par\rule{10cm}{0.4pt} \par {\large Versione \lv - \today} \\
\end{center}\end{titlepage}

\vspace*{0.5in}%pagina del titolo


\begin{center}
\thispagestyle{plain}
\begin{table}[htbp]
\large{
\begin{tabular}{l}
\Large{\textbf{\textsf{Capitolato: ''BR-jsys``}}} \\
\begin{tabular}{|p{6cm}|p{6cm}|}
\hline
\textbf{Data creazione:} & 05/03/2008 \\ \hline
\textbf{Versione:} & \lv \\ \hline
\textbf{Stato del documento:} & Formale, esterno \\ \hline
% ----------------------------------------------------------------------------autori
\textbf{Revisione RQ} & \\ \hline
\textbf{Redazione:} & \MT \\ \hline
\textbf{Revisione:} &  \\ \hline
\textbf{Approvazione:}  &  \\ \hline
\end{tabular} \\
\end{tabular}
}
\end{table}

\begin{table}[hbtp]
\large{
\begin{tabular}{l}
\Large{\textbf{\textsf{Lista di distribuzione}}} \\
\begin{tabular}{|p{6cm}|p{6cm}|} \hline
%  -------------------------------------------------------------lista di distribuzione
{HappyCode inc}& Gruppo di lavoro \\ \hline
{Tullio Vardanega, Renato Conte}& Committenti \\ \hline 
{Zucchetti S.r.l}& Azienda proponente\\ \hline
\end{tabular} \\
\end{tabular}
}
\end{table}
\begin{table}[hbtp]

\Large{\textbf{\textsf{Diario delle modifiche}}} \\
\begin{small}
\begin{tabular}[t]{|p{1,2cm}|p{1.9cm}|p{2.9cm}|p{5cm}|} \hline
Versione & Data & Autore & Descrizione \\ \hline
%-------------------------------------------------------------------------------diario modifiche
0.1 & 05/03/2008 & \MT & Stesura iniziale del documento.\\ \hline

\end{tabular} \\
\end{small}


\end{table}
\end{center}
\newpage
\tableofcontents

\chapter{Introduzione}

\section{Scopo del documento}
Il presente documento viene redatto al fine di illustrare la progettazione dei \textit{test case} nel dettaglio indicando un insieme di input e il corrispondente insieme di \textit{output attesi}. Verranno inoltre riportati nel presente documento gli esiti dei test eseguiti sul sistema ``Br-jsys'' al fine di non appesantire la lettura del documento \PdQ .

\section{Definizioni, acronimi, abbreviazioni}
Nella tabella di seguito vengono riportate tutte le abbreviazioni utlizzate nel documento, comprese le sigle utilizzate per l'identificazione dei test.
\begin{center}
\begin{tabular}{||p{3.0cm}||p{8.5cm}||} \hline
\textbf{Abbreviazione} & \textbf{Significato} \\ \hline

VT\textit{\#} & Con VT seguito da un numero intero sono indicati i test che riguardano la ``macrocomponente validatore''.\\ \hline
CT\textit{\#} & Con CT seguito da un numero intero sono indicati i test che riguardano la ``macrocomponente comunicatore''.\\ \hline

\end{tabular} \\
\end{center}
\subsection{Glossario}
Per gli altri termini non citati nella tabella soprastante si fa riferimento al file esterno \G.

\chapter{Descrizione dei test case}
Vengono di seguito presentati i \textit{test case}. Ogni test case \`e composto da un input, un output atteso. Ogni test case \`e inoltre accompagnato da un breve testo che identifica lo scopo del test, e dall'elenco dei requisiti soddisfatti dal test.

\section{Test sul validatore}
La macrocomponente validatore 
\begin{Large}\textbf{VT1}\end{Large}
\begin{center}
\begin{tabular}{|p{5cm}|p{6cm}|} \hline
\textbf{Attributo \br} & \textbf{Valore} \\ \hline
Nome & Con dfgdfgfg.\\ \hline
Commento & dfgdfgfgf.\\ \hline
Business object associato & dfgdfgfgf.\\ \hline
Regola & dfgdfgfgf.\\ \hline
\end{tabular} \\
\end{center}
\begin{center}
\begin{tabular}{|p{11cm}|} \hline
\textbf{Output atteso}\\ \hline
ewerafdjkofsdlkfjsdl;fjdlsjfdkfjdkfj\\
sdfdffd\\
 \hline
\end{tabular} \\
\end{center}

\section{Test sul comunicatore}

\begin{Large}\textbf{CT1}\end{Large}
\begin{center}
\begin{tabular}{|p{11cm}|} \hline
\textbf{Query eseguita}\\ \hline
Business object associato & dfgdfgfgf.\\ \hline
\end{tabular} \\
\end{center}
\begin{center}
\begin{tabular}{|p{11cm}|} \hline
\textbf{Output atteso}\\ \hline
ewerafdjkofsdlkfjsdl;fjdlsjfdkfjdkfj\\
sdfdffd\\
 \hline
\end{tabular} \\
\end{center}

\chapter{Esiti dei test}
\section{Automatizzazione dei test}
\section{Test sul validatore}

\section{Test sul comunicatore}
\end{document}
