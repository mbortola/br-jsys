
\documentclass[11pt,titlepage,a4paper]{report}

\usepackage[italian]{babel}
\usepackage{fancyhdr}
\usepackage{graphicx}
\usepackage{hyperref}

%\usepackage{lastpage} % total page count

\usepackage{color}
\usepackage{lastpage} % total page count

\graphicspath{{./pics/}} % cartella di salvataggio immagini

\pagestyle{fancy}
\renewcommand{\sectionmark}[1]{\markright{\thesection.\ #1}}
\lhead{\nouppercase{\rightmark}}
\rhead{\nouppercase{\leftmark}}
\renewcommand{\chaptermark}[1]{%
\markboth{\thechapter.\ #1}{}}


\fancypagestyle{plain}{%
	\lhead{\includegraphics[height=50pt]{logo.eps}}
	\chead{}
	\rhead{HappyCode inc \\ happycodeinc@gmail.com}
	\lfoot{BR-jsys}
	\cfoot{\thepage\ / \pageref{LastPage}}
	\rfoot{\dt - \lv}
	\renewcommand{\headrulewidth}{1pt}
	\renewcommand{\footrulewidth}{1pt}
}
	\lhead{\includegraphics[height=50pt]{logo.eps}}
	\chead{}
	\rhead{HappyCode inc \\ happycodeinc@gmail.com}
	\lfoot{BR-jsys}
	\cfoot{\thepage\ / \pageref{LastPage}}
	\rfoot{\dt - \lv}
	\renewcommand{\headrulewidth}{1pt}
	\renewcommand{\footrulewidth}{1pt}

\hypersetup{
    colorlinks=true,       % false: boxed links; true: colored links
   linkcolor=[rgb]{0.11,0.55,0.83},          % color of internal links
    urlcolor=cyan           % color of external links
}
\definecolor{err}{rgb}{0.9,0.1,0.1}

% fine layout% layout
\begin{document}
%definizione variabili 
\newcommand{\lv}{ 0.1 } % latest version
\newcommand{\dt}{ Piano Consuntivo }% Document title
\newcommand{\Glossario}{ Glossario.1.8.pdf }
%fine definizione variabili
%common variables
\newcommand{\br}{\underline{business rule}}
\newcommand{\brs}{\underline{business rules}}
\newcommand{\bo}{\underline{business object}}
\newcommand{\bos}{\underline{business objects}}
\newcommand{\rp}{\underline{repository}}
\newcommand{\brp}{BusinessRuleParser}
\newcommand{\brl}{BusinessRuleLexer}
\newcommand{\BR}{\underline{BusinessRule}}

%nomi dei componenti
\newcommand{\AT}{Alessia Trivellato}
\newcommand{\ET}{Elena Trivellato}
\newcommand{\FC}{Filippo Carraro}
\newcommand{\LA}{Luca Appon}
\newcommand{\MB}{Michele Bortolato}
\newcommand{\MT}{Marco Tessarotto}
\newcommand{\MM}{Mattia Meroi}%altre variabili
\hyphenation{
 a-na-lo-go
 as-so-cia-zio-ne
 %attività non si può inserire come tutte le parole accentate che vanno messe nel testo semplice scritte at\-ti\-vi\-tà o come variabile
 coe-ren-za
 com-po-nen-ti
 con-si-glia-bi-le
 des-crit-te
 des-cri-zio-ni
 di-a-gram-ma
 di-a-gram-mi
 e-le-men-to
 e-se-gui-re
 e-si-sten-ti
 es-pli-ci-to
 glo-bal-men-te
 glos-sa-rio
 in-se-ri-men-to
 li-vel-lo
 ne-ces-sa-rio
 per-met-te-re
 re-po-si-to-ry
 re-vi-sio-na-men-to
 ri-chies-te
 se-le-zio-na-ta
 se-gna-la-ta
 va-li-da-zio-ne
 va-ria-bi-li
 ve-ri-fi-ca-re
 vi-sua-liz-za-te
 e-ven-tua-li
 o-pe-ra-zio-ne
 ar-chi-via-zio-ne
 mo-di-fi-ca
 ar-chi-vio
 des-cri-zio-ne
 pa-ren-te-si
 i-ni-zia
}


%sillabazione

\begin{titlepage}\begin{center}
\vspace*{0.5in}
\includegraphics{logo.eps}
\vspace*{0.2in} \\
{\Large \textbf{BR-jsys}}
{\Large \emph{business rules} per sistemi gestionali in architettura J2EE } 
\vspace{2in} \\
\Huge \textsc{ \dt }
\par\rule{10cm}{0.4pt} \par {\large Versione \lv - \today} \\
\end{center}\end{titlepage}

\vspace*{0.5in}%pagina del titolo


\begin{center}
\thispagestyle{plain}
\begin{table}[htbp]
\large{
\begin{tabular}{l}
\Large{\textbf{\textsf{Capitolato: ''BR-jsys``}}} \\
\begin{tabular}{|p{6cm}|p{6cm}|} \hline
\textbf{Data creazione:} & 2007/11/21 \\ \hline
\textbf{Versione:} & \lv \\ \hline
% ----------------------------------------------------------------------------autori
\textbf{Stato del documento:} & formale, esterno \\ \hline
\textbf{Revisione RA} &  \\ \hline
\textbf{Redazione:} & \ET \\ \hline
\textbf{Revisione:} & \MT \\ \hline
\textbf{Approvazione:} & \AT \\ \hline
\end{tabular} \\
\end{tabular}
}
\end{table}

\begin{table}[hbtp]
\large{
\begin{tabular}{l}
\Large{\textbf{\textsf{Lista di distribuzione}}} \\
\begin{tabular}{|p{6cm}|p{6cm}|} \hline
%  -------------------------------------------------------------lista di distribuzione
{HappyCode inc}& Gruppo di lavoro\\ \hline
{Tullio Vardanega, Renato Conte}& Committenti \\ \hline 
{Zucchetti S.r.l}& Azienda proponente\\ \hline
\end{tabular} \\
\end{tabular}
}
\end{table}
\begin{table}[hbtp]

\Large{\textbf{\textsf{Diario delle modifiche}}} \\
\begin{small}
\begin{tabular}[t]{|p{1,2cm}|p{1.9cm}|p{2.9cm}|p{5cm}|} \hline
Versione & Data & Autore & Descrizione \\ \hline

0.1 & 2008/15/03 & \ET & Stesura documento\\ \hline
\end{tabular} \\
\end{small}


\end{table}
\end{center}

\tableofcontents 
\chapter{Ruoli di progetto}
\section{Visione globale}
Riportiamo in questa tabella il numero di ore dedicate complessivamente dai membri della HappyCode per ciascun ruolo di progetto, indicando anche gli scostamenti dalle previsioni.

\begin{table}[hbtp]
\large{
\begin{tabular}{l}
\Large{\textbf{\textsf{Tabella dei Ruoli}}} \\
\begin{tabular}{||p{3cm}||p{1.8cm}||p{1.8cm}||p{1.8cm}||p{1.8cm}||}
\hline 
\textbf{Ruoli} & \textbf{Analisi} & \textbf{Progett.} & \textbf{Sviluppo} & \textbf{Verifica}\\
\hline

{Responsabile}&10&10&7\footnotesize{(-3)} &9 \\ 
\hline 
{Amministratore} &10&10&8\footnotesize{(-2)}&8\footnotesize{(-2)}\\ 
\hline
{Analista}& 60 \footnotesize{(-2)}&20&5&0 \\
\hline
{Progettista}&29 \footnotesize{(+24)}&6 \footnotesize{(-15)}&25&0 \\
\hline
{Programmatore}&0&10 \footnotesize{(+5)}&59\footnotesize{(-1)}&48\footnotesize{(-7)}\\
\hline
{Verificatore}& 30&30&85\footnotesize{(-5)}&95 \\
\hline
{Totale}& 137 \footnotesize{(+22)}&140 \footnotesize{(-10)}&189\footnotesize{(-11)}&160\footnotesize{(-9)} \\
\hline
\end{tabular} \\

\end{tabular}
}

\end{table}
\newpage 

\\
\section{Visione dettagliata della fase di verifica}
Le due tabelle che seguono mostrano i ruoli assunti da ciascun membro del gruppo nella fase di verifica. I numeri tra parentesi indicano gli scostamenti dalle previsioni.
\begin{table}[hbtp]
\large{
\begin{tabular}{l}
\Large{\textbf{\textsf{fase di verifica - 1}}} \\
\begin{tabular}{||p{3.5cm}||p{2cm}||p{2cm}||p{2cm}||p{2cm}||} \hline
\textbf{Membro} & \textbf{Respon.} & \textbf{Ammin.} & \textbf{Analista} & \textbf{Progett.}\\ \hline
{Appon Luca}&2&0&0&0 \\ \hline 
{Bortolato Michele} &0&4\footnotesize{(-1)}&0&0\\ \hline
{Carraro Filippo}&0&0&0&0 \\ \hline
{Meroi Mattia}&0&0&0&0\\ \hline
{Tessarotto Marco} &0&0&0&0\\ \hline
{Trivellato Alessia} &5&0&0&0 \\ \hline
{Trivellato Elena} &2&4\footnotesize{(-1)}&0&0 \\ \hline
{Totale}& 9&8\footnotesize{(-2)}&0&0 \\ \hline
\end{tabular} \\
\end{tabular}
}
\end{table}

\begin{table}[hbtp]
\large{
\begin{tabular}{l}
\Large{\textbf{\textsf{fase di verifica - 2}}} \\
\begin{tabular}{||p{3.5cm}||p{2cm}||p{2cm}||p{2cm}||p{2cm}||} \hline
\textbf{Membro} & \textbf{Program} & \textbf{Verif.} & \textbf{Totale}\\ \hline
{Appon Luca}&5\footnotesize{(-1)}&16&23\footnotesize{(-1)} \\ \hline
{Bortolato Michele} &3&17&24\footnotesize{(-1)}\\ \hline
{Carraro Filippo}&10&14&24 \\ \hline
{Meroi Mattia}&2\footnotesize{(-5)&17&19\footnotesize{(-5)}\\ \hline
{Tessarotto Marco} &7&17&24\\ \hline
{Trivellato Alessia} &10\footnotesize{(-1)}&8&23\footnotesize{(-1)} \\ \hline
{Trivellato Elena} &11&6&24\footnotesize{(-1)} \\ \hline
{Totale} &48\footnotesize{(-7)}&95&160\footnotesize{(-9)} \\ \hline
\end{tabular} \\
\end{tabular}
}
\end{table}

\newpage
\section{Carico totale delle risorse}
Vediamo come lungo tutto il ciclo di vita del nostro prodotto i componenti del gruppo hanno contribuito alla realizzazione dello stesso.
\begin{table}[hbtp]
\large{
\begin{tabular}{l}
\Large{\textbf{\textsf{Carico totale delle risorse - 1}} \\

\begin{tabular}{||p{3.5cm}||p{2cm}||p{2cm}||p{2cm}||p{2cm}||}
\hline
\textbf{Membro} & \textbf{Respon.} & \textbf{Ammin.} & \textbf{Analista}
& \textbf{Progett.}\\
\hline
{Appon Luca}&6&5&12 \footnotesize{(-1)}&16 \footnotesize{(+1)} \\ 
\hline 
{Bortolato Michele} &6&6\footnotesize{(-1)}&11&18 \footnotesize{(+3)}\\ 
\hline
{Carraro Filippo}&3\footnotesize{(-2)}&5&12&19 \footnotesize{(+3)} \\
\hline
{Meroi Mattia}&6&5&12 \footnotesize{(-1)}&14\\
\hline
{Tessarotto Marco} &2\footnotesize{(-3)}&2\footnotesize{(-4)}&13&18 \footnotesize{(+2)}\\
\hline
{Trivellato Alessia} &7\footnotesize{(+2)}&9\footnotesize{(+2)}&11&15\footnotesize{(-3)} \\
\hline
{Trivellato Elena} &6&4\footnotesize{(-1)}&12&14\footnotesize{(-2)} \\
\hline
{Totale}& 36\footnotesize{(-3)}&34\footnotesize{(-4)}&83\footnotesize{(-2)}&114 \footnotesize{(+4)} \\
\hline



\end{tabular} \\
\end{tabular}
}
\end{table}

\begin{table}[hbtp]
\large{
\begin{tabular}{l}
\Large{\textbf{\textsf{Carico totale delle risorse - 2}}} \\

\begin{tabular}{||p{3.5cm}||p{2cm}||p{2cm}||p{2cm}||p{2cm}||}
\hline
\textbf{Membro} & \textbf{Program} & \textbf{Verif.} & \textbf{Totale}\\
\hline
{Appon Luca}&15\footnotesize{(-1)}&33\footnotesize{(-3)}&88\footnotesize{(-4)} \\ 
\hline 
{Bortolato Michele} &17 \footnotesize{(+2)}&37\footnotesize{(+1)}&95 \footnotesize{(+5)}\\ 
\hline
{Carraro Filippo}&20 \footnotesize{(+2)}&32\footnotesize{(-2)}&91\footnotesize{(+1)} \\
\hline
{Meroi Mattia}&14\footnotesize{(-5)}&32\footnotesize{(-1)}&83\footnotesize{(-7)}\\
\hline
{Tessarotto Marco} &15 \footnotesize{(-1)}&42\footnotesize{(+7)}&92 \footnotesize{(+1)}\\
\hline
{Trivellato Alessia} &18&29\footnotesize{(-6)}&89 \footnotesize{(-2)} \\
\hline
{Trivellato Elena} &18&35\footnotesize{(-1)}&89 \footnotesize{(-2)} \\
\hline
{Totale} &117 \footnotesize{(-3)}&240\footnotesize{(-5)}&626 \footnotesize{(-8)} \\
\hline

\end{tabular} \\
\end{tabular}
}
\end{table}

\chapter{Costi}
Vediamo nella tabella che segue come i vari ruoli di progetto hanno contribuito alla formazione del costo totale del nostro prodotto. Come nelle tabelle precedenti, tra parentesi riportiamo lo scostamento rispetto alle previsioni iniziali.\\

\begin{tabular}{l}
\Large{\textbf{\textsf{costi totali}}} \\
\begin{tabular}{||p{4cm}||p{4cm}||p{4cm}||}
\hline
\textbf{Ruoli} & \textbf{Costo Totale}& \textbf{Costo Percentuale}\\
\hline
{Responsabile}&1.080,00\footnotesize{(-90,00)}&9 \footnotesize{(-1)}\\ 
\hline 
{Amministratore} &648,00\footnotesize{(-72,00)}&6\\ 
\hline
{Analista} &2.075,00 \footnotesize{(-50,00)}&18\footnotesize{(+1)} \\
\hline
{Progettista} &2.280,00 \footnotesize{(+180,00)}&20 \footnotesize{(-2)} \\
\hline
{Programmatore} &1.755,00 \footnotesize{(-45,00)}&15\\
\hline
{Verificatore} &3.600,00\footnotesize{(-75,00)}&32 \footnotesize{(+2)} \\
\hline
{Totale} &11.438,00 \footnotesize{(-152,00)}&100 \\
\hline

\end{tabular} \\
\end{tabular}
}
\end{table}

\chapter{Quadro riassuntivo}
Alla revisione di collaudo la HappyCode giunge con un piano consuntivo che evidenzia un costo del prodotto di 152,00 Euro inferiore alle previsioni. Vediamo in dettaglio come siamo giunti a questa situazione attraverso le varie fasi del ciclo di vita del nostro software:
\begin{itemize}
\item \textbf{Fase di Analisi:} da 115 ore previste si \`e passati a 137 spendendo anzich\`e 2530 euro, 2960 euro (430 euro in pi\`u pari al 17\% della somma). Tale differenza \`e dovuta alle ore svolte dai progettisti della HappyCode. Come detto sopra infatti in questa prima fase si \`e iniziata la progettazione del sistema contrariamente alle previsioni che vedevano l'inizio della progettazione nella fase successiva.
\item \textbf{Fase di Progettazione:} assistiamo qui ad una leggera diminuzione delle ore e dei costi passando da 150 ore previste, per un totale di 3005,00 euro a 140 ore per una spesa complessiva di 2780,00 euro con un disavanzo di 225,00 euro. Sono aumentate infatti le ore di programmatore (leggermente) e diminuite quelle di progettista (grazie al lavoro svolto gi\`a nella fase di analisi). Grazie a queste variazioni la HappyCode si ritrovava al termine della Progettazione con un aumento di spesa prevista ridotto rispetto alla fase di 
analisi e ammontante a 205,00 euro (3,7\% della spesa effettiva fino a quel momento).
\item \textbf{Fase di sviluppo:} la HappyCode riporta la situazione in linea con le previsioni passando da 200 ore previste a 189 effettive per questa fase, con una relativa diminuzione di spesa pari a 216,00 euro (ossia lo 0,02\% della spesa effettiva fino a questo momento). Questa diminuzione di costi e ore in fase di sviluppo \`e stata leggermente forzata (limitando al minimo le ore di amministratore e responsabile e rinviando parte dei test alla fase successiva) per rientrare nei limiti di spesa pattuiti con il committente. Giunti a questo punto tuttavia, la situazione risulta pi\`u che accettabile.
\item \textbf{Fase di verifica:} Assistiamo in questa fase ad un'ulteriore diminuzione delle ore dell'amministratore rispetto alle previsioni iniziali. Risparmiamo inoltre ore di verificatore in quanto i test predisposti nella fase precedente non hanno richiesto grandi sforzi (in termini di ore) per essere adattati ad eseguire anche test di convalida del nostro software. Concludiamo perci\`o il nostro lavoro risparmiando globalmente 8 ore di lavoro e 152,00 Euro di spesa.
\end{itemize}
\end{document}
