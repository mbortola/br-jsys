
\documentclass[11pt,titlepage,a4paper]{report}

\usepackage[italian]{babel}
\usepackage{fancyhdr}
\usepackage{graphicx}
\usepackage{hyperref}

%\usepackage{lastpage} % total page count

\usepackage{color}
\usepackage{lastpage} % total page count

\graphicspath{{./pics/}} % cartella di salvataggio immagini

\pagestyle{fancy}
\renewcommand{\sectionmark}[1]{\markright{\thesection.\ #1}}
\lhead{\nouppercase{\rightmark}}
\rhead{\nouppercase{\leftmark}}
\renewcommand{\chaptermark}[1]{%
\markboth{\thechapter.\ #1}{}}


\fancypagestyle{plain}{%
	\lhead{\includegraphics[height=50pt]{logo.eps}}
	\chead{}
	\rhead{HappyCode inc \\ happycodeinc@gmail.com}
	\lfoot{BR-jsys}
	\cfoot{\thepage\ / \pageref{LastPage}}
	\rfoot{\dt - \lv}
	\renewcommand{\headrulewidth}{1pt}
	\renewcommand{\footrulewidth}{1pt}
}
	\lhead{\includegraphics[height=50pt]{logo.eps}}
	\chead{}
	\rhead{HappyCode inc \\ happycodeinc@gmail.com}
	\lfoot{BR-jsys}
	\cfoot{\thepage\ / \pageref{LastPage}}
	\rfoot{\dt - \lv}
	\renewcommand{\headrulewidth}{1pt}
	\renewcommand{\footrulewidth}{1pt}

\hypersetup{
    colorlinks=true,       % false: boxed links; true: colored links
   linkcolor=[rgb]{0.11,0.55,0.83},          % color of internal links
    urlcolor=cyan           % color of external links
}
\definecolor{err}{rgb}{0.9,0.1,0.1}

% fine layout
\begin{document}
%begin variables
\newcommand{\dt}{ Lettera di accompagnamento }%document title
\hyphenation{
 a-na-lo-go
 as-so-cia-zio-ne
 %attività non si può inserire come tutte le parole accentate che vanno messe nel testo semplice scritte at\-ti\-vi\-tà o come variabile
 coe-ren-za
 com-po-nen-ti
 con-si-glia-bi-le
 des-crit-te
 des-cri-zio-ni
 di-a-gram-ma
 di-a-gram-mi
 e-le-men-to
 e-se-gui-re
 e-si-sten-ti
 es-pli-ci-to
 glo-bal-men-te
 glos-sa-rio
 in-se-ri-men-to
 li-vel-lo
 ne-ces-sa-rio
 per-met-te-re
 re-po-si-to-ry
 re-vi-sio-na-men-to
 ri-chies-te
 se-le-zio-na-ta
 se-gna-la-ta
 va-li-da-zio-ne
 va-ria-bi-li
 ve-ri-fi-ca-re
 vi-sua-liz-za-te
 e-ven-tua-li
 o-pe-ra-zio-ne
 ar-chi-via-zio-ne
 mo-di-fi-ca
 ar-chi-vio
 des-cri-zio-ne
 pa-ren-te-si
 i-ni-zia
}


%sillabazione

\begin{titlepage}\begin{center}
\vspace*{0.5in}
\includegraphics{logo.eps}
\vspace*{0.2in} \\
{\Large \textbf{BR-jsys}}
{\Large \emph{business rules} per sistemi gestionali in architettura J2EE } 
\vspace{2in} \\
\Huge \textsc{ \dt }
\par\rule{10cm}{0.4pt} \par {\large Versione \lv - \today} \\
\end{center}\end{titlepage}

\vspace*{0.5in}
%common variables
\newcommand{\br}{\underline{business rule}}
\newcommand{\brs}{\underline{business rules}}
\newcommand{\bo}{\underline{business object}}
\newcommand{\bos}{\underline{business objects}}
\newcommand{\rp}{\underline{repository}}
\newcommand{\brp}{BusinessRuleParser}
\newcommand{\brl}{BusinessRuleLexer}
\newcommand{\BR}{\underline{BusinessRule}}

%nomi dei componenti
\newcommand{\AT}{Alessia Trivellato}
\newcommand{\ET}{Elena Trivellato}
\newcommand{\FC}{Filippo Carraro}
\newcommand{\LA}{Luca Appon}
\newcommand{\MB}{Michele Bortolato}
\newcommand{\MT}{Marco Tessarotto}
\newcommand{\MM}{Mattia Meroi}%altre variabili
% ultime versioni dei documenti da modificare solo alla fine
\newcommand{\AR}{AnalisiDeiRequisiti.2.6.pdf}
\newcommand{\DdP}{DefinizioneDiProdotto.0.9.pdf}
\newcommand{\G}{ Glossario.1.8.pdf }
\newcommand{\NdP}{NormeDiProgetto.2.0.pdf}
\newcommand{\PdQ}{ PianoDiQualifica.1.4.pdf }
\newcommand{\PdP}{ PianoDiProgetto.1.7.pdf }
\newcommand{\ST}{SpecificaTecnica.1.5.pdf}
\newcommand{\TR}{TestReport.0.7.pdf}
\newcommand{\MU}{ManualeUtente.0.3.pdf}%nomi documenti
%end variables



\thispagestyle{plain}
\begin{flushright}
Alla cortese attenzione dei docenti\\
prof. Tullio Vardanega e prof. Renato Conte,\\
rappresentanti del proponente\\
Zucchetti S.r.l.
\end{flushright}
Padova, \today \\

L'azienda HappyCodeInc chiede di partecipare alla prossima revisione (RQ) che si terr\`a l'11 marzo 2008.
Alla presente viene allegata, come da Vostre indicazioni, la documentazione relativa:
\begin{itemize}
\item{}Glossario (\G)
\item{}Manuale Utente (\MU)
\item{}Piano di Progetto (\PdP)
\item{}Piano di Qualifica (\PdQ)
\item{}Report dei Test (\TR)
\end{itemize}
Per una maggiore completezza e chiarezza alleghiamo inoltre i seguenti documenti aggiornati:
\begin{itemize}
\item{}Analisi dei Requisiti (\AR)
\item{}Definizione di Prodotto (\DdP)
\item{}Norme di Progetto (\NdP)
\item{}Specifica tecnica (\ST) 
\end{itemize}
Viene inoltre allegata:
\begin{itemize}
\item{La  cartella 'codice' con all'interno il codice sorgente prodotto finora}
\item{La documentazione 'javadoc' del codice prodotto.}
\end{itemize}

Cordiali saluti.
\begin{flushright}
HappyCodeInc
\end{flushright}
\newpage

\end{document}
