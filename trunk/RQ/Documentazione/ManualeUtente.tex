\documentclass[11pt,titlepage,a4paper]{report}

%INCLUSIONE PACCHETTI
%---------------------------------------------
\usepackage[italian]{babel}
\usepackage{fancyhdr}
\usepackage{graphicx}
\graphicspath{{./pics/}} % cartella di salvataggio immagini

% STILE DI PAGINA
%---------------------------------------------
\pagestyle{fancy}
\renewcommand{\sectionmark}[1]{\markright{\thesection.\ #1}}
\lhead{\nouppercase{\rightmark}}
\rhead{\nouppercase{\leftmark}}
\renewcommand{\chaptermark}[1]{%
\markboth{\thechapter.\ #1}{}}

%Ridefinisco lo stile plain della pagina
\fancypagestyle{plain}{%
	\lhead{\includegraphics[height=50pt]{logo.eps}}
	\chead{}
	\rhead{HappyCode inc \\ happycodeinc@gmail.com}
	\lfoot{BR-jsys}
	\rfoot{\dt - \lv}
	\cfoot{\thepage}
	\renewcommand{\headrulewidth}{1pt}
	\renewcommand{\footrulewidth}{1pt}
}

\begin{document}

%definizione variabili 
\newcommand{\lv}{ 0.3 } % latest version
\newcommand{\dt}{ Manuale Utente }% Document title
\newcommand{\Glossario}{ Glossario.1.4.pdf }
%fine definizione variabili

\hyphenation{glos-sa-rio}
\begin{titlepage}\begin{center}
\vspace*{0.5in}
\includegraphics{logo.eps}
\vspace*{0.2in} \\
{\Large \textbf{BR-jsys}}
{\Large \emph{business rules} per sistemi gestionali in architettura J2EE } 
\vspace{2in} \\
\Huge \textsc{ \dt }
\par\rule{10cm}{0.4pt} \par {\large Versione \lv - \today} \\
\end{center}\end{titlepage}
\vspace*{0.5in}

\begin{center}
\thispagestyle{plain}
\begin{table}[htbp]
\large{
\begin{tabular}{l}
\Large{\textbf{\textsf{Capitolato: ''BR-jsys``}}} \\
\begin{tabular}{||p{6cm}||p{6cm}||}
\hline
\textbf{Data creazione:} & 2008/02/25 \\ \hline
\textbf{Versione:} & \lv \hline
\textbf{Stato del documento:} & Formale, esterno \\ \hline
% ----------------------------------------------------------------------------autori
\textbf{Redazione:} & \\ \hline
\textbf{Revisione:} & \\ \hline
\textbf{Approvazione:} & \\ \hline
\end{tabular} \\
\end{tabular}
}
\end{table}

\begin{table}[hbtp]
\large{
\begin{tabular}{l}
\Large{\textbf{\textsf{Lista di distribuzione}}} \\

\begin{tabular}{||p{6cm}||p{6cm}||} \hline
{HappyCode inc}& Gruppo di lavoro\\ \hline
{Tullio Vardanega, Renato Conte}& Rappresentanti del committente \\ \hline
{Zucchetti S.r.l}& Azienda committente\\ \hline
\end{tabular} \\
\end{tabular}
}
\end{table}
\begin{table}[hbtp]
\large{
\begin{tabular}{l}
\Large{\textbf{\textsf{Diario delle modifiche}}} \\
\begin{tabular}{||p{2cm}||p{3.5cm}||p{6cm}||} \hline
%-------------------------------------------------------------------------------diario modifiche
\textbf{Versione} & \textbf{Data rilascio} & \textbf{Descrizione} \\ \hline
0.1 & 2008/02/25 & Stesura preliminare del documento \\ \hline

\end{tabular} \\
\end{tabular}

}
\end{table}
\end{center}
\newpage

\tableofcontents 

\chapter{Introduzione}
Il presente documento contiene la definizione dell'architettura logica del sistema ``BR-jsys'' con una descrizione pi\`u accurata di ogni singola componente.
\section{Definizione dell'utente del prodotto}
Il prodotto \`e rivolto ad un utente al quale non sono richieste particolari conoscenze nel campo informatico. A tale scopo il software mette a dispozione:
\begin{itemize}
\item un'interfacciafile:///home/1/2004/atrivell/Desktop/Manuale Utente grafica user-friendly;
\item messaggi di testo che aiutano l'utente dando informazione aggiuntive sullo stato del prodotto ``BR-jsys''.
\end{itemize}
\section{Come leggere il manuale}
Questo manuale presenta il sistema ``BR-jsys'' da noi realizzato e insegna come usare tutte le funzioni ad esso correlate. \`E organizzato in tre sezioni in modo da rendere facile il reperimento delle informazioni desiderate:
\begin{itemize}
\item SEZIONE GENERALE: dedicata alla descrizione generale del prodotto. Fornisce informazioni di base sul funzionamento del prodotto.
\item SEZIONE DETTAGLIATA: dedicata alla descrizione dettagliata. Indispensabile per un corretto utilizzo di ogni singola parte del prodotto.
\item SEZIONE APPENDICE: riporta gli errori pi\`u comuni che l'utente pu\`o riscontrare durante l'utilizzo del prodotto e il glossario per chiarire la terminologia da noi adottata.
\end{itemize}
\section{Documenti utili}
Il presente manuale descrive totalmente il sistema e non richiede la lettura di ulteriori documenti. Tuttavia ogni utente deve avere installato nel proprio pc ``eXist'' in quanto vero e proprio requisito di sistema. 
\section{Come riportare problemi e malfunzionamenti}
Il prodotto software \`e dotato di un sistema di archiviazione svn. Tale archivio fornisce una sezione ``issues'' al seguente URL: \\
http://code.google.com/p/br-jsys/issues/list \\
che visualizza in qualsiasi momento la lista aggiornata di tutti i problemi e malfunzionamenti da noi trovati. Attraverso questo sistema ogni componente del gruppo potr\`a creare un ``new issue'', assegnandogli una priorit\`a che varier\`a a seconda della gravit\`a del problema riscontrato. Ogni issues conterr\`a inoltre il nome dell'utente che lo ha aggiunto nella lista, oltre ad uno status (new, accepted, started, verified, etc...) che in qualsiasi momento ogni membro del gruppo potr\`a aggiornare. 
\chapter{Descrizione generale}
% da fare
\section{Caratteristiche del servizio}
% da fare
\section{Requisiti minimi}
Per usufruire del servizio \`e necessario possedere una connessione con il DBMS ``eXist'', come gi\`a citato precedentemente.
\chapter{Istruzioni per l'uso}
% da fare
\section{Descrizione funzionale}
% da fare
\section{Azioni richieste/permesse}
% da fare
\section{Errori probabili e cause possibili}
% da fare
\chapter{Appendice}
\section{Messaggi di errore e loro significato}
% da finire una volta completata 
\section{Glossario}
Viene fornito come documento esterno chiamato \Glossario.

\newpage

\end{document}
