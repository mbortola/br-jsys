
\documentclass[11pt,titlepage,a4paper]{report}

\usepackage[italian]{babel}
\usepackage{fancyhdr}
\usepackage{graphicx}
\usepackage{hyperref}

%\usepackage{lastpage} % total page count

\usepackage{color}
\usepackage{lastpage} % total page count

\graphicspath{{./pics/}} % cartella di salvataggio immagini

\pagestyle{fancy}
\renewcommand{\sectionmark}[1]{\markright{\thesection.\ #1}}
\lhead{\nouppercase{\rightmark}}
\rhead{\nouppercase{\leftmark}}
\renewcommand{\chaptermark}[1]{%
\markboth{\thechapter.\ #1}{}}


\fancypagestyle{plain}{%
	\lhead{\includegraphics[height=50pt]{logo.eps}}
	\chead{}
	\rhead{HappyCode inc \\ happycodeinc@gmail.com}
	\lfoot{BR-jsys}
	\cfoot{\thepage\ / \pageref{LastPage}}
	\rfoot{\dt - \lv}
	\renewcommand{\headrulewidth}{1pt}
	\renewcommand{\footrulewidth}{1pt}
}
	\lhead{\includegraphics[height=50pt]{logo.eps}}
	\chead{}
	\rhead{HappyCode inc \\ happycodeinc@gmail.com}
	\lfoot{BR-jsys}
	\cfoot{\thepage\ / \pageref{LastPage}}
	\rfoot{\dt - \lv}
	\renewcommand{\headrulewidth}{1pt}
	\renewcommand{\footrulewidth}{1pt}

\hypersetup{
    colorlinks=true,       % false: boxed links; true: colored links
   linkcolor=[rgb]{0.11,0.55,0.83},          % color of internal links
    urlcolor=cyan           % color of external links
}
\definecolor{err}{rgb}{0.9,0.1,0.1}

% fine layout% layout
\begin{document}
%definizione variabili 
\newcommand{\lv}{ 0.2 } % latest version
\newcommand{\dt}{ Campagna di Test }% Document title
%common variables
\newcommand{\br}{\underline{business rule}}
\newcommand{\brs}{\underline{business rules}}
\newcommand{\bo}{\underline{business object}}
\newcommand{\bos}{\underline{business objects}}
\newcommand{\rp}{\underline{repository}}
\newcommand{\brp}{BusinessRuleParser}
\newcommand{\brl}{BusinessRuleLexer}
\newcommand{\BR}{\underline{BusinessRule}}

%nomi dei componenti
\newcommand{\AT}{Alessia Trivellato}
\newcommand{\ET}{Elena Trivellato}
\newcommand{\FC}{Filippo Carraro}
\newcommand{\LA}{Luca Appon}
\newcommand{\MB}{Michele Bortolato}
\newcommand{\MT}{Marco Tessarotto}
\newcommand{\MM}{Mattia Meroi}%altre variabili
% ultime versioni dei documenti da modificare solo alla fine
\newcommand{\AR}{AnalisiDeiRequisiti.2.6.pdf}
\newcommand{\DdP}{DefinizioneDiProdotto.0.9.pdf}
\newcommand{\G}{ Glossario.1.8.pdf }
\newcommand{\NdP}{NormeDiProgetto.2.0.pdf}
\newcommand{\PdQ}{ PianoDiQualifica.1.4.pdf }
\newcommand{\PdP}{ PianoDiProgetto.1.7.pdf }
\newcommand{\ST}{SpecificaTecnica.1.5.pdf}
\newcommand{\TR}{TestReport.0.7.pdf}
\newcommand{\MU}{ManualeUtente.0.3.pdf}%nomi documenti
%fine definizione variabili
\hyphenation{
 a-na-lo-go
 as-so-cia-zio-ne
 %attività non si può inserire come tutte le parole accentate che vanno messe nel testo semplice scritte at\-ti\-vi\-tà o come variabile
 coe-ren-za
 com-po-nen-ti
 con-si-glia-bi-le
 des-crit-te
 des-cri-zio-ni
 di-a-gram-ma
 di-a-gram-mi
 e-le-men-to
 e-se-gui-re
 e-si-sten-ti
 es-pli-ci-to
 glo-bal-men-te
 glos-sa-rio
 in-se-ri-men-to
 li-vel-lo
 ne-ces-sa-rio
 per-met-te-re
 re-po-si-to-ry
 re-vi-sio-na-men-to
 ri-chies-te
 se-le-zio-na-ta
 se-gna-la-ta
 va-li-da-zio-ne
 va-ria-bi-li
 ve-ri-fi-ca-re
 vi-sua-liz-za-te
 e-ven-tua-li
 o-pe-ra-zio-ne
 ar-chi-via-zio-ne
 mo-di-fi-ca
 ar-chi-vio
 des-cri-zio-ne
 pa-ren-te-si
 i-ni-zia
}


%sillabazione

\begin{titlepage}\begin{center}
\vspace*{0.5in}
\includegraphics{logo.eps}
\vspace*{0.2in} \\
{\Large \textbf{BR-jsys}}
{\Large \emph{business rules} per sistemi gestionali in architettura J2EE } 
\vspace{2in} \\
\Huge \textsc{ \dt }
\par\rule{10cm}{0.4pt} \par {\large Versione \lv - \today} \\
\end{center}\end{titlepage}

\vspace*{0.5in}%pagina del titolo



\begin{center}
\thispagestyle{plain}
\begin{table}[htbp]
\large{
\begin{tabular}{l}
\Large{\textbf{\textsf{Capitolato: ''BR-jsys``}}} \\
\begin{tabular}{|p{6cm}|p{6cm}|}
\hline
\textbf{Data creazione:} & 17/03/2008 \\
\hline
\textbf{Versione:} & \lv \\ \hline
% ----------------------------------------------------------------------------autori
\textbf{Stato del documento:} & Formale ad uso Interno \\ \hline
\textbf{Revisione RA} & \\ \hline
\textbf{Redazione:} & \AT \\ \hline
\textbf{Revisione:} &  \ET\\ \hline
\textbf{Approvazione:} &  \MT\\ \hline
\end{tabular} \\
\end{tabular}
}
\end{table}

\begin{table}[hbtp]
\large{
\begin{tabular}{l}
\Large{\textbf{\textsf{Lista di distribuzione}}} \\
\begin{tabular}{|p{6cm}|p{6cm}|} \hline
%  -------------------------------------------------------------lista di distribuzione
{Tutta la HappyCode inc}& Gruppo di lavoro \\ \hline
\end{tabular} \\
\end{tabular}
}
\end{table}

\begin{table}[hbtp]

\Large{\textbf{\textsf{Diario delle modifiche}}} \\
\begin{small}
\begin{tabular}[t]{|p{1,2cm}|p{1.9cm}|p{2.9cm}|p{5cm}|} \hline
Versione & Data & Autore & Descrizione \\ \hline
%-------------------------------------------------------------------------------diario modifiche
0.2 & 17/03/2008 & \ET & Correzione documento. \\ \hline
0.1 & 17/03/2008 & \AT & Stesura preliminare del documento. \\ \hline
\end{tabular} \\
\end{small}
\end{table}
\end{center}
\newpage
\tableofcontents 
\chapter{Scopo del documento}
Il presente documento descrive la campagna dei test che effettueremo in sede di collaudo mercoled\`i 19 marzo.
\chapter{Avvio dell'applicazione}
Essendo l'installazione del prodotto ``BR-jsys'' gi\`a effettuata nel pc con cui ci presenteremo al collaudo, procederemo direttamente con l'avvio dell'applicazione. 
\begin{itemize}
\item Avvio del prodotto ``Br-jsys'' senza avviare prima eXist;
\item Avvio di eXist e successivo avvio del prodotto ``Br-jsys''.
\end{itemize}
A questo punto comparir\`a la schermata di login. 
\section{Procedura di login}
Le credenziali per accedere al programma verranno qui inserite.
\begin{itemize}
\item \textbf{username:} inseriremo lo username utilizzato per la registrazione ad eXist
\item \textbf{password:}  inseriremo la password utilizzata per la registrazione ad eXist
Prima di fare ci\`o proveremo ad accedere con dati di login errati per mostrare il relativo messaggio di errore.
\end{itemize}
\chapter{Inserimento}
A questo punto, essendo il repository ancora vuoto, l'unica operazione consentita sar\`a l'inserimento di una o pi\`u business rules.
Verranno effettuati i seguenti inserimenti:
\begin{itemize}
\item inserimento di una business rule lasciando tutti i campi vuoti;
\item inserimento di una business rule sintatticamente corretta;
\item inserimento di una business rule, lasciando il campo nome vuoto;
\item inserimento di una business rule, lasciando il campo message vuoto;
\item inserimento di una business rule con lo stesso nome di un'altra presente nel repository;
\item inserimento di una business rule con lo stesso testo, ma nome diverso, di un'altra regola presente nel repository;
\item inserimento di una business rule, lasciando vuoto il campo del \bo associato;
\item inserimento di una business rule contente gli operatori logici AND e OR;
\item inserimento di una business rule che non contiene gli operatori logici AND e/o OR;
\item inserimento di una business rule contenente uno o pi\`u operatori di confronto;
\item inserimento di una business rule contenenti operazioni tra scalari;
\item inserimento di una business rule contenenti operazioni tra campi dati scalari;
\item inserimento di una business rule contenenti espressioni matematiche tra parentesi;
\item inserimento di una business rule contenente l'operatore di uguaglianza tra stringhe;
\item inserimento di una business rule contenente operazioni tra booleani;
\item inserimento di una business rule contenente le funzioni del nostro linguaggio (SUM(), AVG(), COUNT());
\item inserimento di una business rule non completa.
\end{itemize}
\chapter{Rimozione}
\begin{itemize}
\item rimozione di una business rule selezionata dalla lista;
\item rimozione di una business rule cercata per nome;
\item rimozione di business rules cercate attraverso espressioni regolari;
\item rimozione multipla di business rules selezionate dalla lista.
\end{itemize}
\chapter{Sandbox}
\begin{itemize}
\item ricerca di business rule il cui nome inzia con una determinata stringa; 
\item ricerca di business rule dato l'oggetto associato e prive di commento;
\item ricerca di business rules senza commento;
\end{itemize}


\end{document}