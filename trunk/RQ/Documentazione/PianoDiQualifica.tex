
\documentclass[11pt,titlepage,a4paper]{report}

\usepackage[italian]{babel}
\usepackage{fancyhdr}
\usepackage{graphicx}
\usepackage{hyperref}

%\usepackage{lastpage} % total page count

\usepackage{color}
\usepackage{lastpage} % total page count

\graphicspath{{./pics/}} % cartella di salvataggio immagini

\pagestyle{fancy}
\renewcommand{\sectionmark}[1]{\markright{\thesection.\ #1}}
\lhead{\nouppercase{\rightmark}}
\rhead{\nouppercase{\leftmark}}
\renewcommand{\chaptermark}[1]{%
\markboth{\thechapter.\ #1}{}}


\fancypagestyle{plain}{%
	\lhead{\includegraphics[height=50pt]{logo.eps}}
	\chead{}
	\rhead{HappyCode inc \\ happycodeinc@gmail.com}
	\lfoot{BR-jsys}
	\cfoot{\thepage\ / \pageref{LastPage}}
	\rfoot{\dt - \lv}
	\renewcommand{\headrulewidth}{1pt}
	\renewcommand{\footrulewidth}{1pt}
}
	\lhead{\includegraphics[height=50pt]{logo.eps}}
	\chead{}
	\rhead{HappyCode inc \\ happycodeinc@gmail.com}
	\lfoot{BR-jsys}
	\cfoot{\thepage\ / \pageref{LastPage}}
	\rfoot{\dt - \lv}
	\renewcommand{\headrulewidth}{1pt}
	\renewcommand{\footrulewidth}{1pt}

\hypersetup{
    colorlinks=true,       % false: boxed links; true: colored links
   linkcolor=[rgb]{0.11,0.55,0.83},          % color of internal links
    urlcolor=cyan           % color of external links
}
\definecolor{err}{rgb}{0.9,0.1,0.1}

% fine layout% layout
\begin{document}
%definizione variabili 
\newcommand{\lv}{ 1.5 } % latest version
\newcommand{\dt}{ Piano Di Qualifica }% Document title
%common variables
\newcommand{\br}{\underline{business rule}}
\newcommand{\brs}{\underline{business rules}}
\newcommand{\bo}{\underline{business object}}
\newcommand{\bos}{\underline{business objects}}
\newcommand{\rp}{\underline{repository}}
\newcommand{\brp}{BusinessRuleParser}
\newcommand{\brl}{BusinessRuleLexer}
\newcommand{\BR}{\underline{BusinessRule}}

%nomi dei componenti
\newcommand{\AT}{Alessia Trivellato}
\newcommand{\ET}{Elena Trivellato}
\newcommand{\FC}{Filippo Carraro}
\newcommand{\LA}{Luca Appon}
\newcommand{\MB}{Michele Bortolato}
\newcommand{\MT}{Marco Tessarotto}
\newcommand{\MM}{Mattia Meroi}%altre variabili
% ultime versioni dei documenti da modificare solo alla fine
\newcommand{\AR}{AnalisiDeiRequisiti.2.6.pdf}
\newcommand{\DdP}{DefinizioneDiProdotto.0.9.pdf}
\newcommand{\G}{ Glossario.1.8.pdf }
\newcommand{\NdP}{NormeDiProgetto.2.0.pdf}
\newcommand{\PdQ}{ PianoDiQualifica.1.4.pdf }
\newcommand{\PdP}{ PianoDiProgetto.1.7.pdf }
\newcommand{\ST}{SpecificaTecnica.1.5.pdf}
\newcommand{\TR}{TestReport.0.7.pdf}
\newcommand{\MU}{ManualeUtente.0.3.pdf}%nomi documenti
%fine definizione variabili
\hyphenation{
 a-na-lo-go
 as-so-cia-zio-ne
 %attività non si può inserire come tutte le parole accentate che vanno messe nel testo semplice scritte at\-ti\-vi\-tà o come variabile
 coe-ren-za
 com-po-nen-ti
 con-si-glia-bi-le
 des-crit-te
 des-cri-zio-ni
 di-a-gram-ma
 di-a-gram-mi
 e-le-men-to
 e-se-gui-re
 e-si-sten-ti
 es-pli-ci-to
 glo-bal-men-te
 glos-sa-rio
 in-se-ri-men-to
 li-vel-lo
 ne-ces-sa-rio
 per-met-te-re
 re-po-si-to-ry
 re-vi-sio-na-men-to
 ri-chies-te
 se-le-zio-na-ta
 se-gna-la-ta
 va-li-da-zio-ne
 va-ria-bi-li
 ve-ri-fi-ca-re
 vi-sua-liz-za-te
 e-ven-tua-li
 o-pe-ra-zio-ne
 ar-chi-via-zio-ne
 mo-di-fi-ca
 ar-chi-vio
 des-cri-zio-ne
 pa-ren-te-si
 i-ni-zia
}


%sillabazione

\begin{titlepage}\begin{center}
\vspace*{0.5in}
\includegraphics{logo.eps}
\vspace*{0.2in} \\
{\Large \textbf{BR-jsys}}
{\Large \emph{business rules} per sistemi gestionali in architettura J2EE } 
\vspace{2in} \\
\Huge \textsc{ \dt }
\par\rule{10cm}{0.4pt} \par {\large Versione \lv - \today} \\
\end{center}\end{titlepage}

\vspace*{0.5in}%pagina del titolo



\begin{center}
\thispagestyle{plain}
\begin{table}[htbp]
\large{
\begin{tabular}{l}
\Large{\textbf{\textsf{Capitolato: ''BR-jsys``}}} \\
\begin{tabular}{|p{6cm}|p{6cm}|} \hline
\textbf{Data creazione:} & 19/11/07 \\ \hline
\textbf{Versione:} & \lv \\ \hline
\textbf{Stato del documento:} & Formale, esterno \\ \hline
% ----------------------------------------------------------------------------autori
\textbf{Revisione RR} &    \\ \hline
\textbf{Redazione:} & \LA \\ \hline
\textbf{Revisione:} & \MT   \\ \hline
\textbf{Approvazione:}  & \ET \\ \hline
\textbf{Revisione RPD} &    \\ \hline
\textbf{Redazione:} & \ET , \AT \\ \hline
\textbf{Revisione:} & \MM , \MT \\ \hline
\textbf{Approvazione:}  & \MB \\ \hline
\textbf{Revisione RQ} &    \\ \hline
\textbf{Redazione:} & \\ \hline
\textbf{Revisione:} & \\ \hline
\textbf{Approvazione:}  & \\ \hline

\end{tabular} \\
\end{tabular}
}
\end{table}

\begin{table}[hbtp]
\large{
\begin{tabular}{l}
\Large{\textbf{\textsf{Lista di distribuzione}}} \\
\begin{tabular}{|p{6cm}|p{6cm}|} \hline
{HappyCode inc}& Gruppo di lavoro\\ \hline
{Tullio Vardanega, Renato Conte}& Committenti \\ \hline
{Zucchetti S.r.l}& Azienda proponente\\ \hline
\end{tabular} \\
\end{tabular}
}
\end{table}

\begin{table}[hbtp]

\Large{\textbf{\textsf{Diario delle modifiche}}} \\
\begin{small}
\begin{tabular}[t]{|p{1,2cm}|p{1.9cm}|p{2.9cm}|p{5cm}|} \hline
Versione & Data & Autore & Descrizione \\ \hline
%-------------------------------------------------------------------------------diario modifiche
1.8 & 06/03/2008 & \LA & Aggiunto Sommario.\\ \hline
1.7 & 05/03/2008 & \MM & Evidenziazione dei termini contenuti nel documento \G .\\ \hline
1.6 & 04/03/2008 & \MM & Aggiunta di link ipertestuali all'indice del documento.\\ \hline
1.5 & 04/03/2008 & \MT & Modifica al layout, introduzione totale pagine e autori nel diario delle modifiche.\\ \hline
1.4 & 05/02/2008 & \MT & Correzione del documento \\ \hline
1.3 & 05/02/2008 & \ET & Completamento tabella tracciamento Requisiti-Test \\ \hline
1.2 & 04/02/2008 & \MM & Correzione grammaticale documento \\ \hline
1.1 & 28/01/2008 & \AT & Modifica al ``Processo di ispezione''\\ \hline
1.0 & 25/01/2008 & \AT & Aggiunta della tabella di tracciamento Requisiti-Test\\ \hline
0.6 & 24/01/2008 & \ET & Modifica al capitolo ``Pianificazione ed esecuzione del collaudo''\\ \hline
0.5 & 23/01/2008 & \AT & Modifiche allo ``Scopo del documento''\\ \hline
0.4 & 22/01/2008 & \MT & Modifica al layout dei documenti\\ \hline
0.3 & 21/12/2007 & \MT & Documento sottoposto a revisionamento automatico\\ \hline
0.2 & 06/12/2007 & \MT & Correzione errori \\ \hline
0.1 & 19/11/2007 & \LA & Stesura preliminare del documento \\ \hline
\end{tabular} \\
\end{small}


\end{table}
\end{center}

\tableofcontents
\chapter*{Sommario}
\addcontentsline{toc}{chapter}{ I Sommario}
Lo scopo del presente documento \`e quello di informare il committente riguardo le strategie di validazione e verifica che la HappyCode
applicher\`a al prodotto BR-jsys.

\chapter*{Glossario}
\addcontentsline{toc}{chapter}{II Glossario}
Viene fornito come documento esterno chiamato \G. I termini che sono presenti nel \G sono sottolineati.

\chapter{Introduzione}
\section{Scopo del documento}
Nel presente documento illustreremo le strategie di verifica e validazione adottate al fine di garantire la qualit\`a attesa del nostro prodotto. Si utilizzer\`a l'analisi statica, sotto forma di ispezioni del codice e dei documenti, al fine di individuare errori e difetti negli stessi. Verr\`a inoltre utilizzata l'analisi dinamica del codice, sottoforma di test, per soddisfare i requisiti quali funzionalit\`a, affidabilit\`a e usabilit\`a del prodotto ``\underline{BR-jsys}''.
\begin{itemize}
\item \underline{Funzionalit\`a:} \newline
Il prodotto soddisfer\`a pienamente i requisiti descritti nel documento di ``AnalisiDeiRequisiti''.
\item \underline{Affidabilit\`a:} \newline
Il sistema sar\`a privo di errori in quanto la verifica verr\`a effettuata attraverso opportuni strumenti di controllo (test e ispezioni).
\item \underline{Usabilit\`a:} \newline
L'utente finale del sistema non dovr\`a necessariamente avere grandi competenze nel campo informatico.
\end{itemize}

\section{Scopo del prodotto}
Il prodotto richiesto verr\`a inserito nell'ambito di un progetto pi\`u ampio. Il suo scopo \`e quello di automatizzare il sistema di validazione dei dati in ingresso al database dell'applicazione principale.


\section{Riferimenti}
\begin{itemize}
\item Capitolato d'appalto concorso per sistema ``\underline{BR-jsys}'';
\item Verbale dell'incontro con il proponente ``Incontro2007-11-22.pdf'';
\item Verbale dell'incontro con il proponente ``Incontro2008-02-05.pdf'';
\item ``Ingegneria del Software'' 8a edizione - Ian Sommerville;
\item ``Analisi dei Requisiti'';
\item ``Norme di Progetto'';
\item Piano di Progetto ``\PdP''.
\end{itemize}

\chapter[Strategia di verifica]{Visione generale della strategia di verifica}
\section[Organizzazione, pianificazione, responsabilit\`a]{Organizzazione, pianificazione strategica e temporale, responsabilit\`a}
La HappyCode ha  pianificato delle fasi temporali per definire, sviluppare e validare il prodotto; tali fasi saranno prese in analisi e descritte nel dettaglio nei successivi paragrafi.

\subsection{Ciclo di vita}
Le attivit\`a seguiranno un modello di tipo evolutivo che permetter\`a di intrecciare le attivit\`a di specifica, sviluppo e convalida del software, e di apportare eventuali modifiche ai documenti in tempi diversi. L'obiettivo \`e comprendere al meglio le richieste del cliente e sviluppare dunque una migliore definizione dei requisiti del sistema. Per arrivare ad un prodotto finale sar\`a quindi indispensabile lavorare in stretto contatto con il cliente, sottoponendogli periodicamente versioni parziali o prototipi del prodotto finale. Adotteremo lo sviluppo esplorativo, concentrandoci dapprima sulle parti del sistema che sono ben chiare (requisiti ben compresi) e, solo successivamente verranno aggiunte nuove parti/funzionalit\`a a fronte di chiarimenti da parte del cliente. 
\subsection{Pianificazione delle attivit\`a}
In una prima fase gli analisti discuteranno e cercheranno di comprendere al meglio il problema da risolvere, grazie soprattutto alle varie comunicazioni e incontri con il cliente. Una volta chiariti e consolidati i vari requisiti sar\`a quindi possibile la stesura del documento intitolato ``AnalisiDeiRequisiti'', che sar\`a la base per la fase di progettazione seguente. Queste attivit\`a potranno essere eseguite in modo parallelo su diverse parti del sistema, come anche le attivit\`a di progettazione e verifica, in modo da consentire la riduzione di tempi e costi di consegna del prodotto. Al termine delle attivit\`a di progettazione seguir\`a quella di programmazione, che produrr\`a il prototipo richiesto. L'ultima fase prima del collaudo sar\`a dedicata ad un attenta verifica di tutto il lavoro svolto nelle precedenti fasi. Il verificatore ed il responsabile di progetto sono le figure alle quali verranno affidate le responsabilit\`a pertinenti alle attivit\`a di verifica. Per una descrizione pi\`u dettagliata della pianificazione si veda il documento ``Piano di Progetto''.

\section{Risorse necessarie, risorse disponibili}
Le attivit\`a di verifica necessitano di risorse umane e tecnologiche. Il gruppo \`e composto da sette membri, ognuno dei quali durante tutto il periodo di lavoro dovr\`a assumere, in periodi di tempo diversi (in base alle disponibilit\`a e competenze di ciascuno), tutti i ruoli significativi per lo sviluppo del prodotto. L'amministratore del progetto sar\`a tenuto a supervisionare tutte le fasi di verifica e gestire le varie risorse necessarie per consentire un'attivit\`a di buon livello, senza sprechi od oneri eccessivi. Per la comunicazione tra i componenti \`e stato creato un gruppo Google, accessibile ai soli membri, raggiungibile all'URL: \textit{http://groups.google.com/group/happycodeinc}. Essendo inoltre ogni componente provvisto di un account Gmail, verr\`a utilizzato il sistema di chat locale per la comunicazione interattiva. Per l'archiviazione dei file verr\`a invece utilizzato un server SVN (vedi il documento ``Norme di Progetto'').

\section{Strumenti, tecniche, metodi}
\subsection{Analisi Statica}
Verr\`a utilizzata durante la stesura del codice sorgente per rilevare errori, omissioni o anomalie, oltre ad incongruenze che possono sorgere tra il progetto ed i requisiti; sar\`a quindi applicata sulla struttura delle varie componenti del sistema ``\underline{BR-jsys}''. Essa comprende le seguenti sottofasi:
\begin{itemize}
\item \textbf{Analisi del flusso di controllo:} Verificher\`a una corretta esecuzione del codice.
In particolare si controller\`a che non vi siano statement irraggiungibili, ossia istruzioni la cui condizione di accesso non pu\`o mai essere vera.
\item \textbf{Analisi dell'uso dei dati:} Verificher\`a un corretto utilizzo delle variabili.
In particolare si controller\`a che non vi siano variabili utilizzate prima di essere inizializzate, variabili inutilizzate, variabili sempre vere o sempre false.
\item \textbf{Verifica formale del codice:} Verificher\`a la correttezza del codice scritto. In particolare si constater\`a la correttezza totale di ogni unit\`a, in modo che non conduca mai in uno stato di non terminazione.
\end{itemize}

\subsection{Analisi Dinamica}
Verr\`a applicata durante la progettazione e la stesura del codice, al fine di verificare dinamicamente l'indipendenza delle singole unit\`a rispetto all'integrazione del sistema. Verr\`a testato quindi il sistema in tutti i suoi possibili casi e verranno effettuate prove per verificarne l'integrit\`a. L'analisi avverr\`a:
\begin{itemize}
\item Attraverso l'inserimento di nuove regole business, in un contesto di prova, effettueremo test sulla validazione e li confronteremo con i risultati attesi;
\item tramite opportune query di prova effettueremo test sull'utilizzo del \underline{DBMS};
\item attraverso opportuni driver da noi progettati e sviluppati, testeremo il corretto funzionamento del \underline{DBMS} e del validatore;
\item utilizzando stub per quanto riguarda la Gui. 
\end{itemize}

\chapter[Gestione revisione]{Gestione amministrativa della revisione} 
\section{Processo di ispezione}
Le ispezioni del codice e dei documenti verranno eseguite da una squadra di almeno 4 persone che, analizzeranno sistematicamente il codice e ne individueranno i possibili difetti. Parteciperanno alla riunione di ispezione: 
\begin{itemize}
\item[-]L'autore del documento ispezionato;
\item[-]i membri del gruppo tenuti a ispezionare il codice;
\item[-]il moderatore capo;
\item[-]il segretario che prender\`a nota degli errori scovati.
\end{itemize}
In fase iniziale l'autore del codice presenter\`a alla squadra di ispezione il funzionamento dello stesso. Ognuno dei membri della squadra sar\`a poi tenuto a studiare il codice, al fine di individuarne difetti ed errori; i difetti individuati verranno poi annotati dal segretario durante la riunione di ispezione. 
\section{Comunicazione e risoluzione di anomalie}
Il documento redatto dal segretario dovr\`a elencare le segnalazioni in modo che sia possibile individuare il punto errato in modo semplice e non ambiguo, eventualmente citando le parti non corrette. In dettaglio il documento sar\`a costituito da una tabella contenente:
\begin{itemize}
\item Un descrizione che identifichi in modo univoco l'errore (riga del codice, nome e revisione del file);
\item i passi che hanno portato al verificarsi dell'errore;
\item la gravit\`a dell'errore ed il tempo in cui deve essere corretto.
\end{itemize}
L'autore del codice sottoposto a ispezione, preso atto dei difetti individuati nello stesso, sar\`a tenuto a correggerli nei tempi indicati presentando un documento in cui indica le modifiche apportate.

\section{Trattamento delle discrepanze} 
Se dovessero verificarsi discrepanze tra le necessit\`a del cliente ed i requisiti risultanti dall'analisi, si provveder\`a ad un ulteriore analisi che avr\`a la priorit\`a sulle altre attivit\`a. Verr\`a aggiornato quindi il documento relativo all' ``Analisi dei Requisiti'' e  i documenti/pezzi di codice da esso dipendenti. In ogni caso, se fosse necessario un cambiamento, si dovr\`a prima di tutto tracciarne l'impatto sugli altri requisiti e sul progetto del sistema, valutando l'effetto della modifica proposta. Il costo della modifica verr\`a stimato in base a quanti cambiamenti dovranno essere fatti al documento dei requisiti e, se opportuno, al progetto del sistema e alla sua implementazione. Completata tale analisi, si dovr\`a decidere se procedere o meno con la modifica. Sar\`a quindi compito dell'analista, in collaborazione col progettista e l'amministratore, valutare questo impatto sul lavoro gi\`a svolto e comunicare ai membri interessati le variazioni.

\section{Procedure di controllo di qualit\`a di processo}
Il controllo della qualit\`a del software interessa l'intero processo di sviluppo del prodotto in questione. A tale scopo sono stati adottati degli standard di documentazione che regolano la struttura e la presentazione dei documenti, nonch\`e degli standard di codifica e di processo. Quest'ultimi definiscono i processi da seguire durante tutto lo sviluppo software; includono definizione dei processi di specifica, progettazione e convalida, oltre ad una descrizione dei documenti che dovrebbero essere scritti durante l'esecuzione di questi processi. Il controllo di qualit\`a prevede il monitoraggio del processo software, per garantire che le procedure e gli standard di qualit\`a siano seguiti. Esso comprende inoltre il miglioramento del processo e la soluzione degli eventuali problemi. Quest'ultimo consiste nel comprendere i processi esistenti e modificarli per aumentare la qualit\`a del software e/o diminuire i costi e i tempi di sviluppo. Il suo obiettivo principale \`e quello di concentrarsi sul perfezionamento, per migliorare la qualit\`a del prodotto e, in particolare, per ridurre il numero di difetti nel software consegnato. Una volta ottenuto ci\`o, gli obiettivi primari diventeranno la riduzione dei costi e dei tempi.
Per avere un esteso controllo degli errori, le modifiche apportate ai documenti, in seguito a inconsistenze riscontrate durante la fase di verifica e stesura, verranno notificate secondo alcune convenzioni interne illustrate nel documento ``NormeDiProgetto''. Queste modifiche si troveranno all'inizio di ogni documento nella sezione ``Diario delle modifiche''. I verificatori sono tenuti a controllare parallelamente la documentazione sul lavoro svolto.

\chapter{Resoconto attivit\`a di verifica}
\section{Tracciamento componenti-requisiti}
Per una completa e chiara tracciabilit\`a componenti-requisiti, ogni componente, classe o metodo del prodotto ``\underline{BR-jsys}'' \`e stato ideato come risposta ad ogni singolo requisito. Al fine di tenere traccia di tali corrispondenze \`e stata redatta una tabella, consultabile nel documento ``Specifica Tecnica'', che identifica per ogni componente quali sono i requisiti che la interessano. Ci\`o faciliter\`a la fase di verifica; si attester\`a quindi se ogni componente soddisfi pienamente i requisiti ad esso associato.
\section{Descrizione prove sui Requisiti}
In questa sezione intendiamo elencare la descrizione dei vari test che successivamente effettueremo sui requisiti.

\subsubsection{Requisito F1}
\begin{tabular}{||p{4cm}||p{8cm}||}
\hline
{\textbf {Obiettivo Prova:}}& Creare un linguaggio per le \underline{business rules} \\ \hline
{\textbf{Dipendenze:}}& nessuna \\ \hline
{\textbf{Descrizione Prova:}}& Rappresentazione di una \underline{business rule} attraverso il linguaggio creato \\ \hline
{\textbf{Input:}}&  \underline{Business rule} in linguaggio naturale \\ \hline
{\textbf{Output:}}& \underline{Business rule} scritta nel linguaggio adottato \\ \hline
\end{tabular} \\
\\
\\
\subsubsection{Requisito F2}
\begin{tabular}{||p{4cm}||p{8cm}||}
\hline
{\textbf {Obiettivo Prova:}}& Testare la correttezza sintattica di ogni \underline{business rule} \\ \hline
{\textbf{Dipendenze:}}& F1 \\ \hline
{\textbf{Descrizione Prova:}}& Inserimento di \underline{business rules} d'appoggio (sintatticamente corrette o meno), per testarne il loro stato di accettazione \\ \hline
{\textbf{Input:}}& \underline{Business Rule} di appoggio \\ \hline
{\textbf{Output:}}& La stringa accettata o meno dal validatore \\ \hline
\end{tabular} \\
\\
\\
\subsubsection{Requisito F3}
\begin{tabular}{||p{4cm}||p{8cm}||}
\hline
{\textbf {Obiettivo Prova:}}& Testare il corretto inserimento delle \underline{business rules} nel repository\\ \hline
{\textbf{Dipendenze:}}& F2 \\ \hline
{\textbf{Descrizione Prova}}&  Inserimento di varie \underline{business rules} d'appoggio gi\`a validate e controllo all'interno del repository dell'avvenuto inserimento\\ \hline
{\textbf{Input:}}& \underline{Business Rule} (validata) di appoggio \\ \hline
{\textbf{Output:}}& Lo stato dell'inserimento della \underline{business rule} attraverso la notifica rilasciata (positivo: inserimento avvenuto con successo, negativo: eccezione) \\ \hline
\end{tabular} \\
\\
\\
\subsubsection{Requisito F4}
\begin{tabular}{||p{4cm}||p{8cm}||}
\hline
{\textbf {Obiettivo Prova:}}& Informare l'utente dell'avvenuta o meno validazione delle \underline{business rules} appena inserite e del tempo trascorso, attraverso una notifica il pi\`u possibile chiara ed esaustiva\\ \hline
{\textbf{Dipendenze:}} & F2, F3, NU3 \\ \hline
{\textbf{Descrizione Prova:}}& Inserimento di varie \underline{business rules} d'appoggio (corrette o meno) e controllo della corretta notifica di risposta alla richiesta di inserimento (attraverso timer nel codice) \\ \hline
{\textbf{Input:}}& \underline{Business Rules} (corrette o meno) di appoggio  \\ \hline
{\textbf{Output:}}& Una notifica per ogni \underline{business rule} che si \`e provato ad inserire, con il relativo messaggio riguardante il tempo impiegato \\ \hline
\end{tabular} \\
\\
\\
\subsubsection{Requisito F5}
\begin{tabular}{||p{4cm}||p{8cm}||}
\hline
{\textbf {Obiettivo Prova:}}& Testare l'efficienza in termini di tempo di risposta del \underline{DBMS} di appoggio al repository \\ \hline
{\textbf{Dipendenze:}}& F2, F3, F4\\ \hline
{\textbf{Descrizione Prova:}}& Inserimento/eliminazione/modifica di varie \underline{business rules} d'appoggio (sintatticamente corrette o meno) e valutazione del tempo di risposta impiegato dal \underline{DBMS}. Tale valore viene visualizzato dalla Gui. \\ \hline
{\textbf{Input:}}& \underline{Business Rules} (corrette o meno) di appoggio  \\ \hline
{\textbf{Output:}}& Tempo impiegato \\ \hline
\end{tabular} \\
\\
\\
\subsubsection{Requisito F6}
\begin{tabular}{||p{4cm}||p{8cm}||}
\hline
{\textbf {Obiettivo Prova:}}& Il sistema deve essere in grado di interfacciarsi con l'interprete esterno, fornendo ad esso le \underline{business rules} da eseguire \\ \hline
{\textbf{Dipendenze:}}& nessuna \\ \hline
{\textbf{Descrizione Prova:}}&  Recuperare dal repository tutte le \underline{business rules} associate ad un determinato \underline{business object}\\ \hline
{\textbf{Input:}}&  \underline{Business object} (in stringa) \\ \hline
{\textbf{Output:}}& Le \underline{business rules} associate\\ \hline
\end{tabular} \\
\\
\\
\subsubsection{Requisito F7}
\begin{tabular}{||p{4cm}||p{8cm}||}
\hline
{\textbf {Obiettivo Prova:}}& Testare la correttezza della cancellazione di una \underline{business rule} dal repository e controllo dei risultati ottenuti\\ \hline
{\textbf{Dipendenze:}}& nessuna \\ \hline
{\textbf{Descrizione Prova:}}&  Richiesta al \underline{DBMS} di cancellare varie \underline{business rules} date in input dal repository. Si tester\`a la correttezza dell'eliminazione fatta. Ogni eliminazione riuscita o meno dovr\`a essere accompagnata da una chiara notifica di evento\\ \hline
{\textbf{Input:}}& \underline{Business Rules} da eliminare \\ \hline
{\textbf{Output:}}& Notifica di eliminazione (positivo: cancellazione avvenuta con successo, negativo: eccezione) \\ \hline
\end{tabular} \\
\\
\\
\subsubsection{Requisito F8}
\begin{tabular}{||p{4cm}||p{8cm}||}
\hline
{\textbf {Obiettivo Prova:}}& Creare interfaccia per l'inserimento/rimozione di \underline{Business Rules} e verificare il corretto funzionamento\\ \hline
{\textbf{Dipendenze:}}& F3, F4, F7 \\ \hline
{\textbf{Descrizione Prova:}}& Si creer\`a  una semplice interfaccia in linguaggio Java, con le componenti essenziali per permettere l' inserimento/rimozione di una \underline{business rule} nel repository e la successiva notifica di tale evento \\ \hline
{\textbf{Input:}}& \underline{Business Rules} da eliminare/inserire \\ \hline
{\textbf{Output:}}& Notifica di eliminazione o inserimento (avvenuta o meno) \\ \hline
\end{tabular} \\
\\
\\
\subsubsection{Requisito F9}
\begin{tabular}{||p{4cm}||p{8cm}||}
\hline
{\textbf {Obiettivo Prova:}}& REQUISITO DEPRECATO\\ \hline
{\textbf{Dipendenze:}}& \\ \hline
{\textbf{Descrizione Prova:}}& \\ \hline
{\textbf{Input:}}& \\ \hline
{\textbf{Output:}}&  \\ \hline
\end{tabular} \\
\\
\\
\subsubsection{Requisito F10}
\begin{tabular}{||p{4cm}||p{8cm}||}
\hline
{\textbf {Obiettivo Prova:}}& Consentire al validatore di accedere ai campi del \underline{business object} associato alla \underline{business rule} da validare \\ \hline
{\textbf{Dipendenze:}}& F11\\ \hline
{\textbf{Descrizione Prova:}}&  Inserimento di varie \underline{business rules} d'appoggio, che si riferiscono ai campi dati del \underline{business object} associato, e testare la correttezza della validazione effettuata \\ \hline
{\textbf{Input:}}&  \underline{Business rules} d'appoggio \\ \hline
{\textbf{Output:}}& Esito della validazione \\ \hline
\end{tabular} \\
\\
\\
\subsubsection{Requisito F11}
\begin{tabular}{||p{4cm}||p{8cm}||}
\hline
{\textbf {Obiettivo Prova:}}& Testare che ad ogni \underline{business rule} sia associato un \underline{business object} \\ \hline
{\textbf{Dipendenze:}}& nessuna \\ \hline
{\textbf{Descrizione Prova:}}&  Nella Gui ogni \underline{business rule} ha un campo dati per il \underline{business object} associato. Se tale campo (``associated Object'') viene lasciato vuoto verr\`a lanciata un'eccezione dal sistema\\ \hline
{\textbf{Input:}}&  \underline{Business rule} di appoggio (con campo dati ``associated Object'' pieno o vuoto)\\ \hline
{\textbf{Output:}}& Una eccezione se il campo dati \`e stato lasciato vuoto, altrimenti niente \\ \hline
\end{tabular} \\
\\
\\
\subsubsection{Requisito NU1}
\begin{tabular}{||p{4cm}||p{8cm}||}
\hline
{\textbf {Obiettivo Prova:}}& Creare un linguaggio di alto livello, facile da capire per un utente con scarse conoscenze informatiche\\ \hline
{\textbf{Dipendenze:}}& F1\\ \hline
{\textbf{Descrizione Prova:}}& Creazione di \underline{business rules} complesse attraverso semplici inserimenti da parte dell'utente\\ \hline
{\textbf{Input:}}& \underline{Business rule} di appoggio \\ \hline
{\textbf{Output:}}& \underline{Business rule} scritta nel linguaggio adottato \\ \hline
\end{tabular} \\
\\
\\
\subsubsection{Requisito NU2}
\begin{tabular}{||p{4cm}||p{8cm}||}
\hline
{\textbf {Obiettivo Prova:}}& Il linguaggio deve fornire funzioni primitive per le operazioni base tra i dati, come la somma, la media aritmetica e la negazione logica \\ \hline
{\textbf{Dipendenze:}}& F1 \\ \hline
{\textbf{Descrizione Prova:}}&  Creazione di \underline{business rules} contenenti diverse operazioni base. Se non vengono accettate tali operazioni viene lanciata un'eccezione \\ \hline
{\textbf{Input:}}&  \underline{Business rule} di appoggio contenenti operazioni base \\ \hline
{\textbf{Output:}}& \underline{Business rule} scritta nel linguaggio adottato \\ \hline
\end{tabular} \\
\\
\\
\subsubsection{Requisito NU3}
\begin{tabular}{||p{4cm}||p{8cm}||}
\hline
{\textbf{Obiettivo Prova:}}& Verificare la chiarezza e completezza dei messaggi d'errore\\ \hline
{\textbf{Dipendenze:}}& nessuna \\ \hline
{\textbf{Descrizione Prova:}}& Si tenter\`a di validare vari \underline{business objects} (con errori interni e di ogni genere possibile), valutando l'efficacia del messaggio d'errore \\ \hline
{\textbf{Input:}}&  Business object (in stringa) \\ \hline %%%%%%%%%%%%%%%%%%%%%%%%%%%% CONTROLLARE
{\textbf{Output:}}& Messaggio d'errore comprensibile \\ \hline
\end{tabular} \\
\\
\\
\subsubsection{Requisito NPo1}
\begin{tabular}{||p{4cm}||p{8cm}||}
\hline
{\textbf{Obiettivo Prova:}}& Il formato di salvataggio delle \underline{business rules} e del repository dovr\`a essere XML.\\ \hline
{\textbf{Dipendenze:}}& nessuna \\ \hline
{\textbf{Descrizione Prova:}}&  Creazione e salvataggio di \underline{business rules} in formato XML  \\ \hline
{\textbf{Input:}}&  \underline{Business rule} di appoggio in XML \\ \hline
{\textbf{Output:}}& L'inserimento della \underline{business rule} in formato XML nel repository \\ \hline
\end{tabular} \\
\\
\\
\subsubsection{Requisito NPr1}
\begin{tabular}{||p{4cm}||p{8cm}||}
\hline
{\textbf{Obiettivo Prova:}}& Testare la velocit\`a di interrogazione del \underline{DBMS}\\ \hline
{\textbf{Dipendenze:}}& nessuna \\ \hline
{\textbf{Descrizione Prova:}}& Popolamento del repository attraverso inserimento di \underline{business rules} d'appoggio e interazioni con esso. Si valuter\`a l'efficienza, in termini di tempo di risposta, del \underline{DBMS} usato  \\ \hline
{\textbf{Input:}}& \underline{Business rule} \\ \hline
{\textbf{Output:}}& Valutazione del tempo impiegato in ogni interrogazione \\ \hline
\end{tabular} \\
\\
\\
\subsubsection{Requisito NPr2}
\begin{tabular}{||p{4cm}||p{8cm}||}
\hline
{\textbf{Obiettivo Prova:}}& Rendere univoca l'identificazione delle \underline{business rules} all'interno del repository \\ \hline
{\textbf{Dipendenze:}}& nessuna \\ \hline
{\textbf{Descrizione Prova:}}& Inserimento di varie \underline{business rules} d'appoggio nel repository e controllo dell'unicit\`a della loro indicizzazione \\ \hline
{\textbf{Input:}}& Business object (in stringa) \\ \hline 
{\textbf{Output:}}& L'esito del controllo\\ \hline
\end{tabular} \\
\\
\\
\subsubsection{Requisito NQ1}
\begin{tabular}{||p{4cm}||p{8cm}||}
\hline
{\textbf{Obiettivo Prova:}}& Testare tutte le operazioni consentite dal linguaggio creato e la loro validazione\\ \hline
{\textbf{Dipendenze:}}& F1, F3, F7 \\ \hline
{\textbf{Descrizione Prova:}}& Si applicheranno a \underline{business objects} esistenti tutte le operazioni che il linguaggio permette e si valutera\`a l'esito della validazione di tali operazioni, confrontandoli con i risultati aspettati.  \\ \hline
{\textbf{Input:}}& \underline{Business Object}s \\ \hline
{\textbf{Output:}}& L'esito della validazione delle operazioni tra i \underline{business objects} dati in input\\ \hline
\end{tabular} \\
\\
\\
\subsubsection{Requisito NQ2}
\begin{tabular}{||p{4cm}||p{8cm}||}
\hline
{\textbf {Obiettivo Prova:}}& Testare che il manuale descriva il linguaggio di definizione delle \underline{business rules} \\ \hline
{\textbf{Dipendenze:}}&  F1, F8 \\ \hline
{\textbf{Descrizione Prova:}}&  Creazione di un manuale utente di facile comprensione per qualsiasi tipo di utente \\ \hline
{\textbf{Input:}}&  Richieste di informazioni \\ \hline  
{\textbf{Output:}}& Manuale utente completo e allo stesso tempo semplice di ogni componente del linguaggio di definizione delle \underline{business rules} e delle funzioni della Gui \\ \hline
\end{tabular} \\
\\
\\
\subsubsection{Requisito NQ3}
\begin{tabular}{||p{4cm}||p{8cm}||}
\hline
{\textbf {Obiettivo Prova:}}& Descrivere le API relative al validatore e al \underline{DBMS}\\ \hline
{\textbf{Dipendenze:}}& nessuna \\ \hline
{\textbf{Descrizione Prova:}}&  Creazione di una descrizione dettagliata delle API relative al validatore e al \underline{DBMS}\\ \hline
{\textbf{Input:}}&  Richieste di informazioni \\ \hline  
{\textbf{Output:}}& Il documento completo e semplice con la descrizione delle API  \\ \hline
\end{tabular} \\


\end{document}












