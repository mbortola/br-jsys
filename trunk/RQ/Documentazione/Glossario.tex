
\documentclass[11pt,titlepage,a4paper]{report}

\usepackage[italian]{babel}
\usepackage{fancyhdr}
\usepackage{graphicx}
\usepackage{hyperref}

%\usepackage{lastpage} % total page count

\usepackage{color}
\usepackage{lastpage} % total page count

\graphicspath{{./pics/}} % cartella di salvataggio immagini

\pagestyle{fancy}
\renewcommand{\sectionmark}[1]{\markright{\thesection.\ #1}}
\lhead{\nouppercase{\rightmark}}
\rhead{\nouppercase{\leftmark}}
\renewcommand{\chaptermark}[1]{%
\markboth{\thechapter.\ #1}{}}


\fancypagestyle{plain}{%
	\lhead{\includegraphics[height=50pt]{logo.eps}}
	\chead{}
	\rhead{HappyCode inc \\ happycodeinc@gmail.com}
	\lfoot{BR-jsys}
	\cfoot{\thepage\ / \pageref{LastPage}}
	\rfoot{\dt - \lv}
	\renewcommand{\headrulewidth}{1pt}
	\renewcommand{\footrulewidth}{1pt}
}
	\lhead{\includegraphics[height=50pt]{logo.eps}}
	\chead{}
	\rhead{HappyCode inc \\ happycodeinc@gmail.com}
	\lfoot{BR-jsys}
	\cfoot{\thepage\ / \pageref{LastPage}}
	\rfoot{\dt - \lv}
	\renewcommand{\headrulewidth}{1pt}
	\renewcommand{\footrulewidth}{1pt}

\hypersetup{
    colorlinks=true,       % false: boxed links; true: colored links
   linkcolor=[rgb]{0.11,0.55,0.83},          % color of internal links
    urlcolor=cyan           % color of external links
}
\definecolor{err}{rgb}{0.9,0.1,0.1}

% fine layout% layout
\begin{document}
%definizione variabili 
\newcommand{\lv}{ 1.9 } % latest version
\newcommand{\dt}{ Glossario }% Document title
%common variables
\newcommand{\br}{\underline{business rule}}
\newcommand{\brs}{\underline{business rules}}
\newcommand{\bo}{\underline{business object}}
\newcommand{\bos}{\underline{business objects}}
\newcommand{\rp}{\underline{repository}}
\newcommand{\brp}{BusinessRuleParser}
\newcommand{\brl}{BusinessRuleLexer}
\newcommand{\BR}{\underline{BusinessRule}}

%nomi dei componenti
\newcommand{\AT}{Alessia Trivellato}
\newcommand{\ET}{Elena Trivellato}
\newcommand{\FC}{Filippo Carraro}
\newcommand{\LA}{Luca Appon}
\newcommand{\MB}{Michele Bortolato}
\newcommand{\MT}{Marco Tessarotto}
\newcommand{\MM}{Mattia Meroi}%altre variabili
% ultime versioni dei documenti da modificare solo alla fine
\newcommand{\AR}{AnalisiDeiRequisiti.2.6.pdf}
\newcommand{\DdP}{DefinizioneDiProdotto.0.9.pdf}
\newcommand{\G}{ Glossario.1.8.pdf }
\newcommand{\NdP}{NormeDiProgetto.2.0.pdf}
\newcommand{\PdQ}{ PianoDiQualifica.1.4.pdf }
\newcommand{\PdP}{ PianoDiProgetto.1.7.pdf }
\newcommand{\ST}{SpecificaTecnica.1.5.pdf}
\newcommand{\TR}{TestReport.0.7.pdf}
\newcommand{\MU}{ManualeUtente.0.3.pdf}%nomi documenti
%fine definizione variabili
\hyphenation{
 a-na-lo-go
 as-so-cia-zio-ne
 %attività non si può inserire come tutte le parole accentate che vanno messe nel testo semplice scritte at\-ti\-vi\-tà o come variabile
 coe-ren-za
 com-po-nen-ti
 con-si-glia-bi-le
 des-crit-te
 des-cri-zio-ni
 di-a-gram-ma
 di-a-gram-mi
 e-le-men-to
 e-se-gui-re
 e-si-sten-ti
 es-pli-ci-to
 glo-bal-men-te
 glos-sa-rio
 in-se-ri-men-to
 li-vel-lo
 ne-ces-sa-rio
 per-met-te-re
 re-po-si-to-ry
 re-vi-sio-na-men-to
 ri-chies-te
 se-le-zio-na-ta
 se-gna-la-ta
 va-li-da-zio-ne
 va-ria-bi-li
 ve-ri-fi-ca-re
 vi-sua-liz-za-te
 e-ven-tua-li
 o-pe-ra-zio-ne
 ar-chi-via-zio-ne
 mo-di-fi-ca
 ar-chi-vio
 des-cri-zio-ne
 pa-ren-te-si
 i-ni-zia
}


%sillabazione

\begin{titlepage}\begin{center}
\vspace*{0.5in}
\includegraphics{logo.eps}
\vspace*{0.2in} \\
{\Large \textbf{BR-jsys}}
{\Large \emph{business rules} per sistemi gestionali in architettura J2EE } 
\vspace{2in} \\
\Huge \textsc{ \dt }
\par\rule{10cm}{0.4pt} \par {\large Versione \lv - \today} \\
\end{center}\end{titlepage}

\vspace*{0.5in}%pagina del titolo


\begin{center}
\thispagestyle{plain}
\begin{table}[htbp]
\large{
\begin{tabular}{l}
\Large{\textbf{\textsf{Capitolato: ''BR-jsys``}}} \\
\begin{tabular}{|p{6cm}|p{6cm}|}
\hline
\textbf{Data creazione:} & 03/12/2007 \\ \hline
\textbf{Versione:} & \lv \\ \hline
\textbf{Stato del documento:} & Formale, esterno \\ \hline
% ----------------------------------------------------------------------------autori
\textbf{Revisione RR} & \\ \hline
\textbf{Redazione:} & \MB \\ \hline
\textbf{Revisione:} & \MT \\ \hline
\textbf{Approvazione:}  & \ET \\ \hline
\textbf{Revisione RPD} & \\ \hline
\textbf{Redazione:} & \LA \\ \hline
\textbf{Revisione:} & \AT, \MB  \\ \hline
\textbf{Approvazione:}  & \MB \\ \hline
\end{tabular} \\
\end{tabular}
}
\end{table}

\begin{table}[hbtp]
\large{
\begin{tabular}{l}
\Large{\textbf{\textsf{Lista di distribuzione}}} \\
\begin{tabular}{|p{6cm}|p{6cm}|} \hline
%  -------------------------------------------------------------lista di distribuzione
{HappyCode inc}& Gruppo di lavoro \\ \hline
{Tullio Vardanega, Renato Conte}& Committenti \\ \hline 
{Zucchetti S.r.l}& Azienda proponente\\ \hline
\end{tabular} \\
\end{tabular}
}
\end{table}
\begin{table}[hbtp]

\Large{\textbf{\textsf{Diario delle modifiche}}} \\
\begin{small}
\begin{tabular}[t]{|p{1,2cm}|p{1.9cm}|p{2.9cm}|p{5cm}|} \hline
Versione & Data & Autore & Descrizione \\ \hline
%-------------------------------------------------------------------------------diario modifiche
2.0 & 04/03/2008 & \MM & Modifica al layout, inserimento dei collegamenti ipertestuali.\\ \hline
1.9 & 04/03/2008 & \MT & Modifica al layout, introduzione totale pagine e autori nel diario delle modifiche.\\ \hline
1.8 & 15/02/2008 & \MB & Inserimento di alcuni riferimenti e nuovi termini.\\ \hline
1.7 & 13/02/2008 & \MM & Inserimento di XQuery, XPath, Sandbox, FLWOR.\\ \hline
1.6 & 11/02/2008 & \LA & Inserimento di AST e Token.\\ \hline
1.5 & 05/02/2008 & \MT & Aggiunta del nome del file nel modello di documento.\\ \hline
1.4 & 04/02/2008 & \AT & Correzione del documento.\\ \hline
1.3 & 24/01/2008 & \LA & Inserimento di Antlr ,eXist, Use Case Diagram, Database, DBMS, JUnit, Stakeholder, Subversion, UML.\\ \hline
1.2 & 22/01/2008 & \MT & Modifica al layout dei documenti.\\ \hline
1.1 & 09/01/2008 & \LA & Aggiunto Middleware. Cancellati perch\`e non utilizzati i termini Calculator, Java Reflection, Oggetto Master, Oggetto Detail, Oggetto Master Detail, Projector.\\ \hline
1.0 & 21/12/2007 & \MT & Documento sottoposto a revisionamento automatico.\\ \hline
0.4 & 2007/12/05 & \MT & Correzione del documento. \\ \hline
0.3 & 2007/12/05 & \MB & Inserimento documento nel modello unico di layout. \\ \hline
0.2 & 2007/12/04 & \MB & Aggiunta nuovi termini. \\ \hline
0.1 & 2007/12/03 & \MT & Stesura preliminare del documento. \\ \hline
\end{tabular} \\
\end{small}


\end{table}
\end{center}
\newpage
\tableofcontents

\chapter{Introduzione}
\section{Scopo del documento}
Il seguente documento ha lo scopo di fornire una breve lista dei termini, usati nei vari documenti, che possono sembrare ambigui oppure necessitano di una definizione o di una migliore spiegazione. In alcuni casi, per maggiori informazioni, verranno messi a disposizione dei collegamenti web.
\section{Riferimenti}
\begin{itemize}
\item Sommerville Ian - Ingegneria del software
\item Design Patterns - E. Gamma, R. Helm, R. Johnson, J. Vlissides
\item UML2 and the Unified Process - Jim Arlow, Ila Neustadt
\item http://en.wikipedia.org
\item http://exist.sourceforge.net/
\item http://www.uml.org/
\item http://java.sun.com/
\item http://www.w3.org/XML/
\item Altre Risorse Web


\end{itemize}
\chapter{A}
\section{Account Gmail:}
Account personale fornito gratuitamente da Google e attivato da ognuno dei membri della HappyCodeInc.
\section{Albero sintattico:}
\`E una struttura che interpreta un'espressione utilizzando una grammatica CFG.
\section{Antlr:}
``Another Tool for Language Recognition''. \`E uno strumento generatore di parser e traduttori che permette di definire grammatiche nella sintassi Antlr.
\section{Architettura J2EE:}
Vedi \hyperlink{J2EE}{J2EE}.
\section{AST:} 
``Abstract Syntax Tree''.\`E una struttura che racchiude le stesse informazioni degli alberi sintattici, eliminandone per\`o l'informazione ridondante.

\chapter{B}
\section{BR-jsys:}
Per BR-jsys si intende un sistema software per la gestione di business rules in ambito gestionale.
\hypertarget{Business Object}{}
\section{Business Object:}
Anche chiamato Oggetto Business. \`E un oggetto, in un linguaggio di programmazione orientato agli oggetti, che astrae le entit\`a solitamente presenti in un database relazionale.
Per esempio, un acquisto ha bisogno di un acquirente, di un fornitore e di merci di scambio. Questi possono essere rappresentati in un database relazionale tramite entit\`a o in un programma sviluppato con un linguaggio orientato agli oggetti (come java o C++) tramite business objects.
\hypertarget{Business rule}{}
\section{Business rule:}
Anche chiamata Regola Business. \`E una regola atta a definire dei vincoli particolari all'interno della gestione di un azienda. Un esempio di Business Rule potrebbe essere questo: ``a tutti i prodotti con prezzo superiore a 600 euro  viene applicato uno sconto dell' 5\%''.

\chapter{C}
\section{CFG:}
Grammatica libera da contesto (context-free). 
\section{Classe wrapper:}
Vedi definizione di \hyperlink{wrapper}{``wrapper''}.

\chapter{D}
\section{Database:}
Archivio di dati riguardanti uno stesso argomento o pi\`u argomenti correlati tra loro, strutturato in modo tale da consentire la gestione dei dati stessi da parte di applicazioni software.
\section{DBMS:}
``Database Management System''. \`E un sistema software progettato per consentire la creazione e la manipolazione efficiente di database, solitamentente da parte di pi\`u utenti.
\section{Discussioni:}
Risorsa disponibile su Google Groups. Permette lo scambio di pareri (similmente a un forum) all'interno del gruppo stesso.
\section{Driver:}
Componente attiva fittizia che serve per pilotare un modulo.

\chapter{E}
\section{eXist:}
eXist-db \`e un DBMS open source interamente basato sulla tecnologia XML. Salva i dati in formato XML e permette un' interrogazione molto efficiente del database grazie a \hyperlink{XQuery}{XQuery}.

\chapter{F}
\hypertarget{FLWOR}{}
\section{FLWOR:}
XQuery utilizza una serie di istruzioni che possono essere raccolte ed identificate con la sigla "FLOWR" dove ogni lettera corrisponde alla lettera iniziale di ogni istruzione:\\
For \\
Let \\
Order by \\
Where \\
Return \\
XQuery permette di fare query su dati semi-strutturati e/o destrutturati come XML. Pu\`o lavorare direttamente sull' XML Data Type, sia come variabile che come colonna, utilizzando sia query XPath che FLWOR.

\chapter{G}
\hypertarget{Google Groups}{}
\section{Google Groups:}
Vedi \hyperlink{Spazio Web in Google Groups}{Spazio Web in Google Groups}.

\chapter{J}
\section{JUnit:}
\`E un unit test framework per il linguaggio di programmazione Java.
\hypertarget{J2EE}{}
\section{J2EE:}
Il termine J2EE \`e l'acronimo di Java 2 Enterprise Edition, ossia la versione enterprise della piattaforma Java utilizzata per lo sviluppo di applicazioni server.

\chapter{M}
\section{Mail del gruppo:}
L'indirizzo di posta elettronica \textit{happycodeinc@gmail.com} creata appositamente come supporto allo sviluppo del progetto.
\section{Middleware:}
Programma informatico che funge da intermediario tra diverse applicazioni.

\chapter{O}
\section{Oggetto Business:}
Vedi \hyperlink{Business Object}{Business Object}.

\chapter{P}
\section{Pagine Statiche:}
Risorsa disponibile su Google Groups. Permette di definire documenti modificabili e consultabili da parte di tutto il gruppo.

\chapter{R}
\section{Regola Business:}
Vedi \hyperlink{Business Rule}{Business Rule}.
\section{Rejector:}
Tipologia di business rule che consente di bloccare un operazione se i dati analizzati non sono accettabili.
\section{Repository:} 
Per repository si intende un archivio (nel nostro caso un file in linguaggio XML) in cui vengono memorizzati e mantenuti dei dati.
\section{Revisione}
La revisione, numero intero che parte da 1, indica se il documento \`e stato aggiornato dal suo ultimo rilascio. Un aumento nell'indice di revisione indica che il documento \`e stato modificato e/o corretto in una delle sue parti.

\chapter{S}
\section{Sandbox:}
Identifica un ambiente in cui si possono effettuare sperimentazioni (nel nostro caso mediante il linguaggio \hyperlink{XQuery}{XQuery}) che hanno lo scopo di investigare sugli effetti di eventuali modifiche o sviluppi all' interno del database. Tali sperimentazioni hanno inoltre lo scopo di favorire un rapido apprendimento del linguaggio \hyperlink{XQuery}{XQuery}.
\hypertarget{Spazio web in Google Groups}{}
\section{Spazio web in Google Groups:}
Insieme delle risorse messe a disposizione da Google Groups. Sono raggiungibili dai soli membri della HappyCodeInc all'URL\\ \textit{http://groups.google.com/group/happycodeinc}. \`E comprensivo delle seguenti risorse: spazio web per la condivisione di file, gestione delle discussioni, gestione delle pagine statiche.
\section{Stakeholder:}
Soggetti ``portatori di interessi'' nei confronti di un'iniziativa economica, sia essa un'azienda o un progetto. Fanno, ad esempio, parte di questo insieme: i clienti, i fornitori, i finanziatori, i collaboratori, ma anche gruppi di interesse esterni, come i residenti di aree limitrofe all'azienda o gruppi di interesse locali.
\section{Stub:}
Componente passiva fittizia per simulare un modulo.
\section{Svn:}
``Subversion'' \`e un sistema di controllo versione automatico. 

\chapter{T}
\section{Token:}
Un blocco di testo categorizzato, scritto in qualsiasi linguaggio e normalmente costituito da caratteri indivisibili chiamati lessemi. L'analXQueryizzatore lessicale legge in un flusso di lessemi e li categorizza in token: se esso trova un token non valido, restituisce un errore.

\chapter{U}
\section{UML:}
``Unified Modeling Language'' (linguaggio di modellazione unificato). \`E un linguaggio di modellazione e specifica basato sul paradigma object-oriented.
\section{Use Case Diagram:}
In UML, gli Use Case Diagram (UCD o diagrammi dei casi d'uso) sono diagrammi dedicati alla descrizione delle funzioni o servizi offerti da un sistema, cos\`i come sono percepiti e utilizzati dagli attori che interagiscono col sistema stesso. 

\chapter{V}
\section{Versione:}
La versione, numero intero che parte da 0, serve a stabilire se il documento \`e stato modificato dal suo ultimo rilascio. Un aumento nell'indice della versione indica l'aggiunta di una nuova sezione al documento.

\chapter{W}
\hypertarget{Wrapper}{}
\section{Wrapper:}
Nel package java.lang sono definite delle classi wrapper, che servono da involucro per i tipi primitivi, e permettono di trattarli come oggetti. Un oggetto della classe wrapper avvolge un singolo valore primitivo, che non pu\`o essere pi\`u modificato.

\chapter{X}
\hypertarget{XQuery}{}
\section{XQuery:}
``XML Query Language'' \`e un linguaggio destinato ad interrogare documenti e basi di dati XML.
XQuery usa la sintassi delle espressioni di \hyperlink{XPath}{XPath} per la selezione di specifiche porzioni di documenti XML, con l'aggiunta delle cosiddette espressioni \hyperlink{FLWOR}{FLWOR} per la formulazione di query complesse.
\hypertarget{XPath}{}
\section{XPath:}
\`E un linguaggio parte della famiglia XML che permette di individuare i nodi all'interno di un documento XML. Le espressioni XPath, a differenza delle espressioni XML, non servono a identificare la struttura di un documento, bens\`i a localizzarne con precisione i nodi.


\end{document}
